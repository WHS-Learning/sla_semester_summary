\renewcommand{\abstractname}{Einleitung}
\begin{abstract}

Die vorliegende Aufgabensammlung wurde von Frau Prof. Dr. Anderle dankenswerterweise 
zur Verfügung gestellt. Sie dient der Vertiefung und Anwendung der 
Inhalte der Vorlesung \textit{Stochastik und Lineare Algebra}.

Die Aufgaben decken alle für die Klausur 2025 relevanten Themen ab und sollen Ihnen die 
Möglichkeit geben, Ihr Verständnis zu festigen und sich gezielt auf die Prüfung vorzubereiten.

Obwohl die Aufgaben mit Sorgfalt ausgewählt und aufbereitet wurden, kann \textbf{für die vollständige 
Richtigkeit und Fehlerfreiheit keine Gewähr übernommen werden}. Wir empfehlen, die Lösungswege 
kritisch zu reflektieren und bei Unklarheiten Rücksprache zu halten.

Diese Aufgabensammlung dient als Hilfestellung. Sie sollten versuchen, die Aufgaben 
selbstständig und nur mit den erlaubten Hilfsmitteln zu bearbeiten und erst nach 
deren Bearbeitung Ihren eigenen Lösungsweg mit den hier 
vorliegenden Lösungsvorschlägen zu überprüfen.

Diese Aufgabensammlung enthält auch Lösungen zu den Bonustests. Da diese Tests teilweise viele Variationen 
aufweisen, ist es wahrscheinlich, dass Ihre konkreten Zahlenwerte und Ihr Ergebnis von den hier 
dargestellten abweichen werden. Beachten Sie des Weiteren, dass auch die Reihenfolge der Bonustests variieren kann. 
Achten Sie deswegen bitte auf den Titel des je\-weiligen Tests, um die korrekte Lösung zuzuordnen.

Die hier vorgestellten Rechenwege sind der Verständlichkeit halber bewusst sehr detailliert gehalten. 
In der Prüfung sollten Sie aus zeitlichen Gründen eine kompaktere Darstellung wählen. 
Es empfiehlt sich, Ihren Professor oder Ihre Professorin mit einer Ihrer Beispielrechnungen zu konsultieren. 
So können Sie sicherstellen, dass Ihre Notation korrekt ist, keine wesentlichen Schritte fehlen und Ihre Ausführungen 
den Anforderungen entsprechen, ohne unnötig ausführlich zu sein.

Haben Sie des Weiteren keine Bedenken, einen anderen Lösungsweg zu verfolgen. Die hier 
aufgezeigten Lösungswege sind lediglich Vorschläge und sind keineswegs die einzig korrekten Wege.

Wir wünschen Ihnen viel Erfolg bei der Bearbeitung und eine gute Klausurvorbereitung!

\end{abstract}