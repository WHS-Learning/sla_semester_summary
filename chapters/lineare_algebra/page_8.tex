\chapter{Übungsblatt 8}

\section{Aufgabe 1}

\subsection{a}
Bestimmen Sie Skalarprodukt und Kreuzprodukt der Einheitsvektoren $e_1 = \begin{pmatrix}1 \\ 0 \\ 0\end{pmatrix}$ und $e_2 = \begin{pmatrix}0 \\ 1 \\ 0\end{pmatrix}$.

\subsubsection*{Skalarprodukt}
\begin{align*}
    \left\langle e_1, e_2\right\rangle \\
    = 1 \cdot 0 + 0 \cdot 1 + 0 \cdot 0 \\
    = 0 + 0 + 0 \\
    = 0
\end{align*}

\subsubsection*{Kreuzprodukt}
\begin{align*}
    e_1 \times e_2 \\
    =\begin{pmatrix}
        0 \cdot 0 - 0 \cdot 1 \\
        0 \cdot 0 - 1 \cdot 0 \\
        1 \cdot 1 - 0 \cdot 0
    \end{pmatrix} \\
    = \begin{pmatrix}
        0 \\ 0 \\ 1
    \end{pmatrix}
\end{align*}

\subsection{b}
Sei $v \in \mathbb{R}^n$ beliebig. Sei $e_i \in \mathbb{R}^n (i \in \{1, \dots, n\})$ der i-te Einheitsvektor in $\mathbb{R}^n$. Bestimmen Sie $\langle v, e_i \rangle$ und (für den Fall $n = 3$) $v \times e_i$.

\subsection*{Skalarprodukt}

\begin{align*}
    \left\langle v, e_1 \right\rangle \\
    = v_1 \cdot 1 + v_2 \cdot 0 + v_3 \cdot 0\\
    = v_1
\end{align*}

allgemein

\begin{align*}
    \left\langle v, e_i \right\rangle \\
    = v_1 \cdot e_1 + v_2 \cdot e_2 + \dots + v_n \cdot e_i\\
    = v_i
\end{align*}

\subsubsection*{Kreuzprodukt}
\begin{align*}
    v \times e_1 \\
    = \begin{pmatrix}
        v_2 \cdot 0 - v_3 \cdot 0 \\
        v_3 \cdot 1 - v_1 \cdot 0 \\
        v_1 \cdot 0 - v_2 \cdot 1
    \end{pmatrix} \\
    = \begin{pmatrix}
        0 \\ v_3 \\ -v_2
    \end{pmatrix}
\end{align*}

\begin{align*}
    v \times e_2 \\
    = \begin{pmatrix}
        v_2 \cdot 0 - v_3 \cdot 1 \\
        v_3 \cdot 0 - v_1 \cdot 0 \\
        v_1 \cdot 1 - v_2 \cdot 0
    \end{pmatrix} \\
    = \begin{pmatrix}
        -v_3 \\ 0 \\ v_1
    \end{pmatrix}
\end{align*}

\begin{align*}
    v \times e_3 \\
    = \begin{pmatrix}
        v_2 \cdot 1 - v_3 \cdot 0 \\
        v_3 \cdot 0 - v_1 \cdot 1 \\
        v_1 \cdot 0 - v_2 \cdot 0
    \end{pmatrix} \\
    = \begin{pmatrix}
        v_2 \\ -v_1 \\ 0
    \end{pmatrix}
\end{align*}

\subsection{c}
Bestimmen Sie einen Vektor, der auf der von $\begin{pmatrix}5 \\ 1 \\ 9\end{pmatrix}$ und $\begin{pmatrix}1 \\ 2 \\ 10\end{pmatrix}$ aufgespannten Ebene senkrecht steht.

\begin{align*}
    span\left\{\begin{pmatrix}
        5 \\ 1 \\ 9
    \end{pmatrix}, \begin{pmatrix}
        1 \\ 2 \\ 10
    \end{pmatrix}\right\} \\
    cos(\theta) = \frac{\left\langle a, b\right\rangle}{\left|a\right| \cdot \left|b\right|}
\end{align*}

\section{Aufgabe 2}

\subsection{a}
Sei $A = \begin{pmatrix}6 & 17 & 24 \\ 9 & 32 & 6 \\ -5 & -1 & 47\end{pmatrix}$.

Bestimmen Sie $det(A)$

\begin{enumerate}
    \item mittels der Saurrus-Regel
    \item mittels der Leibniz-Formel
    \item mittels der Kästchenregel
\end{enumerate}

\subsection{b}
Sei $B = \begin{pmatrix}
    1 & 0 & 2 & 0 \\ 1 & 2 & 9 & 0 \\ 2 & 0 & 1 & 5 \\ 1 & 1 & 1 & 1
\end{pmatrix}$. 

Bestimmen Sie $det(B)$

\subsection{c}
Sei $C = \begin{pmatrix}
    1 & 0 & 2 & 0 \\ 1 & 2 & 9 & 0 \\ 2 & 0 & 1 & 5 \\ 1 & 2 & 9 & 0
\end{pmatrix}$. Bestimmen Sie ohne Rechung $det(C)$.

\section{Aufgabe 3}

\subsection{a}
Prüfen Sie die Vektoren $\begin{pmatrix}5 \\ 1 \\ 9\end{pmatrix}, \begin{pmatrix}1 \\2 \\ 3\end{pmatrix}$ und $\begin{pmatrix}4 \\ 4 \\ 1\end{pmatrix}$ auf lineare Unabhängigkeit.

\subsection{b}
Ist das lineare Gleichungssystem $Ax = b$ mit $b = \begin{pmatrix}5 \\ 10 \\ 1\end{pmatrix}$ und $A = \begin{pmatrix}1 & 0 & 2 \\ 1 & 2 & 0 \\ 2 & 0 & 1\end{pmatrix}$ eindeutig lösbar?

