\chapter{Übungsblatt 8}

\section{Aufgabe 1}

\subsection{a}
Bestimmen Sie Skalarprodukt und Kreuzprodukt der Einheitsvektoren $e_1 = \begin{pmatrix}1 \\ 0 \\ 0\end{pmatrix}$ und $e_2 = \begin{pmatrix}0 \\ 1 \\ 0\end{pmatrix}$.

\subsubsection*{Skalarprodukt}
\begin{align*}
    \left\langle e_1, e_2\right\rangle  \\
    = 1 \cdot 0 + 0 \cdot 1 + 0 \cdot 0 \\
    = 0 + 0 + 0                         \\
    = 0
\end{align*}

\subsubsection*{Kreuzprodukt}
\begin{align*}
    e_1 \times e_2         \\
    =\begin{pmatrix}
         0 \cdot 0 - 0 \cdot 1 \\
         0 \cdot 0 - 1 \cdot 0 \\
         1 \cdot 1 - 0 \cdot 0
     \end{pmatrix} \\
    = \begin{pmatrix}
          0 \\ 0 \\ 1
      \end{pmatrix}
\end{align*}

\subsection{b}
Sei $v \in \mathbb{R}^n$ beliebig. Sei $e_i \in \mathbb{R}^n (i \in \{1, \dots,
    n\})$ der i-te Einheitsvektor in $\mathbb{R}^n$. Bestimmen Sie $\langle v, e_i
    \rangle$ und (für den Fall $n = 3$) $v \times e_i$.

\subsection*{Skalarprodukt}

\begin{align*}
    \left\langle v, e_1 \right\rangle         \\
    = v_1 \cdot 1 + v_2 \cdot 0 + v_3 \cdot 0 \\
    = v_1
\end{align*}

allgemein

\begin{align*}
    \left\langle v, e_i \right\rangle                        \\
    = v_1 \cdot e_1 + v_2 \cdot e_2 + \cdots + v_n \cdot e_i \\
    = v_i
\end{align*}

\subsubsection*{Kreuzprodukt}
\begin{align*}
    v \times e_1                \\
    = \begin{pmatrix}
          v_2 \cdot 0 - v_3 \cdot 0 \\
          v_3 \cdot 1 - v_1 \cdot 0 \\
          v_1 \cdot 0 - v_2 \cdot 1
      \end{pmatrix} \\
    = \begin{pmatrix}
          0 \\ v_3 \\ -v_2
      \end{pmatrix}
\end{align*}

\begin{align*}
    v \times e_2                \\
    = \begin{pmatrix}
          v_2 \cdot 0 - v_3 \cdot 1 \\
          v_3 \cdot 0 - v_1 \cdot 0 \\
          v_1 \cdot 1 - v_2 \cdot 0
      \end{pmatrix} \\
    = \begin{pmatrix}
          -v_3 \\ 0 \\ v_1
      \end{pmatrix}
\end{align*}

\begin{align*}
    v \times e_3                \\
    = \begin{pmatrix}
          v_2 \cdot 1 - v_3 \cdot 0 \\
          v_3 \cdot 0 - v_1 \cdot 1 \\
          v_1 \cdot 0 - v_2 \cdot 0
      \end{pmatrix} \\
    = \begin{pmatrix}
          v_2 \\ -v_1 \\ 0
      \end{pmatrix}
\end{align*}

\subsection{c}
Bestimmen Sie einen Vektor, der auf der von $\begin{pmatrix}5 \\ 1 \\ 9\end{pmatrix}$ und $\begin{pmatrix}1 \\ 2 \\ 10\end{pmatrix}$ aufgespannten Ebene senkrecht steht.

Das Kreuzprodukt zweier Vektoren steht orthogonal auf der von diesen Vektoren
aufgespannten Ebene.

\begin{align*}
    \begin{pmatrix}
        5 \\ 1 \\ 9
    \end{pmatrix} \times \begin{pmatrix}
                             1 \\ 2 \\ 10
                         \end{pmatrix} = \begin{pmatrix}
                                             1 \cdot 10 - 9 \cdot 2 \\
                                             9 \cdot 1 - 5 \cdot 10 \\
                                             5 \cdot 2 - 1 \cdot 1
                                         \end{pmatrix} = \begin{pmatrix}
                                                             10 - 18 \\
                                                             9 - 50  \\
                                                             10 - 1
                                                         \end{pmatrix} = \begin{pmatrix}
                                                                             -8 \\ -41 \\ 9
                                                                         \end{pmatrix}
\end{align*}

\section{Aufgabe 2}

\subsection{a}
Sei $A = \begin{pmatrix}6 & 17 & 24 \\ 9 & 32 & 6 \\ -5 & -1 & 47\end{pmatrix}$.

Bestimmen Sie $det(A)$

\begin{enumerate}
    \item mittels der Saurrus-Regel
    \item mittels der Leibniz-Formel
    \item mittels der Kästchenregel
\end{enumerate}

\subsubsection*{Saurrus Regel}

\begin{tikzpicture}[
        every node/.style={minimum size=0.7cm},
        positive/.style={blue, thick},
        negative/.style={red, thick, dashed}
    ]
    \matrix (m) [matrix of math nodes,
        nodes in empty cells,
        left delimiter=(, right delimiter=),
        column sep=0.7cm, row sep=0.7cm, % Abstand zwischen Zellen
        ampersand replacement=\&] % Wichtig für tikz-matrix
    {
        6 \& 17 \& 24  \\
        9 \& 32 \& 6   \\
        -5 \& -1 \& 47 \\
    };

    % Erweiterte Spalten (nur für die Visualisierung der Linien)
    \node at ($(m-1-3.east)!0.5!(m-1-3.east) + (0.7cm,0)$) (m-1-4) {$6$};
    \node at ($(m-2-3.east)!0.5!(m-2-3.east) + (0.7cm,0)$) (m-2-4) {$9$};
    \node at ($(m-3-3.east)!0.5!(m-3-3.east) + (0.7cm,0)$) (m-3-4) {$-5$};

    \node at ($(m-1-4.east)!0.5!(m-1-4.east) + (0.7cm,0)$) (m-1-5) {$17$};
    \node at ($(m-2-4.east)!0.5!(m-2-4.east) + (0.7cm,0)$) (m-2-5) {$32$};
    \node at ($(m-3-4.east)!0.5!(m-3-4.east) + (0.7cm,0)$) (m-3-5) {$-1$};

    % Vertikale Trennlinie (optional)
    \draw[gray, thin] ($(m-1-3.east)!0.5!(m-2-3.east) + (0.35cm,0.35cm)$) -- ($(m-3-3.east) + (0.35cm,-0.35cm)$);

    % Positive Diagonalen (blau, durchgezogen)
    \draw[positive] (m-1-1.center) -- (m-2-2.center) -- (m-3-3.center); % a-e-i
    \draw[positive] (m-1-2.center) -- (m-2-3.center) -- (m-3-4.center); % b-f-g (auf erweiterter Spalte)
    \draw[positive] (m-1-3.center) -- (m-2-4.center) -- (m-3-5.center); % c-d-h (auf erweiterter Spalte)

    % Negative Diagonalen (rot, gestrichelt)
    \draw[negative] (m-1-3.center) -- (m-2-2.center) -- (m-3-1.center); % c-e-g
    \draw[negative] (m-1-4.center) -- (m-2-3.center) -- (m-3-2.center); % a-f-h (auf erweiterter Spalte)
    \draw[negative] (m-1-5.center) -- (m-2-4.center) -- (m-3-3.center); % b-d-i (auf erweiterter Spalte)

\end{tikzpicture}

\bigskip % Etwas Abstand

Die Zahlen an den Linien werden Multipliziert. Blaue Linien werden miteinander
addiert und Rote subtrahiert.

\begin{align*}
    6 \cdot 32 \cdot 47 + 17 \cdot 6 \cdot -5 + 24 \cdot 9 \cdot -1   \\
    - -5 \cdot 32 \cdot 24 - -1 \cdot 6 \cdot 6 - 47 \cdot 9 \cdot 17 \\
    = 4983
\end{align*}

\subsubsection*{Leibniz-Formel}

Muss ich nachher machen

\subsubsection*{Kästchenregel}

\begin{longtable}{p{10cm}}
    \hline
    \multicolumn{1}{c}{\textbf{Linearkombination}}                                         \\
    \hline
    \endfirsthead

    \hline
    \multicolumn{1}{c}{\tablename\ \thetable\ -- \textit{Fortführung von vorherier Seite}} \\
    \hline
    \multicolumn{1}{c}{\textbf{Linearkombination}}                                         \\
    \hline
    \endhead

    \hline
    \multicolumn{1}{r}{\textit{Fortsetzung siehe nächste Seite}}                           \\
    \endfoot

    \hline
    \endlastfoot

    $\displaystyle\begin{matrix}
                          6  & 17 & 24 \\
                          9  & 32 & 6  \\
                          -5 & -1 & 47
                      \end{matrix}$                                                            \\\hline
    Operation: III + $\frac{5}{6}$I                                                        \\\hline\pagebreak[0]

    $\displaystyle\begin{matrix}
                          6 & 17           & 24 \\
                          9 & 32           & 6  \\
                          0 & \frac{79}{6} & 67
                      \end{matrix}$                                                    \\\hline

    Operation: II - $\frac{3}{2}$I                                                         \\\hline\pagebreak[0]

    $\displaystyle\begin{matrix}
                          6 & 17           & 24  \\
                          0 & \frac{13}{2} & -30 \\
                          0 & \frac{79}{6} & 67
                      \end{matrix}$                                                   \\\hline

\end{longtable}

\begin{align*}
    6 \cdot \det\left(\begin{pmatrix}
                              \frac{13}{2} & -30 \\ \frac{79}{6} & 67
                          \end{pmatrix}\right) \\
    = 6 \cdot \frac{13}{2} \cdot 67 - \frac{79}{6} \cdot -30 \\
    = 4983
\end{align*}

\subsection{b}

Sei $B = \begin{pmatrix}
        1 & 0 & 2 & 0 \\ 1 & 2 & 9 & 0 \\ 2 & 0 & 1 & 5 \\ 1 & 1 & 1 & 1
    \end{pmatrix}$.

Bestimmen Sie $det(B)$

\begin{longtable}{p{10cm}}

    \hline
    \multicolumn{1}{c}{\textbf{Linearkombination}}                                         \\
    \hline
    \endfirsthead

    \hline
    \multicolumn{1}{c}{\tablename\ \thetable\ -- \textit{Fortführung von vorherier Seite}} \\
    \hline
    \multicolumn{1}{c}{\textbf{Linearkombination}}                                         \\
    \hline
    \endhead

    \hline
    \multicolumn{1}{r}{\textit{Fortsetzung siehe nächste Seite}}                           \\
    \endfoot

    \hline
    \endlastfoot

    $\displaystyle\begin{matrix}
                          1 & 0 & 2 & 0 \\
                          1 & 2 & 9 & 0 \\
                          2 & 0 & 1 & 5 \\
                          1 & 1 & 1 & 1
                      \end{matrix}$                                                            \\\hline
    I und IV tauschen                                                                      \\\hline\pagebreak[0]

    $\displaystyle\begin{matrix}
                          1 & 1 & 1 & 1 \\
                          1 & 2 & 9 & 0 \\
                          2 & 0 & 1 & 5 \\
                          1 & 0 & 2 & 0
                      \end{matrix}$                                                            \\\hline
    IV - 2I                                                                                \\\hline\pagebreak[0]

    $\displaystyle\begin{matrix}
                          1  & 1 & 1 & 1  \\
                          -1 & 0 & 7 & -2 \\
                          2  & 0 & 1 & 5  \\
                          1  & 0 & 2 & 0
                      \end{matrix}$                                                          \\\hline

    IV + II                                                                                \\\hline\pagebreak[0]

    $\displaystyle\begin{matrix}
                          1  & 1 & 1 & 1  \\
                          -1 & 0 & 7 & -2 \\
                          2  & 0 & 1 & 5  \\
                          0  & 0 & 9 & -2
                      \end{matrix}$                                                          \\\hline

    III + 2II                                                                              \\\hline\pagebreak[0]

    $\displaystyle\begin{matrix}
                          1  & 1 & 1  & 1  \\
                          -1 & 0 & 7  & -2 \\
                          0  & 0 & 15 & 1  \\
                          0  & 0 & 9  & -2
                      \end{matrix}$                                                         \\\hline

\end{longtable}

\[
    B = \left( \begin{array}{cc|cc}
            1  & 1 & 1  & 1  \\
            -1 & 0 & 7  & -2 \\
            \hline
            0  & 0 & 15 & 1  \\
            0  & 0 & 9  & -2
        \end{array} \right)
    = \begin{pmatrix} A & B \\ 0 & D \end{pmatrix}
\]

wobei $A = \begin{pmatrix} 1 & 1 \\ -1 & 0 \end{pmatrix}$ und $D = \begin{pmatrix} 15 & 1 \\ 9 & -2 \end{pmatrix}$.
Die Determinante ist $\det(M) = \det(A) \cdot \det(D)$.

\bigskip

\begin{tikzpicture}[
        every node/.style={minimum size=0.7cm},
        positive/.style={blue, thick},
        negative/.style={red, thick, dashed},
        block/.style={draw, very thick, rounded corners, inner sep=3pt}
    ]
    \matrix (m) [matrix of math nodes,
        nodes in empty cells,
        column sep=0.7cm, row sep=0.7cm,
        ampersand replacement=\&]
    {
        1 \& 1 \& 1 \& 1   \\
        -1 \& 0 \& 7 \& -2 \\
        0 \& 0 \& 15 \& 1  \\
        0 \& 0 \& 9 \& -2  \\
    };

    \draw[gray, thin] (m-2-1.south west) -- (m-2-4.south east);
    \draw[gray, thin] (m-1-2.north east) -- (m-4-2.south east);

    \node[block, fit=(m-1-1) (m-2-2), label={[yshift=-0.1cm]below:$\det(A)$}] (blockA) {};
    \node[block, fit=(m-3-3) (m-4-4), label={[yshift=-0.1cm]below:$\det(D)$}] (blockD) {};

    \draw[positive, shorten >=2pt, shorten <=2pt] (m-1-1.center) -- (m-2-2.center);
    \draw[negative, shorten >=2pt, shorten <=2pt] (m-1-2.center) -- (m-2-1.center);

    \draw[positive, shorten >=2pt, shorten <=2pt] (m-3-3.center) -- (m-4-4.center);
    \draw[negative, shorten >=2pt, shorten <=2pt] (m-3-4.center) -- (m-4-3.center);

    \node[anchor=west, text width=6cm, at={($(m-2-4.east)+(1cm,0)$)}] (formel)
    {
        $\det(B) = \underbrace{(1 \cdot 0 - 1 \cdot -1)}_{\det(A)} \cdot \underbrace{(15 \cdot -2 - 9 \cdot 1)}_{\det(D)} = (0 - 1) \cdot (-30 - 9) = -1 \cdot -39 = 39$
    };
\end{tikzpicture}

\subsection{c}
Sei $C = \begin{pmatrix}
        1 & 0 & 2 & 0 \\ 1 & 2 & 9 & 0 \\ 2 & 0 & 1 & 5 \\ 1 & 2 & 9 & 0
    \end{pmatrix}$.

Bestimmen Sie ohne Rechung $det(C)$.

Die Matrix ist Linear abhängig. Daher ist die Determinante = 0.

\section{Aufgabe 3}

\subsection{a}
Prüfen Sie die Vektoren $\begin{pmatrix}5 \\ 1 \\ 9\end{pmatrix}, \begin{pmatrix}1 \\2 \\ 3\end{pmatrix}$ und $\begin{pmatrix}4 \\ 4 \\ 1\end{pmatrix}$ auf lineare Unabhängigkeit.

\begin{align*}
    det\left(\begin{pmatrix}
                 5 & 1 & 4 \\
                 1 & 2 & 4 \\
                 9 & 3 & 1
             \end{pmatrix}\right)                               \\
    =5 \cdot 2 \cdot 1 + 1 \cdot 4 \cdot 9 + 4 \cdot 1 \cdot 3  \\
    - 9 \cdot 2 \cdot 4 - 3 \cdot 4 \cdot 5 - 1 \cdot 1 \cdot 1 \\
    = 10 + 36 + 12 - 72 - 60 - 1                                \\
    = -75
\end{align*}

\subsection{b}
Ist das lineare Gleichungssystem $Ax = b$ mit $b = \begin{pmatrix}5 \\ 10 \\ 1\end{pmatrix}$ und $A = \begin{pmatrix}1 & 0 & 2 \\ 1 & 2 & 0 \\ 2 & 0 & 1\end{pmatrix}$ eindeutig lösbar?

\begin{align*}
    \det \left(\begin{pmatrix}
                   1 & 0 & 2 \\
                   1 & 2 & 0 \\
                   2 & 0 & 1
               \end{pmatrix}\right)                             \\
    = 1 \cdot 2 \cdot 1 + 0 \cdot 0 \cdot 2 + 2 \cdot 1 \cdot 0 \\
    - 2 \cdot 2 \cdot 2 - 0 \cdot 0 \cdot 1 - 1 \cdot 1 \cdot 0 \\
    = 2 + 0 + 0 - 8 - 0 - 0                                     \\
    = -6
\end{align*}

Da die Matrix Linear unabhängig ist, ist das Gleichungssystem eindeutig lösbar.