\chapter{Übungsblatt 9}

\section{Aufgabe 1}

\subsection{a}
Seien die Geraden $G_1$ und $G_2$ definiert durch:
\begin{align*}
G_1 &:= \left\{\begin{pmatrix}
1 \\ 2 \\ 3
\end{pmatrix} + \lambda\begin{pmatrix}
1 \\ 1 \\ 1
\end{pmatrix}: \lambda \in \mathbb{R}\right\} \\
G_2 &:= \left\{\begin{pmatrix}
3 \\ 3 \\ 3
\end{pmatrix} + \mu \begin{pmatrix}
1 \\ 1 \\ 1
\end{pmatrix}: \mu \in \mathbb{R}\right\}
\end{align*}
Wie liegen $G_1$ und $G_2$ zueinander im Raum? Bestimmen Sie, je nach Lage, Schnittpunkt oder Abstand der beiden Geraden.

\subsection{b}

Es sei $E$ definiert durch:
\begin{align*}
    E := \left\{\begin{pmatrix}
    1 \\ 0 \\1
    \end{pmatrix} + \lambda \begin{pmatrix}
        1 \\ 1 \\ 1
    \end{pmatrix} + \mu \begin{pmatrix}
        2 \\ 3 \\ 4
    \end{pmatrix}: \lambda, \mu \in \mathbb{R}\right\}
\end{align*}
Geben Sie $E$ in Normalenform an.

\section{Aufgabe 2}

\subsection{a}
Sei $v = \begin{pmatrix}
    1 \\ 1 \\ 0
\end{pmatrix}$ und $w = \begin{pmatrix}
    1 \\ 1 \\ 0
\end{pmatrix}$ und $G := \left\{\lambda w: \lambda \in \mathbb{R}\right\}$. Bestimmen Sie die Projektion $v_w$ von 
$v$ auf die von $w$ aufgespannte Gerade $G$.
Bestimmen Sie weiterhin die Matrix, die die lineare Abbildung $P: \mathbb{R}^3 \rightarrow G, a \mapsto aw$ dastellt. 
Was ist der Rang dieser Matrix? Was ist ihr Kern? Was ihr Bild?

\subsection{b}
(Transferfrage) Sei $E := \left\{\lambda\begin{pmatrix}
    1 \\ 0 \\ 1
\end{pmatrix} + \mu \begin{pmatrix}
    1 \\ 1 \\ 0
\end{pmatrix}: \lambda \in \mathbb{R}\right\}$. Sei $v := \begin{pmatrix}
    1 \\ 1 \\ 2
\end{pmatrix}$. Wie kann man die orthogonale Projektion von $v$ auf $E$ bestimmen? Welcher Vektor kommt dabei heraus?

\section{Aufgabe 3}

\subsection{a}
Sei $D_\pi: \mathbb{R}^2 \rightarrow \mathbb{R}^2$ die lineare Abbildung, die jeden Vektor aus $\mathbb{R}^2$ um 
$\pi$ dreht. Sei $v := \begin{pmatrix}1 \\2\end{pmatrix}$. Sellen Sie die Matrix zu $D_\pi$ auf und bestimmen Sie 
$D_\pi(v)$. Bestimmen Sie $\langle v, D_\pi(v)\rangle$.

\subsection{b}
Sei $D : \mathbb{R}^3 \rightarrow \mathbb{R}^3$ die Drehung, die den Vektor $(1, 0, 0)^t$ auf den Vektor 
$\frac{1}{\sqrt{3}}(1, 1, 1)$ abbildet. Bestimmen Sie die Drehmatrix, die D darstellt.

\subsection{c}

Sei $\varphi \in [0, 2\pi)$. Bestimmen Sie für die Drehmatrix $A_\varphi \in \mathbb{R}^{2 \times 2}$ und 
$A_{3, \varphi} \in \mathbb{R}^{3 \times 3}$ (Drehung um die $x_3$-Achse) die Determinante. Bestimmen Sie 
weiterhin das Produkt $A^t_\varphi A^t_\varphi$ bzw. $A^t_{3, \varphi} A_{3, \varphi}$. Was ist also die Inverse
$A^{-1}_{\varphi}$ bzw. $A^{-1}_{3, \varphi}$