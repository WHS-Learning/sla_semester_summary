\chapter{Übungsblatt 9}

\section{Aufgabe 1}

\subsection{a}
Seien die Geraden $G_1$ und $G_2$ definiert durch:
\begin{align*}
G_1 &:= \left\{\begin{pmatrix}
1 \\ 2 \\ 3
\end{pmatrix} + \lambda \begin{pmatrix}
1 \\ 1 \\ 1
\end{pmatrix}: \lambda \in \mathbb{R}\right\} \\
G_2 &:= \left\{\begin{pmatrix}
3 \\ 3 \\ 3
\end{pmatrix} + \mu \begin{pmatrix}
1 \\ 1 \\ 1
\end{pmatrix}: \mu \in \mathbb{R}\right\}
\end{align*}
Wie liegen $G_1$ und $G_2$ zueinander im Raum? Bestimmen Sie, je nach Lage, Schnittpunkt oder Abstand der beiden Geraden.

Da beide Vektoren die selben Richtungsvektoren besitzen, sind sie entweder echt parallel oder identisch. Um zu prüfen, ob die beiden Vektoren Identisch sind, muss das Gleichungssystem $G_1 = G_2 + \mu \begin{pmatrix}
    1 \\ 1 \\ 1
\end{pmatrix}$ gelöst werden.

\begin{align*}
    \begin{pmatrix}
        1 \\ 2 \\ 3
    \end{pmatrix} = \begin{pmatrix}
        3 \\ 3 \\ 3
    \end{pmatrix} + \mu \cdot \begin{pmatrix}
        1 \\ 1 \\ 1
    \end{pmatrix} \\
    \begin{cases}
        \text{I:\@} & 1 = 3 + \mu \quad | -3 \Leftrightarrow -2 = \mu\\
        \text{II:\@} & 2 = 3 + \mu \quad | -3 \Leftrightarrow -1 = \mu\\
        \text{III:\@} & 3 = 3 + \mu \quad | -3 \Leftrightarrow 0 = \mu
    \end{cases}
\end{align*}

Da das resultierende Gleichungssystem nicht lösbar ist, sind die Vektoren nicht identisch. Nun muss geprüft werden, ob sie echt parallel zueinander stehen. Hierfür muss der Abstand $d$ der Vektoren $G_1: r = p_1 + \lambda u$ und $G_2: r = p_2 + \mu u$ bestimmt werden, welcher sich über die Formel $d = \frac{|(p_2 - p_1) \times u|}{|u|}$ berechnen lässt.

\begin{align*}
    d = \frac{\left|\left(p_2 - p_1\right) \times u\right|}{\left|u\right|} \\
    d = \frac{\left|\left(\begin{pmatrix}
        3 \\ 3 \\ 3
    \end{pmatrix} - \begin{pmatrix}
        1 \\ 2 \\ 3
    \end{pmatrix}\right) \times \begin{pmatrix}
        1 \\ 1 \\ 1
    \end{pmatrix}\right|}{\left|\begin{pmatrix}
        1 \\ 1 \\ 1
    \end{pmatrix}\right|} \\
    d = \frac{\left|\begin{pmatrix}
        2 \\ 1 \\ 0
    \end{pmatrix} \times \begin{pmatrix}
        1 \\ 1 \\ 1
    \end{pmatrix}\right|}{\left|\begin{pmatrix}
        1 \\ 1 \\ 1
    \end{pmatrix}\right|} \\
    d = \frac{\left|\begin{pmatrix}
        1 \\ -2 \\ 1
    \end{pmatrix}\right|}{\left|\begin{pmatrix}
        1 \\ 1 \\ 1
    \end{pmatrix}\right|} \\
    d = \frac{\sqrt{6}}{\sqrt{3}} \\
    d = \sqrt{2} \\
\end{align*}

Der Abstand zwischen den beiden Vektoren beträgt $\sqrt{2}$. Das bedeutet, dass diese echt parallel zueinander sind.

\subsection{b}

Es sei $E$ definiert durch:
\begin{align*}
    E := \left\{\begin{pmatrix}
    1 \\ 0 \\1
    \end{pmatrix} + \lambda \begin{pmatrix}
        1 \\ 1 \\ 1
    \end{pmatrix} + \mu \begin{pmatrix}
        2 \\ 3 \\ 4
    \end{pmatrix}: \lambda, \mu \in \mathbb{R}\right\}
\end{align*}
Geben Sie $E$ in Normalenform an.

Nicht Klausurrelevant?

\section{Aufgabe 2 - nicht klausurrelevant?}

\subsection{a}
Sei $v = \begin{pmatrix}
    1 \\ 1 \\ 0
\end{pmatrix}$ und $w = \begin{pmatrix}
    1 \\ 1 \\ 0
\end{pmatrix}$ und $G := \left\{\lambda w: \lambda \in \mathbb{R}\right\}$. Bestimmen Sie die Projektion $v_w$ von 
$v$ auf die von $w$ aufgespannte Gerade $G$.
Bestimmen Sie weiterhin die Matrix, die die lineare Abbildung $P: \mathbb{R}^3 \rightarrow G, a \mapsto aw$ dastellt. 
Was ist der Rang dieser Matrix? Was ist ihr Kern? Was ihr Bild?

\subsection{b}
(Transferfrage) Sei $E := \left\{\lambda\begin{pmatrix}
    1 \\ 0 \\ 1
\end{pmatrix} + \mu \begin{pmatrix}
    1 \\ 1 \\ 0
\end{pmatrix}: \lambda \in \mathbb{R}\right\}$. Sei $v := \begin{pmatrix}
    1 \\ 1 \\ 2
\end{pmatrix}$. Wie kann man die orthogonale Projektion von $v$ auf $E$ bestimmen? Welcher Vektor kommt dabei heraus?

\section{Aufgabe 3}

\subsection{a}
Sei $D_\pi: \mathbb{R}^2 \rightarrow \mathbb{R}^2$ die lineare Abbildung, die jeden Vektor aus $\mathbb{R}^2$ um 
$\pi$ dreht. Sei $v := \begin{pmatrix}1 \\2\end{pmatrix}$. Sellen Sie die Matrix zu $D_\pi$ auf und bestimmen Sie 
$D_\pi(v)$. Bestimmen Sie $\langle v, D_\pi(v)\rangle$.

Die 2x2 Matrix zur Drehung ist $D_\alpha = \begin{pmatrix}
    cos(\alpha) & -sin(\alpha) \\ sin(\alpha) & cos(\alpha)
\end{pmatrix}$. Die Drehmatrix, welche um $\pi$ dreht, ist somit $D_\pi = \begin{pmatrix}
    cos(\pi) & -sin(\pi) \\ sin(\pi) & \cos(\pi)
\end{pmatrix} = \begin{pmatrix}
    -1 & 0 \\ 0 & -1
\end{pmatrix}$.

\begin{align*}
    D_\pi \cdot \begin{pmatrix}
        1 \\ 2
    \end{pmatrix} \\
    = \begin{pmatrix}
        -1 & 0 \\ 0 & -1
    \end{pmatrix} \cdot \begin{pmatrix}
        1 \\ 2
    \end{pmatrix} \\
    = \begin{pmatrix}
        -1 \cdot 1 + 0 \cdot 2 \\ 0 \cdot 1 + (-1) \cdot 2
    \end{pmatrix} \\
    = \begin{pmatrix}
        -1 \\ -2
    \end{pmatrix}
\end{align*}

\begin{align*}
    \left\langle v, D_\pi(v) \right\rangle \\
    = \left\langle \begin{pmatrix}
        1 \\ 2
    \end{pmatrix}, \begin{pmatrix}
        -1 \\ -2
    \end{pmatrix} \right\rangle \\
    = 1 \cdot -1 + 2 \cdot -2 \\
    = -1 + (-4) \\
    = -5
\end{align*}

\subsection{b}
Sei $D : \mathbb{R}^3 \rightarrow \mathbb{R}^3$ die Drehung, die den Vektor $(1, 0, 0)^t$ auf den Vektor 
$\frac{1}{\sqrt{3}}(1, 1, 1)$ abbildet. Bestimmen Sie die Drehmatrix, die D darstellt.

\begin{align*}
    D = \begin{pmatrix}
        c_1 & c_2 & c_3 \\
        c_1 & c_2 & c_3 \\
        c_1 & c_2 & c_3
    \end{pmatrix}\\
    D\begin{pmatrix}
        1 \\ 0 \\ 0
    \end{pmatrix} \\
    = \frac{1}{\sqrt{3}} \begin{pmatrix}
        1 \\ 1 \\ 1
    \end{pmatrix} \\
    = \begin{pmatrix}
        c_1 \\ c_1 \\ c_1
    \end{pmatrix} \\\\
    \text{Erste Bedingung an } c_2 \\
    \left\langle c_1, c_2 \right\rangle = 0 \\
    \left\langle \begin{pmatrix}
        \frac{1}{\sqrt{3}} \\
        \frac{1}{\sqrt{3}} \\
        \frac{1}{\sqrt{3}} 
    \end{pmatrix}, \begin{pmatrix}
        x_1 \\ x_2 \\ x_3
    \end{pmatrix} \right\rangle = 0 \\
    \frac{1}{\sqrt{3}}x_1 + \frac{1}{\sqrt{3}}x_2 + \frac{1}{\sqrt{3}}x_3 = 0 \quad \frac{1}{\sqrt{3}} \\
    x_1 + x_2 + x_3 = 0 \\\\
    \text{Zweite Bedingung an } c_2 \\
    \left|c_2\right| = 1 \\
    \left|\begin{pmatrix}
        x_1 \\x_2 \\ x_3
    \end{pmatrix}\right| = 1 \\
    x_1^2 + x_2^2 + x_3^2 = 1 \\\\
    \text{Wähle } x_3 = 0 \\
    \text{In erste Bedingung einsetzen} \\
    x_1 + x_2 + 0 = 0 \quad | - x_2\\
    \Leftrightarrow x_1 = -x_2 \\\\
    \text{In zweite Bedingung einsetzen} \\
    x_1^2 + (- x_1^2) + 0^2 = 1 \\
    x_1^2 + x_1^2 = 1 \\
    2x_1^2 = 1 \quad | : 2\\
    \Leftrightarrow x_1^2 = \frac{1}{2} \quad | \sqrt{}\\
    \Leftrightarrow x_1 = \frac{1}{\sqrt{2}} \\
    x_2 = -x_1 \\
    x_2 = -\frac{1}{\sqrt{2}} \\
    c_2 = \begin{pmatrix}
        \frac{1}{\sqrt{2}} \\
        - \frac{1}{\sqrt{2}} \\
        0 \\
    \end{pmatrix} \\\\
    c_3 = c_1 \times c_2 \\
    c_3 = \begin{pmatrix}
        \frac{1}{\sqrt{3}} \\
        \frac{1}{\sqrt{3}} \\
        \frac{1}{\sqrt{3}} 
    \end{pmatrix} \times \begin{pmatrix}
        \frac{1}{\sqrt{2}} \\ -\frac{1}{\sqrt{2}} \\ 0
    \end{pmatrix} \\
    c_3 = \begin{pmatrix}
        \frac{1}{\sqrt{3}} \cdot 0 - \frac{1}{\sqrt{3}} \cdot -\frac{1}{\sqrt{2}} \\
        \frac{1}{\sqrt{3}} \cdot \frac{1}{\sqrt{2}} - \frac{1}{\sqrt{3}} \cdot 0 \\
        \frac{1}{\sqrt{3}} \cdot -\frac{1}{\sqrt{2}} - \frac{1}{\sqrt{3}} \cdot \frac{1}{\sqrt{2}}
    \end{pmatrix} \\
    c_3 = \begin{pmatrix}
        \frac{\sqrt{6}}{6} \\ \frac{\sqrt{6}}{6} \\ -\frac{\sqrt{6}}{3}
    \end{pmatrix} \\\\D = \begin{pmatrix}
        \frac{1}{\sqrt{3}} & \frac{1}{\sqrt{2}} & \frac{\sqrt{6}}{6} \\
        \frac{1}{\sqrt{3}} & -\frac{1}{\sqrt{2}} & \frac{\sqrt{6}}{6} \\
        \frac{1}{\sqrt{3}} & 0 & -\frac{\sqrt{6}}{3} \\
    \end{pmatrix}
\end{align*}

\subsection{c}

Sei $\varphi \in [0, 2\pi)$. Bestimmen Sie für die Drehmatrix $A_\varphi \in \mathbb{R}^{2 \times 2}$ und 
$A_{3, \varphi} \in \mathbb{R}^{3 \times 3}$ (Drehung um die $x_3$-Achse) die Determinante. Bestimmen Sie 
weiterhin das Produkt $A^t_\varphi A^t_\varphi$ bzw. $A^t_{3, \varphi} A_{3, \varphi}$. Was ist also die Inverse
$A^{-1}_{\varphi}$ bzw. $A^{-1}_{3, \varphi}$?

