\chapter{Übungsblatt 5}

\section{Aufgabe 1}
Bestimmen Sie die Ergebnisse der folgenden Rechenoperationen.

\subsection{a}
\begin{align*}
  3\begin{pmatrix} 1 \\ 1 \end{pmatrix} + 5 \begin{pmatrix} 4 \\ 1 \end{pmatrix}
  &= \begin{pmatrix} 3 \cdot 1 \\ 3 \cdot 1 \end{pmatrix} + \begin{pmatrix} 5 \cdot 4 \\ 5 \cdot 1 \end{pmatrix} \\
  &= \begin{pmatrix} 3 \\ 3 \end{pmatrix} + \begin{pmatrix} 20 \\ 5 \end{pmatrix} \\
  &= \begin{pmatrix} 3+20 \\ 3+5 \end{pmatrix} \\  
  &= \begin{pmatrix} 23 \\ 8 \end{pmatrix}
\end{align*}

\subsection{b}
\begin{align*}
  10\begin{pmatrix} 5 \\ 4 \\ 3 \end{pmatrix} + 4 \begin{pmatrix} 1 \\ 1 \\ 0 \end{pmatrix} + \begin{pmatrix} 3 \\ 2 \\ 1 \end{pmatrix}
  &= \begin{pmatrix} 10 \cdot 5 \\ 10 \cdot 4 \\ 10 \cdot 3 \end{pmatrix} + \begin{pmatrix} 4 \cdot 1 \\ 4 \cdot 1 \\ 4 \cdot 0 \end{pmatrix} + \begin{pmatrix} 3 \\ 2 \\ 1 \end{pmatrix} \\
  &= \begin{pmatrix} 50 \\ 40 \\ 30 \end{pmatrix} + \begin{pmatrix} 4 \\ 4 \\ 0 \end{pmatrix} + \begin{pmatrix} 3 \\ 2 \\ 1 \end{pmatrix} \\
  &= \begin{pmatrix} 50 + 4 + 3 \\ 40 + 4 + 2 \\ 30 + 0 + 1 \end{pmatrix} \\
  &= \begin{pmatrix} 57 \\ 46 \\ 31 \end{pmatrix}
\end{align*}

\hrulefill{}

\section{Aufgabe 2}

Sind die folgenden Mengen von Vektoren linear unabhängig? Können sie durch Entfernen eines Vektors linear unabhängig gemacht werden?

Um auf lineare Unabhängigkeit zu prüfen, gibt es mehrere Möglichkeiten. Hier zwei gängige Ansätze:

1.  Prüfung über die Determinante:
    Man bildet aus den Vektoren eine quadratische Matrix $A$. Die Vektoren sind linear unabhängig, wenn die Determinante dieser Matrix 
    ungleich null ist ($\det(A) \neq 0$). Ist die Determinante gleich null ($\det(A) = 0$), sind die Vektoren linear abhängig. 
    (Diese Methode ist direkt nur anwendbar, wenn die Anzahl der Vektoren der Dimension des Raumes entspricht, z.B. 2 Vektoren im $\mathbb{R}^2$ oder 3 Vektoren im $\mathbb{R}^3$).

2.  Prüfung über die Definition der linearen Unabhängigkeit:
    Eine Menge von Vektoren $\{v_1, v_2, \dots, v_n\}$ ist linear unabhängig, wenn die einzige Lösung der Vektorgleichung
    \[ x_1v_1 + x_2v_2 + \cdots + x_n v_n = \mathbf{0} \]
    die sogenannte triviale Lösung ist, bei der alle Skalare $x_1, x_2, \dots, x_n$ gleich null sind ($x_1 = x_2 = \cdots = x_n = 0$).
    Wenn es mindestens eine nicht-triviale Lösung gibt (d.\ h.\ mindestens ein $x_i \neq 0$), dann sind die Vektoren linear abhängig.

\subsection{a}
\[ \left\{ \begin{pmatrix} 1 \\ 1 \end{pmatrix}, \begin{pmatrix} 1 \\ 0 \end{pmatrix} \right\} \]

\subsubsection*{Prüfung für a über Determinante}
1.  Matrix aus den Vektoren erstellen:
    \[ A = \begin{pmatrix} 1 & 1 \\ 1 & 0 \end{pmatrix} \]

2. Determinante der Matrix A berechnen
    \begin{align*}
        \det(A) = A_{1,1} \cdot A_{2,2} - A_{2,1} \cdot A_{1,2} = 1 \cdot 0 - 1 \cdot 1 = 0 - 1 = -1
    \end{align*}
    Da die Determinante ungleich null ist, ist die Linearkombination linear unabhängig.

\subsubsection{Prüfung für a über Linearkombination}

\begin{align*}
    x_1 \cdot \begin{pmatrix}
        1 \\1
    \end{pmatrix} + x_2 \cdot \begin{pmatrix}
        1 \\0
    \end{pmatrix} = \begin{pmatrix}
        0 \\0
    \end{pmatrix} \\
        \begin{pmatrix}
        1 \cdot x_1 \\1 \cdot x_1
    \end{pmatrix} +\begin{pmatrix}
        1 \cdot x_2 \\0 \cdot x_2
    \end{pmatrix} = \begin{pmatrix}
        0 \\0
    \end{pmatrix} \\
            \begin{pmatrix}
        1 \cdot x_1 + 1 \cdot x_2 \\1 \cdot x_1 + 0 \cdot x_2
    \end{pmatrix} = \begin{pmatrix}
        0 \\0
    \end{pmatrix} \\
    \begin{cases}
       \text{I:\@} & 1 \cdot x_1 + 1  \cdot x_2 = 0 \\
       \text{II:\@} & 1 \cdot x_1 + 0 \cdot x_2 = 0
    \end{cases} \\
    \begin{cases}
       \text{I:\@} & x_1 + x_2 = 0 \\
       \text{II:\@} & x_1 \phantom{{} + x_2} = 0
    \end{cases}\\
    \text{in I einsetzen} \\
    0 + x_2 = 0 \\
    x_2 = 0
\end{align*}

Die einzige Lösung des linearen Gleichungssystem $x_1 = x_2 = 0$ ist, sind die Vektoren linear unabhängig.

\subsection{b}
\[ \left\{ \begin{pmatrix} 1 \\ 1 \\ 1 \end{pmatrix},
\begin{pmatrix} 1 \\ 0 \\ 1 \end{pmatrix}, \begin{pmatrix} 2 \\ 1 \\ 1 \end{pmatrix},
\begin{pmatrix} 1 \\ 2 \\ 2 \end{pmatrix} \right\} \]

Es können nur drei Vektoren aus $\mathbb{R}^3$ linear unabhängig sein. Jede Kombination aus mehr als drei Vektoren aus $\mathbb{R}^3$ ist linear abhängig. Daher sind die vier Vektoren linear abhängig.

Um auf lineare Unabhängigkeit zu prüfen, wird ein beliebigen Vektor entfernt.\ hier wird der vierte Vektor entfernt.

Im schlimmsten Fall kann es passieren, dass vier Linearkombinationen auf lineare Unabhängigkeit prüfen müssen, bis wir eine Linearkombination gefunden wird, welche linear unabhängig ist.

\subsubsection{Prüfung für b über Determinante}
1.  Matrix aus den Vektoren erstellen:
    \[ B = \begin{pmatrix} 1 & 1 & 2 \\ 1 & 0 & 1 \\ 1 & 1 & 1 \end{pmatrix} \]

2. Determinante der Matrix B berechnen
    \begin{align*}
        \det(B) = B_{1,1} \cdot B_{2,2} \cdot B_{3,3} + B_{1,2} \cdot B_{2,3} \cdot B_{3,1} + B_{1,3} \cdot B_{2,1} \cdot B_{3,2} \\ - B_{3,1} \cdot B_{2,2} \cdot B_{1,3} - B_{3,2} \cdot B_{2,3} \cdot B_{1,1} - B_{3,3} \cdot B_{2,1} \cdot B_{1,2}\\
        = 1 \cdot 0 \cdot 1 + 1 \cdot 1 \cdot 1 + 2 \cdot 1 \cdot 1 - 1 \cdot 0 \cdot 2 - 1 \cdot 1 \cdot 1 - 1 \cdot 1 \cdot 1 \\
        = 0 + 1 + 2 - 0 - 1 - 1 \\
        = 1
    \end{align*}
    Da die Determinante ungleich null ist, kann die Vektormenge durch entfernen des vierten Vektors linear unabhängig gemacht werden.

\subsubsection{Prüfung für b über Linearkombination}
\begin{align*}
    x_1 \cdot \begin{pmatrix}
        1 \\ 1 \\ 1
    \end{pmatrix} + x_2 \cdot \begin{pmatrix}
        1 \\ 0 \\ 1
    \end{pmatrix} + x_3 \cdot \begin{pmatrix}
        2 \\ 1 \\ 1
    \end{pmatrix} = \begin{pmatrix}
        0 \\ 0 \\ 0
    \end{pmatrix} \\
    \begin{pmatrix}
        1 \cdot x_1 \\ 1 \cdot x_1 \\ 1 \cdot x_1
    \end{pmatrix} + \begin{pmatrix}
        1 \cdot x_2\\ 0 \cdot x_2 \\ 1 \cdot x_2
    \end{pmatrix} + \begin{pmatrix}
        2 \cdot x_3 \\ 1 \cdot x_3 \\ 1 \cdot x_3
    \end{pmatrix} = \begin{pmatrix}
        0 \\ 0 \\ 0
    \end{pmatrix} \\
    \begin{cases}
       \text{I:\@} & 1 \cdot x_1 + 1 \cdot x_2 + 2 \cdot x_3 = 0 \\
       \text{II:\@} & 1 \cdot x_1 + 0 \cdot x_2 + 1 \cdot x_3 = 0 \\
       \text{III:\@} & 1 \cdot x_1 + 1 \cdot x_2 + 1 \cdot x_3 = 0
    \end{cases} \\
    \begin{cases}
        \text{I:\@} & x_1 + x_2 + 2x_3 = 0 \\
        \text{II:\@} & x_1 \phantom{{}+x_2} + \phantom{{}2}x_3 = 0 \quad | -x_3 \Leftrightarrow x_1 = -x_3\\
        \text{III:\@} & x_1 + x_2 + \phantom{{}2}x_3 = 0
    \end{cases} \\
    x_1 \text{ in III einsetzen} \\
    -x_3 + x_2 + x_3 = 0 \\ 
    \Leftrightarrow x_2 = 0 \\
    x_1 \text{ und } x_2 \text{ in I einsetzen} \\
    -x_3 + 0 + 2x_3 = 0 \\
    \Leftrightarrow -x_3 + 2x_3 = 0 \\
    \Leftrightarrow x_3 = 0 \\
    x_3 \text{ in II einsetzen}\\
    x_1 + 0 = 0 \\
    \Leftrightarrow x_1 = 0
\end{align*}
Die einzige Lösung des linearen Gleichungssystem ist $x_1 = x_2 = x_3 = 0$, das heißt, dass die Vektormenge ohne den vierten Vektor linear unabhängig ist.


\hrulefill{}
\section{Aufgabe 3}
Bestimmen Sie die Dimension des Untervektorraums
\[ V := \text{span}\left\{ \begin{pmatrix} 1 \\ 1 \\ 1 \end{pmatrix}, \begin{pmatrix} 1 \\ 1 \\ 0 \end{pmatrix} \right\} \]

Die Dimension eines Untervektorraums ist die anzahl der Basisvektoren. Um herauszufinden, ob die gegebenen Vektoren eines Basis des $\mathbb{R}^3$ bilden, muss auf lineare unabhängigkeit geprüft werden.

Hier bietet es sich nicht an, dies über die Determinante zu errechnen, da nur quadratische Matzitzen eine Determinante besitzen.

\begin{align*}
    x_1 \cdot \begin{pmatrix}
        1 \\ 1 \\1
    \end{pmatrix} + x_2 \cdot \begin{pmatrix}
        1 \\ 1 \\ 0
    \end{pmatrix} = \begin{pmatrix}
        0 \\ 0 \\ 0
    \end{pmatrix} \\
    \begin{pmatrix}
        1 \cdot x_1 \\ 1 \cdot x_1 \\1 \cdot x_1
    \end{pmatrix} + \begin{pmatrix}
        1 \cdot x_2 \\ 1 \cdot x_2 \\ 0 \cdot x_2
    \end{pmatrix} = \begin{pmatrix}
        0 \\ 0 \\ 0
    \end{pmatrix} \\
    \begin{cases}
        \text{I:\@} & 1 \cdot x_1 + 1 \cdot x_2 = 0 \\
        \text{II:\@} & 1 \cdot x_1 + 1 \cdot x_2 = 0 \\
        \text{III:\@} & 1 \cdot x_1 + 0 \cdot x_2 = 0
    \end{cases} \\
    \begin{cases}
        \text{I:\@} & x_1 + x_2 = 0 \\
        \text{II:\@} & x_1 + x_2 = 0 \\
        \text{III:\@} & x_1 \phantom{{} + x_2} = 0
    \end{cases} \\
    x_1 \text{ in I einsetzen} \\
    0 + x_2 = 0 \\
    \Leftrightarrow x_2 = 0
\end{align*}

Da $x_1 = x_2 = 0$ ist, ist die Vektormenge linear unabhängig. Diese zwei linear unabhängigen Vektoren spannen den Untervektorraum V auf und bilden somit eine Basis dieses Untervektorraums V. Da diese Basis aus zwei Vektoren besteht, ist die Dimension des Untervektorraums V gleich 2.