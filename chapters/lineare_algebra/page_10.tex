\chapter{Übungsblatt 10}

\section{Aufgabe 1}

Bestimmen Sie sämtliche Eigenwerte der folgenden Matrizen:

\subsection{a}

\begin{align*}
    A := \begin{pmatrix}
        1 & 2 & 3 \\
        4 & 5 & 6 \\ 
        0 & 0 & 0
    \end{pmatrix}
\end{align*}

\begin{align*}
    A - \lambda \cdot I \\
    = \begin{pmatrix}
        1 & 2 & 3 \\ 
        4 & 5 & 6 \\ 
        0 & 0 & 0
    \end{pmatrix} - \lambda \cdot \begin{pmatrix}
        1 & 0 & 0 \\
        0 & 1 & 0 \\
        0 & 0 & 1
    \end{pmatrix} \\
    = \begin{pmatrix}
        1 & 2 & 3 \\ 
        4 & 5 & 6 \\ 
        0 & 0 & 0
    \end{pmatrix}\begin{pmatrix}
        \lambda & 0 & 0 \\
        0 & \lambda & 0 \\
        0 & 0 & \lambda
    \end{pmatrix} \\
    = \begin{pmatrix}
        1 - \lambda & 2 & 3 \\
        4 & 5 - \lambda & 6 \\
        0 & 0 & -\lambda
    \end{pmatrix} =: A_\lambda \\
    \det(A_\lambda) = (1 - \lambda) \cdot (5 - \lambda) \cdot (-\lambda) \\
        + 2 \cdot 6 \cdot 0  + 3 \cdot 4 \cdot 0 - 0 \cdot (5 - \lambda) \cdot 3\\
        - 0 \cdot 6 \cdot (1 - \lambda) - (-\lambda) \cdot 4 \cdot 2 \\
    = (5 - \lambda - 5\lambda + \lambda^2) \cdot (-\lambda) + 0 \\
        + 0 - 0 - 0 + 4\lambda \cdot 2 \\
    = -5\lambda +\lambda^2 +5\lambda^2 -\lambda^3 + 8\lambda \\
    = -\lambda^3 + 6\lambda^2 + 3\lambda \\
    \text{$\lambda$ Ausklammern} \\
    \lambda (-\lambda^2 + 6\lambda + 3) \\\\
    \lambda (-\lambda^2 + 6\lambda + 3) = 0\\
    \text{Ein Produkt ist null, wenn ein Faktor gleich null ist} \\
    \begin{cases}
        \lambda = 0 & \text{Oder} \\
        -\lambda^2 + 6\lambda + 3 = 0
    \end{cases} \\
    \lambda_1 = 0 \\
    -\lambda^2 + 6 \lambda + 3 = 0 \quad |\cdot (-1) \\
    \Leftrightarrow \lambda^2 - 6 \lambda - 3 = 0 \quad | \text{PQ-Formel} \\
    p = 6 \quad q = -3 \\
    \lambda_{2,3} = \frac{-6}{2} \pm \sqrt{\left(\frac{6}{2}\right)^2 + 3} \\
    \lambda_{2,3}= -3 \pm \sqrt{\left(3\right)^2 + 3} \\
    \lambda_{2,3}= -3 \pm \sqrt{9 + 3} \\
    \lambda_{2,3}= -3 \pm \sqrt{12} \\\\
    \lambda_1 = 0, \quad \lambda_2 = -3 + \sqrt{12}, \quad \lambda_3 = -3 - \sqrt{12}
\end{align*}

\subsection{b}

\begin{align*}
    B := \begin{pmatrix}
        1 & 1 & 1 \\
        1 & 1 & 1 \\
        1 & 1 & 1
    \end{pmatrix}
\end{align*}

\begin{align*}
    B - \lambda \cdot I \\
    \begin{pmatrix}
        1 & 1 & 1 \\
        1 & 1 & 1 \\
        1 & 1 & 1
    \end{pmatrix} - \lambda \cdot \begin{pmatrix}
        1 & 0 & 0 \\
        0 & 1 & 0 \\
        0 & 0 & 1
    \end{pmatrix} \\
        \begin{pmatrix}
        1 & 1 & 1 \\
        1 & 1 & 1 \\
        1 & 1 & 1
    \end{pmatrix} - \begin{pmatrix}
        \lambda & 0 & 0 \\
        0 & \lambda & 0 \\
        0 & 0 & \lambda
    \end{pmatrix} \\
    \begin{pmatrix}
        1 - \lambda & 1 & 1 \\
        1 & 1 - \lambda & 1 \\
        1 & 1 & 1 - \lambda
    \end{pmatrix} =: B_\lambda \\
    \det(B_\lambda) = (1 - \lambda) \cdot (1 - \lambda) \cdot (1 - \lambda) \\
                    + 1 \cdot 1 \cdot 1 + 1 \cdot 1 \cdot 1 - 1 \cdot (1 - \lambda) \cdot 1 \\
                    - 1 \cdot 1 \cdot (1 - \lambda) - (1 - \lambda) \cdot 1 \cdot 1 \\
    = (1 -\lambda - \lambda + \lambda^2) \cdot (1 - \lambda) + 1 \\
        + 1 - (1 - \lambda) - (1 - \lambda) - (1 - \lambda) \\
    = (1 - 2\lambda + \lambda^2) \cdot (1 - \lambda) + 2 - 3(1 - \lambda) \\
    = 1 - 2\lambda + \lambda^2 - \lambda + 2\lambda^2 - \lambda^3 + 2 - 3 + 3\lambda \\
    = -\lambda^3 + 3\lambda^2 \\\\
    -\lambda^3 + 3\lambda^2 = 0 \\
    \text{$\lambda$ Ausklammern} \\
    \lambda (-\lambda^2 + 3\lambda) = 0 \\
    \text{Ein Produkt ist null, wenn mindestens ein Faktor gleich null ist.} \\
    \begin{cases}
        \lambda = 0 & \text{Oder} \\
        -\lambda^2 + 3 \lambda = 0
    \end{cases} \\
    \lambda_1 = 0 \\
    \text{$\lambda$ Ausklammern} \\
    \lambda(\lambda + 3) = 0 \\
    \text{Ein Produkt ist null, wenn mindestens ein Faktor gleich null ist.} \\
    \begin{cases}
        \lambda = 0 & \text{Oder} \\
        \lambda + 3 = 0
    \end{cases} \\
    \lambda_2 = 0 \\
    \lambda_3 + 3 = 0 \quad | -3 \\
    \lambda_3 = -3 \\\\
    \lambda_1 = 0, \quad \lambda_2 = 0, \quad \lambda_3 = -3
\end{align*}

\subsection{c}

\begin{align*}
    C := \begin{pmatrix}
        0 & 0 & 0 \\
        0 & 0 & 0 \\
        0 & 0 & 0
    \end{pmatrix}
\end{align*}

\begin{align*}
    C - \lambda \cdot I \\
    \begin{pmatrix}
        0 & 0 & 0 \\
        0 & 0 & 0 \\
        0 & 0 & 0
    \end{pmatrix} - \lambda \cdot \begin{pmatrix}
        1 & 0 & 0 \\
        0 & 1 & 0 \\
        0 & 0 & 1
    \end{pmatrix} \\
    \begin{pmatrix}
        0 & 0 & 0 \\
        0 & 0 & 0 \\
        0 & 0 & 0
    \end{pmatrix} - \begin{pmatrix}
        \lambda & 0 & 0 \\
        0 & \lambda & 0 \\
        0 & 0 & \lambda
    \end{pmatrix} \\
    \begin{pmatrix}
        -\lambda & 0 & 0 \\
        0 & -\lambda & 0 \\
        0 & 0 & -\lambda
    \end{pmatrix} =: C_\lambda \\
    \det(C_\lambda) = -\lambda \cdot -\lambda \cdot -\lambda \\
    = -\lambda^3 \\\\
    -\lambda^3 = 0 \quad |\cdot (-1) \\
    \Leftrightarrow \lambda^3 = 0 \quad | \sqrt[3]{0} \\
    \Leftrightarrow \lambda_{1, 2, 3} = 0
\end{align*}

\subsection{d}

\begin{align*}
    D := \begin{pmatrix}
        0 & 1 \\
        2 & 3
    \end{pmatrix}
\end{align*}

\begin{align*}
    D - \lambda \cdot I \\
        \begin{pmatrix}
        0 & 1 \\
        2 & 3
    \end{pmatrix} - \lambda \cdot \begin{pmatrix}
        1 & 0 \\
        0 & 1
    \end{pmatrix} \\
    \begin{pmatrix}
        0 & 1 \\
        2 & 3
    \end{pmatrix} - \begin{pmatrix}
        \lambda & 0 \\
        0 & \lambda
    \end{pmatrix} \\
    \begin{pmatrix}
        -\lambda & 1 \\
        2 & 3 - \lambda
    \end{pmatrix} =: D_\lambda\\\\
    \det(D_\lambda) = -\lambda \cdot (3 - \lambda) - 2 \cdot 1 \\
    = \lambda^2 - 3 \lambda - 2 \\\\
    \lambda^2 - 3 \lambda - 2 = 0 \quad | PQ\\
    p = -3 \quad q = -2 \\
    -\frac{-3}{2} \pm \sqrt{\left(\frac{-3}{2}\right)^2 + 2} \\
    \frac{3}{2} \pm \sqrt{\left(\frac{9}{4}\right) + 2} \\
    \frac{3}{2} \pm \sqrt{\frac{17}{4}} \\
    \lambda_1 = \frac{3}{2} + \sqrt{\frac{17}{4}}, \quad \lambda_2 = \frac{3}{2} - \sqrt{\frac{17}{4}}
\end{align*}

\section{Aufgabe 2}

\subsection{a}

Ist die Matrix $A := \begin{pmatrix}
    1 & 2 & 3 \\
    4 & 5 & 6 \\
    7 & 8 & 9
\end{pmatrix}$ positiv definit?

\subsection{b}
Wie kann man das Hauptminorenkriterium nutzen, um zu zeigen, dass eine Matrix negativ definit ist? Nutzen Sie Ihr Ergebnis, um zu zeigen, dass die Matrix $B := \begin{pmatrix}
    -1 & 1 \\
    -2 & 1
\end{pmatrix}$

\section{Aufgabe 3}

\subsection{a}

Für welche Winkel $\varphi \in [0, 2\pi)$ hat eine Drehmatrix $A_\varphi \in \mathbb{R}^{2 \times 2}$ reele Eigenwerte? Lösen Sie diese Aufgabe einerseits durch geometrische Argumentation, andererseits rechnerisch.

\subsection{b}

Bestimmen Sie mindestens eine Matrix, die in $O(n) \backslash SO(n)$ enthalten ist.