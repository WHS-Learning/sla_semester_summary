\chapter{Übungsblatt 10}

\section{Aufgabe 1}

Bestimmen Sie sämtliche Eigenwerte der folgenden Matrizen:

\subsection{a}

\begin{align*}
    A := \begin{pmatrix}
        1 & 2 & 3 \\
        4 & 5 & 6 \\ 
        0 & 0 & 0
    \end{pmatrix}
\end{align*}

\subsection{b}

\begin{align*}
    B := \begin{pmatrix}
        1 & 1 & 1 \\
        1 & 1 & 1 \\
        1 & 1 & 1
    \end{pmatrix}
\end{align*}

\subsection{c}

\begin{align*}
    C := \begin{pmatrix}
        0 & 0 & 0 \\
        0 & 0 & 0 \\
        0 & 0 & 0
    \end{pmatrix}
\end{align*}

\subsection{d}

\begin{align*}
    B = \begin{pmatrix}
        0 & 1 \\
        2 & 3
    \end{pmatrix}
\end{align*}

\section{b}

\subsection{a}
Ist die Matrix $A := \begin{pmatrix}
    1 & 2 & 3 \\
    4 & 5 & 6 \\
    7 & 8 & 9
\end{pmatrix}$ positiv definit?

\subsection{b}
Wie kann man das Hauptminorenkriterium nutzen, um zu zeigen, dass eine Matrix negativ definit ist? Nutzen Sie Ihr Ergebnis, um zu zeigen, dass die Matrix $B := \begin{pmatrix}
    -1 & 1 \\
    -2 & 1
\end{pmatrix}$

\section{Aufgabe 3}

\subsection{a}

Für welche Winkel $\varphi \in [0, 2\pi)$ hat eine Drehmatrix $A_\varphi \in \mathbb{R}^{2 \times 2}$ reele Eigenwerte? Lösen Sie diese Aufgabe einerseits durch geometrische Argumentation, andererseits rechnerisch.

\subsection{b}

Bestimmen Sie mindestens eine Matrix, die in $O(n) \backslash SO(n)$ enthalten ist.