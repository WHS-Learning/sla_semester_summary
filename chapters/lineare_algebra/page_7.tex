\allowdisplaybreaks

\chapter{Übungsblatt 7}

\section{Aufgabe 1}

\subsection{a}
\begin{align*}
    A = \begin{pmatrix}
            1 & 2 & 3 \\ 1 & 5 & 0 \\ -1 & -1 & 7
        \end{pmatrix} \cdot \begin{pmatrix}
                                4 & 2 & 1 \\ 1 & 6 & 1 \\ 0 & 1 & 7
                            \end{pmatrix}                                                    \\
    = \begin{pmatrix}
          1 \cdot 4 + 2 \cdot 1 + 3 \cdot 0 & 1 \cdot 2 + 2 \cdot 6 + 3 \cdot 1  & 1 \cdot 1 + 2 \cdot 1 + 3 \cdot 7  \\
          1 \cdot 4 + 5 \cdot 1 + 0 \cdot 0 & 1 \cdot 2 + 5 \cdot 6 + 0 \cdot 1  & 1 \cdot 1 + 5 \cdot 1 + 0 \cdot 7  \\
          -1 \cdot 4 -1 \cdot 1 + 7 \cdot 4 & -1 \cdot 2 - 1 \cdot 6 + 7 \cdot 1 & -1 \cdot 1 - 1 \cdot 1 + 7 \cdot 7
      \end{pmatrix} \\
    \begin{pmatrix}
        6 & 17 & 24 \\ 9 & 32 & 6 \\ -5 & -1 & 47
    \end{pmatrix}
\end{align*}

\subsection{b}
\begin{align*}
    w = \begin{pmatrix}
            0 & 4 & 3 \\ 5 & 5 & 9 \\ 1 & 0 & 7
        \end{pmatrix} \cdot \begin{pmatrix}
                                1 \\ 0 \\ 5
                            \end{pmatrix}                                                                       \\
    = \begin{pmatrix}
          0 \cdot 1 + 4 \cdot 0 + 3 \cdot 5 \\ 5 \cdot 1 + 5 \cdot 0 + 9 \cdot 5 \\ 1 \cdot 1 + 0 \cdot 0 + 7 \cdot 5
      \end{pmatrix} \\
    = \begin{pmatrix}
          15 \\ 50 \\ 36
      \end{pmatrix}
\end{align*}

\subsection{c}
\begin{align*}
    A \cdot w                                \\
    = \begin{pmatrix}
          6 \cdot 15 + 17 \cdot 50 + 24 \cdot 36 \\
          9 \cdot 15 + 32 \cdot 50 + 6 \cdot 36  \\
          -5 \cdot 15 - 1 \cdot 50 + 47 \cdot 36
      \end{pmatrix} \\
    = \begin{pmatrix}
          1804 \\ 1951 \\ 1567
      \end{pmatrix}
\end{align*}

\section{Aufgabe 2}

Gegeben seien die Vektoren
\[
    u_1 = \begin{pmatrix}
        1 \\ 2 \\ 0
    \end{pmatrix}, u_2 = \begin{pmatrix}
        1 \\ -1 \\ 0
    \end{pmatrix}, u_3 = \begin{pmatrix}
        0 \\ 0 \\ 1
    \end{pmatrix}, u_4 = \begin{pmatrix}
        4 \\ 2 \\ 2
    \end{pmatrix}, w = \begin{pmatrix}
        2 \\ 3 \\ 5
    \end{pmatrix}
\]
sowie die lineare Abbildung $\varphi: \mathbb{R}^3 \rightarrow \mathbb{R}^3$
mit:
\[
    \varphi(u_1) = \begin{pmatrix}
        0 \\ 1 \\ 0
    \end{pmatrix},
    \varphi(u_2) = \begin{pmatrix}
        1 \\ 2 \\ 3
    \end{pmatrix},
    \varphi(u_3) = \begin{pmatrix}
        2 \\ -1 \\ 7
    \end{pmatrix}
\]

\subsection{a}
Zeigen Sie, dass die Vektoren $\{u_1, u_2, u_3\}$ eine Basis des $\mathbb{R}^3$
bilden.

Prüfen auf lineare Unabhängigkeit

\begin{align*}
    x_1u_1 + x_2u_2 + x_3u_3 = 0                                                            \\
    \Leftrightarrow x_1\begin{pmatrix}
                           1 \\ 2 \\ 0
                       \end{pmatrix} + x_2\begin{pmatrix}
                                              1 \\ -1 \\ 0
                                          \end{pmatrix} + x_3\begin{pmatrix}
                                                                 0 \\ 0 \\ 1
                                                             \end{pmatrix} = \begin{pmatrix}
                                                                                 0 \\ 0 \\ 0
                                                                             \end{pmatrix} \\
    \Leftrightarrow \begin{pmatrix}
                        x_1 \\ 2x_1 \\ 0
                    \end{pmatrix} + \begin{pmatrix}
                                        x_2 \\ -x_2 \\ 0
                                    \end{pmatrix} + \begin{pmatrix}
                                                        0 \\ 0 \\ x_3
                                                    \end{pmatrix} = \begin{pmatrix}
                                                                        0 \\ 0 \\ 0
                                                                    \end{pmatrix}          \\
    \begin{cases}
        \text{I:\@}   & x_1 + x_2 = 0  \\
        \text{II:\@}  & 2x_1 - x_2 = 0 \\
        \text{III:\@} & x_3 = 0        \\
    \end{cases}                                                          \\
    \text{I = II}                                                                           \\
    x_1 + x_2 = 2x_1 - x_2 \quad | -x_1 \quad | +x_2                                        \\
    \Leftrightarrow  2x_2 = x_1                                                             \\
    \Leftrightarrow x_1 = 2x_2                                                              \\
    \text{in I einsetzen}                                                                   \\
    2x_2 + x_2 = 0                                                                          \\
    \Leftrightarrow 3x_2 = 0 \quad | : 3                                                    \\
    \Leftrightarrow x_2 = 0                                                                 \\
    x_1 = 2 \cdot 0                                                                         \\
    \Leftrightarrow x_1 = 0
\end{align*}

Die einzige Lösung des linearen Gleichungssystems ist $x_1 = x_2 = x_3 = 0$.
Das bedeutet, dass die Vektoren $\left\{u_1, u_2, u_3\right\}$ linear
unabhängig sind. Dementsprechend bilden sie eine Basis des $\mathbb{R}^3$

\subsection{b}
Berechnen Sie $\varphi(u_4)$

\begin{align*}
    \begin{pmatrix}
        1 & 1 & 0 \\ 2 & -1 & 0 \\ 0 & 0 & 1
    \end{pmatrix} \cdot \begin{pmatrix}
                            x_1 \\ x_2 \\ x_3
                        \end{pmatrix} = \begin{pmatrix}
                                            4 \\ 2 \\ 2
                                        \end{pmatrix}          \\
    \begin{cases}
        \text{I:\@}   & x_1 + x_2 = 4  \\
        \text{II:\@}  & 2x_1 - x_2 = 2 \\
        \text{III:\@} & x_3 = 2
    \end{cases}                              \\
    \text{I + II}
    \begin{cases}
        \text{I:\@}   & 3x_1 = 6 \quad | :3 \Leftrightarrow x_1 = 2 \\
        \text{II:\@}  & 2x_1 - x_2 = 2                              \\
        \text{III:\@} & x_3 = 2
    \end{cases} \\
    \text{in II einsetzen}                                      \\
    2 \cdot 2 - x_2 = 2                                         \\
    \Leftrightarrow 4 - x_2 = 2 \quad | -4 \quad | \cdot(-1)    \\
    \Leftrightarrow x_2 = 2
\end{align*}

$u_4$ ist linear abhängig zu den Vektoren $\left\{u_1, u_2, u_3\right\}$ mit den Faktor 2.

\begin{align*}
    u_1, u_2, u_3 = u_4                                                                  \\
    \varphi(u_1), \varphi(u_2), \varphi(u_3) = \varphi(u_4)                              \\
    \begin{pmatrix}
        0 & 1 & 2 \\ 1 & 2 & -1 \\ 0 & 3 & 7
    \end{pmatrix} \cdot \begin{pmatrix}
                            2 \\ 2 \\ 2
                        \end{pmatrix} = \varphi(u_4)                                     \\
    2 \cdot \begin{pmatrix}
                0 \\ 1 \\ 0
            \end{pmatrix} + 2 \cdot \begin{pmatrix}
                                        1 \\ 2 \\ 3
                                    \end{pmatrix} + 2 \cdot \begin{pmatrix}
                                                                2 \\ -1 \\ 7
                                                            \end{pmatrix} = \varphi(u_4) \\
    = \begin{pmatrix}
          0 \\ 2 \\ 0
      \end{pmatrix} + \begin{pmatrix}
                          2 \\ 4 \\ 6
                      \end{pmatrix} + \begin{pmatrix}
                                          4 \\ -2 \\ 14
                                      \end{pmatrix} = \varphi(u_4)                       \\
    = \begin{pmatrix}
          6 \\ 4 \\ 20
      \end{pmatrix} = \varphi(u_4)
\end{align*}

\subsection{c}
Geben Sie einen Vektor $u_5$ an, mit $\varphi(u_5) = w$.

\begin{align*}
    \varphi(u_5) = x_1 \cdot \varphi(u_1) + x_2 \cdot \varphi(u_2) + x_3 \cdot \varphi(u_3)                     \\
    \varphi(u_5) = x_1 \cdot \begin{pmatrix}
                                 0 \\ 1 \\ 0
                             \end{pmatrix} + x_2 \cdot \begin{pmatrix}
                                                           1 \\ 2 \\ 3
                                                       \end{pmatrix} + x_3 \cdot \begin{pmatrix}
                                                                                     2 \\ -1 \\ 7
                                                                                 \end{pmatrix} = \begin{pmatrix}
                                                                                                     2 \\ 3 \\ 5
                                                                                                 \end{pmatrix} \\
    \begin{cases}
        \text{I:\@}   & x_2 + 2x_3 = 2       \\
        \text{II:\@}  & x_1 + 2x_2 - x_3 = 3 \\
        \text{III:\@} & 3x_2 + 7x_3 = 5
    \end{cases}
\end{align*}

\begin{longtable}{p{4cm}|p{3cm}}

    \hline
    \multicolumn{1}{c|}{\textbf{Linearkombination}} & \multicolumn{1}{c}{\textbf{Konstanten}} \\
    \hline
    \endfirsthead

    \hline
    \multicolumn{2}{c}{\tablename\ \thetable\ -- \textit{Fortführung von vorherier Seite}}    \\
    \hline
    \multicolumn{1}{c|}{\textbf{Linearkombination}} & \multicolumn{1}{c}{\textbf{Konstanten}} \\
    \hline
    \endhead

    \hline
    \multicolumn{2}{r}{\textit{Fortsetzung siehe nächste Seite}}                              \\
    \endfoot

    \hline
    \endlastfoot

    $\displaystyle\begin{matrix}
                          0 & 1 & 2  \\
                          1 & 2 & -1 \\
                          0 & 3 & 7
                      \end{matrix}$                    &
    $\displaystyle\begin{matrix}
                          2 \\ 3 \\ 5
                      \end{matrix}$                                                               \\\hline

    \multicolumn{2}{p{\dimexpr4cm+3cm+2\tabcolsep\relax}}{Operation: I und II tauschen}       \\\hline\pagebreak[0]

    $\displaystyle\begin{matrix}
                          1 & 2 & -1 \\
                          0 & 1 & 2  \\
                          0 & 3 & 7
                      \end{matrix}$                    &
    $\displaystyle\begin{matrix}
                          3 \\ 2 \\ 5
                      \end{matrix}$                                                               \\\hline

    \multicolumn{2}{p{\dimexpr4cm+3cm+2\tabcolsep\relax}}{Operation: III - 3II}               \\\hline\pagebreak[0]

    $\displaystyle\begin{matrix}
                          1 & 2 & -1 \\
                          0 & 1 & 2  \\
                          0 & 0 & 1
                      \end{matrix}$                    &
    $\displaystyle\begin{matrix}
                          3 \\ 2 \\ -1
                      \end{matrix}$                                                               \\\hline

    \multicolumn{2}{p{\dimexpr4cm+3cm+2\tabcolsep\relax}}{Operation: II - 2III}               \\\hline\pagebreak[0]

    $\displaystyle\begin{matrix}
                          1 & 2 & -1 \\
                          0 & 1 & 0  \\
                          0 & 0 & 1
                      \end{matrix}$                    &
    $\displaystyle\begin{matrix}
                          3 \\ 4 \\ -1
                      \end{matrix}$                                                               \\\hline

    \multicolumn{2}{p{\dimexpr4cm+3cm+2\tabcolsep\relax}}{Operation: I + III}                 \\\hline\pagebreak[0]

    $\displaystyle\begin{matrix}
                          1 & 2 & 0 \\
                          0 & 1 & 0 \\
                          0 & 0 & 1
                      \end{matrix}$                    &
    $\displaystyle\begin{matrix}
                          2 \\ 4 \\ -1
                      \end{matrix}$                                                               \\\hline

    \multicolumn{2}{p{\dimexpr4cm+3cm+2\tabcolsep\relax}}{Operation: I - 2II}                 \\\hline\pagebreak[0]

    $\displaystyle\begin{matrix}
                          1 & 0 & 0 \\
                          0 & 1 & 0 \\
                          0 & 0 & 1
                      \end{matrix}$                    &
    $\displaystyle\begin{matrix}
                          -6 \\ 4 \\ -1
                      \end{matrix}$                                                               \\\hline
\end{longtable}

\begin{align*}
    x_1 = -6, \quad x_2 = 4, \quad x_3 = -1                                       \\
    u_5 = -6 \cdot \begin{pmatrix}
                       1 \\ 2 \\ 0
                   \end{pmatrix} + 4 \cdot \begin{pmatrix}
                                               1 \\ -1 \\ 0
                                           \end{pmatrix} - 1 \cdot \begin{pmatrix}
                                                                       0 \\ 0 \\ 1
                                                                   \end{pmatrix} \\
    = \begin{pmatrix}
          -6 \\ -12 \\ 0
      \end{pmatrix} + \begin{pmatrix}
                          4 \\-4 \\ 0
                      \end{pmatrix} + \begin{pmatrix}
                                          0 \\ 0 \\ -1
                                      \end{pmatrix}                              \\
    = \begin{pmatrix}
          -2 \\ -16 \\ -1
      \end{pmatrix} = u_5
\end{align*}

\subsection{d}
Geben Sie die lineare Abbildung $\varphi$ in der Form $\varphi(x) = Ax$ an.

\begin{align*}
    c_{1,1} \cdot \begin{pmatrix}
                      1 \\ 2 \\ 0
                  \end{pmatrix} + c_{2, 1} \cdot \begin{pmatrix}
                                                     1 \\ -1 \\ 0
                                                 \end{pmatrix} + c_{3, 1} \cdot \begin{pmatrix}
                                                                                    0 \\ 0 \\ 1
                                                                                \end{pmatrix} = \begin{pmatrix}
                                                                                                    1 \\ 0 \\ 0
                                                                                                \end{pmatrix}  \\
    \begin{cases}
        \text{I:\@}   & c_{1,1} + c_{2,1} = 1  \\
        \text{II:\@}  & 2c_{1,1} - c_{2,1} = 0 \\
        \text{III:\@} & c_{3,1} = 0
    \end{cases}                                                                      \\
    \text{I + II}                                                                                               \\
    \begin{cases}
        \text{I:\@}   & 3c_{1,1}  = 1 \quad | : 3 \Leftrightarrow c_{1,1} = \frac{1}{3} \\
        \text{II:\@}  & 2c_{1,1} - c_{2,1} = 0                                          \\
        \text{III:\@} & c_{3,1} = 0
    \end{cases}                             \\
    \text{in II einsetzen}                                                                                      \\
    \begin{cases}
        \text{I:\@}   & c_{1,1} = \frac{1}{3}                                                                       \\
        \text{II:\@}  & 2 \cdot \frac{1}{3} - c_{2, 1} = 0 \quad | +c_{2, 1} \Leftrightarrow \frac{2}{3} = c_{2, 1} \\
        \text{III:\@} & c_{3,1} = 0
    \end{cases} \\
\end{align*}

\begin{align*}
    c_{1,2} \cdot \begin{pmatrix}
                      1 \\ 2 \\ 0
                  \end{pmatrix} + c_{2, 2} \cdot \begin{pmatrix}
                                                     1 \\ -1 \\ 0
                                                 \end{pmatrix} + c_{3, 2} \cdot \begin{pmatrix}
                                                                                    0 \\ 0 \\ 1
                                                                                \end{pmatrix} = \begin{pmatrix}
                                                                                                    0 \\ 1 \\ 0
                                                                                                \end{pmatrix}                \\
    \begin{cases}
        \text{I:\@}   & c_{1,2} + c_{2,2} = 0  \\
        \text{II:\@}  & 2c_{1,2} - c_{2,2} = 1 \\
        \text{III:\@} & c_{3,2} = 0
    \end{cases}                                                                                    \\
    \text{I + II}                                                                                                             \\
    \begin{cases}
        \text{I:\@}   & 3c_{1,2}  = 1 \quad | : 3 \Leftrightarrow c_{1,2} = \frac{1}{3} \\
        \text{II:\@}  & 2c_{1,2} - c_{2,2} = 1                                          \\
        \text{III:\@} & c_{3,2} = 0
    \end{cases}                                           \\
    \text{in II einsetzen}                                                                                                    \\
    \begin{cases}
        \text{I:\@}   & c_{1,2} = \frac{1}{3}                                                                                     \\
        \text{II:\@}  & 2 \cdot \frac{1}{3} - c_{2, 2} = 1 \quad | + c_{2, 2}  \quad | -1 \Leftrightarrow -\frac{1}{3} = c_{2, 2} \\
        \text{III:\@} & c_{3,2} = 0
    \end{cases} \\
\end{align*}

\begin{align*}
    c_{1,3} \cdot \begin{pmatrix}
                      1 \\ 2 \\ 0
                  \end{pmatrix} + c_{2, 3} \cdot \begin{pmatrix}
                                                     1 \\ -1 \\ 0
                                                 \end{pmatrix} + c_{3, 3} \cdot \begin{pmatrix}
                                                                                    0 \\ 0 \\ 1
                                                                                \end{pmatrix} = \begin{pmatrix}
                                                                                                    0 \\ 0 \\ 1
                                                                                                \end{pmatrix} \\
    \begin{cases}
        \text{I:\@}   & c_{1,3} + c_{2,3} = 0  \\
        \text{II:\@}  & 2c_{1,3} - c_{2,3} = 0 \\
        \text{III:\@} & c_{3,3} = 1
    \end{cases}                                                                     \\
    \text{I + II}                                                                                              \\
    \begin{cases}
        \text{I:\@}   & 3c_{1,3}  = 0 \quad | : 3 \Leftrightarrow c_{1,3} = 0 \\
        \text{II:\@}  & 2c_{1,3} - c_{2,3} = 0                                \\
        \text{III:\@} & c_{3,3} = 1
    \end{cases}                                      \\
    \text{in II einsetzen}                                                                                     \\
    \begin{cases}
        \text{I:\@}   & c_{1,3} = 0                                                              \\
        \text{II:\@}  & 2 \cdot 0 - c_{2, 3} = 0 \quad | \cdot (-1) \Leftrightarrow c_{2, 3} = 0 \\
        \text{III:\@} & c_{3,3} = 1
    \end{cases}                   \\
\end{align*}

\begin{align*}
    \varphi(e_1) = \frac{1}{3} \cdot \varphi(u_1) + \frac{2}{3} \varphi(u_2)          \\
    \varphi(e_1) = \frac{1}{3} \cdot \begin{pmatrix}
                                         0 \\ 1 \\ 0
                                     \end{pmatrix} + \frac{2}{3} \cdot \begin{pmatrix}
                                                                           1 \\ 2 \\ 3
                                                                       \end{pmatrix} \\
    \varphi(e_1) = \begin{pmatrix}
                       0 \\ \frac{1}{3} \\ 0
                   \end{pmatrix} + \begin{pmatrix}
                                       \frac{2}{3} \\ \frac{4}{3} \\ 2
                                   \end{pmatrix}                     \\
    \varphi(e_1) = \begin{pmatrix}
                       \frac{2}{3} \\ \frac{5}{3} \\ 2
                   \end{pmatrix}                                    \\
    \varphi(e_2) = \frac{1}{3} \cdot \varphi(u_1) - \frac{1}{3} \cdot \varphi(u_2)    \\
    \varphi(e_2) = \frac{1}{3} \cdot \begin{pmatrix}
                                         0 \\ 1 \\ 0
                                     \end{pmatrix} -\frac{1}{3} \cdot \begin{pmatrix}
                                                                          1 \\ 2 \\ 3
                                                                      \end{pmatrix}  \\
    \varphi(e_2) = \begin{pmatrix}
                       0 \\ \frac{1}{3} \\ 0
                   \end{pmatrix} + \begin{pmatrix}
                                       -\frac{1}{3} \\ -\frac{2}{3} \\ -1
                                   \end{pmatrix}                  \\
    \varphi(e_2) = \begin{pmatrix}
                       -\frac{1}{3} \\ -\frac{1}{3} \\ -1
                   \end{pmatrix}                                  \\
    \varphi(e_3) = \varphi(u_3)                                                       \\
    \varphi(e_3) = \begin{pmatrix}
                       2 \\ -1 \\ 7
                   \end{pmatrix}                                                     \\
    A = \begin{pmatrix}
            \frac{2}{3} & -\frac{1}{3} & 2  \\
            \frac{5}{3} & -\frac{2}{3} & -1 \\
            2           & -1           & 7
        \end{pmatrix}                                               \\
    \varphi(x) = \begin{pmatrix}
                     \frac{2}{3} & -\frac{1}{3} & 2  \\
                     \frac{5}{3} & -\frac{2}{3} & -1 \\
                     2           & -1           & 7
                 \end{pmatrix} \cdot \begin{pmatrix}
                                         x_1 \\ x_2 \\ x_3
                                     \end{pmatrix}
\end{align*}

\section{Aufgabe 3}

\subsection{a}
Bestimmen Sie eine lineare Abbildung $T: \mathbb{R}^3 \rightarrow
    \mathbb{R}^3$, deren Kern nur den Nullvektor enthält. Bestimmen Sie weiterhin
eine lineare Abbildung $S: \mathbb{R}^3 \rightarrow \mathbb{R}^3$, deren Kern
der gesamte Raum $\mathbb{R}^3$ ist.

\[
    T = \begin{pmatrix}
        1 & 0 & 0 \\
        0 & 1 & 0 \\
        0 & 0 & 1
    \end{pmatrix}
\]

\[
    S = \begin{pmatrix}
        0 & 0 & 0 \\
        0 & 0 & 0 \\
        0 & 0 & 0
    \end{pmatrix}
\]

\subsection{b}
Bestimmen Sie Bild und Kern der Matrix
\[
    A=\begin{pmatrix}
        2 & 2 & 3 \\ 1 & 1 & 0 \\ -1 & -1 & 1
    \end{pmatrix}
\]

\subsection*{Bild}

\begin{align*}
    n = \dim(Bild(A)) + \dim(Kern(A)) \\
    3 = \dim(Bild(A)) + 1 \quad |-1   \\
    \Leftrightarrow  2 = \dim(Bild(A))
\end{align*}

\subsection*{Kern}
\begin{align*}
    \begin{pmatrix}
        2 & 2 & 3 \\ 1 & 1 & 0 \\ -1 & -1 & 1
    \end{pmatrix} \cdot \begin{pmatrix}
                            x_1 \\ x_2 \\ x_3
                        \end{pmatrix} = \begin{pmatrix}
                                            0 \\ 0 \\ 0
                                        \end{pmatrix}                          \\\pagebreak[0]
    \begin{cases}
        \text{I:\@}   & 2x_1 + 2x_2 + 3x_3 = 0 \\
        \text{II:\@}  & x_1 + x_2 = 0          \\
        \text{III:\@} & -x_1 - x_2 + x_3 = 0
    \end{cases}                                      \\\pagebreak[0]
    \text{III + II}                                                             \\\pagebreak[0]
    \begin{cases}
        \text{I:\@}   & 2x_1 + 2x_2 + 3x_3 = 0 \\
        \text{II:\@}  & x_1 + x_2 = 0          \\
        \text{III:\@} & x_3 = 0
    \end{cases}                                      \\
    \text{in I einsetzen}                                                       \\\pagebreak[0]
    \begin{cases}
        \text{I:\@}   & 2x_1 + 2x_2 + 3 \cdot 0 = 0 \Leftrightarrow 2x_1 + 2x_2 = 0 \\
        \text{II:\@}  & x_1 + x_2 = 0 \quad | -x_2 \Leftrightarrow x_1 = -x_2       \\
        \text{III:\@} & x_3 = 0
    \end{cases} \\\pagebreak[0]
    \text{in I einsetzen}                                                       \\\pagebreak[0]
    \begin{cases}
        \text{I:\@}   & 2 \cdot -x_2 + 2x_2 = 0 \Leftrightarrow 0 = 0 \\
        \text{II:\@}  & x_1 = -x_2                                    \\
        \text{III:\@} & x_3 = 0
    \end{cases}               \\\pagebreak[0]
    \begin{pmatrix}
        x_1 \\ x_2 \\ x_3
    \end{pmatrix} = \begin{pmatrix}
                        -x_2 \\ x_2 \\ 0
                    \end{pmatrix} \text{, wobei } x_2 \in \mathbb{R}            \\
    span\left\{\begin{pmatrix}
                   -1 \\ 1 \\ 0
               \end{pmatrix}\right\}
\end{align*}

\section{Aufgabe 4}

\subsection{a}
Es sei $T: \mathbb{R}^3 \rightarrow \mathbb{R}^3$ gegeben durch: $T \begin{pmatrix} x \\ y \\ z \end{pmatrix} = \begin{pmatrix}2x \\ 4x-y \\ 2x+3y\end{pmatrix}$. Zeigen Sie, dass $T$ invertierbar ist, und geben Sie eine Formel für $T^{-1}$ an.

\subsubsection*{Prüfen, ob Matrix invertierbar ist}
Dass eine Matrix invertierbar ist, muss sie linear unabhängig sein. $\rightarrow \det(T) \neq 0$.
\begin{align*}
    \det\left(\begin{pmatrix}
                  2 & 0  & 0 \\
                  4 & -1 & 0 \\
                  2 & 3  & 0
              \end{pmatrix}\right) = 2 \cdot -1 \cdot 0 + 0 \cdot 0 \cdot 0 + 0 \cdot 4 \cdot 3 \\
    - 2 \cdot -1 \cdot 0 - 3 \cdot 0 \cdot 0 - 0 \cdot 4 \cdot 0                                \\
    \det\left(\begin{pmatrix}
                  2 & 0  & 0 \\
                  4 & -1 & 0 \\
                  2 & 3  & 0
              \end{pmatrix}\right) = 0
\end{align*}

Die Matrix ist nicht invertierbar.

\subsection{b}
Bestimmen Sie die inverse Matrix für

\[
    A = \begin{pmatrix}
        2 & 2 & 3 \\ 1 & -1 & 0 \\ -1 & 2 & 1
    \end{pmatrix}
\]

\subsubsection*{Prüfen, ob Matrix invertierbar ist}

\begin{align*}
    \det\left(
    \begin{pmatrix}
        2  & 2  & 3 \\
        1  & -1 & 0 \\
        -1 & 2  & 1
    \end{pmatrix}
    \right) = 2 \cdot -1 \cdot 1 + 2 \cdot 0 \cdot -1 + 3 \cdot 1 \cdot 2 \\
    - -1 \cdot -1 \cdot 3 - 2 \cdot 0 \cdot 2 - 1 \cdot 1 \cdot 2         \\
    =-2 + 0 + 6 - 3 - 0 - 2                                               \\
    = -1
\end{align*}

Die Matrix ist invertierbar.

\begin{longtable}{p{4cm}|p{3cm}}

    \hline
    \multicolumn{1}{c|}{\textbf{Matrix}} & \multicolumn{1}{c}{\textbf{Einheitsmatrix}}     \\
    \hline
    \endfirsthead

    \hline
    \multicolumn{2}{c}{\tablename\ \thetable\ -- \textit{Fortführung von vorherier Seite}} \\
    \hline
    \multicolumn{1}{c|}{\textbf{Matrix}} & \multicolumn{1}{c}{\textbf{Einheitsmatrix}}     \\
    \hline
    \endhead

    \hline
    \multicolumn{2}{r}{\textit{Fortsetzung siehe nächste Seite}}                           \\
    \endfoot

    \hline
    \endlastfoot

    $\displaystyle\begin{matrix}
                          2  & 2  & 3 \\
                          1  & -1 & 0 \\
                          -1 & 2  & 1
                      \end{matrix}$         &
    $\displaystyle\begin{matrix}
                          1 & 0 & 0 \\
                          0 & 1 & 0 \\
                          0 & 0 & 1
                      \end{matrix}$                                                            \\\hline

    \multicolumn{2}{p{\dimexpr4cm+3cm+2\tabcolsep\relax}}{Operation: II + III}             \\\hline\pagebreak[0]

    $\displaystyle\begin{matrix}
                          2  & 2 & 3 \\
                          0  & 1 & 1 \\
                          -1 & 2 & 1
                      \end{matrix}$         &
    $\displaystyle\begin{matrix}
                          1 & 0 & 0 \\
                          0 & 1 & 1 \\
                          0 & 0 & 1
                      \end{matrix}$                                                            \\\hline

    \multicolumn{2}{p{\dimexpr4cm+3cm+2\tabcolsep\relax}}{Operation: I + III}              \\\hline\pagebreak[0]

    $\displaystyle\begin{matrix}
                          1  & 4 & 4 \\
                          0  & 1 & 1 \\
                          -1 & 2 & 1
                      \end{matrix}$         &
    $\displaystyle\begin{matrix}
                          1 & 0 & 1 \\
                          0 & 1 & 1 \\
                          0 & 0 & 1
                      \end{matrix}$                                                            \\\hline

    \multicolumn{2}{p{\dimexpr4cm+3cm+2\tabcolsep\relax}}{Operation: III + I}              \\\hline\pagebreak[0]

    $\displaystyle\begin{matrix}
                          1 & 4 & 4 \\
                          0 & 1 & 1 \\
                          0 & 6 & 5
                      \end{matrix}$         &
    $\displaystyle\begin{matrix}
                          1 & 0 & 1 \\
                          0 & 1 & 1 \\
                          1 & 0 & 2
                      \end{matrix}$                                                            \\\hline

    \multicolumn{2}{p{\dimexpr4cm+3cm+2\tabcolsep\relax}}{Operation: III - 6II}            \\\hline\pagebreak[0]

    $\displaystyle\begin{matrix}
                          1 & 4 & 4  \\
                          0 & 1 & 1  \\
                          0 & 0 & -1
                      \end{matrix}$         &
    $\displaystyle\begin{matrix}
                          1 & 0  & 1  \\
                          0 & 1  & 1  \\
                          1 & -6 & -4
                      \end{matrix}$                                                            \\\hline

    \multicolumn{2}{p{\dimexpr4cm+3cm+2\tabcolsep\relax}}{Operation: I - 4II}              \\\hline\pagebreak[0]

    $\displaystyle\begin{matrix}
                          1 & 0 & 0  \\
                          0 & 1 & 1  \\
                          0 & 0 & -1
                      \end{matrix}$         &
    $\displaystyle\begin{matrix}
                          1 & -4 & -3 \\
                          0 & 1  & 1  \\
                          1 & -6 & -4
                      \end{matrix}$                                                            \\\hline

    \multicolumn{2}{p{\dimexpr4cm+3cm+2\tabcolsep\relax}}{Operation: II + III}             \\\hline\pagebreak[0]

    $\displaystyle\begin{matrix}
                          1 & 0 & 0  \\
                          0 & 1 & 0  \\
                          0 & 0 & -1
                      \end{matrix}$         &
    $\displaystyle\begin{matrix}
                          1 & -4 & -3 \\
                          1 & -5 & -3 \\
                          1 & -6 & -4
                      \end{matrix}$                                                            \\\hline

    \multicolumn{2}{p{\dimexpr4cm+3cm+2\tabcolsep\relax}}{Operation: III $\cdot$ (-1)}     \\\hline\pagebreak[0]

    $\displaystyle\begin{matrix}
                          1 & 0 & 0 \\
                          0 & 1 & 0 \\
                          0 & 0 & 1
                      \end{matrix}$         &
    $\displaystyle\begin{matrix}
                          1  & -4 & -3 \\
                          1  & -5 & -3 \\
                          -1 & 6  & 4
                      \end{matrix}$                                                            \\\hline
\end{longtable}

Die Inverse der Matrix ist also $\begin{pmatrix}
        1  & -4 & -3 \\
        1  & -5 & -3 \\
        -1 & 6  & 4
    \end{pmatrix}.$