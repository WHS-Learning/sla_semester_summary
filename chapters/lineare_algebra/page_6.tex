\chapter{Übungsblatt 6}

\section{Aufgabe 1}
Bestimmen Sie die Lösungsmengen der folgenden linearen Gleichungssysteme mit
Hilfe des Gauß-Jordan-Algorithmus.

\subsection{a}
\begin{align*}
  \begin{cases}
    3x_1 + 2x_2 = 1 \\
    5x_1 + 4x_2 = 5
  \end{cases}
\end{align*}

\begin{longtable}{p{4cm}|p{3cm}}

  \hline
  \multicolumn{1}{c|}{\textbf{Linearkombination}} & \multicolumn{1}{c}{\textbf{Konstanten}} \\
  \hline
  \endfirsthead

  \hline
  \multicolumn{2}{c}{\tablename\ \thetable\ -- \textit{Fortführung von vorherier Seite}}    \\
  \hline
  \multicolumn{1}{c|}{\textbf{Linearkombination}} & \multicolumn{1}{c}{\textbf{Konstanten}} \\
  \hline
  \endhead

  \hline
  \multicolumn{2}{r}{\textit{Fortsetzung siehe nächste Seite}}                              \\
  \endfoot

  \hline
  \endlastfoot

  $\displaystyle\begin{matrix}
                    3 & 2 \\ 5 & 4
                  \end{matrix}$                    &
  $\displaystyle\begin{matrix}
                    1 \\ 5
                  \end{matrix}$                                                               \\\hline

  \multicolumn{2}{p{\dimexpr4cm+3cm+2\tabcolsep\relax}}{Operation: 3II - 5I}                \\\hline\pagebreak[0]

  $\displaystyle\begin{matrix}
                    3 & 2 \\ 0 & 2
                  \end{matrix}$                    &
  $\displaystyle\begin{matrix}
                    1 \\ 10
                  \end{matrix}$                                                               \\\hline

  \multicolumn{2}{p{\dimexpr4cm+3cm+2\tabcolsep\relax}}{Operation: I - II}                  \\\hline\pagebreak[0]

  $\displaystyle\begin{matrix}
                    3 & 0 \\ 0 & 2
                  \end{matrix}$                    &
  $\displaystyle\begin{matrix}
                    -9 \\ 10
                  \end{matrix}$                                                               \\\hline

  \multicolumn{2}{p{\dimexpr4cm+3cm+2\tabcolsep\relax}}{Operation: I : 3}                   \\\hline\pagebreak[0]
  \multicolumn{2}{p{\dimexpr4cm+3cm+2\tabcolsep\relax}}{Operation: II : 2}                  \\\hline\pagebreak[0]

  $\displaystyle\begin{matrix}
                    1 & 0 \\ 0 & 1
                  \end{matrix}$                    &
  $\displaystyle\begin{matrix}
                    -3 \\ 5
                  \end{matrix}$                                                               \\\hline

\end{longtable}

$x_1 = -3 \quad x_2 = 5$

\subsection{b}
\begin{align*}
  \begin{cases}
    \begin{aligned}
      x_1   & + 2x_2 & - x_3  & = 10 \\
      15x_1 & + x_2  & - 4x_3 & = 10 \\
      -x_1  &        & - 3x_3 & = 10
    \end{aligned}
  \end{cases}
\end{align*}

\begin{longtable}{p{4cm}|p{3cm}}

  \hline
  \multicolumn{1}{c|}{\textbf{Linearkombination}} & \multicolumn{1}{c}{\textbf{Konstanten}} \\
  \hline
  \endfirsthead

  \hline
  \multicolumn{2}{c}{\tablename\ \thetable\ -- \textit{Fortführung von vorherier Seite}}    \\
  \hline
  \multicolumn{1}{c|}{\textbf{Linearkombination}} & \multicolumn{1}{c}{\textbf{Konstanten}} \\
  \hline
  \endhead

  \hline
  \multicolumn{2}{r}{\textit{Fortsetzung siehe nächste Seite}}                              \\
  \endfoot

  \hline
  \endlastfoot

  $\displaystyle\begin{matrix}
                    1 & 2 & -1 \\ 15 & 1 & -4 \\ -1 & 0 & -3
                  \end{matrix}$        &
  $\displaystyle\begin{matrix}
                    10 \\ 10 \\ 10
                  \end{matrix}$                                                               \\\hline

  \multicolumn{2}{p{\dimexpr4cm+3cm+2\tabcolsep\relax}}{Operation: II + 15III}              \\\hline\pagebreak[0]

  $\displaystyle\begin{matrix}
                    1 & 2 & -1 \\ 0 & 1 & -49 \\ -1 & 0 & -3
                  \end{matrix}$        &
  $\displaystyle\begin{matrix}
                    10 \\ 160 \\ 10
                  \end{matrix}$                                                              \\\hline

  \multicolumn{2}{p{\dimexpr4cm+3cm+2\tabcolsep\relax}}{Operation: III + I}                 \\\hline\pagebreak[0]

  $\displaystyle\begin{matrix}
                    1 & 2 & -1 \\ 0 & 1 & -49 \\ 0 & 2 & -4
                  \end{matrix}$         &
  $\displaystyle\begin{matrix}
                    10 \\ 160 \\ 20
                  \end{matrix}$                                                              \\\hline

  \multicolumn{2}{p{\dimexpr4cm+3cm+2\tabcolsep\relax}}{Operation: III - 2II}               \\\hline\pagebreak[0]

  $\displaystyle\begin{matrix}
                    1 & 2 & -1 \\ 0 & 1 & -49 \\ 0 & 0 & 94
                  \end{matrix}$         &
  $\displaystyle\begin{matrix}
                    10 \\ 160 \\ -300
                  \end{matrix}$                                                            \\\hline

  \multicolumn{2}{p{\dimexpr4cm+3cm+2\tabcolsep\relax}}{Operation: III : 94}                \\\hline\pagebreak[0]

  $\displaystyle\begin{matrix}
                    1 & 2 & -1 \\ 0 & 1 & -49 \\ 0 & 0 & 1
                  \end{matrix}$          &
  $\displaystyle\begin{matrix}
                    10 \\ 160 \\ -\dfrac{150}{47}
                  \end{matrix}$                                                \\\hline

  \multicolumn{2}{p{\dimexpr4cm+3cm+2\tabcolsep\relax}}{Operation: I + III}                 \\\hline\pagebreak[0]

  $\displaystyle\begin{matrix}
                    1 & 2 & 0 \\ 0 & 1 & -49 \\ 0 & 0 & 1
                  \end{matrix}$           &
  $\displaystyle\begin{matrix}
                    \dfrac{320}{47} \\ 160 \\ -\dfrac{150}{47}
                  \end{matrix}$                                   \\\hline

  \multicolumn{2}{p{\dimexpr4cm+3cm+2\tabcolsep\relax}}{Operation: II + 49III}              \\\hline\pagebreak[0]

  $\displaystyle\begin{matrix}
                    1 & 2 & 0 \\ 0 & 1 & 0 \\ 0 & 0 & 1
                  \end{matrix}$             &
  $\displaystyle\begin{matrix}
                    \dfrac{320}{47} \\ \dfrac{170}{47} \\ -\dfrac{150}{47}
                  \end{matrix}$                       \\\hline

  \multicolumn{2}{p{\dimexpr4cm+3cm+2\tabcolsep\relax}}{Operation: I - 2II}                 \\\hline\pagebreak[0]

  $\displaystyle\begin{matrix}
                    1 & 0 & 0 \\ 0 & 1 & 0 \\ 0 & 0 & 1
                  \end{matrix}$             &
  $\displaystyle\begin{matrix}
                    -\dfrac{20}{47} \\ \dfrac{170}{47} \\ -\dfrac{150}{47}
                  \end{matrix}$                       \\\hline

\end{longtable}

$x_1 = -\frac{20}{47}\quad x_2 = \frac{170}{47}\quad x_3 = -\frac{150}{47}$

\subsection{c}
\begin{align*}
  \begin{cases}
    \begin{aligned}
      x_1 - x_2 - x_3 = 1 \\
      x_1 + x_2 - x_3 = 2
    \end{aligned}
  \end{cases}
\end{align*}

\begin{longtable}{p{4cm}|p{3cm}}

  \hline
  \multicolumn{1}{c|}{\textbf{Linearkombination}} & \multicolumn{1}{c}{\textbf{Konstanten}} \\
  \hline
  \endfirsthead

  \hline
  \multicolumn{2}{c}{\tablename\ \thetable\ -- \textit{Fortführung von vorherier Seite}}    \\
  \hline
  \multicolumn{1}{c|}{\textbf{Linearkombination}} & \multicolumn{1}{c}{\textbf{Konstanten}} \\
  \hline
  \endhead

  \hline
  \multicolumn{2}{r}{\textit{Fortsetzung siehe nächste Seite}}                              \\
  \endfoot

  \hline
  \endlastfoot

  $\displaystyle\begin{matrix}
                    1 & -1 & -1 \\
                    1 & 1  & -1
                  \end{matrix}$                    &
  $\displaystyle\begin{matrix}
                    1 \\ 2
                  \end{matrix}$                                                               \\\hline

  \multicolumn{2}{p{\dimexpr4cm+3cm+2\tabcolsep\relax}}{Operation: II - I}                  \\\hline\pagebreak[0]

  $\displaystyle\begin{matrix}
                    1 & -1 & -1 \\
                    0 & 2  & 0
                  \end{matrix}$                    &
  $\displaystyle\begin{matrix}
                    1 \\ 1
                  \end{matrix}$                                                               \\\hline

  \multicolumn{2}{p{\dimexpr4cm+3cm+2\tabcolsep\relax}}{Operation: 2I + II}                 \\\hline\pagebreak[0]

  $\displaystyle\begin{matrix}
                    2 & 0 & -2 \\
                    0 & 2 & 0
                  \end{matrix}$                    &
  $\displaystyle\begin{matrix}
                    3 \\ 1
                  \end{matrix}$                                                               \\\hline

  \multicolumn{2}{p{\dimexpr4cm+3cm+2\tabcolsep\relax}}{Operation: I : 2}                   \\\hline\pagebreak[0]
  \multicolumn{2}{p{\dimexpr4cm+3cm+2\tabcolsep\relax}}{Operation: II : 2}                  \\\hline\pagebreak[0]

  $\displaystyle\begin{matrix}
                    1 & 0 & -1 \\
                    0 & 1 & 0
                  \end{matrix}$                    &
  $\displaystyle\begin{matrix}
                    \frac{3}{2} \\ \frac{1}{2}
                  \end{matrix}$                                                   \\\hline

\end{longtable}

$x_2 = \frac{1}{2}$

$x_1 - x_3 = \frac{3}{2}$

$x_3 =: t \quad t \in \mathbb{R}$

$x_1 - t = \frac{3}{2} \quad | + t$

$\Leftrightarrow x_1 = \frac{3}{2} + t$

$\begin{pmatrix}
    x_1 \\ x_2 \\ x_3
  \end{pmatrix} = \begin{pmatrix}
    \frac{3}{2} + t \\ \frac{1}{2} \\ t
  \end{pmatrix}$ wobei $t \in \mathbb{R}$

\section{Aufgabe 2}

\subsection{a}
Wenn fünf Ochsen und zwei Schafe acht Taels Gold kosten, sowie zwei Ochsen und
acht Schafe auch acht Taels, was ist dann der Preis eines Tieres? (Chiu-Chang
Suan-Chu, 300 n.Chr.)

$x_1 :=$ Ochsen \quad $x_2 :=$ Schafe

\begin{align*}
  \begin{cases}
    5x_1 + 2x_2 = 8 \\
    2x_1 + 8x_2 = 8
  \end{cases}
\end{align*}

\begin{longtable}{p{4cm}|p{3cm}}

  \hline
  \multicolumn{1}{c|}{\textbf{Linearkombination}} & \multicolumn{1}{c}{\textbf{Konstanten}} \\
  \hline
  \endfirsthead

  \hline
  \multicolumn{2}{c}{\tablename\ \thetable\ -- \textit{Fortführung von vorherier Seite}}    \\
  \hline
  \multicolumn{1}{c|}{\textbf{Linearkombination}} & \multicolumn{1}{c}{\textbf{Konstanten}} \\
  \hline
  \endhead

  \hline
  \multicolumn{2}{r}{\textit{Fortsetzung siehe nächste Seite}}                              \\
  \endfoot

  \hline
  \endlastfoot

  $\displaystyle\begin{matrix}
                    5 & 2 \\
                    2 & 8
                  \end{matrix}$                    &
  $\displaystyle\begin{matrix}
                    8 \\ 8
                  \end{matrix}$                                                               \\\hline

  \multicolumn{2}{p{\dimexpr4cm+3cm+2\tabcolsep\relax}}{Operation: 5II - 2I}                \\\hline\pagebreak[0]

  $\displaystyle\begin{matrix}
                    5 & 2  \\
                    0 & 36
                  \end{matrix}$                    &
  $\displaystyle\begin{matrix}
                    8 \\ 24
                  \end{matrix}$                                                               \\\hline

  \multicolumn{2}{p{\dimexpr4cm+3cm+2\tabcolsep\relax}}{Operation: 36I - 2II}               \\\hline\pagebreak[0]

  $\displaystyle\begin{matrix}
                    180 & 0  \\
                    0   & 36
                  \end{matrix}$                    &
  $\displaystyle\begin{matrix}
                    240 \\ 24
                  \end{matrix}$                                                               \\\hline

  \multicolumn{2}{p{\dimexpr4cm+3cm+2\tabcolsep\relax}}{Operation: I: 180}                  \\\hline\pagebreak[0]
  \multicolumn{2}{p{\dimexpr4cm+3cm+2\tabcolsep\relax}}{Operation: II: 36}                  \\\hline\pagebreak[0]

  $\displaystyle\begin{matrix}
                    1 & 0 \\
                    0 & 1
                  \end{matrix}$                    &
  $\displaystyle\begin{matrix}
                    \frac{4}{3} \\ \frac{3}{2}
                  \end{matrix}$                                                   \\\hline

\end{longtable}

\subsection{b}
Ein 9-Tupel $(x1, . . . , x9)$ nennt man \enquote{magisches Quadrat der Ordnung
  3}, wenn gilt:
\begin{align*}
  x1 + x2 + x3 = x4 + x5 + x6 = x7 + x8 + x9 = x1 + x4 + x7 \\ = x2 + x5 + x8 = x3 + x6 + x9 = x1 + x5 + x9 = x3 + x5 + x7
\end{align*}

Stellen Sie ein lineares Gleichungssystem auf, das zu diesen sieben Bedingungen
äquivalent ist, und bestimmen Sie den Lösungsraum in $\mathbb{R}^9$. Wie kann
man die Menge der rationalen Lösungen (also der $(x_1, \dots, x_9) \in
  \mathbb{Q}^9)$ beschrieben? Gibt es auch eine Lösung in $\mathbb{Z}^9$? Oder
gar in $\mathbb{N}^9$? (siehe J. W. von Goethe 1: Faust. Der Tragödie erster
Teil, Hexenküche).

(Diese Aufgaben sind entnommen aus: \textit{Peter Knaber, Wolf P. Barth:} Lineare Algebra. Aufgaben und Lösungen. \textit{Springer Verlag,} 2017. Seite 4.)

\begin{align*}
  \begin{cases}
    x_1 + x_2 + x_3 - x_4 - x_5 - x_6 = 0                                               \\
    x_4 + x_5 + x_6 - x_7 - x_8 - x_9 = 0                                               \\
    x_7 + x_8 + x_9 - x_1 - x_4 - x_7 = 0 \Rightarrow -x_1 -x_4 +x_8 +x_9 = 0           \\
    x_1 + x_4 + x_7 - x_2 - x_5 - x_8 = 0 \Rightarrow x_1 - x_2 + x_4 - x_5 + x_7 - x_8 \\
    x_2 + x_5 + x_8 - x_3 - x_6 - x_9 = 0 \Rightarrow x_2 - x_3 + x_5 - x_6 + x_8 - x_9 \\
    x_3 + x_6 + x_9 - x_1 - x_5 - x_9 = 0 \Rightarrow - x_1 + x_3 - x_5 + x_6           \\
    x_1 + x_5 + x_9 - x_3 - x_5 - x_7 = 0 \Rightarrow x_1 - x_3 - x_7 + x_9
  \end{cases} \\
\end{align*}

\begin{longtable}{p{10cm}}

  \hline
  \multicolumn{1}{c}{\textbf{Linearkombination}}                                            \\
  \hline
  \endfirsthead

  \hline
  \multicolumn{1}{c}{\tablename\ \thetable\ -- \textit{Fortführung von vorherier Seite}}    \\
  \hline
  \multicolumn{1}{c}{\textbf{Linearkombination}}                                            \\
  \hline
  \endhead

  \hline
  \multicolumn{1}{r}{\textit{Fortsetzung siehe nächste Seite}}                              \\
  \endfoot

  \hline
  \endlastfoot

  $\displaystyle\begin{matrix}
                    I   & 1  & 1  & 1  & -1 & -1 & -1 & 0  & 0  & 0  \\
                    II  & 0  & 0  & 0  & 1  & 1  & 1  & -1 & -1 & -1 \\
                    III & -1 & 0  & 0  & -1 & 0  & 0  & 0  & 1  & 1  \\
                    IV  & 1  & -1 & 0  & 1  & -1 & 0  & 1  & -1 & 0  \\
                    V   & 0  & 1  & -1 & 0  & 1  & -1 & 0  & 1  & -1 \\
                    VI  & -1 & 0  & 1  & 0  & -1 & 1  & 0  & 0  & 0  \\
                    VII & 1  & 0  & -1 & 0  & 0  & 0  & -1 & 0  & 1
                  \end{matrix}$                            \\\hline
  Ziel: erste Spalte bereinigen                                                             \\\hline\pagebreak[0]
  Operation: III + I                                                                        \\\hline\pagebreak[0]
  $\displaystyle\begin{matrix}
                    I   & 1  & 1  & 1  & -1 & -1 & -1 & 0  & 0  & 0  \\
                    II  & 0  & 0  & 0  & 1  & 1  & 1  & -1 & -1 & -1 \\
                    III & 0  & 1  & 1  & -2 & -1 & -1 & 0  & 1  & 1  \\
                    IV  & 1  & -1 & 0  & 1  & -1 & 0  & 1  & -1 & 0  \\
                    V   & 0  & 1  & -1 & 0  & 1  & -1 & 0  & 1  & -1 \\
                    VI  & -1 & 0  & 1  & 0  & -1 & 1  & 0  & 0  & 0  \\
                    VII & 1  & 0  & -1 & 0  & 0  & 0  & -1 & 0  & 1
                  \end{matrix}$                            \\\hline
  Operation: IV - I                                                                         \\\hline\pagebreak[0]
  $\displaystyle\begin{matrix}
                    I   & 1  & 1  & 1  & -1 & -1 & -1 & 0  & 0  & 0  \\
                    II  & 0  & 0  & 0  & 1  & 1  & 1  & -1 & -1 & -1 \\
                    III & 0  & 1  & 1  & -2 & -1 & -1 & 0  & 1  & 1  \\
                    IV  & 0  & -2 & -1 & 2  & 0  & 1  & 1  & -1 & 0  \\
                    V   & 0  & 1  & -1 & 0  & 1  & -1 & 0  & 1  & -1 \\
                    VI  & -1 & 0  & 1  & 0  & -1 & 1  & 0  & 0  & 0  \\
                    VII & 1  & 0  & -1 & 0  & 0  & 0  & -1 & 0  & 1
                  \end{matrix}$                            \\\hline
  Operation: VI + I                                                                         \\\hline\pagebreak[0]
  $\displaystyle\begin{matrix}
                    I   & 1 & 1  & 1  & -1 & -1 & -1 & 0  & 0  & 0  \\
                    II  & 0 & 0  & 0  & 1  & 1  & 1  & -1 & -1 & -1 \\
                    III & 0 & 1  & 1  & -2 & -1 & -1 & 0  & 1  & 1  \\
                    IV  & 0 & -2 & -1 & 2  & 0  & 1  & 1  & -1 & 0  \\
                    V   & 0 & 1  & -1 & 0  & 1  & -1 & 0  & 1  & -1 \\
                    VI  & 0 & 1  & 2  & -1 & -2 & 0  & 0  & 0  & 0  \\
                    VII & 1 & 0  & -1 & 0  & 0  & 0  & -1 & 0  & 1
                  \end{matrix}$                             \\\hline
  Operation: VII - I                                                                        \\\hline\pagebreak[0]
  $\displaystyle\begin{matrix}
                    I   & 1 & 1  & 1  & -1 & -1 & -1 & 0  & 0  & 0  \\
                    II  & 0 & 0  & 0  & 1  & 1  & 1  & -1 & -1 & -1 \\
                    III & 0 & 1  & 1  & -2 & -1 & -1 & 0  & 1  & 1  \\
                    IV  & 0 & -2 & -1 & 2  & 0  & 1  & 1  & -1 & 0  \\
                    V   & 0 & 1  & -1 & 0  & 1  & -1 & 0  & 1  & -1 \\
                    VI  & 0 & 1  & 2  & -1 & -2 & 0  & 0  & 0  & 0  \\
                    VII & 0 & -1 & -2 & 1  & 1  & 1  & -1 & 0  & 1
                  \end{matrix}$                             \\\hline
  Ziel: zweite Spalte bereinigen                                                            \\\hline\pagebreak[0]
  Operation: II und III tauschen                                                            \\\hline\pagebreak[0]
  $\displaystyle\begin{matrix}
                    I   & 1 & 1  & 1  & -1 & -1 & -1 & 0  & 0  & 0  \\
                    II  & 0 & 1  & 1  & -2 & -1 & -1 & 0  & 1  & 1  \\
                    III & 0 & 0  & 0  & 1  & 1  & 1  & -1 & -1 & -1 \\
                    IV  & 0 & -2 & -1 & 2  & 0  & 1  & 1  & -1 & 0  \\
                    V   & 0 & 1  & -1 & 0  & 1  & -1 & 0  & 1  & -1 \\
                    VI  & 0 & 1  & 2  & -1 & -2 & 0  & 0  & 0  & 0  \\
                    VII & 0 & -1 & -2 & 1  & 1  & 1  & -1 & 0  & 1
                  \end{matrix}$                             \\\hline
  Operation: IV + 2V                                                                        \\\hline\pagebreak[0]
  $\displaystyle\begin{matrix}
                    I   & 1 & 1  & 1  & -1 & -1 & -1 & 0  & 0  & 0  \\
                    II  & 0 & 1  & 1  & -2 & -1 & -1 & 0  & 1  & 1  \\
                    III & 0 & 0  & 0  & 1  & 1  & 1  & -1 & -1 & -1 \\
                    IV  & 0 & 0  & -3 & 2  & 2  & -1 & 1  & 1  & -2 \\
                    V   & 0 & 1  & -1 & 0  & 1  & -1 & 0  & 1  & -1 \\
                    VI  & 0 & 1  & 2  & -1 & -2 & 0  & 0  & 0  & 0  \\
                    VII & 0 & -1 & -2 & 1  & 1  & 1  & -1 & 0  & 1
                  \end{matrix}$                             \\\hline
  Operation: V - VI                                                                         \\\hline\pagebreak[0]
  $\displaystyle\begin{matrix}
                    I   & 1 & 1  & 1  & -1 & -1 & -1 & 0  & 0  & 0  \\
                    II  & 0 & 1  & 1  & -2 & -1 & -1 & 0  & 1  & 1  \\
                    III & 0 & 0  & 0  & 1  & 1  & 1  & -1 & -1 & -1 \\
                    IV  & 0 & 0  & -3 & 2  & 2  & -1 & 1  & 1  & -2 \\
                    V   & 0 & 0  & -3 & 1  & 3  & -1 & 0  & 1  & -1 \\
                    VI  & 0 & 1  & 2  & -1 & -2 & 0  & 0  & 0  & 0  \\
                    VII & 0 & -1 & -2 & 1  & 1  & 1  & -1 & 0  & 1
                  \end{matrix}$                             \\\hline
  Operation: VI + VII                                                                       \\\hline\pagebreak[0]
  $\displaystyle\begin{matrix}
                    I   & 1 & 1  & 1  & -1 & -1 & -1 & 0  & 0  & 0  \\
                    II  & 0 & 1  & 1  & -2 & -1 & -1 & 0  & 1  & 1  \\
                    III & 0 & 0  & 0  & 1  & 1  & 1  & -1 & -1 & -1 \\
                    IV  & 0 & 0  & -3 & 2  & 2  & -1 & 1  & 1  & -2 \\
                    V   & 0 & 0  & -3 & 1  & 3  & -1 & 0  & 1  & -1 \\
                    VI  & 0 & 0  & 0  & 0  & -1 & 1  & -1 & 0  & 1  \\
                    VII & 0 & -1 & -2 & 1  & 1  & 1  & -1 & 0  & 1
                  \end{matrix}$                             \\\hline
  Operation: VII + II                                                                       \\\hline\pagebreak[0]
  $\displaystyle\begin{matrix}
                    I   & 1 & 1 & 1  & -1 & -1 & -1 & 0  & 0  & 0  \\
                    II  & 0 & 1 & 1  & -2 & -1 & -1 & 0  & 1  & 1  \\
                    III & 0 & 0 & 0  & 1  & 1  & 1  & -1 & -1 & -1 \\
                    IV  & 0 & 0 & -3 & 2  & 2  & -1 & 1  & 1  & -2 \\
                    V   & 0 & 0 & -3 & 1  & 3  & -1 & 0  & 1  & -1 \\
                    VI  & 0 & 0 & 0  & 0  & -1 & 1  & -1 & 0  & 1  \\
                    VII & 0 & 0 & -1 & -1 & 0  & 0  & -1 & 1  & 2
                  \end{matrix}$                              \\\hline
  Ziel: dritte Spalte bereinigen                                                            \\\hline\pagebreak[0]
  Operation: III und VII tauschen                                                           \\\hline\pagebreak[0]
  $\displaystyle\begin{matrix}
                    I   & 1 & 1 & 1  & -1 & -1 & -1 & 0  & 0  & 0  \\
                    II  & 0 & 1 & 1  & -2 & -1 & -1 & 0  & 1  & 1  \\
                    III & 0 & 0 & -1 & -1 & 0  & 0  & -1 & 1  & 2  \\
                    IV  & 0 & 0 & -3 & 2  & 2  & -1 & 1  & 1  & -2 \\
                    V   & 0 & 0 & -3 & 1  & 3  & -1 & 0  & 1  & -1 \\
                    VI  & 0 & 0 & 0  & 0  & -1 & 1  & -1 & 0  & 1  \\
                    VII & 0 & 0 & 0  & 1  & 1  & 1  & -1 & -1 & -1
                  \end{matrix}$                              \\\hline
  Operation: IV - V                                                                         \\\hline\pagebreak[0]
  $\displaystyle\begin{matrix}
                    I   & 1 & 1 & 1  & -1 & -1 & -1 & 0  & 0  & 0  \\
                    II  & 0 & 1 & 1  & -2 & -1 & -1 & 0  & 1  & 1  \\
                    III & 0 & 0 & -1 & -1 & 0  & 0  & -1 & 1  & 2  \\
                    IV  & 0 & 0 & 0  & 1  & -1 & 0  & 1  & 0  & -1 \\
                    V   & 0 & 0 & -3 & 1  & 3  & -1 & 0  & 1  & -1 \\
                    VI  & 0 & 0 & 0  & 0  & -1 & 1  & -1 & 0  & 1  \\
                    VII & 0 & 0 & 0  & 1  & 1  & 1  & -1 & -1 & -1
                  \end{matrix}$                              \\\hline
  Operation: V - 3III                                                                       \\\hline\pagebreak[0]
  $\displaystyle\begin{matrix}
                    I   & 1 & 1 & 1  & -1 & -1 & -1 & 0  & 0  & 0  \\
                    II  & 0 & 1 & 1  & -2 & -1 & -1 & 0  & 1  & 1  \\
                    III & 0 & 0 & -1 & -1 & 0  & 0  & -1 & 1  & 2  \\
                    IV  & 0 & 0 & 0  & 1  & -1 & 0  & 1  & 0  & -1 \\
                    V   & 0 & 0 & 0  & 4  & 3  & -1 & 3  & -2 & -7 \\
                    VI  & 0 & 0 & 0  & 0  & -1 & 1  & -1 & 0  & 1  \\
                    VII & 0 & 0 & 0  & 1  & 1  & 1  & -1 & -1 & -1
                  \end{matrix}$                              \\\hline
  Operation: III $\cdot$ (-1)                                                               \\\hline\pagebreak[0]
  $\displaystyle\begin{matrix}
                    I   & 1 & 1 & 1 & -1 & -1 & -1 & 0  & 0  & 0  \\
                    II  & 0 & 1 & 1 & -2 & -1 & -1 & 0  & 1  & 1  \\
                    III & 0 & 0 & 1 & 1  & 0  & 0  & 1  & -1 & -2 \\
                    IV  & 0 & 0 & 0 & 1  & -1 & 0  & 1  & 0  & -1 \\
                    V   & 0 & 0 & 0 & 4  & 3  & -1 & 3  & -2 & -7 \\
                    VI  & 0 & 0 & 0 & 0  & -1 & 1  & -1 & 0  & 1  \\
                    VII & 0 & 0 & 0 & 1  & 1  & 1  & -1 & -1 & -1
                  \end{matrix}$                               \\\hline
  Ziel: vierte Spalte bereinigen                                                            \\\hline\pagebreak[0]
  Operation: V - 4VII                                                                       \\\hline\pagebreak[0]
  $\displaystyle\begin{matrix}
                    I   & 1 & 1 & 1 & -1 & -1 & -1 & 0  & 0  & 0  \\
                    II  & 0 & 1 & 1 & -2 & -1 & -1 & 0  & 1  & 1  \\
                    III & 0 & 0 & 1 & 1  & 0  & 0  & 1  & -1 & -2 \\
                    IV  & 0 & 0 & 0 & 1  & -1 & 0  & 1  & 0  & -1 \\
                    V   & 0 & 0 & 0 & 0  & -1 & -5 & 7  & 2  & -3 \\
                    VI  & 0 & 0 & 0 & 0  & -1 & 1  & -1 & 0  & 1  \\
                    VII & 0 & 0 & 0 & 1  & 1  & 1  & -1 & -1 & -1
                  \end{matrix}$                               \\\hline
  Operation: VII - IV                                                                       \\\hline\pagebreak[0]
  $\displaystyle\begin{matrix}
                    I   & 1 & 1 & 1 & -1 & -1 & -1 & 0  & 0  & 0  \\
                    II  & 0 & 1 & 1 & -2 & -1 & -1 & 0  & 1  & 1  \\
                    III & 0 & 0 & 1 & 1  & 0  & 0  & 1  & -1 & -2 \\
                    IV  & 0 & 0 & 0 & 1  & -1 & 0  & 1  & 0  & -1 \\
                    V   & 0 & 0 & 0 & 0  & -1 & -5 & 7  & 2  & -3 \\
                    VI  & 0 & 0 & 0 & 0  & -1 & 1  & -1 & 0  & 1  \\
                    VII & 0 & 0 & 0 & 0  & 2  & 1  & -2 & -1 & 0
                  \end{matrix}$                               \\\hline
  Ziel: fünfte Spalte bereinigen                                                            \\\hline\pagebreak[0]
  Operation: 2VI + VII                                                                      \\\hline\pagebreak[0]
  $\displaystyle\begin{matrix}
                    I   & 1 & 1 & 1 & -1 & -1 & -1 & 0  & 0  & 0  \\
                    II  & 0 & 1 & 1 & -2 & -1 & -1 & 0  & 1  & 1  \\
                    III & 0 & 0 & 1 & 1  & 0  & 0  & 1  & -1 & -2 \\
                    IV  & 0 & 0 & 0 & 1  & -1 & 0  & 1  & 0  & -1 \\
                    V   & 0 & 0 & 0 & 0  & -1 & -5 & 7  & 2  & -3 \\
                    VI  & 0 & 0 & 0 & 0  & -1 & 1  & -1 & 0  & 1  \\
                    VII & 0 & 0 & 0 & 0  & 0  & 3  & -4 & -1 & 2
                  \end{matrix}$                               \\\hline
  Operation: VI - V                                                                         \\\hline\pagebreak[0]
  $\displaystyle\begin{matrix}
                    I   & 1 & 1 & 1 & -1 & -1 & -1 & 0  & 0  & 0  \\
                    II  & 0 & 1 & 1 & -2 & -1 & -1 & 0  & 1  & 1  \\
                    III & 0 & 0 & 1 & 1  & 0  & 0  & 1  & -1 & -2 \\
                    IV  & 0 & 0 & 0 & 1  & -1 & 0  & 1  & 0  & -1 \\
                    V   & 0 & 0 & 0 & 0  & -1 & -5 & 7  & 2  & -3 \\
                    VI  & 0 & 0 & 0 & 0  & 0  & 6  & -8 & -2 & 4  \\
                    VII & 0 & 0 & 0 & 0  & 0  & 3  & -4 & -1 & 2
                  \end{matrix}$                               \\\hline
  Operation: V $\cdot$ (-1)                                                                 \\\hline\pagebreak[0]
  $\displaystyle\begin{matrix}
                    I   & 1 & 1 & 1 & -1 & -1 & -1 & 0  & 0  & 0  \\
                    II  & 0 & 1 & 1 & -2 & -1 & -1 & 0  & 1  & 1  \\
                    III & 0 & 0 & 1 & 1  & 0  & 0  & 1  & -1 & -2 \\
                    IV  & 0 & 0 & 0 & 1  & -1 & 0  & 1  & 0  & -1 \\
                    V   & 0 & 0 & 0 & 0  & 1  & 5  & -7 & -2 & 3  \\
                    VI  & 0 & 0 & 0 & 0  & 0  & 6  & -8 & -2 & 4  \\
                    VII & 0 & 0 & 0 & 0  & 0  & 3  & -4 & -1 & 2
                  \end{matrix}$                               \\\hline
  Ziel: sechste Spalte bereinigen                                                           \\\hline\pagebreak[0]
  Operation: 2VII - VI                                                                      \\\hline\pagebreak[0]
  $\displaystyle\begin{matrix}
                    I   & 1 & 1 & 1 & -1 & -1 & -1 & 0  & 0  & 0  \\
                    II  & 0 & 1 & 1 & -2 & -1 & -1 & 0  & 1  & 1  \\
                    III & 0 & 0 & 1 & 1  & 0  & 0  & 1  & -1 & -2 \\
                    IV  & 0 & 0 & 0 & 1  & -1 & 0  & 1  & 0  & -1 \\
                    V   & 0 & 0 & 0 & 0  & 1  & 5  & -7 & -2 & 3  \\
                    VI  & 0 & 0 & 0 & 0  & 0  & 6  & -8 & -2 & 4  \\
                    VII & 0 & 0 & 0 & 0  & 0  & 0  & 0  & 0  & 0
                  \end{matrix}$                               \\\hline
  Operation: VI : 6                                                                         \\\hline\pagebreak[0]
  $\displaystyle\begin{matrix}
                    I   & 1 & 1 & 1 & -1 & -1 & -1 & 0            & 0            & 0           \\
                    II  & 0 & 1 & 1 & -2 & -1 & -1 & 0            & 1            & 1           \\
                    III & 0 & 0 & 1 & 1  & 0  & 0  & 1            & -1           & -2          \\
                    IV  & 0 & 0 & 0 & 1  & -1 & 0  & 1            & 0            & -1          \\
                    V   & 0 & 0 & 0 & 0  & 1  & 5  & -7           & -2           & 3           \\
                    VI  & 0 & 0 & 0 & 0  & 0  & 1  & -\frac{4}{3} & -\frac{1}{3} & \frac{2}{3} \\
                    VII & 0 & 0 & 0 & 0  & 0  & 0  & 0            & 0            & 0
                  \end{matrix}$  \\\hline
  Die siebte Zeile (VII) besteht ausschließlich aus nullen. Das bedeutet,
  dass die ursprüngliche siebte Gleichung von den anderen linear abhängig
  war und keine neuen Informationen liefert. Diese Zeile wird daher im
  Folgenden nicht mehr berücksichtigt.                                                      \\\hline\pagebreak[0]
  $\displaystyle\begin{matrix}
                    I   & 1 & 1 & 1 & -1 & -1 & -1 & 0            & 0            & 0           \\
                    II  & 0 & 1 & 1 & -2 & -1 & -1 & 0            & 1            & 1           \\
                    III & 0 & 0 & 1 & 1  & 0  & 0  & 1            & -1           & -2          \\
                    IV  & 0 & 0 & 0 & 1  & -1 & 0  & 1            & 0            & -1          \\
                    V   & 0 & 0 & 0 & 0  & 1  & 5  & -7           & -2           & 3           \\
                    VI  & 0 & 0 & 0 & 0  & 0  & 1  & -\frac{4}{3} & -\frac{1}{3} & \frac{2}{3}
                  \end{matrix}$  \\\hline
  Ziel: Einträge in Spalte sechs oberhalb des Pivots eleminieren                            \\\hline\pagebreak[0]
  Operation: V - 5VI                                                                        \\\hline\pagebreak[0]
  $\displaystyle\begin{matrix}
                    I   & 1 & 1 & 1 & -1 & -1 & -1 & 0            & 0            & 0            \\
                    II  & 0 & 1 & 1 & -2 & -1 & -1 & 0            & 1            & 1            \\
                    III & 0 & 0 & 1 & 1  & 0  & 0  & 1            & -1           & -2           \\
                    IV  & 0 & 0 & 0 & 1  & -1 & 0  & 1            & 0            & -1           \\
                    V   & 0 & 0 & 0 & 0  & 1  & 0  & -\frac{1}{3} & -\frac{1}{3} & -\frac{1}{3} \\
                    VI  & 0 & 0 & 0 & 0  & 0  & 1  & -\frac{4}{3} & -\frac{1}{3} & \frac{2}{3}
                  \end{matrix}$ \\\hline
  Operation: II + VI                                                                        \\\hline\pagebreak[0]
  $\displaystyle\begin{matrix}
                    I   & 1 & 1 & 1 & -1 & -1 & -1 & 0            & 0            & 0            \\
                    II  & 0 & 1 & 1 & -2 & -1 & 0  & -\frac{4}{3} & \frac{2}{3}  & \frac{5}{3}  \\
                    III & 0 & 0 & 1 & 1  & 0  & 0  & 1            & -1           & -2           \\
                    IV  & 0 & 0 & 0 & 1  & -1 & 0  & 1            & 0            & -1           \\
                    V   & 0 & 0 & 0 & 0  & 1  & 0  & -\frac{1}{3} & -\frac{1}{3} & -\frac{1}{3} \\
                    VI  & 0 & 0 & 0 & 0  & 0  & 1  & -\frac{4}{3} & -\frac{1}{3} & \frac{2}{3}
                  \end{matrix}$ \\\hline
  Operation: I + VI                                                                         \\\hline\pagebreak[0]
  $\displaystyle\begin{matrix}
                    I   & 1 & 1 & 1 & -1 & -1 & 0 & -\frac{4}{3} & -\frac{1}{3} & \frac{2}{3}  \\
                    II  & 0 & 1 & 1 & -2 & -1 & 0 & -\frac{4}{3} & \frac{2}{3}  & \frac{5}{3}  \\
                    III & 0 & 0 & 1 & 1  & 0  & 0 & 1            & -1           & -2           \\
                    IV  & 0 & 0 & 0 & 1  & -1 & 0 & 1            & 0            & -1           \\
                    V   & 0 & 0 & 0 & 0  & 1  & 0 & -\frac{1}{3} & -\frac{1}{3} & -\frac{1}{3} \\
                    VI  & 0 & 0 & 0 & 0  & 0  & 1 & -\frac{4}{3} & -\frac{1}{3} & \frac{2}{3}
                  \end{matrix}$  \\\hline
  Ziel: Einträge in Spalte fünf oberhalb des Pivots eleminieren                             \\\hline\pagebreak[0]
  Operation: IV + V                                                                         \\\hline\pagebreak[0]
  $\displaystyle\begin{matrix}
                    I   & 1 & 1 & 1 & -1 & -1 & 0 & -\frac{4}{3} & -\frac{1}{3} & \frac{2}{3}  \\
                    II  & 0 & 1 & 1 & -2 & -1 & 0 & -\frac{4}{3} & \frac{2}{3}  & \frac{5}{3}  \\
                    III & 0 & 0 & 1 & 1  & 0  & 0 & 1            & -1           & -2           \\
                    IV  & 0 & 0 & 0 & 1  & 0  & 0 & \frac{2}{3}  & -\frac{1}{3} & -\frac{4}{3} \\
                    V   & 0 & 0 & 0 & 0  & 1  & 0 & -\frac{1}{3} & -\frac{1}{3} & -\frac{1}{3} \\
                    VI  & 0 & 0 & 0 & 0  & 0  & 1 & -\frac{4}{3} & -\frac{1}{3} & \frac{2}{3}
                  \end{matrix}$  \\\hline
  Operation: II + V                                                                         \\\hline\pagebreak[0]
  $\displaystyle\begin{matrix}
                    I   & 1 & 1 & 1 & -1 & -1 & 0 & -\frac{4}{3} & -\frac{1}{3} & \frac{2}{3}  \\
                    II  & 0 & 1 & 1 & -2 & 0  & 0 & -\frac{5}{3} & \frac{1}{3}  & \frac{4}{3}  \\
                    III & 0 & 0 & 1 & 1  & 0  & 0 & 1            & -1           & -2           \\
                    IV  & 0 & 0 & 0 & 1  & 0  & 0 & \frac{2}{3}  & -\frac{1}{3} & -\frac{4}{3} \\
                    V   & 0 & 0 & 0 & 0  & 1  & 0 & -\frac{1}{3} & -\frac{1}{3} & -\frac{1}{3} \\
                    VI  & 0 & 0 & 0 & 0  & 0  & 1 & -\frac{4}{3} & -\frac{1}{3} & \frac{2}{3}
                  \end{matrix}$  \\\hline
  Operation: I + V                                                                          \\\hline\pagebreak[0]
  $\displaystyle\begin{matrix}
                    I   & 1 & 1 & 1 & -1 & 0 & 0 & -\frac{5}{3} & -\frac{2}{3} & \frac{1}{3}  \\
                    II  & 0 & 1 & 1 & -2 & 0 & 0 & -\frac{5}{3} & \frac{1}{3}  & \frac{4}{3}  \\
                    III & 0 & 0 & 1 & 1  & 0 & 0 & 1            & -1           & -2           \\
                    IV  & 0 & 0 & 0 & 1  & 0 & 0 & \frac{2}{3}  & -\frac{1}{3} & -\frac{4}{3} \\
                    V   & 0 & 0 & 0 & 0  & 1 & 0 & -\frac{1}{3} & -\frac{1}{3} & -\frac{1}{3} \\
                    VI  & 0 & 0 & 0 & 0  & 0 & 1 & -\frac{4}{3} & -\frac{1}{3} & \frac{2}{3}
                  \end{matrix}$   \\\hline
  Ziel: Einträge in Spalte vier oberhalb des Pivots eleminieren                             \\\hline\pagebreak[0]
  Operation: III - IV                                                                       \\\hline\pagebreak[0]
  $\displaystyle\begin{matrix}
                    I   & 1 & 1 & 1 & -1 & 0 & 0 & -\frac{5}{3} & -\frac{2}{3} & \frac{1}{3}  \\
                    II  & 0 & 1 & 1 & -2 & 0 & 0 & -\frac{5}{3} & \frac{1}{3}  & \frac{4}{3}  \\
                    III & 0 & 0 & 1 & 0  & 0 & 0 & \frac{1}{3}  & -\frac{2}{3} & -\frac{2}{3} \\
                    IV  & 0 & 0 & 0 & 1  & 0 & 0 & \frac{2}{3}  & -\frac{1}{3} & -\frac{4}{3} \\
                    V   & 0 & 0 & 0 & 0  & 1 & 0 & -\frac{1}{3} & -\frac{1}{3} & -\frac{1}{3} \\
                    VI  & 0 & 0 & 0 & 0  & 0 & 1 & -\frac{4}{3} & -\frac{1}{3} & \frac{2}{3}
                  \end{matrix}$   \\\hline
  Operation: II + 2IV                                                                       \\\hline\pagebreak[0]
  $\displaystyle\begin{matrix}
                    I   & 1 & 1 & 1 & -1 & 0 & 0 & -\frac{5}{3} & -\frac{2}{3} & \frac{1}{3}  \\
                    II  & 0 & 1 & 1 & 0  & 0 & 0 & -\frac{1}{3} & -\frac{1}{3} & -\frac{4}{3} \\
                    III & 0 & 0 & 1 & 0  & 0 & 0 & \frac{1}{3}  & -\frac{2}{3} & -\frac{2}{3} \\
                    IV  & 0 & 0 & 0 & 1  & 0 & 0 & \frac{2}{3}  & -\frac{1}{3} & -\frac{4}{3} \\
                    V   & 0 & 0 & 0 & 0  & 1 & 0 & -\frac{1}{3} & -\frac{1}{3} & -\frac{1}{3} \\
                    VI  & 0 & 0 & 0 & 0  & 0 & 1 & -\frac{4}{3} & -\frac{1}{3} & \frac{2}{3}
                  \end{matrix}$   \\\hline
  Operation: I + IV                                                                         \\\hline\pagebreak[0]
  $\displaystyle\begin{matrix}
                    I   & 1 & 1 & 1 & 0 & 0 & 0 & -1           & -1           & -1           \\
                    II  & 0 & 1 & 1 & 0 & 0 & 0 & -\frac{1}{3} & -\frac{1}{3} & -\frac{4}{3} \\
                    III & 0 & 0 & 1 & 0 & 0 & 0 & \frac{1}{3}  & -\frac{2}{3} & -\frac{2}{3} \\
                    IV  & 0 & 0 & 0 & 1 & 0 & 0 & \frac{2}{3}  & -\frac{1}{3} & -\frac{4}{3} \\
                    V   & 0 & 0 & 0 & 0 & 1 & 0 & -\frac{1}{3} & -\frac{1}{3} & -\frac{1}{3} \\
                    VI  & 0 & 0 & 0 & 0 & 0 & 1 & -\frac{4}{3} & -\frac{1}{3} & \frac{2}{3}
                  \end{matrix}$    \\\hline
  Ziel: Einträge in Spalte drei oberhalb des Pivots eleminieren                             \\\hline\pagebreak[0]
  Operation: II - III                                                                       \\\hline\pagebreak[0]
  $\displaystyle\begin{matrix}
                    I   & 1 & 1 & 1 & 0 & 0 & 0 & -1           & -1           & -1           \\
                    II  & 0 & 1 & 0 & 0 & 0 & 0 & -\frac{2}{3} & \frac{1}{3}  & -\frac{2}{3} \\
                    III & 0 & 0 & 1 & 0 & 0 & 0 & \frac{1}{3}  & -\frac{2}{3} & -\frac{2}{3} \\
                    IV  & 0 & 0 & 0 & 1 & 0 & 0 & \frac{2}{3}  & -\frac{1}{3} & -\frac{4}{3} \\
                    V   & 0 & 0 & 0 & 0 & 1 & 0 & -\frac{1}{3} & -\frac{1}{3} & -\frac{1}{3} \\
                    VI  & 0 & 0 & 0 & 0 & 0 & 1 & -\frac{4}{3} & -\frac{1}{3} & \frac{2}{3}
                  \end{matrix}$    \\\hline
  Operation: I - III                                                                        \\\hline\pagebreak[0]
  $\displaystyle\begin{matrix}
                    I   & 1 & 1 & 0 & 0 & 0 & 0 & -\frac{4}{3} & -\frac{1}{3} & -\frac{1}{3} \\
                    II  & 0 & 1 & 0 & 0 & 0 & 0 & -\frac{2}{3} & \frac{1}{3}  & -\frac{2}{3} \\
                    III & 0 & 0 & 1 & 0 & 0 & 0 & \frac{1}{3}  & -\frac{2}{3} & -\frac{2}{3} \\
                    IV  & 0 & 0 & 0 & 1 & 0 & 0 & \frac{2}{3}  & -\frac{1}{3} & -\frac{4}{3} \\
                    V   & 0 & 0 & 0 & 0 & 1 & 0 & -\frac{1}{3} & -\frac{1}{3} & -\frac{1}{3} \\
                    VI  & 0 & 0 & 0 & 0 & 0 & 1 & -\frac{4}{3} & -\frac{1}{3} & \frac{2}{3}
                  \end{matrix}$    \\\hline
  Ziel: Einträge in Spalte zwei oberhalb des Pivots eleminieren                             \\\hline\pagebreak[0]
  Operation: I - II                                                                         \\\hline\pagebreak[0]
  $\displaystyle\begin{matrix}
                    I   & 1 & 0 & 0 & 0 & 0 & 0 & -\frac{2}{3} & -\frac{2}{3} & \frac{1}{3}  \\
                    II  & 0 & 1 & 0 & 0 & 0 & 0 & -\frac{2}{3} & \frac{1}{3}  & -\frac{2}{3} \\
                    III & 0 & 0 & 1 & 0 & 0 & 0 & \frac{1}{3}  & -\frac{2}{3} & -\frac{2}{3} \\
                    IV  & 0 & 0 & 0 & 1 & 0 & 0 & \frac{2}{3}  & -\frac{1}{3} & -\frac{4}{3} \\
                    V   & 0 & 0 & 0 & 0 & 1 & 0 & -\frac{1}{3} & -\frac{1}{3} & -\frac{1}{3} \\
                    VI  & 0 & 0 & 0 & 0 & 0 & 1 & -\frac{4}{3} & -\frac{1}{3} & \frac{2}{3}
                  \end{matrix}$    \\\hline

\end{longtable}

Hierraus lässt sich jetzt Schließen, dass
\begin{align*}
  x_1 -\frac{2}{3}x_7 - \frac{2}{3}x_8 + \frac{1}{3}x_9 = 0 \Leftrightarrow x_1 = \frac{2}{3}x_7 + \frac{2}{3}x_8 - \frac{1}{3}x_9,    \\
  x_2 -\frac{2}{3}x_7 + \frac{1}{3}x_8 - \frac{2}{3}x_9 = 0 \Leftrightarrow x_2 = \frac{2}{3}x_7 - \frac{1}{3}x_8 + \frac{2}{3}x_9,    \\
  x_3 + \frac{1}{3}x_7 - \frac{2}{3}x_8 - \frac{2}{3}x_9 = 0 \Leftrightarrow x_3 = - \frac{1}{3}x_7 + \frac{2}{3}x_8 + \frac{2}{3}x_9, \\
  x_4 + \frac{2}{3}x_7 - \frac{1}{3}x_8 - \frac{4}{3}x_9 = 0 \Leftrightarrow x_4 = - \frac{2}{3}x_7 + \frac{1}{3}x_8 + \frac{4}{3}x_9, \\
  x_5 -\frac{1}{3}x_7 - \frac{1}{3}x_8 - \frac{1}{3}x_9 = 0 \Leftrightarrow x_5 = \frac{1}{3}x_7 + \frac{1}{3}x_8 + \frac{1}{3}x_9,    \\
  x_6 - \frac{4}{3}x_7 - \frac{1}{3}x_8 + \frac{2}{3}x_9 = 0 \Leftrightarrow x_6 = \frac{4}{3}x_7 + \frac{1}{3}x_8 - \frac{2}{3}x_9
\end{align*}

Also gilt
\begin{align*}
  \begin{pmatrix}
    x_1 \\ x_2 \\ x_3 \\ x_4 \\ x_5 \\ x_6 \\ x_7 \\ x_8 \\ x_9
  \end{pmatrix} \displaybreak[2]     \\ =
  \begin{pmatrix}
    \frac{2}{3}x_7 + \frac{2}{3}x_8 - \frac{1}{3}x_9   \\
    \frac{2}{3}x_7 - \frac{1}{3}x_8 + \frac{2}{3}x_9   \\
    - \frac{1}{3}x_7 + \frac{2}{3}x_8 + \frac{2}{3}x_9 \\
    - \frac{2}{3}x_7 + \frac{1}{3}x_8 + \frac{4}{3}x_9 \\
    \frac{1}{3}x_7 + \frac{1}{3}x_8 + \frac{1}{3}x_9   \\
    \frac{4}{3}x_7 + \frac{1}{3}x_8 - \frac{2}{3}x_9   \\
    x_7                                                \\
    x_8                                                \\
    x_9
  \end{pmatrix} \displaybreak[2]             \\ =
  \begin{pmatrix}
    \frac{2}{3}x_7   \\
    \frac{2}{3}x_7   \\
    - \frac{1}{3}x_7 \\
    - \frac{2}{3}x_7 \\
    \frac{1}{3}x_7   \\
    \frac{4}{3}x_7   \\
    1x_7             \\
    0x_7             \\
    0x_7
  \end{pmatrix} + \begin{pmatrix}
                    \frac{2}{3}x_8  \\
                    -\frac{1}{3}x_8 \\
                    \frac{2}{3}x_8  \\
                    \frac{1}{3}x_8  \\
                    \frac{1}{3}x_8  \\
                    \frac{1}{3}x_8  \\
                    0x_8            \\
                    1x_8            \\
                    0x_8
                  \end{pmatrix} + \begin{pmatrix}
                                    - \frac{1}{3}x_9 \\
                                    \frac{2}{3}x_9   \\
                                    \frac{2}{3}x_9   \\
                                    \frac{4}{3}x_9   \\
                                    \frac{1}{3}x_9   \\
                                    - \frac{2}{3}x_9 \\
                                    0x_9             \\
                                    0x_9             \\
                                    1x_9
                                  \end{pmatrix} \displaybreak[2] \\ = x_7\begin{pmatrix}
    \frac{2}{3}   \\
    \frac{2}{3}   \\
    - \frac{1}{3} \\
    - \frac{2}{3} \\
    \frac{1}{3}   \\
    \frac{4}{3}   \\
    1             \\
    0             \\
    0
  \end{pmatrix} + x_8\begin{pmatrix}
    \frac{2}{3}  \\
    -\frac{1}{3} \\
    \frac{2}{3}  \\
    \frac{1}{3}  \\
    \frac{1}{3}  \\
    \frac{1}{3}  \\
    0            \\
    1            \\
    0
  \end{pmatrix} + x_9\begin{pmatrix}
    - \frac{1}{3} \\
    \frac{2}{3}   \\
    \frac{2}{3}   \\
    \frac{4}{3}   \\
    \frac{1}{3}   \\
    - \frac{2}{3} \\
    0             \\
    0             \\
    1
  \end{pmatrix}
\end{align*}

wobei $x_7, x_8, x_9 \in \mathbb{R}$

Der Lösungsraum des Tupels in $\mathbb{R}^9$ ist also

\[
  L = \left\{
  x_7\begin{pmatrix}
    \frac{2}{3}   \\
    \frac{2}{3}   \\
    - \frac{1}{3} \\
    - \frac{2}{3} \\
    \frac{1}{3}   \\
    \frac{4}{3}   \\
    1             \\
    0             \\
    0
  \end{pmatrix} + x_8\begin{pmatrix}
    \frac{2}{3}  \\
    -\frac{1}{3} \\
    \frac{2}{3}  \\
    \frac{1}{3}  \\
    \frac{1}{3}  \\
    \frac{1}{3}  \\
    0            \\
    1            \\
    0
  \end{pmatrix} + x_9\begin{pmatrix}
    - \frac{1}{3} \\
    \frac{2}{3}   \\
    \frac{2}{3}   \\
    \frac{4}{3}   \\
    \frac{1}{3}   \\
    - \frac{2}{3} \\
    0             \\
    0             \\
    1
  \end{pmatrix} \Bigg| x_7, x_8, x_9 \in \mathbb{R}
  \right\}
\]

Die Menge der Rationalen Lösungen kann eingach beschrieben werden also

\[
  L = \left\{
  x_7\begin{pmatrix}
    \frac{2}{3}   \\
    \frac{2}{3}   \\
    - \frac{1}{3} \\
    - \frac{2}{3} \\
    \frac{1}{3}   \\
    \frac{4}{3}   \\
    1             \\
    0             \\
    0
  \end{pmatrix} + x_8\begin{pmatrix}
    \frac{2}{3}  \\
    -\frac{1}{3} \\
    \frac{2}{3}  \\
    \frac{1}{3}  \\
    \frac{1}{3}  \\
    \frac{1}{3}  \\
    0            \\
    1            \\
    0
  \end{pmatrix} + x_9\begin{pmatrix}
    - \frac{1}{3} \\
    \frac{2}{3}   \\
    \frac{2}{3}   \\
    \frac{4}{3}   \\
    \frac{1}{3}   \\
    - \frac{2}{3} \\
    0             \\
    0             \\
    1
  \end{pmatrix} \Bigg| x_7, x_8, x_9 \in \mathbb{Q}
  \right\}
\]

Um eine Lösung in $\mathbb{Z}^9$ zu erhalten kann sichergestellt werden, dass
alle Komponenten ein vielfaches von drei sind. Hierfür kann beispielsweise die
hinreichende, aber nicht notwendige Bedingung dass $x_7 = 3a, x_8 = 3b, x_9 =
  3c$ mit $a, b, c \in \mathbb{Z}$ aufgestellt werden.

Man wähle beispielsweise $a = 1, b = 1, c = 1$, so erhält an das magische
Quadrat
\[
  \begin{pmatrix}
    3 & 3 & 3 \\ 3 & 3 & 3 \\ 3 & 3 & 3
  \end{pmatrix}
\]

Dieses ist auch eine Lösung für ein magisches Quadrat aus $\mathbb{N}^9$.

\section{Aufgabe 3}
Schreiben Sie den Gauß-Jordan-Algorithmus in Pseudocode auf.
\begin{lstlisting}
def gauss_jordan(matrix: list[list[float]], constants: list[list[float]] | None) -> list[list[float]]:
  n = len(matrix)
  m = len(matrix[0])
  i = 0

  while i < n and i < m:
    pivot_zeile = i

    for zeile in range(i + 1, n): # beste Zeile zum tauschen finden
      if abs(matrix[zeile][i]) > abs(matrix[pivot_zeile][i]):
        pivot_zeile = zeile

    if pivot_zeile != i: # Zeilen ggf. tauschen
      matrix[i], matrix[pivot_zeile] = matrix[pivot_zeile], matrix[i]  
      if constants:
        constants[i], constants[pivot_zeile] = constants[pivot_zeile], constants[i]

    if matrix[i][i] == 0:
      i += 1
      continue # Es gab keinen geeigneten Tauschkanidat. Eleminierung wird uebersprungen.
        
    pivot_wert = matrix[i][i]
    for k_norm in range(i, m):
      matrix[i][k_norm] = matrix[i][k_norm] / pivot_wert
    if constants and i < len(constants) and constants[i] is not None:
      for c_col in range(len(constants[i])):
        constants[i][c_col] = constants[i][c_col]  / pivot_wert

    for j in range(0, i): # Nullen oberhalb
        factor = matrix[j][i] / matrix[i][i]
        for k in range(i, m):
            matrix[j][k] = matrix[j][k] - matrix[i][k] * factor
        if constants and len(constants[0]) > 0:
            num_const_cols = len(constants[0])
            for c_col in range(num_const_cols):
                constants[j][c_col] = constants[j][c_col] - constants[i][c_col] * factor

    for j in range(i + 1, n): # Nullen unterhalb
        factor = matrix[j][i] / matrix[i][i]
        for k in range(i, m):
            matrix[j][k] = matrix[j][k] - matrix[i][k] * factor
        if constants and len(constants[0]) > 0:
            num_const_cols = len(constants[0])
            for c_col in range(num_const_cols):
                constants[j][c_col] = constants[j][c_col] - constants[i][c_col] * factor
    
    i += 1

  return matrix, constants
  
\end{lstlisting}