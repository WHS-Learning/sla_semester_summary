\chapter{Übungsblatt 11}

\section{Aufgabe 1}
Bestimmen Sie jeweilts sämtliche Eigenvektoren der folgenden Matzitzen. Hinweis: Die Eigenwerte haben Sie bereits auf dem letzten Übungsblatt bestimmt:

\subsection{a}

\begin{align*}
    A := \begin{pmatrix}
        1 & 2 & 3 \\
        4 & 5 & 6 \\ 
        0 & 0 & 0
    \end{pmatrix} \\
    \lambda_1 = 0, \quad \lambda_2 = 3 + \sqrt{12}, \quad \lambda_3 = 3 - \sqrt{12} \\
    \begin{pmatrix}
        1 - \lambda & 2 & 3 \\
        4 & 5 - \lambda & 6 \\
        0 & 0 & -\lambda
    \end{pmatrix} \\\\
    \text{Für } \lambda_1 = 0 \\
    \begin{pmatrix}
        1 - 0 & 2 & 3 \\
        4 & 5 - 0 & 6 \\
        0 & 0 & 0 - 0
    \end{pmatrix} \\
    \begin{pmatrix}
        1 & 2 & 3 \\
        4 & 5 & 6 \\
        0 & 0 & 0
    \end{pmatrix} \\
        \begin{pmatrix}
        1 & 2 & 3 \\
        4 & 5 & 6 \\
        0 & 0 & 0
    \end{pmatrix} \cdot \begin{pmatrix}
        x_1 \\ x_2 \\ x_3
    \end{pmatrix} = \begin{pmatrix}
        0 \\ 0 \\ 0
    \end{pmatrix}\\
    \begin{cases}
        \text{I:\@} & 1 \cdot x_1 + 2 \cdot x_2 + 3 \cdot x_3 = 0 \\
        \text{II:\@} & 4 \cdot x_1 + 5 \cdot x_2 + 6 \cdot x_3 = 0 \\
        \text{III:\@} & 0 \cdot x_1 + 0 \cdot x_2 + 0 \cdot x_3 = 0
    \end{cases} \\
    \begin{cases}
        \text{I:\@} & 1x_1 + 2x_2 + 3x_3 = 0 \\
        \text{II:\@} & 4x_1 + 5x_2 + 6x_3 = 0 \\
        \text{III:\@} & 0 = 0
    \end{cases}
\end{align*}

\begin{longtable}{p{10cm}}
    \hline
    \multicolumn{1}{c}{\textbf{Linearkombination}} \\
    \hline
    \endfirsthead

    \hline
    \multicolumn{1}{c}{\tablename\ \thetable\ -- \textit{Fortführung von vorherier Seite}} \\
    \hline
    \multicolumn{1}{c}{\textbf{Linearkombination}} \\
    \hline
    \endhead

    \hline
    \multicolumn{1}{r}{\textit{Fortsetzung siehe nächste Seite}} \\
    \endfoot

    \hline
    \endlastfoot

    $\displaystyle\begin{matrix}
        1 & 2 & 3 \\
        4 & 5 & 6
    \end{matrix}$\\\hline
    II - 2I \\\hline\pagebreak[0]
    $\displaystyle\begin{matrix}
        1 & 2 & 3 \\
        2 & 1 & 0
    \end{matrix}$\\\hline
    I - 2II \\\hline\pagebreak[0]
    $\displaystyle\begin{matrix}
        -3 & 0 & 3 \\
        2 & 1 & 0
    \end{matrix}$\\\hline
    I : (-3) \\\hline\pagebreak[0]
    $\displaystyle\begin{matrix}
        1 & 0 & -1 \\
        2 & 1 & 0
    \end{matrix}$\\\hline
    II - 2I \\\hline\pagebreak[0]
    $\displaystyle\begin{matrix}
        1 & 0 & -1 \\
        0 & 1 & -2
    \end{matrix}$\\\hline
\end{longtable}

\begin{align*}
    \begin{cases}
        \text{I:\@} & x_1 - x_3 = 0 \quad | + x_3 \Leftrightarrow x_1 = x_3 \\
        \text{II:\@} & x_2 - 2x_3 = 0 \quad | +2x_3 \Leftrightarrow x_2 = 2x_3
    \end{cases} \\
    x_3 = t, \text{ mit } t \in \mathbb{R} \\
    \begin{pmatrix}
        t \\ 2t \\ t
    \end{pmatrix} \\
    L = \left\{t \cdot \begin{pmatrix}
        1 \\ 2 \\ 1
    \end{pmatrix}\right\} \\
    span = \left\{\begin{pmatrix}
        1 \\ 2 \\ 1
    \end{pmatrix}\right\}
\end{align*}

\begin{align*}
    \text{Für } \lambda_2 = 3 + \sqrt{12} \\
    \begin{pmatrix}
        1 - \lambda & 2 & 3 \\
        4 & 5 - \lambda & 6 \\
        0 & 0 & 0 - \lambda
    \end{pmatrix} \\
    \begin{pmatrix}
        1 - (3 + \sqrt{12}) & 2 & 3 \\
        4 & 5 - (3 + \sqrt{12}) & 6 \\
        0 & 0 & 0 - (3 + \sqrt{12})
    \end{pmatrix} \\
    \begin{pmatrix}
        1 - 3 - \sqrt{12} & 2 & 3 \\
        4 & 5 - 3 - \sqrt{12} & 6 \\
        0 & 0 & 0 - 3 - \sqrt{12}
    \end{pmatrix} \\
    \begin{pmatrix}
        -2 - \sqrt{12} & 2 & 3 \\
        4 & 2 - \sqrt{12} & 6 \\
        0 & 0 & -3 - \sqrt{12}
    \end{pmatrix} \\
    \begin{pmatrix}
        -2 - \sqrt{12} & 2 & 3 \\
        4 & 2 - \sqrt{12} & 6 \\
        0 & 0 & -3 - \sqrt{12}
    \end{pmatrix} \cdot \begin{pmatrix}
        x_1 \\ x_2 \\ x_3
    \end{pmatrix} = \begin{pmatrix}
        0 \\ 0 \\ 0
    \end{pmatrix} \\
    \begin{cases}
        \text{I:\@} & (-2 - \sqrt{12})x_1 + 2x_2 + 3x_3 = 0 \\
        \text{II:\@} & 4x_1 + (2 - \sqrt{12})x_2 + 6x_3 = 0 \\
        \text{III:\@} & 0x_1 + 0x_2 + (-3 - \sqrt{12})x_3 = 0
    \end{cases}
\end{align*}

\begin{longtable}{p{10cm}}
    \hline
    \multicolumn{1}{c}{\textbf{Linearkombination}} \\
    \hline
    \endfirsthead

    \hline
    \multicolumn{1}{c}{\tablename\ \thetable\ -- \textit{Fortführung von vorherier Seite}} \\
    \hline
    \multicolumn{1}{c}{\textbf{Linearkombination}} \\
    \hline
    \endhead

    \hline
    \multicolumn{1}{r}{\textit{Fortsetzung siehe nächste Seite}} \\
    \endfoot

    \hline
    \endlastfoot

    $\displaystyle\begin{matrix}
        -2 - \sqrt{12} & 2 & 3 \\
        4 & 2 - \sqrt{12} & 6 \\
        0 & 0 & -3 - \sqrt{12}
    \end{matrix}$\\\hline
    III : (-3 - $\sqrt{12}$) \\\hline\pagebreak[0]
    $\displaystyle\begin{matrix}
        -2 - \sqrt{12} & 2 & 3 \\
        4 & 2 - \sqrt{12} & 6 \\
        0 & 0 & 1
    \end{matrix}$\\\hline
    (-2 - $\sqrt{12}$)II - 4I \\\hline\pagebreak[0]
    $\displaystyle\begin{matrix}
        -2 - \sqrt{12} & 2 & 3 \\
        0 & 0 & -24 - 12\sqrt{3} \\
        0 & 0 & 1
    \end{matrix}$\\\hline
    II - $(-24 - 12\sqrt{3})$III \\\hline\pagebreak[0]
    $\displaystyle\begin{matrix}
        -2 - \sqrt{12} & 2 & 3 \\
        0 & 0 & 0 \\
        0 & 0 & 1
    \end{matrix}$\\\hline
    I - 3III \\\hline\pagebreak[0]
    $\displaystyle\begin{matrix}
        -2 - \sqrt{12} & 2 & 0 \\
        0 & 0 & 0 \\
        0 & 0 & 1
    \end{matrix}$\\\hline
    I : (-2 - $\sqrt{12}$) \\\hline\pagebreak[0]
    $\displaystyle\begin{matrix}
        1 & \frac{2}{-2-\sqrt{12}} & 0 \\
        0 & 0 & 0 \\
        0 & 0 & 1
    \end{matrix}$\\\hline

\end{longtable}

\begin{align*}
    x_2 = t, \text{ mit } t \in \mathbb{R} \\
    \begin{cases}
        \text{I:\@} & x_1 + \frac{2}{-2-\sqrt{12}}t = 0 \quad | - \frac{2}{-2-\sqrt{12}}t \Leftrightarrow x_1 = -\frac{2}{-2-\sqrt{12}}t\\
        \text{II:\@} & 0 = 0 \\
        \text{III:\@} & x_3 = 0
    \end{cases} \\
    \begin{pmatrix}
        -\frac{2}{-2-\sqrt{12}}t \\
        t \\
        0
    \end{pmatrix} \\
    L = \left\{t \cdot \begin{pmatrix}
        -\frac{2}{-2-\sqrt{12}} \\
        1 \\
        0
    \end{pmatrix}\right\} \\
    span = \left\{\begin{pmatrix}
        -\frac{2}{-2-\sqrt{12}} \\
        1 \\
        0
    \end{pmatrix}\right\}
\end{align*}

\subsection{b}

\begin{align*}
    B := \begin{pmatrix}
        1 & 1 & 1 \\
        1 & 1 & 1 \\
        1 & 1 & 1
    \end{pmatrix}
\end{align*}

\subsection{c}

\begin{align*}
    C := \begin{pmatrix}
        0 & 0 & 0 \\
        0 & 0 & 0 \\
        0 & 0 & 0
    \end{pmatrix}
\end{align*}

\subsection{d}

\begin{align*}
    B = \begin{pmatrix}
        0 & 1 \\
        2 & 3
    \end{pmatrix}
\end{align*}