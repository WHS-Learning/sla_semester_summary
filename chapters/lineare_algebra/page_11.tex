\chapter{Übungsblatt 11}

\section{Aufgabe 1}
Bestimmen Sie jeweilts sämtliche Eigenvektoren der folgenden Matzitzen.
Hinweis: Die Eigenwerte haben Sie bereits auf dem letzten Übungsblatt bestimmt:

\subsection{a}

\begin{align*}
    A := \begin{pmatrix}
             1 & 2 & 3 \\
             4 & 5 & 6 \\
             0 & 0 & 0
         \end{pmatrix}                                                             \\
    \lambda_1 = 0, \quad \lambda_2 = 3 + \sqrt{12}, \quad \lambda_3 = 3 - \sqrt{12} \\
    \begin{pmatrix}
        1 - \lambda & 2           & 3        \\
        4           & 5 - \lambda & 6        \\
        0           & 0           & -\lambda
    \end{pmatrix}                                            \\\\
    \text{Für } \lambda_1 = 0                                                       \\
    \begin{pmatrix}
        1 - 0 & 2     & 3     \\
        4     & 5 - 0 & 6     \\
        0     & 0     & 0 - 0
    \end{pmatrix}                                                           \\
    \begin{pmatrix}
        1 & 2 & 3 \\
        4 & 5 & 6 \\
        0 & 0 & 0
    \end{pmatrix}                                                                  \\
    \begin{pmatrix}
        1 & 2 & 3 \\
        4 & 5 & 6 \\
        0 & 0 & 0
    \end{pmatrix} \cdot \begin{pmatrix}
                            x_1 \\ x_2 \\ x_3
                        \end{pmatrix} = \begin{pmatrix}
                                            0 \\ 0 \\ 0
                                        \end{pmatrix}                              \\
    \begin{cases}
        \text{I:\@}   & 1 \cdot x_1 + 2 \cdot x_2 + 3 \cdot x_3 = 0 \\
        \text{II:\@}  & 4 \cdot x_1 + 5 \cdot x_2 + 6 \cdot x_3 = 0 \\
        \text{III:\@} & 0 \cdot x_1 + 0 \cdot x_2 + 0 \cdot x_3 = 0
    \end{cases}                     \\
    \begin{cases}
        \text{I:\@}   & 1x_1 + 2x_2 + 3x_3 = 0 \\
        \text{II:\@}  & 4x_1 + 5x_2 + 6x_3 = 0 \\
        \text{III:\@} & 0 = 0
    \end{cases}
\end{align*}

\begin{longtable}{p{10cm}}
    \hline
    \multicolumn{1}{c}{\textbf{Linearkombination}}                                         \\
    \hline
    \endfirsthead

    \hline
    \multicolumn{1}{c}{\tablename\ \thetable\ -- \textit{Fortführung von vorherier Seite}} \\
    \hline
    \multicolumn{1}{c}{\textbf{Linearkombination}}                                         \\
    \hline
    \endhead

    \hline
    \multicolumn{1}{r}{\textit{Fortsetzung siehe nächste Seite}}                           \\
    \endfoot

    \hline
    \endlastfoot

    $\displaystyle\begin{matrix}
                          1 & 2 & 3 \\
                          4 & 5 & 6
                      \end{matrix}$                                                            \\\hline
    II - 2I                                                                                \\\hline\pagebreak[0]
    $\displaystyle\begin{matrix}
                          1 & 2 & 3 \\
                          2 & 1 & 0
                      \end{matrix}$                                                            \\\hline
    I - 2II                                                                                \\\hline\pagebreak[0]
    $\displaystyle\begin{matrix}
                          -3 & 0 & 3 \\
                          2  & 1 & 0
                      \end{matrix}$                                                            \\\hline
    I : (-3)                                                                               \\\hline\pagebreak[0]
    $\displaystyle\begin{matrix}
                          1 & 0 & -1 \\
                          2 & 1 & 0
                      \end{matrix}$                                                            \\\hline
    II - 2I                                                                                \\\hline\pagebreak[0]
    $\displaystyle\begin{matrix}
                          1 & 0 & -1 \\
                          0 & 1 & -2
                      \end{matrix}$                                                            \\\hline
\end{longtable}

\begin{align*}
    \begin{cases}
        \text{I:\@}  & x_1 - x_3 = 0 \quad | + x_3 \Leftrightarrow x_1 = x_3   \\
        \text{II:\@} & x_2 - 2x_3 = 0 \quad | +2x_3 \Leftrightarrow x_2 = 2x_3
    \end{cases} \\
    x_3 = t, \text{ mit } t \in \mathbb{R}                                 \\
    \begin{pmatrix}
        t \\ 2t \\ t
    \end{pmatrix}                                                         \\
    L = \left\{t \cdot \begin{pmatrix}
                           1 \\ 2 \\ 1
                       \end{pmatrix}\right\}                               \\
    span = \left\{\begin{pmatrix}
                      1 \\ 2 \\ 1
                  \end{pmatrix}\right\}
\end{align*}

\begin{align*}
    \text{Für } \lambda_2 = 3 + \sqrt{12}                           \\
    \begin{pmatrix}
        1 - \lambda & 2           & 3           \\
        4           & 5 - \lambda & 6           \\
        0           & 0           & 0 - \lambda
    \end{pmatrix}                         \\
    \begin{pmatrix}
        1 - (3 + \sqrt{12}) & 2                   & 3                   \\
        4                   & 5 - (3 + \sqrt{12}) & 6                   \\
        0                   & 0                   & 0 - (3 + \sqrt{12})
    \end{pmatrix} \\
    \begin{pmatrix}
        1 - 3 - \sqrt{12} & 2                 & 3                 \\
        4                 & 5 - 3 - \sqrt{12} & 6                 \\
        0                 & 0                 & 0 - 3 - \sqrt{12}
    \end{pmatrix}       \\
    \begin{pmatrix}
        -2 - \sqrt{12} & 2             & 3              \\
        4              & 2 - \sqrt{12} & 6              \\
        0              & 0             & -3 - \sqrt{12}
    \end{pmatrix}                 \\
    \begin{pmatrix}
        -2 - \sqrt{12} & 2             & 3              \\
        4              & 2 - \sqrt{12} & 6              \\
        0              & 0             & -3 - \sqrt{12}
    \end{pmatrix} \cdot \begin{pmatrix}
                            x_1 \\ x_2 \\ x_3
                        \end{pmatrix} = \begin{pmatrix}
                                            0 \\ 0 \\ 0
                                        \end{pmatrix}              \\
    \begin{cases}
        \text{I:\@}   & (-2 - \sqrt{12})x_1 + 2x_2 + 3x_3 = 0 \\
        \text{II:\@}  & 4x_1 + (2 - \sqrt{12})x_2 + 6x_3 = 0  \\
        \text{III:\@} & 0x_1 + 0x_2 + (-3 - \sqrt{12})x_3 = 0
    \end{cases}
\end{align*}

\begin{longtable}{p{10cm}}
    \hline
    \multicolumn{1}{c}{\textbf{Linearkombination}}                                         \\
    \hline
    \endfirsthead

    \hline
    \multicolumn{1}{c}{\tablename\ \thetable\ -- \textit{Fortführung von vorherier Seite}} \\
    \hline
    \multicolumn{1}{c}{\textbf{Linearkombination}}                                         \\
    \hline
    \endhead

    \hline
    \multicolumn{1}{r}{\textit{Fortsetzung siehe nächste Seite}}                           \\
    \endfoot

    \hline
    \endlastfoot

    $\displaystyle\begin{matrix}
                          -2 - \sqrt{12} & 2             & 3              \\
                          4              & 2 - \sqrt{12} & 6              \\
                          0              & 0             & -3 - \sqrt{12}
                      \end{matrix}$                          \\\hline
    III : (-3 - $\sqrt{12}$)                                                               \\\hline\pagebreak[0]
    $\displaystyle\begin{matrix}
                          -2 - \sqrt{12} & 2             & 3 \\
                          4              & 2 - \sqrt{12} & 6 \\
                          0              & 0             & 1
                      \end{matrix}$                                       \\\hline
    (-2 - $\sqrt{12}$)II - 4I                                                              \\\hline\pagebreak[0]
    $\displaystyle\begin{matrix}
                          -2 - \sqrt{12} & 2 & 3                \\
                          0              & 0 & -24 - 12\sqrt{3} \\
                          0              & 0 & 1
                      \end{matrix}$                                    \\\hline
    II - $(-24 - 12\sqrt{3})$III                                                           \\\hline\pagebreak[0]
    $\displaystyle\begin{matrix}
                          -2 - \sqrt{12} & 2 & 3 \\
                          0              & 0 & 0 \\
                          0              & 0 & 1
                      \end{matrix}$                                                   \\\hline
    I - 3III                                                                               \\\hline\pagebreak[0]
    $\displaystyle\begin{matrix}
                          -2 - \sqrt{12} & 2 & 0 \\
                          0              & 0 & 0 \\
                          0              & 0 & 1
                      \end{matrix}$                                                   \\\hline
    I : (-2 - $\sqrt{12}$)                                                                 \\\hline\pagebreak[0]
    $\displaystyle\begin{matrix}
                          1 & \frac{2}{-2-\sqrt{12}} & 0 \\
                          0 & 0                      & 0 \\
                          0 & 0                      & 1
                      \end{matrix}$                              \\\hline

\end{longtable}

\begin{align*}
    x_2 = t, \text{ mit } t \in \mathbb{R}                                                                                                                                    \\
    \begin{cases}
        \text{I:\@}   & x_1 + \frac{2}{-2-\sqrt{12}}t = 0 \quad | - \frac{2}{-2-\sqrt{12}}t \Leftrightarrow x_1 = -\frac{2}{-2-\sqrt{12}}t \\
        \text{II:\@}  & 0 = 0                                                                                                              \\
        \text{III:\@} & x_3 = 0
    \end{cases} \\
    \begin{pmatrix}
        -\frac{2}{-2-\sqrt{12}}t \\
        t                        \\
        0
    \end{pmatrix}                                                                                                                                     \\
    L = \left\{t \cdot \begin{pmatrix}
                           -\frac{2}{-2-\sqrt{12}} \\
                           1                       \\
                           0
                       \end{pmatrix}\right\}                                                                                                                   \\
    span = \left\{\begin{pmatrix}
                      -\frac{2}{-2-\sqrt{12}} \\
                      1                       \\
                      0
                  \end{pmatrix}\right\}
\end{align*}

\begin{align*}
    \text{Für } \lambda_3 = 3 - \sqrt{12}                           \\
    \begin{pmatrix}
        1 - \lambda & 2           & 3           \\
        4           & 5 - \lambda & 6           \\
        0           & 0           & 0 - \lambda
    \end{pmatrix}                         \\
    \begin{pmatrix}
        1 - (3 - \sqrt{12}) & 2                   & 3                   \\
        4                   & 5 - (3 - \sqrt{12}) & 6                   \\
        0                   & 0                   & 0 - (3 - \sqrt{12})
    \end{pmatrix} \\
    \begin{pmatrix}
        -2 + \sqrt{12} & 2             & 3              \\
        4              & 2 + \sqrt{12} & 6              \\
        0              & 0             & -3 + \sqrt{12}
    \end{pmatrix}                 \\
    \begin{pmatrix}
        -2 + \sqrt{12} & 2             & 3              \\
        4              & 2 + \sqrt{12} & 6              \\
        0              & 0             & -3 + \sqrt{12}
    \end{pmatrix} \cdot \begin{pmatrix}
                            x_1 \\ x_2 \\ x_3
                        \end{pmatrix} = \begin{pmatrix}
                                            0 \\ 0 \\ 0
                                        \end{pmatrix}              \\
    \begin{cases}
        \text{I:\@}   & (-2 + \sqrt{12})x_1 + 2x_2 + 3x_3 = 0 \\
        \text{II:\@}  & 4x_1 + (2 + \sqrt{12})x_2 + 6x_3 = 0  \\
        \text{III:\@} & (-3 + \sqrt{12})x_3 = 0
    \end{cases}
\end{align*}

\begin{longtable}{p{10cm}}
    \hline
    \multicolumn{1}{c}{\textbf{Linearkombination}}                                         \\
    \hline
    \endfirsthead

    \hline
    \multicolumn{1}{c}{\tablename\ \thetable\ -- \textit{Fortführung von vorherier Seite}} \\
    \hline
    \multicolumn{1}{c}{\textbf{Linearkombination}}                                         \\
    \hline
    \endhead

    \hline
    \multicolumn{1}{r}{\textit{Fortsetzung siehe nächste Seite}}                           \\
    \endfoot

    \hline
    \endlastfoot

    $\displaystyle\begin{matrix}
                          -2 + \sqrt{12} & 2             & 3              \\
                          4              & 2 + \sqrt{12} & 6              \\
                          0              & 0             & -3 + \sqrt{12}
                      \end{matrix}$                          \\\hline
    III : (-3 + $\sqrt{12}$)                                                               \\\hline\pagebreak[0]
    $\displaystyle\begin{matrix}
                          -2 + \sqrt{12} & 2             & 3 \\
                          4              & 2 + \sqrt{12} & 6 \\
                          0              & 0             & 1
                      \end{matrix}$                                       \\\hline
    I - 3III                                                                               \\\hline\pagebreak[0]
    II - 6III                                                                              \\\hline\pagebreak[0]
    $\displaystyle\begin{matrix}
                          -2 + \sqrt{12} & 2             & 0 \\
                          4              & 2 + \sqrt{12} & 0 \\
                          0              & 0             & 1
                      \end{matrix}$                                       \\\hline
    (-2 + $\sqrt{12}$)II - 4I                                                              \\\hline\pagebreak[0]
    $\displaystyle\begin{matrix}
                          -2 + \sqrt{12} & 2 & 0 \\
                          0              & 0 & 0 \\
                          0              & 0 & 1
                      \end{matrix}$                                                   \\\hline
    I : (-2 + $\sqrt{12}$)                                                                 \\\hline\pagebreak[0]
    $\displaystyle\begin{matrix}
                          1 & \frac{2}{-2 + \sqrt{12}} & 0 \\
                          0 & 0                        & 0 \\
                          0 & 0                        & 1
                      \end{matrix}$                            \\

\end{longtable}

\begin{align*}
    \begin{cases}
        \text{I:\@}   & x_1 + \frac{2}{-2 + \sqrt{12}}x_2 = 0 \quad |- \frac{2}{-2 + \sqrt{12}}x_2 \Leftrightarrow x_1 =  \frac{-1+\sqrt{3}}{2}x_2 \\
        \text{II:\@}  & 0 = 0                                                                                                                      \\
        \text{III:\@} & x_3 = 0
    \end{cases} \\
    x_2 = t \text{ wobei } t \in \mathbb{R}                                                                                                                                            \\
    \begin{pmatrix}
        \frac{2}{-2 + \sqrt{12}}t \\
        t                         \\
        0
    \end{pmatrix}                                                                                                                                             \\
    t \cdot \begin{pmatrix}
                \frac{2}{-2 + \sqrt{12}} \\
                1                        \\
                0
            \end{pmatrix}                                                                                                                                      \\
    span \left\{\begin{pmatrix}
                    \frac{2}{-2 + \sqrt{12}} \\
                    1                        \\
                    0
                \end{pmatrix}\right\}
\end{align*}

\subsection{b}

\begin{align*}
    B := \begin{pmatrix}
             1 & 1 & 1 \\
             1 & 1 & 1 \\
             1 & 1 & 1
         \end{pmatrix}
\end{align*}

\begin{align*}
    \lambda_1 = 0, \quad \lambda_2 = 0, \quad \lambda_3 = -3 \\
    \text{Für } \lambda_1 = \lambda_2 = 0                    \\
    \begin{pmatrix}
        1 - \lambda & 1           & 1           \\
        1           & 1 - \lambda & 1           \\
        1           & 1           & 1 - \lambda
    \end{pmatrix}                  \\
    \begin{pmatrix}
        1 - 0 & 1     & 1     \\
        1     & 1 - 0 & 1     \\
        1     & 1     & 1 - 0
    \end{pmatrix}                                    \\
    \begin{pmatrix}
        1 & 1 & 1 \\
        1 & 1 & 1 \\
        1 & 1 & 1
    \end{pmatrix}                                           \\
    \begin{pmatrix}
        1 & 1 & 1 \\
        1 & 1 & 1 \\
        1 & 1 & 1
    \end{pmatrix} \cdot \begin{pmatrix}
                            x_1 \\ x_2 \\ x_3
                        \end{pmatrix} = \begin{pmatrix}
                                            0 \\ 0 \\ 0
                                        \end{pmatrix}       \\
    \begin{cases}
        \text{I:\@}   & x_1 + x_2 + x_3 = 0 \\
        \text{II:\@}  & x_1 + x_2 + x_3 = 0 \\
        \text{III:\@} & x_1 + x_2 + x_3 = 0 \\
    \end{cases}
\end{align*}

\begin{longtable}{p{10cm}}
    \hline
    \multicolumn{1}{c}{\textbf{Linearkombination}}                                         \\
    \hline
    \endfirsthead

    \hline
    \multicolumn{1}{c}{\tablename\ \thetable\ -- \textit{Fortführung von vorherier Seite}} \\
    \hline
    \multicolumn{1}{c}{\textbf{Linearkombination}}                                         \\
    \hline
    \endhead

    \hline
    \multicolumn{1}{r}{\textit{Fortsetzung siehe nächste Seite}}                           \\
    \endfoot

    \hline
    \endlastfoot

    $\displaystyle\begin{matrix}
                          1 & 1 & 1 \\
                          1 & 1 & 1 \\
                          1 & 1 & 1
                      \end{matrix}$                                                            \\\hline
    II - I                                                                                 \\\hline\pagebreak[0]
    $\displaystyle\begin{matrix}
                          1 & 1 & 1 \\
                          0 & 0 & 0 \\
                          1 & 1 & 1
                      \end{matrix}$                                                            \\\hline
    III - I                                                                                \\\hline\pagebreak[0]
    $\displaystyle\begin{matrix}
                          1 & 1 & 1 \\
                          0 & 0 & 0 \\
                          0 & 0 & 0
                      \end{matrix}$                                                            \\\hline
\end{longtable}

\begin{align*}
    \begin{cases}
        \text{I:\@}   & x_1 + x_2 + x_3 = 0 \quad | -x_2 - x_3 \Leftrightarrow x_1 = -x_2 - x_3 \\
        \text{II:\@}  & 0 = 0                                                                   \\
        \text{III:\@} & 0 = 0
    \end{cases} \\
    x_2 = t \text{ wobei } t \in \mathbb{R}                                                 \\
    x_3 = p \text{ wobei } p \in \mathbb{R}                                                 \\
    \begin{pmatrix}
        -t -p \\ t \\ p
    \end{pmatrix}                                                                          \\
    t \begin{pmatrix}
          -1 -p \\ 1 \\ p
      \end{pmatrix}                                                                        \\
    t \begin{pmatrix}
          -1 \\ 1 \\ 0
      \end{pmatrix} + p \begin{pmatrix}
                            -1 \\ 0 \\ 1
                        \end{pmatrix}                                                      \\
    span = \left\{\begin{pmatrix}
                      -1 \\ 1 \\ 0
                  \end{pmatrix}, \begin{pmatrix}
                                     -1 \\ 0 \\ 1
                                 \end{pmatrix}\right\}
\end{align*}

\begin{align*}
    \text{Für } \lambda_3 = -3                         \\
    \begin{pmatrix}
        1 - \lambda & 1           & 1           \\
        1           & 1 - \lambda & 1           \\
        1           & 1           & 1 - \lambda
    \end{pmatrix}            \\
    \begin{pmatrix}
        1 - -3 & 1      & 1      \\
        1      & 1 - -3 & 1      \\
        1      & 1      & 1 - -3
    \end{pmatrix}                           \\
    \begin{pmatrix}
        4 & 1 & 1 \\
        1 & 4 & 1 \\
        1 & 1 & 4
    \end{pmatrix}                                     \\
    \begin{pmatrix}
        4 & 1 & 1 \\
        1 & 4 & 1 \\
        1 & 1 & 4
    \end{pmatrix} \cdot \begin{pmatrix}
                            x_1 \\ x_2 \\ x_3
                        \end{pmatrix} = \begin{pmatrix}
                                            0 \\ 0 \\ 0
                                        \end{pmatrix} \\
    \begin{cases}
        \text{I:\@}   & 4x_1 + x_2 + x_3 = 0 \\
        \text{II:\@}  & x_1 + 4x_2 + x_3 = 0 \\
        \text{III:\@} & x_1 + x_2 + 4x_3 = 0
    \end{cases}
\end{align*}

\begin{longtable}{p{10cm}}
    \hline
    \multicolumn{1}{c}{\textbf{Linearkombination}}                                         \\
    \hline
    \endfirsthead

    \hline
    \multicolumn{1}{c}{\tablename\ \thetable\ -- \textit{Fortführung von vorherier Seite}} \\
    \hline
    \multicolumn{1}{c}{\textbf{Linearkombination}}                                         \\
    \hline
    \endhead

    \hline
    \multicolumn{1}{r}{\textit{Fortsetzung siehe nächste Seite}}                           \\
    \endfoot

    \hline
    \endlastfoot

    $\displaystyle\begin{matrix}
                          4 & 1 & 1 \\
                          1 & 4 & 1 \\
                          1 & 1 & 4
                      \end{matrix}$                                                            \\\hline
    II - III                                                                               \\\hline\pagebreak[0]
    $\displaystyle\begin{matrix}
                          4 & 1 & 1  \\
                          0 & 3 & -3 \\
                          1 & 1 & 4
                      \end{matrix}$                                                            \\\hline
    4III - I                                                                               \\\hline\pagebreak[0]
    $\displaystyle\begin{matrix}
                          4 & 1  & 1   \\
                          0 & 3  & -3  \\
                          0 & -3 & -15
                      \end{matrix}$                                                            \\\hline
    III + II                                                                               \\\hline\pagebreak[0]
    $\displaystyle\begin{matrix}
                          4 & 1 & 1   \\
                          0 & 3 & -3  \\
                          0 & 0 & -18
                      \end{matrix}$                                                            \\\hline
    II : 3                                                                                 \\\hline\pagebreak[0]
    III : -18                                                                              \\\hline\pagebreak[0]
    $\displaystyle\begin{matrix}
                          4 & 1 & 1  \\
                          0 & 1 & -1 \\
                          0 & 0 & 1
                      \end{matrix}$                                                            \\\hline
    II + III                                                                               \\\hline\pagebreak[0]
    $\displaystyle\begin{matrix}
                          4 & 1 & 1 \\
                          0 & 1 & 0 \\
                          0 & 0 & 1
                      \end{matrix}$                                                            \\\hline
    I - III                                                                                \\\hline\pagebreak[0]
    $\displaystyle\begin{matrix}
                          4 & 1 & 0 \\
                          0 & 1 & 0 \\
                          0 & 0 & 1
                      \end{matrix}$                                                            \\\hline
    II - III                                                                               \\\hline\pagebreak[0]
    $\displaystyle\begin{matrix}
                          4 & 0 & 0 \\
                          0 & 1 & 0 \\
                          0 & 0 & 1
                      \end{matrix}$                                                            \\\hline
    I : 4                                                                                  \\\hline\pagebreak[0]
    $\displaystyle\begin{matrix}
                          1 & 0 & 0 \\
                          0 & 1 & 0 \\
                          0 & 0 & 1
                      \end{matrix}$                                                            \\
\end{longtable}

\begin{align*}
    \begin{cases}
        \text{I:\@}   & x_1 = 0 \\
        \text{II:\@}  & x_2 = 0 \\
        \text{III:\@} & x_3 = 0
    \end{cases} \\
    \begin{pmatrix}
        0 \\ 0 \\ 0
    \end{pmatrix}          \\
    span = \left\{\begin{pmatrix}
                      0 \\ 0 \\ 0
                  \end{pmatrix}\right\}
\end{align*}

\subsection{c}

\begin{align*}
    C := \begin{pmatrix}
             0 & 0 & 0 \\
             0 & 0 & 0 \\
             0 & 0 & 0
         \end{pmatrix}
\end{align*}

\begin{align*}
    \lambda_{1, 2, 3} = 0                                                  \\
    \text{Für } \lambda_1 = \lambda_2 = \lambda_3 = 0                      \\
    \begin{pmatrix}
        0 - \lambda & 0           & 0           \\
        0           & 0 - \lambda & 0           \\
        0           & 0           & 0 - \lambda
    \end{pmatrix}                                \\
    \begin{pmatrix}
        0 & 0 & 0 \\
        0 & 0 & 0 \\
        0 & 0 & 0
    \end{pmatrix}                                                         \\
    \begin{pmatrix}
        0 & 0 & 0 \\
        0 & 0 & 0 \\
        0 & 0 & 0
    \end{pmatrix} \cdot \begin{pmatrix}
                            x_1 \\ x_2 \\ x_3
                        \end{pmatrix} = \begin{pmatrix}
                                            0 \\ 0 \\ 0
                                        \end{pmatrix}                     \\
    \begin{cases}
        \text{I:\@}   & 0 = 0 \\
        \text{II:\@}  & 0 = 0 \\
        \text{III:\@} & 0 = 0
    \end{cases}                                                  \\
    x_1 = t \text{ wobei } t \in \mathbb{R}                                \\
    x_2 = p \text{ wobei } p \in \mathbb{R}                                \\
    x_3 = u \text{ wobei } u \in \mathbb{R}                                \\
    \begin{pmatrix}
        t \\ p \\ u
    \end{pmatrix}                                                         \\
    t \cdot \begin{pmatrix}
                1 \\ 0 \\ 0
            \end{pmatrix} + p \cdot \begin{pmatrix}
                                        0 \\ 1 \\ 0
                                    \end{pmatrix} + u \cdot \begin{pmatrix}
                                                                0 \\ 0 \\ 1
                                                            \end{pmatrix} \\
    span = \left\{\begin{pmatrix}
                      1 \\ 0 \\ 0
                  \end{pmatrix}, \begin{pmatrix}
                                     0 \\ 1 \\ 0
                                 \end{pmatrix}, \begin{pmatrix}
                                                    0 \\ 0 \\ 1
                                                \end{pmatrix}\right\}
\end{align*}

\subsection{d}

\begin{align*}
    B = \begin{pmatrix}
            0 & 1 \\
            2 & 3
        \end{pmatrix}
\end{align*}

\begin{align*}
    \lambda_1 = \frac{3}{2} + \sqrt{\frac{17}{4}}, \quad \lambda_2 = \frac{3}{2} - \sqrt{\frac{17}{4}}                                                                   \\
    \text{Für } \lambda_1 = \frac{3}{2} + \sqrt{\frac{17}{4}}                                                                                                            \\
    \begin{pmatrix}
        0 - \lambda & 1           \\
        2           & 3 - \lambda
    \end{pmatrix}                                                                                                                                            \\
    \begin{pmatrix}
        0 - \left(\frac{3}{2} + \sqrt{\frac{17}{4}}\right) & 1                                                  \\
        2                                                  & 3 - \left(\frac{3}{2} + \sqrt{\frac{17}{4}}\right)
    \end{pmatrix} \\
    \begin{pmatrix}
        0 - \frac{3}{2} - \sqrt{\frac{17}{4}} & 1                                    \\
        2                                     & 3 -\frac{3}{2} - \sqrt{\frac{17}{4}}
    \end{pmatrix}                                                                              \\
    \begin{pmatrix}
        -\frac{3}{2} - \sqrt{\frac{17}{4}} & 1                                 \\
        2                                  & \frac{3}{2} - \sqrt{\frac{17}{4}}
    \end{pmatrix}                                                                                    \\
    \begin{pmatrix}
        -\frac{3}{2} - \sqrt{\frac{17}{4}} & 1                                 \\
        2                                  & \frac{3}{2} - \sqrt{\frac{17}{4}}
    \end{pmatrix} \begin{pmatrix}
                      x_1 \\ x_2
                  \end{pmatrix} = \begin{pmatrix}
                                      0 \\ 0
                                  \end{pmatrix}                                                                                    \\
    \begin{cases}
        \text{I:\@}  & -\frac{3}{2} - \sqrt{\frac{17}{4}}x_1 + x_2 = 0 \\
        \text{II:\@} & 2x_1 + \frac{3}{2} - \sqrt{\frac{17}{4}}x_2 = 0
    \end{cases}
\end{align*}

\begin{longtable}{p{10cm}}
    \hline
    \multicolumn{1}{c}{\textbf{Linearkombination}}                                                  \\
    \hline
    \endfirsthead

    \hline
    \multicolumn{1}{c}{\tablename\ \thetable\ -- \textit{Fortführung von vorherier Seite}}          \\
    \hline
    \multicolumn{1}{c}{\textbf{Linearkombination}}                                                  \\
    \hline
    \endhead

    \hline
    \multicolumn{1}{r}{\textit{Fortsetzung siehe nächste Seite}}                                    \\
    \endfoot

    \hline
    \endlastfoot

    $\displaystyle\begin{matrix}
                          -\frac{3}{2} - \sqrt{\frac{17}{4}} & 1                                 \\
                          2                                  & \frac{3}{2} - \sqrt{\frac{17}{4}}
                      \end{matrix}$ \\\hline
    ($-\frac{3}{2} - \sqrt{\frac{17}{4}}$)II - 2I                                                   \\\hline\pagebreak[0]
    $\displaystyle\begin{matrix}
                          -\frac{3}{2} - \sqrt{\frac{17}{4}} & 1 \\
                          0                                  & 0
                      \end{matrix}$                                 \\\hline
\end{longtable}

\begin{align*}
    \begin{cases}
        \text{I:\@}  & -\frac{3}{2} - \sqrt{\frac{17}{4}}x_1 + x_2 = 0 \quad -x_2 \\
        \text{II:\@} & 0 = 0
    \end{cases} \\
    \begin{cases}
        \text{I:\@}  & -\frac{3}{2} - \sqrt{\frac{17}{4}}x_1 = -x_2 \quad -x_2 \\
        \text{II:\@} & 0 = 0
    \end{cases}    \\
    x_2 = t \text{ wobei } t \in \mathbb{R}                                              \\
    \begin{pmatrix}
        -\frac{3}{2} - \sqrt{\frac{17}{4}} \\ t
    \end{pmatrix}                                    \\
    t \cdot \begin{pmatrix}
                -\frac{3}{2} - \sqrt{\frac{17}{4}} \\ 1
            \end{pmatrix}                            \\
    span = \left\{\begin{pmatrix}
                      -\frac{3}{2} - \sqrt{\frac{17}{4}} \\ 1
                  \end{pmatrix}\right\}
\end{align*}

\begin{align*}
    \text{Für } \lambda_2 = \frac{3}{2} - \sqrt{\frac{17}{4}}                                    \\
    \begin{pmatrix}
        0 - \lambda & 1           \\
        2           & 3 - \lambda
    \end{pmatrix}                                                                    \\
    \begin{pmatrix}
        0 - (\frac{3}{2} - \sqrt{\frac{17}{4}}) & 1                                       \\
        2                                       & 3 - (\frac{3}{2} - \sqrt{\frac{17}{4}})
    \end{pmatrix} \\
    \begin{pmatrix}
        0 -\frac{3}{2} + \sqrt{\frac{17}{4}} & 1                                    \\
        2                                    & 3 -\frac{3}{2} + \sqrt{\frac{17}{4}}
    \end{pmatrix}       \\
    \begin{pmatrix}
        -\frac{3}{2} + \sqrt{\frac{17}{4}} & 1                                 \\
        2                                  & \frac{3}{2} + \sqrt{\frac{17}{4}}
    \end{pmatrix}            \\
\end{align*}

\begin{longtable}{p{10cm}}
    \hline
    \multicolumn{1}{c}{\textbf{Linearkombination}}                                                  \\
    \hline
    \endfirsthead

    \hline
    \multicolumn{1}{c}{\tablename\ \thetable\ -- \textit{Fortführung von vorherier Seite}}          \\
    \hline
    \multicolumn{1}{c}{\textbf{Linearkombination}}                                                  \\
    \hline
    \endhead

    \hline
    \multicolumn{1}{r}{\textit{Fortsetzung siehe nächste Seite}}                                    \\
    \endfoot

    \hline
    \endlastfoot

    $\displaystyle\begin{matrix}
                          -\frac{3}{2} + \sqrt{\frac{17}{4}} & 1                                 \\
                          2                                  & \frac{3}{2} + \sqrt{\frac{17}{4}}
                      \end{matrix}$ \\\hline
    ($-\frac{3}{2} + \sqrt{\frac{17}{4}}$)II - 2I                                                   \\\hline\pagebreak[0]
    $\displaystyle\begin{matrix}
                          -\frac{3}{2} + \sqrt{\frac{17}{4}} & 1 \\
                          0                                  & 0
                      \end{matrix}$                                 \\
\end{longtable}

\begin{align*}
    \begin{cases}
        \text{I:\@}  & -\frac{3}{2} + \sqrt{\frac{17}{4}}x_1 + x_2 = 0 \quad -x_2 \\
        \text{II:\@} & 0 = 0
    \end{cases} \\
    \begin{cases}
        \text{I:\@}  & -\frac{3}{2} + \sqrt{\frac{17}{4}}x_1 = -x_2 \quad -x_2 \\
        \text{II:\@} & 0 = 0
    \end{cases}    \\
    x_2 = t \text{ wobei } t \in \mathbb{R}                                              \\
    \begin{pmatrix}
        -\frac{3}{2} + \sqrt{\frac{17}{4}} \\ t
    \end{pmatrix}                                    \\
    t \cdot \begin{pmatrix}
                -\frac{3}{2} + \sqrt{\frac{17}{4}} \\ 1
            \end{pmatrix}                            \\
    span = \left\{\begin{pmatrix}
                      -\frac{3}{2} + \sqrt{\frac{17}{4}} \\ 1
                  \end{pmatrix}\right\}
\end{align*}