\chapter{Übungsblatt 9}

\section{Aufgabe 1}

\subsection{a}

Die größenverteilung von Männern in Deutschland kann laut einer Studie als normalverteilt angenommen werden mit Mittelwert $176.69cm$ und Standardabweichung $6959 cm$. Wie groß ist nach diesen Daten die Wahrscheinlichkeit dafür, dass ein zufällig ausgewähler Mann größer als $200 cm$ ist?

\subsection{b}

Gemäß einer Definition von Armut gilt als arm, wer ein monatliches Haushaltseinkommen von weniger als 60\% des Medians der monatlichen Haushaltseinkommen eines Landes zur Verfügung hat. Angenommen, die Haushaltseinkommen in einem Land seien normalverteilt mit einem Mittelwert von $3382 €$ (das entsprocht den mittleren Haushaltseinkommen in Deutschland 2019) und einer Standardabweichung von $1000 €$. Bis zu welcher Einkommensgrenze gilt ein Mensch als arm? Was hinkt an dieser Definition?

\section{Aufgabe 2}

Da erfahrungsgemäß etwa 4\% der Fluggäste nicht zum Abflug erschienen, werden Flugzeuge systematisch überbucht. Für eine Maschine mit 98 Sitzplätzen werden 10 Tickets verkauft. Bestimmen Sie die Wahrscheinlichkeit dafür, dass mehr Ticketinhaber zum Abflug erscheinen als Pläte vorhanden sind, unter der Annahme, dass das Erscheinen der Ticketinhaber paarweise stochastisch unabhängig ist

\subsection{a}

exakt an Hand der Binomialverteilung

\subsection{b}

näherungsweise mit dem Satz von Moivere-Laplace ohne Stetigkeitskorrektur

\subsection{c}

näherungsweise mit dem Satz von Moivre-Laplace mit Stetigkeitskorrektur.

\section{Aufgabe 3}

Wie oft müssen Sie eine faire Münze werfen, damit mit einer Wahrscheinlichkeit von $0.95$ die relative Häufigkeit der Würfe mit dem Ergebnis "Kopf" zwischen $0.49$ und $0.51$ liegt? Arbeiten Sie mit der Näherungsformel aus dem Satz von de Moivre-Laplace.

