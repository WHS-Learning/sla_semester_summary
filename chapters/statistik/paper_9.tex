\chapter{Übungsblatt 9}

\section{Aufgabe 1}

\subsection{a}

Die größenverteilung von Männern in Deutschland kann laut einer Studie als
normalverteilt angenommen werden mit Mittelwert $176.69cm$ und
Standardabweichung $6.959 cm$. Wie groß ist nach diesen Daten die
Wahrscheinlichkeit dafür, dass ein zufällig ausgewähler Mann größer als $200
    cm$ ist?

\begin{align*}
    X = \text{Größe der Männer in cm}                              \\
    E(x) = 176.69                                                  \\
    \sigma = 6.959                                                 \\
    P(X > 200)                                                     \\
    1 - P(X \leq 200)                                              \\
    1 - P(\frac{X - E(X)}{\sigma} \leq \frac{200 - 176.69}{6.959}) \\
    = 1 - \Phi(\frac{200 - 176.69}{6.959})                         \\
    \approx 1 - \Phi(3.35)                                         \\
    \overset{Tabelle}{\approx} 1 - 0.99960                         \\
    = 0.0004                                                       \\
    = 0.04\%                                                       \\
    \text{Zum spaß an der Freude}                                  \\
    P(X > 300)                                                     \\
    1 - P(X \leq 300)                                              \\
    1 - P(\frac{X - E(X)}{\sigma} \leq \frac{300 - 176.69}{6.959}) \\
    = 1 - \Phi(\frac{300 - 176.69}{6.959})                         \\
    = 1 - \Phi(17.72)                                              \\
    \overset{Tabelle}{\approx} 1 - 1                               \\
    = 0                                                            \\
    = 0\%                                                          \\
\end{align*}

\subsection{b}

Gemäß einer Definition von Armut gilt als arm, wer ein monatliches
Haushaltseinkommen von weniger als 60\% des Medians der monatlichen
Haushaltseinkommen eines Landes zur Verfügung hat. Angenommen, die
Haushaltseinkommen in einem Land seien normalverteilt mit einem Mittelwert von
$3382$€ (das entspricht den mittleren Haushaltseinkommen in Deutschland 2019)
und einer Standardabweichung von $1000$€. Bis zu welcher Einkommensgrenze gilt
ein Mensch als arm? Was hinkt an dieser Definition?

\begin{align*}
    3382 \cdot 0.6 = 2029.2??                                         \\\\
    P(X) = 0.6                                                        \\
    \frac{X - 3382}{1000} = 1 - 0.26 \quad | \cdot 1000 \quad |+ 3382 \\
    x = (1 - 0.26) \cdot 1000 + 3382                                  \\
    x = 4122???
\end{align*}

Das Haushaltseinkommen kann nicht normalverteilt sein, da die Glockensymmetrie
nicht gegeben sein kann.

\section{Aufgabe 2}

Da erfahrungsgemäß etwa 4\% der Fluggäste nicht zum Abflug er\-schei\-nen,
wer\-den Flugzeuge systematisch überbucht. Für eine Maschine mit 98 Sitzplätzen
werden 100 Tickets verkauft. Bestimmen Sie die Wahrscheinlichkeit dafür, dass
mehr Ticketinhaber zum Abflug erscheinen als Pläte vorhanden sind, unter der
Annahme, dass das Erscheinen der Ticketinhaber paarweise stochastisch
unabhängig ist

\subsection{a}

exakt an Hand der Binomialverteilung

\begin{align*}
    X = \text{Anzahl Fluggäste}                                                              \\
    X \sim B(100, 0.96)                                                                      \\
    \sum_{i = 99}^{100} \left(\begin{pmatrix}
                                      100 \\ i
                                  \end{pmatrix} \cdot 0.96^i \cdot {(1 - 0.96)}^{100 - i}\right) \\
    \approx 0.0872                                                                           \\
    8.72\%
\end{align*}

\subsection{b}

näherungsweise mit dem Satz von Moivere-Laplace ohne Stetigkeitskorrektur

\begin{align*}
    X = \text{Anzahl Fluggäste}                                               \\
    X \sim B(100, 0.96)                                                       \\
    E(X) = 100 \cdot 0.96 = 96                                                \\
    Var(X) = 96 \cdot 0.04 = 3.84                                             \\
    \sigma = \sqrt{3.84}
    P(X \geq 99)                                                              \\
    = 1 - P(X < 99)                                                           \\
    = 1 - P\left(\frac{X - E(X)}{\sigma} < \frac{99 - 96}{\sqrt{3.84}}\right) \\
    \approx 1 - \Phi(1.53)                                                    \\
    \overset{Tabelle}{\approx} 1 - 0.93699                                    \\
    \approx 0.0630                                                            \\
    = 6.30\%
\end{align*}

\subsection{c}

näherungsweise mit dem Satz von Moivre-Laplace mit Stetigkeitskorrektur.

\begin{align*}
    X = \text{Anzahl Fluggäste}                                               \\
    X \sim B(100, 0.96)                                                       \\
    E(X) = 100 \cdot 0.96 = 96                                                \\
    Var(X) = 96 \cdot 0.04 = 3.84                                             \\
    \sigma = \sqrt{3.84}
    P(X \geq 99)                                                              \\
    P(X \geq 98.5)                                                            \\
    1 - P(X < 98.5)                                                           \\
    1 - P\left(\frac{X - E(X)}{\sigma} < \frac{98.5 - 96}{\sqrt{3.84}}\right) \\
    \approx 1 - \Phi(1.28)                                                    \\
    \overset{Tabelle}{\approx} 1 - 0.89973                                    \\
    \approx 0.1003                                                            \\
    = 10.03 \%
\end{align*}

\section{Aufgabe 3}

Wie oft müssen Sie eine faire Münze werfen, damit mit einer Wahrscheinlichkeit
von $0.95$ die relative Häufigkeit der Würfe mit dem Ergebnis `Kopf' zwischen
$0.49$ und $0.51$ liegt? Arbeiten Sie mit der Näherungsformel aus dem Satz von
de Moivre-Laplace.

\begin{align*}
    X = \text{Münzwurf mit Kopf, } \begin{cases}
                                       0 & \text{bei Kof, } \\
                                       1 & \text{sonst}
                                   \end{cases} \\
    P(0.49 \leq X \leq 0.51)                            \\
    \Phi(1 - 0.49) = \Phi(0.51)                         \\
    \Phi(0.51) = 0.03                                   \\
    \frac{n - n \cdot \frac{1}{2}}{\sqrt{n \cdot \frac{1}{2} \cdot \frac{1}{2}}} = 0.03\\
    \Leftrightarrow \frac{\frac{n}{2}}{\sqrt{\frac{n}{4}}} = 0.03 \\
    \Leftrightarrow \frac{\frac{n}{2}}{\frac{\sqrt{n}}{2}} = 0.03 \\
    \Leftrightarrow \frac{n}{2} \cdot \frac{2}{\sqrt{n}} = 0.03 \\
    \Leftrightarrow \frac{2n}{2\sqrt{n}} = 0.03 \\
    \Leftrightarrow \frac{2n}{2\sqrt{n}} = 0.03 \\
    \Leftrightarrow \frac{n}{\sqrt{n}} = 0.03 \quad | \cdot \sqrt{n} \\
    \Leftrightarrow n = 0.03\sqrt{n}\\
\end{align*}