\chapter{Übungsblatt 7}

\section{Aufgabe 1}

Herr Huber hat eine Alarmanlage in seinem Auto installiert. Es werden die Ereignisse A: "Alarmanlage springt an" und E: "Jemand versucht, das Auto aufzubrechen" betrachtet. Beschreiben Sie die folgenden bedingten Wahrscheinlichkeiten mit Worten: $P(A|E), P(E|A), P(A|E^C), P(E|A^C)$. Welche dieser bedingten Wahrscheinlichkeiten möglichst hoch bzw. niedrig seien?

\begin{itemize}
    \item $P(A|E)$ \texttt{Die Alarmanlage springt an, unter der Bedingung dass jemand versucht, das Auto aufzubrechen}: Die Wahrscheinlichkeit hierfür sollte möglichst hoch sein, da dies der sinn der Alarmanlage sein. Im idealfall sollte die Wahrscheinlichkeit hierfür gegen 100\% sein.
    \item $P(E|A)$ \texttt{Jemand versucht, in das Auto einzubrechen, unter der Bedingung, dass die Alarmanlage angeht}: Die Wahrscheinlichkeit hierfür sollte hoch sein, aber nicht 100\%, da die Alarmanlage den Dieb verschrecken sollte.
    \item $P(A|E^C)$ \texttt{Die Alarmanlage springt an, unter der Bedingung, dass niemand versucht, das Auto aufzubrechen (Fehlarlarm)}: Die Wahrscheinlichkeit hierfür sollte möglichst niedrig sein, da niemand versucht in das Auto einzubrechen.
    \item $P(E|A^C)$ \texttt{Jemand versucht, in das Auto einzubrechen, unter der Bedingung, dass die Alarmanlage \textbf{nicht} angeht}: Die Wahrscheinlichkeit hierfür sollte sehr niedrig sein.
\end{itemize}

\section{Aufgabe 2}

Bei einer Sportveranstaltung wird ein Dopingtest durchgeführt. Wenn ein Sportler gedopt hat, dann fällt der Test zu 99\% positiv aus. Hat ein Sportler aber nicht gedopt, zeigt der Test trotzdem zu 5\% ein positives Ergebnis an. Aus Erfahrung weiß man das 20\% der Sportler gedopt sind.

\begin{center}    
    \begin{forest}
    prob/.style={
        edge label={node[midway, auto, sloped, font=\small]{#1}}
    },
    for tree={
        draw,
        circle,
        minimum size=1.8em,
        l sep=1.5cm,
        s sep=1.5cm,
    }
    [
        [D, prob={$20\%$}
            [P, prob={$99\%$}]
            [$\overline{P}$, prob={$1\%$}]
        ]
        [$\overline{D}$, prob={$80\%$}
            [P, prob={$5\%$}]
            [$\overline{P}$, prob={$95\%$}]
        ]
    ]
    \end{forest}
\end{center}

\subsection{a}

Wie groß ist die Wahrscheinlichkeit dafür, dass ein Dopingtest positiv ausfällt?

\begin{align*}
    0.2 \cdot 0.99 + 0.8 \cdot 0.05 = 0.238 = 23.8\%
\end{align*}

\subsection{b}

Wie groß ist die Wahrscheinlichkeit dafür, dass der Test negativ ausfällt, obwohl der Sportler gedopt hat?

\begin{align*}
    1\%??
\end{align*}

\subsection{c}

Wie groß ist die Wahrscheinlichkeit dafür, dass ein Sportler gedopt hat, falls sein Dopingtest negativ ausgefallen ist?

\begin{align*}
    \frac{0.2 \cdot 0.01}{0.8 \cdot 0.95} = \frac{0.002}{0.76} = \frac{1}{380} \approx 0.0026 = 0.26\%
\end{align*}

\section{Aufgabe 3}

Herr Mayer kommt im Durchschnitt an 10 von 100 Tagen zu spät für den Vorlesungsbeginn zur Hochschule. Er fährt manchmal mit dem eigenen Auto zur W-HS, an 60\% aller Vorlesungstage nimmt er jedoch öffentliche Verkehrsmittel. Er hat beobachtet, dass er durchschnittlich in 5\% aller Fälle mit dem Auto unterwegs ist und zu spät zur Hochschule kommt. Sind das Zu-Spät-Kommen und die Nutzung des eigenen Autos voneinander stochastisch unabhängig?

Das Zuspät kommen ist stochastisch abhängig, da die Nutzung des Autos direkt in verbindung mit der Wahrscheinlichkeit, dass Herr Mayer zu spät kommt, steht.

\section{Aufgabe 4}

Seien $(\Omega, \mathcal{P}(\Omega), P)$ ein Wahrscheinlichkeitsraum und seinen $A, B, C \subset \Omega$ drei Ergebnisse. Man sagt, $A, B$ und $C$ seien \textit{Vollständig stochastisch unabhängig}, wenn gilt, dass

\begin{itemize}
    \item $P(A \cap B) = P(A) \cdot P(B)$
    \item $P(A \cap C) = P(A) \cdot P(C)$
    \item $P(B \cap C) = P(B) \cdot P(C)$
    \item $P(A \cap B \cap C) = P(A) \cdot P(B) \cdot P(C)$
\end{itemize}

Beantworten Sie die folgenden Fragen:

\subsection{a}

Was ist der Unterschied zwischen vollständiger stochastischer Unabhängigkeit und paarweiser stochastischer Unabhängigkeit?

Paarweise Stochastische unabhängigkeit ist gegeben, wenn die Bedingung $P(A \cap B \cap C) = P(A) \cdot P(B) \cdot P(C)$ \textit{nicht} gegeben ist, aber alle anderen Bedingungen gegeben sind.

\subsection{b}

Wie kann man diese Definition auf beliebig viele Ereignisse verallgemeinern?

\subsection{c}

Finden Sie ein Beispiel für drei Ereignisse, die paarweise stochastisch unabhängig sind, aber nicht vollständig stochastisch unabhängig.