\chapter{Übungsblatt 10}

\section{Aufgabe 1}

Einer Handelskette wurde vertraglich zugesichert, dass maximal $1\%$ der Becher einen Defekten Deckel besitzen. Normalerweise kann dieser Qualitätsstandard leicht eingehalten werden. Eines Tages stellt sich bei einer Qualitätskontrolle in der Molkerei heraus, dass $4\%$ der Joghurtbecher einen Defekten Deckel aufweisen. Bei einem schon beladenen Lkw ist ungewiss, ob die Joghurtbecher bereits aus der Produktion mit dem erhöhten Anteil an defekten Deckeln stammen. Deshalb wird der Ladung eine Stichprobe entnommen und untersucht.

\subsection{a}

Falls bei einer Stichprobe aus 100 Beckern mindestens zwei Becker einen defekten Deckel haben, wird der Lkw in der Molkerei wieder entladen, andernfalls wird die Lieferung freigegeben. Wie groß ist die Wahrscheinlichkeit, dass die Lieferung freigegeben. Wie groß ist die Wahrscheinlichkeit, dass die Lieferung freigegeben wird, obwohl sie einen erhöhten Anteil an Joghurtbechern mit defekten Deckel aufweist? Wie groß kann die Wahrscheinlichkeit für ein unnötiges Entladen des Lkws bei Einhaltung des zugesicherten Qualitätsstandards maximal werden?

\subsection{b}

Nun soll die Wahrscheinlichkeit eines unnötigen Ausladens des Lkws auf $5\%$ festgelegtz werden und damit das Konfidenzniveau des beabsichtigten Tests auf $95\%$. Bestimmen Sie eine Anzahl $z \in \mathbb{N}$ an Joghurtbechern mit defekten Deckel, ab der die Hypothese, dass die Ladung dem Qualitätsstandard entspricht, abgelehnt werden soll.

\subsection{c}

Um das Risiko einer fälschlichen Auslieferung noch kleiner zu machen, soll die Lieferung nur dann freigegeben werden, wenn sich kein defekter Deckel in einer Stichprobe der Länge $n$ befindet. Bestimmen Sie $n$ so, dass dieses Risiko nach der neuen Regel höchstens $1\%$ beträgt.

\section{Aufgabe 2}

\subsection{a}

Die erwartete Jahresrendite eines Aktieninvestments $A$ beträgt $5\%$ und die Standardabwichung der Jahresrendite ebenfalls $5\%$. Weiterhin sei die Aktienrendite normalverteilt. Außerdem gibt es eine von $A$ unabhängige, risikolose Anlagemöglichkeit $R$, die eine Jahresrendite von $2\%$ einbringt. Mit welcher Wahrscheinlichkeit ist die Rendite des Aktieninvestments höher als die der risikolosen Anlage?

\subsection{b}

Ein Gartencenter bietet Gurkensamen an, die zu $95\%$ keimen, Berechen Sie mit Hilfe des Satzes von de Moivre-Laplace näherungsweise die Wahrscheinlichkeit, dass von 600 ausgesätzten samen

\begin{itemize}
    \item höchstens $565$ keimen
    \item weniger als $570$ keimen
    \item mindestens $565$ keimen
    \item $565$ bis $575$ keimen
\end{itemize}