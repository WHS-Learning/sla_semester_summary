\chapter{Übungsblatt 10}

\section{Aufgabe 1}

Einer Handelskette wurde vertraglich zugesichert, dass maximal $1\%$ der Becher
einen Defekten Deckel besitzen. Normalerweise kann dieser Qualitätsstandard
leicht eingehalten werden. Eines Tages stellt sich bei einer
Qualitäts\-kontrolle in der Molkerei heraus, dass $4\%$ der Joghurtbecher einen
Defekten Deckel aufweisen. Bei einem schon beladenen Lkw ist ungewiss, ob die
Joghurtbecher bereits aus der Produktion mit dem erhöhten Anteil an defekten
Deckeln stammen. Deshalb wird der Ladung eine Stichprobe entnommen und
untersucht.

\subsection{a}

Falls bei einer Stichprobe aus 100 Bechern mindestens zwei Becher einen
defekten Deckel haben, wird der Lkw in der Molkerei wieder entladen,
andernfalls wird die Lieferung freigegeben. Wie groß ist die
Wahrscheinlichkeit, dass die Lieferung freigegeben wird, obwohl sie einen
erhöhten Anteil an Joghurtbechern mit defektem Deckel aufweist? Wie groß kann
die Wahrscheinlichkeit für ein unnötiges Entladen des Lkws bei Einhaltung des
zugesicherten Qualitätsstandards maximal werden?

\begin{align*}
    X \sim B(100, p)                                                                                      \\
    X = \text{Anzahl defekter Becher}                                                                     \\
    H_0 = p \leq 0.01 \text{, da maximal 1\% der Becher defekt sein dürfen}                               \\
    H_1 = p > 0.01 \text{, wenn mehr als 2 Becher defekt sind }H_0 \text{ ablehnen}                       \\
    E(X) = 100 \cdot 0.01 = 1                                                                             \\
    Var(X) = 1 \cdot 0.99 = 0.99                                                                          \\
    \sigma = \sqrt{0.99}                                                                                  \\
    P(X < 2) = P\left(\frac{X - E(X)}{\sigma} < \frac{2 - 1}{\sqrt{0.99}}\right)                          \\
    \overset{Stetigkeitskorrektur}{=} P\left(\frac{X - E(X)}{\sigma} < \frac{1.5 - 1}{\sqrt{0.99}}\right) \\
    \Phi\left(\frac{1.5 - 1}{\sqrt{0.99}}\right)                                                          \\
    \approx \Phi(0.50) \overset{Tabelle}{\approx} 0.69146 \approx 69.15 \%                                \\\\
\end{align*}

\subsection{b}

Nun soll die Wahrscheinlichkeit eines unnötigen Ausladens des Lkws auf $5\%$
festgelegtz werden und damit das Konfidenzniveau des beabsichtigten Tests auf
$95\%$. Bestimmen Sie eine Anzahl $z \in \mathbb{N}$ an Joghurtbechern mit
defekten Deckel, ab der die Hypothese, dass die Ladung dem Qualitätsstandard
entspricht, abgelehnt werden soll.

\begin{align*}
    X \sim B(100, 0.01)                                                          \\
    X = \text{Anzahl defekter Becher}                                            \\
    P(X > z) = P\left(\frac{X - E(X)}{\sigma} < \frac{z - 1}{\sqrt{0.99}}\right) \\
    \Phi\left(\frac{z - 1}{\sqrt{0.99}}\right) = 0.95                            \\
    \frac{z - 1}{\sqrt{0.99}} = 1.65 \quad | \cdot \sqrt{0.99} \quad | + 1       \\
    z = 1.65 \cdot \sqrt{0.99} + 1                                               \\
    z \approx 2.64 \Rightarrow z = 3
\end{align*}

\subsection{c}

Um das Risiko einer fälschlichen Auslieferung noch kleiner zu machen, soll die
Lieferung nur dann freigegeben werden, wenn sich kein defekter Deckel in einer
Stichprobe der Länge $n$ befindet. Bestimmen Sie $n$ so, dass dieses Risiko
nach der neuen Regel höchstens $1\%$ beträgt.

\begin{align*}
    X \sim B(n, 0.01)                                                                          \\
    X = \text{Anzahl defekter Becker}                                                          \\
    P(X = 0) = P\left(\frac{X - E(X)}{\sigma} < \frac{0 - n \cdot 0.01}{\sqrt{0.0099n}}\right) \\
    \Phi\left(\frac{0 - 0.01n}{\sqrt{\sqrt{0.0099n}}}\right)                                   \\
    = \Phi\left(\frac{0 - 0.01n}{\sqrt{0.0099n}}\right) = 0.01                                 \\
    = \Phi\left(\frac{0 - 0.01n}{\sqrt{0.0099n}}\right) = 1 - 0.99                             \\
    = \frac{0 - 0.01n}{\sqrt{0.0099n}} = -2.33                                                 \\
    = \frac{-0.01n}{\sqrt{0.0099n}} = -2.33                                                    \\
    = \frac{-0.01n}{0.0994987 \cdot \sqrt{n}} = -2.33                                          \\
    = \frac{-0.01}{0.0994987} \cdot \sqrt{n} = -2.33  \quad | \cdot 0.0994987                  \\
    = -0.01 \cdot \sqrt{n} = -0.231831971 \quad | : -0.01                                      \\
    \sqrt{n} = 23.1831971 \quad | \phantom{}^2                                                 \\
    = 576 = n
\end{align*}

\section{Aufgabe 2}

\subsection{a}

Die erwartete Jahresrendite eines Aktieninvestments $A$ beträgt $5\%$ und die
Standardabwichung der Jahresrendite ebenfalls $5\%$. Weiterhin sei die
Aktienrendite normalverteilt. Außerdem gibt es eine von $A$ unabhängige,
risikolose Anlagemöglichkeit $R$, die eine Jahresrendite von $2\%$ einbringt.
Mit welcher Wahrscheinlichkeit ist die Rendite des Aktieninvestments höher als
die der risikolosen Anlage?

\begin{align*}
    X \sim N(5\%, 5\%^2)                                                 \\
    X = \text{Jahresrendite}
    P(X > 2\%) = 1 - P(X \leq 2\%) =                                     \\
    1 - P\left(\frac{X - E(X)}{\sigma} \leq \frac{2\% - 5\%}{5\%}\right) \\
    = 1 - \Phi\left(\frac{2\% - 5\%}{5\%}\right) \approx 1 - \Phi(-0.6)  \\
    \overset{Tabelle}{\approx} 1 - (1 - 0.72575)                         \\
    = 0.72575 \approx 72.58\%
\end{align*}

\subsection{b}

Ein Gartencenter bietet Gurkensamen an, die zu $95\%$ keimen, Berechen Sie mit
Hilfe des Satzes von de Moivre-Laplace näherungsweise die Wahrscheinlichkeit,
dass von 600 ausgesätzten samen

\begin{itemize}
    \item höchstens $565$ keimen
    \item weniger als $570$ keimen
    \item mindestens $565$ keimen
    \item $565$ bis $575$ keimen
\end{itemize}

\begin{align*}
    X \sim B(600, 0.95)                                                                   \\
    X = \text{Anzahl keimender Gurkensamen}                                               \\
    E(X) = 600 \cdot 0.95 = 570                                                           \\
    Var(X) = 28.5                                                                         \\
    \sigma = \sqrt{28.5}                                                                  \\
    P(X < 565) = P\left(\frac{X - E(X)}{\sigma} < \frac{565 - 570}{\sqrt{28.5}}\right)    \\
    \Phi\left(\frac{565 - 570}{\sqrt{28.5}}\right)                                        \\
    \approx \Phi(-0.94) = 1 - \Phi(0.94) = 0.82639 = 82.64\%                              \\\\
    P(X > 570) = P(X \leq 570)                                                            \\
    = P\left(\frac{X - E(X)}{\sigma} \leq \frac{570 - 570}{\sqrt{28.5}}\right)            \\
    \Phi(0) \overset{Tabelle}{\approx} 0.5 = 50\%                                         \\\\
    P(565 < X < 575) = P(X < 575) - P(X < 565)                                            \\
    P(X < 575) = P\left(\frac{X - E(X)}{\sigma} \leq \frac{575 - 570}{\sqrt{28.5}}\right) \\
    \Phi\left(\frac{575 - 570}{\sqrt{28.5}}\right)                                        \\
    \approx \Phi(0.94) \overset{Tabelle}{\approx} 0.82639 \approx 82.64 \%                \\
    P(X < 565) = P\left(\frac{X - E(X)}{\sigma} < \frac{565 - 570}{\sqrt{28.5}}\right)    \\
    \Phi\left(\frac{565 - 570}{\sqrt{28.5}}\right)                                        \\
    \approx \Phi(-0.94) = 1 - \Phi(0.94)                                                  \\
    \overset{Tabelle}{\approx} 1 - 0.82639 = 0.17361 = 17.36\%                            \\
    P(565 < X < 575) = 0.82639 - 0.17361 = 0.65278 = 65.28\%
\end{align*}