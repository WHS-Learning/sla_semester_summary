\chapter{Übungsblatt 5}

\section{Aufgabe 1}

\subsection{a}
Wie viele Möglichkeiten gibt es, mit der rechten Hand eine zwei zu zeigen (also
zwei Finger ausgestreckt, die andere eingeklappt zu lassen)? (Anatomisch
schwierige Kombinationen und Verrenkungen werden mitgezählt!)

\begin{align*}
    \frac{5!}{2!(5-2)!} \\
    = \frac{5!}{2!3!}   \\
    = \frac{120}{2*6}   \\
    = 10
\end{align*}

Es gibt 10 Möglichkeiten, eine zwei mit einer Hand zu zeigen.

\subsection{b}
Wie viele Möglichkeiten gibt es, irgendeine Zahl zwischen 0 und 5 mit einer
Hand zu zeigen? (Auch hier: missverständliche Handzeichen und Verrenkungen
werden mitgezählt)

\begin{align*}
    \frac{5!}{0!(5-0)!} + \frac{5!}{1!(5-1)!} + \frac{5!}{2!(2-5)!}   \\
    + \frac{5!}{3!(5-3)!} + \frac{5!}{4!(5-4)!} + \frac{5!}{5!(5-5)!} \\
    = 32
\end{align*}

\section{Aufgabe 2}
Eine Freundin hat Ihnen zum Geburtstag sämtliche Buchstaben Ihres Vor- und
Nachnamens aus Beton gegossen. Wie (unterscheidbare) viele Wörter können Sie
damit bilden, ohne Buchstaben wegzulassen? Wie viele davon enthalten Ihren
Vornamen?

Mit Vorname mit 4 unterschiedlichen und Nachname mit 6 unterschiedlichen
Buchstaben (Alle sind unterschiedlich):

\begin{align*}
    (4 + 6)! = 3628800
\end{align*}

\section{Aufgabe 3}
Sie werfen eine (faire) Münze dreimal.

\subsection{a}
Was ist ein passender Ereignisraum $\Omega$?

Ein passender Ereignisraum listet alle 8 möglichen, geordneten Ergebnisse auf.
Wir kürzen Kopf mit K und Zahl mit Z ab.
\begin{align*}
    \Omega = \{
    \text{KKK, }
    \text{KKZ, }
    \text{KZK, }
    \text{KZZ, }
    \text{ZKK, }
    \text{ZKZ, }
    \text{ZZK, }
    \text{ZZZ}
    \}
\end{align*}

\subsection{b}
Wie viele Elemente hat die Potenzmenge $\mathcal{P}(\Omega)$?

Die Mächtigkeit (Anzahl der Elemente) des Ereignisraums beträgt $|\Omega| = 8$.
Die Mächtigkeit der Potenzmenge $\mathcal{P}(\Omega)$ ist $2$ hoch die
Mächtigkeit von $\Omega$.
\begin{align*}
    |\mathcal{P}(\Omega)| = 2^{|\Omega|} = 2^8 = 256
\end{align*}
Die Potenzmenge hat somit 256 Elemente.

\subsection{c}
Wie groß ist die Wahrscheinlichkeit dafür, dass Sie dreimal dasselbe Ergebnis
erziehlen?

\begin{align*}
    \frac{1}{2}^3 + \frac{1}{2}^3 \\
    = \frac{1}{8} + \frac{1}{8}   \\
    = \frac{2}{8} = \frac{1}{4} = 25\%
\end{align*}