\chapter{Übungsblatt 5}

\section{Aufgabe 1}

\subsection{a} 
Wie viele Möglichkeiten gibt es, mit der rechten Hand eine zwei zu zeigen (also zwei Finger ausgestreckt, die andere eingeklappt zu lassen)? (Anatomisch schwierige Kombinationen und Verrenkungen werden mitgezählt!)

\subsection{b}
Wie viele Möglichkeiten gibt es, irgendeine Zahl zwischen 0 und 5 mit einer Hand zu zeigen? (Auch hier: missverständliche Handzeichen und Verrenkungen werden mitgezählt)

\section{Aufgabe 2}
Eine Freundin hat Ihnen zum Geburtstag sämtliche Buchstaben Ihres Vor- und Nachnamens aus Beton gegossen. Wie (unterscheidbare) viele Wörter können Sie damit bilden, ohne Buchstaben wegzulassen? Wie viele davon enthalten Ihren Vornamen?

\section{Aufgabe 3}
Sie werfen eine (faire) Münze dreimal.

\subsection{a}
Was ist ein passender Ereignisraum $\Omega$?

\subsection{b}
Wie viele Elemente hat die Potenzmente $\mathcal{P}(\Omega)$?

\subsection{c}
Wie groß ist die Wahrscheinlichkeit dafür, dass Sie dreimal dasselbe Ergebnis erziehlen?