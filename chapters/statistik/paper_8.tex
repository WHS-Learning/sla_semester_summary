\chapter{Übungsblatt 8}

\section{Aufgabe 1}

Zeigen Sie die folgenden Rechenregeln, die Sie in der Vorlesung kennen gelernt haben. Seien dabei jeweils $X : \Omega \rightarrow \mathbb{R}$ und $Y : \Omega \rightarrow \mathbb{R}$ Zufallsvariablen und $a, b \in \mathbb{R}$.

\subsection{a}

$E(aX + bX) = aE(X) + bE(Y)$

\subsection{b}

$E(X) \leq E(Y), \text{ falls } P(X \leq Y) = 1$

\subsection{c}

$Var(X + Y) = Var(X) + Var(Y), \text{ falls } X \text{ und } Y \text{ stochastisch unabhängig sind}.$

\section{Aufgabe 2}

Sei $X$ die Anzahl der "Zahl"-Würfe beim zehnmaligen Münzwurf. Berechnen Sie $E(Y)$ und $Var(Y)$.

\subsection{a}
von Hand

\subsection{b}
mittels der Formel für Erwartungswwert und Varianz der Binomialverteilung

Sei Y die Wartezeit bis zum ersten "Zahl"-Wurf. Berechnen Sie $E(Y)$ und $Var(Y)$

\section{Aufgabe 3}

Sie kaufen ein Paket mit 300 Schrauben. Laut Herstellerangaben sind $1\%$ der verkauften Schrauben Ausschluss und somit nicht verwendbar. Für ein Bauprojekt benötigen Sie $298$ Schrauben. Bestimmen Sie die Wahrscheinlichkeit dafür, dass die verwendbaren Schrauben ausreichen

\begin{itemize}
    \item exakt mit Hilfe der Binomialverteilung und
    \item näherungsweise an Hand des Satzes von de Moivere-Laplace.
\end{itemize}