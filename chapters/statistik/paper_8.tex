\chapter{Übungsblatt 8}

\section{Aufgabe 1}

Zeigen Sie die folgenden Rechenregeln, die Sie in der Vorlesung kennen gelernt
haben. Seien dabei jeweils $X : \Omega \rightarrow \mathbb{R}$ und $Y : \Omega
    \rightarrow \mathbb{R}$ Zufallsvariablen und $a, b \in \mathbb{R}$.

\subsection{a}

$E(aX + bY) = aE(X) + bE(Y)$

\subsection{b}

$E(X) \leq E(Y), \text{ falls } P(X \leq Y) = 1$

\subsection{c}

$Var(X + Y) = Var(X) + Var(Y), \text{ falls } X \text{ und } Y \text{ stochastisch unabhängig sind}.$

\section{Aufgabe 2}

Sei $X$ die Anzahl der 'Zahl'-Würfe beim zehnmaligen Münzwurf. Berechnen Sie
$E(X)$ und $Var(X)$.

\subsection{a}
von Hand

\subsection{b}
mittels der Formel für Erwartungswert und Varianz der Binomialverteilung

Sei Y die Wartezeit bis zum ersten 'Zahl'-Wurf. Berechnen Sie $E(Y)$ und
$Var(Y)$

\section{Aufgabe 3}

Sie kaufen ein Paket mit 300 Schrauben. Laut Herstellerangaben sind $1\%$ der
verkauften Schrauben Ausschluss und somit nicht verwendbar. Für ein Bauprojekt
benötigen Sie $298$ Schrauben. Bestimmen Sie die Wahrscheinlichkeit dafür, dass
die verwendbaren Schrauben ausreichen

\begin{itemize}
    \item exakt mit Hilfe der Binomialverteilung und
    \item näherungsweise an Hand des Satzes von de Moivere-Laplace.
\end{itemize}

\textbf{Binomialverteilung}

\begin{align*}
    \begin{pmatrix}
        n \\ k
    \end{pmatrix} \cdot p^k \cdot {(1 - p)}^{n - k}                                             \\
    \sum_{i = 298}^{300} \left(\begin{pmatrix}
                                   300 \\ i
                               \end{pmatrix} \cdot 0.99^{i} \cdot {(1 - 0.99)}^{300 - i}\right) \\
    \approx 0.4221                                                                              \\
    = 42.21\%
\end{align*}

Die Schrauben reichen zu 42.21\% aus.

\textbf{Satz von Moivere-Laplace}

\begin{align*}
    X = \text{Anzahl der funktionierenden Schrauben}                 \\
    X \sim B(300, 0.99)                                              \\
    E(X) = 300 \cdot 0.99 = 297                                      \\
    Var(X) = 297 \cdot 0.01 = 2.97                                   \\
    \sigma = \sqrt{2.97}                                             \\
    P(X \geq 298) \overset{Stetigkeitskorrektur}{=} P(X \geq 297.5)  \\
    1 - P(X < 297.5)                                                 \\
    1 - P(\frac{X - E(X)}{\sigma} < \frac{297.5 - 297}{\sqrt{2.97}}) \\
    1 - \phi(\frac{297.5 - 297}{\sqrt{2.97}})                        \\
    1 - \phi(0.29)                                                   \\
    \overset{Tabelle}{\approx} 1 - 0.61409                           \\
    0.38501 = 38.50\%
\end{align*}