\chapter{Moodle Übungsaufgaben}

\section{Aufgabe: Sterbewahrscheinlichkeit}

Die Wahrscheinlichkeit einer zufällig gewählten 65-jährigen Person in der
Bundesrepublik Deutschland, im Laufe der kommenden zwölf Monate zu sterben, ist
$p = 0.01$. Eine kleine Pensionsversicherung hat $n = 400$ Verscicherte dieses
Alters.

Wie viele von ihnen werden in den kommenden zwölf Monaten sterben?

\subsection{a}
Geben Sie zunächst an, welche Verteilung diejenige Zufallsvariable $X$ haben
könnte, die diese Anzahl beschreibt.

Antwort: Binomialverteilung

\subsection{b}

Bitte geben Sie die Wahrscheinlichkeitsfunktion der Zufallsvariable $X$ an.

$P(K) = \begin{pmatrix}
        400 \\ k
    \end{pmatrix}  \cdot 0.01^k \cdot 0.99^{400 - k}$

\subsection{c}

Bestimmen Sie den Erwartungswert von X. Bitte geben Sie einen exakten,
ungerundeten Wert an.

$E(X) = 400 \cdot 0.01$

\subsection{d}

Bestimmen Sie die Varianz von X. Bitte geben Sie einen exakten, ungerundeten
Wert an.

$Var(X) = 400 \cdot 0.01 \cdot 0.99$

\subsection{e}

Wie groß ist die Wahrscheinlichkeit, dass es mindestens 4 Todesfälle in dieser
Altersgruppe gibt?

\begin{align*}
    1 - P(X \leq 3) =                                                                \\
    1 - \sum_{i = 0}^{3}\left(\begin{pmatrix}
                                  400 \\ i
                              \end{pmatrix} \cdot 0.01^i \cdot 0.99^{400 - i}\right) \\
    = 1 - 0.432487956                                                                \\
    = 0.567512044
\end{align*}

\subsection{f}

Wie groß ist die Wahrscheinlichkeit, dass es in der Altersgruppe keinen
Todesfall gibt?

\begin{align*}
    P(X = 0) = \begin{pmatrix}
                   400 \\ 0
               \end{pmatrix} \cdot 0.01^0 \cdot 0.99^{400} \\
    = 0.017950553
\end{align*}

\section{Aufgabe: Nachrichtenkanal}

Über einen Nachrichtenkanal wird ein Zeichen mit einer Wahrscheinlichkeit von $0.96$ korrekt
übertragen. Eine Nachricht besteht aus $75$ Zeichen. Wir bezeichnen die Zahl der falsch
übertragenen Zeichen mit $X$.

\subsection{a}

Welche Wahrscheinlichkeitsverteilung eignet sich am besten zur Modelierung von
$X$?

$Bin(75, 0.04)$

\subsection{b}

Bestimmen Sie den Erwartungswert und die Varianz von $X$.

\begin{align*}
    E(X) = 75 \cdot 0.04 = 3 \\
    Var(X) = 75 \cdot 0.04 \cdot 0.96 = 2.88
\end{align*}

\subsection{c}

Berechnen Sie die Wahrscheinlichkeit $P(X \geq 3)$

\begin{align*}
    \sum_{i = 3}^{75} \left(\begin{pmatrix}
                                75 \\ i
                            \end{pmatrix} \cdot 0.04^i \cdot 0.96^{75 - i}\right) \\
    = 0.581388125
\end{align*}

\section{Aufgabe: Brettspiel}

Beim Brettspiel `Mensch ärgere dich nicht' darf man zu Beginn dreimal würfeln
und kann starten, wenn man dabei mindestens eine 6 würfelt. Ansonsten muss man
warten, bis man das nächste mal an der Reihe ist und dann wieder dreimal
würfeln darf.

\subsection{a}

Wie groß ist die Wahrscheinlichkeit, bei den drei Versuchen zu Beginn
mindestens eine 6 zu erhalten?

\begin{align*}
    P(X \geq 1)                                                 \\
    = P(X = 1) + P(X = 2) + P(X = 3)                            \\
    = 1 - P(X = 0)                                              \\
    = 1 - \begin{pmatrix}
              3 \\ 0
          \end{pmatrix} \cdot \frac{1}{6}^0 \cdot \frac{5}{6}^3 \\
    = 1 - 0.578703704                                           \\
    = 0.421296
\end{align*}

\subsection{b}

Wir bezeichnen mit $X$ Die Anzahl der Runden, die man warten muss, bevor man
starten kann. Bestimmen Sie die Verteilung von $X$

Geometrische Verteilung

\subsection{c}

Bitte geben Sie den Parameter der Verteilung an

0.42??

\subsection{d}

Bestimmen Sie den Erwartungswert von X

$E(X) = \frac{1 - \frac{91}{216}}{\frac{91}{216}} \approx 1.38$

\subsection{e}

Bestimmen Sie die Vaianz von X.

$Var(X) = \frac{1 - \frac{91}{216}}{\frac{91}{216}^2} \approx 3.287981$

\section{Aufgabe: Würfel konstruieren}

Sei $X$ eine Zufallsvariable, die die gewürfelte Augenzahl beim Wurf mit einem
(unfairen) sechsseitigen Würfel beschreibt. Der Erwartungswert dieser
Zufallsvariable soll bei $E(X) = 2.6$ liegen. Geben Sie ein Beispiel für einen
Würfel mit dieser Eigenschaft an, indem Sie die Verteilung von $X$ (d.h. die
Wahrscheinlichkeit $P(X = k)$ für $k = 1, \dots, 6$) angeben. Wählen Sie die
Verteilung so, dass $P(X = k) = 0$ für höchstens $k$ gilt.

\begin{align*}
    P_1 + P_2 + P_3 + P_4 + P_5 + P_6 = 1        \\
    P_1 + 2P_2 + 3P_3 + 4P_4 + 5P_5 + 6P_6 = 2.6 \\
\end{align*}

\begin{longtable}{p{4cm}|p{3cm}}

    \hline
    \multicolumn{1}{c|}{\textbf{Linearkombination}} & \multicolumn{1}{c}{\textbf{Konstanten}} \\
    \hline
    \endfirsthead

    \hline
    \multicolumn{2}{c}{\tablename\ \thetable\ -- \textit{Fortführung von vorherier Seite}}    \\
    \hline
    \multicolumn{1}{c|}{\textbf{Linearkombination}} & \multicolumn{1}{c}{\textbf{Konstanten}} \\
    \hline
    \endhead

    \hline
    \multicolumn{2}{r}{\textit{Fortsetzung siehe nächste Seite}}                              \\
    \endfoot

    \hline
    \endlastfoot

    $\displaystyle\begin{matrix}
                          1 & 1 & 1 & 1 & 1 & 1 \\
                          1 & 2 & 3 & 4 & 5 & 6 \\
                      \end{matrix}$                    &
    $\displaystyle\begin{matrix}
                          1 \\
                          2.6
                      \end{matrix}$                                                               \\\hline
    \multicolumn{2}{p{\dimexpr4cm+3cm+2\tabcolsep\relax}}{Operation: II - I}                  \\\hline\pagebreak[0]
    $\displaystyle\begin{matrix}
                          1 & 1 & 1 & 1 & 1 & 1 \\
                          0 & 1 & 2 & 3 & 4 & 5 \\
                      \end{matrix}$                    &
    $\displaystyle\begin{matrix}
                          1 \\
                          1.6
                      \end{matrix}$                                                               \\\hline
    \multicolumn{2}{p{\dimexpr4cm+3cm+2\tabcolsep\relax}}{Operation: I - II}                  \\\hline\pagebreak[0]
    $\displaystyle\begin{matrix}
                          1 & 0 & -1 & -2 & -3 & -4 \\
                          0 & 1 & 2  & 3  & 4  & 5  \\
                      \end{matrix}$                    &
    $\displaystyle\begin{matrix}
                          -0.6 \\
                          1.6
                      \end{matrix}$                                                               \\\hline
\end{longtable}

\begin{align*}
    P_1 - P_3 - 2P_4 - 3P_5 - 4P_6 = -0.6                                                                 \\
    P_1 = -0.6 + P_3 + 2P_4 + 3P_5 + 4P_6                                                                 \\
    P_2 + 2P_3 + 3P_4 + 4P_5 + 5P_6 = 1.6                                                                 \\
    P_2 = 1.6 - 2P_3 - 3P_4 - 4P_5 - 5P_6                                                                 \\\\
    P_1 > 0                                                                                               \\
    -0.6 + P_3 + 2P_4 + 3P_5 + 4P_6 > 0 \quad | + 0.6                                                     \\
    P_3 + 2P_4 + 3P_5 + 4P_6 > 0.6
    P_2 > 0                                                                                               \\
    1.6 - 2P_3 - 3P_4 - 4P_5 - 5P_6 > 0 \quad | -1.6                                                      \\
    - 2P_3 - 3P_4 - 4P_5 - 5P_6 > -1.6 \quad |\cdot (-1)                                                  \\
    2P_3 + 3P_4 + 4P_5 + 5P_6 < 1.6                                                                       \\
    \text{Wähle: } P_3 = P_4 = P_5 = \frac{1}{10}                                                         \\ \\
    \frac{1}{10} + 2 \cdot \frac{1}{10} + 3 \cdot \frac{1}{10} + 4P_6 > 0.6                               \\
    \frac{3}{5} + 4P_6 > 0.6 \quad | - \frac{3}{5}                                                        \\
    4P_6 > 0 \quad | : 4                                                                                  \\
    P_6 > 0                                                                                               \\\\
    2 \cdot \frac{1}{10} + 3 \cdot \frac{1}{10} + 4 \cdot \frac{1}{10} + 5P_6 < 1.6                       \\
    \frac{9}{10} + 5P_6 < 1.6 \quad | -\frac{9}{10}                                                       \\
    5P_6 < \frac{7}{10} \quad | : 5                                                                       \\
    5P_6 < \frac{7}{10} \quad | : 5                                                                       \\
    P_6 < \frac{7}{50} \quad | : 5                                                                        \\\\
    \text{Wähle} P_6 = \frac{5}{50}                                                                       \\
    P_1 = -0.6 + \frac{1}{10} + 2 \cdot \frac{1}{10} + 3 \cdot \frac{1}{10} + 4 \cdot \frac{5}{50}        \\
    P_1 = \frac{2}{5}                                                                                     \\
    P_2 = 1.6 - 2 \cdot \frac{1}{10} - 3 \cdot \frac{1}{10} - 4 \cdot \frac{1}{10} - 5 \cdot \frac{5}{50} \\
    P_2 = \frac{1}{5}
\end{align*}

\section{Aufgabe: Normalapproximation}

Beim Paketversand kann es bei jedem Paket unabhängig voneinander mit Wahrscheinlichkeit $p=\frac{1}{12}$ zu Lieferverzögerungen kommen. Täglich werden $n=240$ Pakete verschickt. Sei $X$ die zufällige Anzahl der Pakete, die nicht rechtzeitig zugestellt werden können. 

\subsection{a}

Berechnen Sie exakt die Wahrscheinlichkeit, dass es an einem Tag bei mindestens 19, aber gleichzeitig nur höchstens 23 Paketen zu Lieferverzögerungen kommt.

\begin{align*}
    X = \text{Anzahl Verspäteter Pakete} \\
    X \sim B(n, p) \\
    P(19 \leq X \leq 23) = P(X \leq 23) - P(X < 19) \\
    = \sum_{i = 0}^{23}\left(\begin{pmatrix}
        240 \\ i
    \end{pmatrix} \cdot \frac{1}{12}^i \cdot \frac{11}{12}^{240 - i}\right) - \\
    \sum_{i = 0}^{18}\left(\begin{pmatrix}
        240 \\ i
    \end{pmatrix} \cdot \frac{1}{12}^i \cdot \frac{11}{12}^{240 - i}\right) \\
    = 0.422379171
\end{align*}

\subsection{b}

Wir können die gesuchte Wahrscheinlichkeit auch approximativ berechnen, indem wir die Normalapproximation verwenden, welche auf dem zentralen Grenzwertsatz beruht. Berechnen Sie die Wahrscheinlichkeit, dass es an einem Tag bei mindestens 19, aber gleichzeitig nur höchstens 23 Paketen zu Lieferverzögerungen kommt, approximativ mittels einer Normalapproximation ohne Stetigkeitskorrektur. 

\begin{align*}
    X = \text{Anzahl Verspäteter Pakete} \\
    X \sim B(n, p) \\
    E(X) = 240 \cdot \frac{1}{12} = 20 \\
    Var(X) = 240 \cdot \frac{1}{12} \cdot \frac{11}{12} = \frac{55}{3} \\
    \sigma(X) = \sqrt{\frac{55}{3}}
    P(19 \geq X \geq 23) = P(X \leq 23) - P(X < 19) \\
    P(X \leq 23) = P\left(\frac{X - E(X)}{\sigma} \leq \frac{23 - 20}{\sqrt{\frac{55}{3}}}\right) \sim N(0, 1)\\
    \varphi\left(\frac{23 - 20}{\sqrt{\frac{55}{3}}}\right) \\
    \approx \varphi\left(\frac{23 - 20}{\sqrt{\frac{55}{3}}}\right)0.70 \\
    \overset{Tabelle}{\approx} 0.75804 = 75.80\% \\
    P(X < 19) = P\left(\frac{X - E(X)}{\sigma} < \frac{19 - 20}{\sqrt{\frac{55}{3}}}\right) \sim N(0, 1) \\
    \varphi \left(\frac{19 - 20}{\sqrt{\frac{55}{3}}}\right) \\
    \approx \varphi\left(-0.23\right) = \\
    1 - \varphi \left(0.23\right) = 1 - 0.59095 = \\
    0.40905 \approx 40.91\%
    P(19 \geq X \geq 23) = 0.75804 - 0.40905 = 0.34899
\end{align*}
