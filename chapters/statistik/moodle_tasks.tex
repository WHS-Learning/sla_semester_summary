\chapter{Moodle Übungsaufgaben}

\section{Aufgabe: Sterbewahrscheinlichkeit}

Die Wahrscheinlichkeit einer zufällig gewählten 65-jährigen Person in der
Bundesrepublik Deutschland, im Laufe der kommenden zwölf Monate zu sterben, ist
$p = 0.01$. Eine kleine Pensionsversicherung hat $n = 400$ Verscicherte dieses
Alters.

Wie viele von ihnen werden in den kommenden zwölf Monaten sterben?

\subsection{a}
Geben Sie zunächst an, welche Verteilung diejenige Zufallsvariable $X$ haben
könnte, die diese Anzahl beschreibt.

Antwort: Binomialverteilung

\subsection{b}

Bitte geben Sie die Wahrscheinlichkeitsfunktion der Zufallsvariable $X$ an.

$P(K) = \begin{pmatrix}
        400 \\ k
    \end{pmatrix}  \cdot 0.01^k \cdot 0.99^{400 - k}$

\subsection{c}

Bestimmen Sie den Erwartungswert von X. Bitte geben Sie einen exakten,
ungerundeten Wert an.

$E(X) = 400 \cdot 0.01$

\subsection{d}

Bestimmen Sie die Varianz von X. Bitte geben Sie einen exakten, ungerundeten
Wert an.

$Var(X) = 400 \cdot 0.01 \cdot 0.99$

\subsection{e}

Wie groß ist die Wahrscheinlichkeit, dass es mindestens 4 Todesfälle in dieser
Altersgruppe gibt?

\begin{align*}
    1 - P(X \leq 3) =                                                                \\
    1 - \sum_{i = 0}^{3}\left(\begin{pmatrix}
                                  400 \\ i
                              \end{pmatrix} \cdot 0.01^i \cdot 0.99^{400 - i}\right) \\
    = 1 - 0.432487956                                                                \\
    = 0.567512044
\end{align*}

\subsection{f}

Wie groß ist die Wahrscheinlichkeit, dass es in der Altersgruppe keinen
Todesfall gibt?

\begin{align*}
    P(X = 0) = \begin{pmatrix}
                   400 \\ 0
               \end{pmatrix} \cdot 0.01^0 \cdot 0.99^{400} \\
    = 0.017950553
\end{align*}

\section{Aufgabe: Nachrichtenkanal}

Über einen Nachrichtenkanal wird ein Zeichen mit einer Wahrscheinlichkeit von $0.96$ korrekt
übertragen. Eine Nachricht besteht aus $75$ Zeichen. Wir bezeichnen die Zahl der falsch
übertragenen Zeichen mit $X$.

\subsection{a}

Welche Wahrscheinlichkeitsverteilung eignet sich am besten zur Modelierung von
$X$?

$Bin(75, 0.04)$

\subsection{b}

Bestimmen Sie den Erwartungswert und die Varianz von $X$.

\begin{align*}
    E(X) = 75 \cdot 0.04 = 3 \\
    Var(X) = 75 \cdot 0.04 \cdot 0.96 = 2.88
\end{align*}

\subsection{c}

Berechnen Sie die Wahrscheinlichkeit $P(X \geq 3)$

\begin{align*}
    \sum_{i = 3}^{75} \left(\begin{pmatrix}
                                75 \\ i
                            \end{pmatrix} \cdot 0.04^i \cdot 0.96^{75 - i}\right) \\
    = 0.581388125
\end{align*}

\section{Aufgabe: Brettspiel}

Beim Brettspiel `Mensch ärgere dich nicht' darf man zu Beginn dreimal würfeln
und kann starten, wenn man dabei mindestens eine 6 würfelt. Ansonsten muss man
warten, bis man das nächste mal an der Reihe ist und dann wieder dreimal
würfeln darf.

\subsection{a}

Wie groß ist die Wahrscheinlichkeit, bei den drei Versuchen zu Beginn
mindestens eine 6 zu erhalten?

\begin{align*}
    P(X \geq 1)                                                 \\
    = P(X = 1) + P(X = 2) + P(X = 3)                            \\
    = 1 - P(X = 0)                                              \\
    = 1 - \begin{pmatrix}
              3 \\ 0
          \end{pmatrix} \cdot \frac{1}{6}^0 \cdot \frac{5}{6}^3 \\
    = 1 - 0.578703704                                           \\
    = 0.421296
\end{align*}

\subsection{b}

Wir bezeichnen mit $X$ Die Anzahl der Runden, die man warten muss, bevor man
starten kann. Bestimmen Sie die Verteilung von $X$

Geometrische Verteilung

\subsection{c}

Bitte geben Sie den Parameter der Verteilung an

0.42??

\subsection{d}

Bestimmen Sie den Erwartungswert von X

$E(X) = \frac{1 - \frac{91}{216}}{\frac{91}{216}} \approx 1.38$

\subsection{e}

Bestimmen Sie die Vaianz von X.

$Var(X) = \frac{1 - \frac{91}{216}}{\frac{91}{216}^2} \approx 3.287981$

\section{Aufgabe: Würfel konstruieren}

Sei $X$ eine Zufallsvariable, die die gewürfelte Augenzahl beim Wurf mit einem
(unfairen) sechsseitigen Würfel beschreibt. Der Erwartungswert dieser
Zufallsvariable soll bei $E(X) = 2.6$ liegen. Geben Sie ein Beispiel für einen
Würfel mit dieser Eigenschaft an, indem Sie die Verteilung von $X$ (d.h. die
Wahrscheinlichkeit $P(X = k)$ für $k = 1, \dots, 6$) angeben. Wählen Sie die
Verteilung so, dass $P(X = k) = 0$ für höchstens $k$ gilt.

\begin{align*}
    P_1 + P_2 + P_3 + P_4 + P_5 + P_6 = 1        \\
    P_1 + 2P_2 + 3P_3 + 4P_4 + 5P_5 + 6P_6 = 2.6 \\
\end{align*}

\begin{longtable}{p{4cm}|p{3cm}}

    \hline
    \multicolumn{1}{c|}{\textbf{Linearkombination}} & \multicolumn{1}{c}{\textbf{Konstanten}} \\
    \hline
    \endfirsthead

    \hline
    \multicolumn{2}{c}{\tablename\ \thetable\ -- \textit{Fortführung von vorherier Seite}}    \\
    \hline
    \multicolumn{1}{c|}{\textbf{Linearkombination}} & \multicolumn{1}{c}{\textbf{Konstanten}} \\
    \hline
    \endhead

    \hline
    \multicolumn{2}{r}{\textit{Fortsetzung siehe nächste Seite}}                              \\
    \endfoot

    \hline
    \endlastfoot

    $\displaystyle\begin{matrix}
                          1 & 1 & 1 & 1 & 1 & 1 \\
                          1 & 2 & 3 & 4 & 5 & 6 \\
                      \end{matrix}$                    &
    $\displaystyle\begin{matrix}
                          1 \\
                          2.6
                      \end{matrix}$                                                               \\\hline
    \multicolumn{2}{p{\dimexpr4cm+3cm+2\tabcolsep\relax}}{Operation: II - I}                  \\\hline\pagebreak[0]
    $\displaystyle\begin{matrix}
                          1 & 1 & 1 & 1 & 1 & 1 \\
                          0 & 1 & 2 & 3 & 4 & 5 \\
                      \end{matrix}$                    &
    $\displaystyle\begin{matrix}
                          1 \\
                          1.6
                      \end{matrix}$                                                               \\\hline
    \multicolumn{2}{p{\dimexpr4cm+3cm+2\tabcolsep\relax}}{Operation: I - II}                  \\\hline\pagebreak[0]
    $\displaystyle\begin{matrix}
                          1 & 0 & -1 & -2 & -3 & -4 \\
                          0 & 1 & 2  & 3  & 4  & 5  \\
                      \end{matrix}$                    &
    $\displaystyle\begin{matrix}
                          -0.6 \\
                          1.6
                      \end{matrix}$                                                               \\\hline
\end{longtable}

\begin{align*}
    P_1 - P_3 - 2P_4 - 3P_5 - 4P_6 = -0.6                                                                 \\
    P_1 = -0.6 + P_3 + 2P_4 + 3P_5 + 4P_6                                                                 \\
    P_2 + 2P_3 + 3P_4 + 4P_5 + 5P_6 = 1.6                                                                 \\
    P_2 = 1.6 - 2P_3 - 3P_4 - 4P_5 - 5P_6                                                                 \\\\
    P_1 > 0                                                                                               \\
    -0.6 + P_3 + 2P_4 + 3P_5 + 4P_6 > 0 \quad | + 0.6                                                     \\
    P_3 + 2P_4 + 3P_5 + 4P_6 > 0.6
    P_2 > 0                                                                                               \\
    1.6 - 2P_3 - 3P_4 - 4P_5 - 5P_6 > 0 \quad | -1.6                                                      \\
    - 2P_3 - 3P_4 - 4P_5 - 5P_6 > -1.6 \quad |\cdot (-1)                                                  \\
    2P_3 + 3P_4 + 4P_5 + 5P_6 < 1.6                                                                       \\
    \text{Wähle: } P_3 = P_4 = P_5 = \frac{1}{10}                                                         \\ \\
    \frac{1}{10} + 2 \cdot \frac{1}{10} + 3 \cdot \frac{1}{10} + 4P_6 > 0.6                               \\
    \frac{3}{5} + 4P_6 > 0.6 \quad | - \frac{3}{5}                                                        \\
    4P_6 > 0 \quad | : 4                                                                                  \\
    P_6 > 0                                                                                               \\\\
    2 \cdot \frac{1}{10} + 3 \cdot \frac{1}{10} + 4 \cdot \frac{1}{10} + 5P_6 < 1.6                       \\
    \frac{9}{10} + 5P_6 < 1.6 \quad | -\frac{9}{10}                                                       \\
    5P_6 < \frac{7}{10} \quad | : 5                                                                       \\
    5P_6 < \frac{7}{10} \quad | : 5                                                                       \\
    P_6 < \frac{7}{50} \quad | : 5                                                                        \\\\
    \text{Wähle} P_6 = \frac{5}{50}                                                                       \\
    P_1 = -0.6 + \frac{1}{10} + 2 \cdot \frac{1}{10} + 3 \cdot \frac{1}{10} + 4 \cdot \frac{5}{50}        \\
    P_1 = \frac{2}{5}                                                                                     \\
    P_2 = 1.6 - 2 \cdot \frac{1}{10} - 3 \cdot \frac{1}{10} - 4 \cdot \frac{1}{10} - 5 \cdot \frac{5}{50} \\
    P_2 = \frac{1}{5}
\end{align*}

\section{Aufgabe: Normalapproximation}

Beim Paketversand kann es bei jedem Paket unabhängig voneinander mit
Wahrscheinlichkeit $p=\frac{1}{12}$ zu Lieferverzögerungen kommen. Täglich
werden $n=240$ Pakete verschickt. Sei $X$ die zufällige Anzahl der Pakete, die
nicht rechtzeitig zugestellt werden können.

\subsection{a}

Berechnen Sie exakt die Wahrscheinlichkeit, dass es an einem Tag bei mindestens
19, aber gleichzeitig nur höchstens 23 Paketen zu Lieferverzögerungen kommt.

\begin{align*}
    X = \text{Anzahl Verspäteter Pakete}                                                               \\
    X \sim B(n, p)                                                                                     \\
    P(19 \leq X \leq 23) = P(X \leq 23) - P(X < 19)                                                    \\
    = \sum_{i = 0}^{23}\left(\begin{pmatrix}
                                 240 \\ i
                             \end{pmatrix} \cdot \frac{1}{12}^i \cdot \frac{11}{12}^{240 - i}\right) - \\
    \sum_{i = 0}^{18}\left(\begin{pmatrix}
                               240 \\ i
                           \end{pmatrix} \cdot \frac{1}{12}^i \cdot \frac{11}{12}^{240 - i}\right)     \\
    = 0.422379171
\end{align*}

\subsection{b}

Wir können die gesuchte Wahrscheinlichkeit auch approximativ berechnen, indem
wir die Normalapproximation verwenden, welche auf dem zentralen Grenzwertsatz
beruht. Berechnen Sie die Wahrscheinlichkeit, dass es an einem Tag bei
mindestens 19, aber gleichzeitig nur höchstens 23 Paketen zu
Lieferverzögerungen kommt, approximativ mittels einer Normalapproximation ohne
Stetigkeitskorrektur.

\begin{align*}
    X = \text{Anzahl Verspäteter Pakete}                                                          \\
    X \sim B(n, p)                                                                                \\
    E(X) = 240 \cdot \frac{1}{12} = 20                                                            \\
    Var(X) = 240 \cdot \frac{1}{12} \cdot \frac{11}{12} = \frac{55}{3}                            \\
    \sigma(X) = \sqrt{\frac{55}{3}}                                                               \\
    P(19 \geq X \geq 23) = P(X \leq 23) - P(X \leq 18)                                            \\\\
    P(X \leq 23) = P\left(\frac{X - E(X)}{\sigma} \leq \frac{23 - 20}{\sqrt{\frac{55}{3}}}\right) \\
    \varphi\left(\frac{23 - 20}{\sqrt{\frac{55}{3}}}\right)                                       \\
    \approx \varphi\left(\frac{23 - 20}{\sqrt{\frac{55}{3}}}\right) = \varphi(0.70)               \\
    \overset{Tabelle}{\approx} = 0.75804 = 75.80\%                                                \\\\
    P(X \leq 18) = P\left(\frac{X - E(X)}{\sigma} < \frac{18 - 20}{\sqrt{\frac{55}{3}}}\right)    \\
    \varphi \left(\frac{18 - 20}{\sqrt{\frac{55}{3}}}\right)                                      \\
    \approx \varphi\left(-0.47\right) =                                                           \\
    1 - \varphi \left(0.47\right) = 1 - 0.68082 =                                                 \\
    0.31918 \approx 31.92\%                                                                       \\
    P(19 \geq X \geq 23)                                                                          \\
    = 0.75804 - 0.31918                                                                           \\
    = 0.43886 = 43.89\%
\end{align*}

\subsection{c}

Bei Zufallsvariablen mit Werten in den ganzen Zahlen kann die
Normalapproximation um eine sogenannte Stetigkeitskorrektur erweitert werden.
Dabei werden die (ganzzahligen) Intervallgrenzen um $0.5$ erhöht bzw. um $0.5$
verringert, sodass die mit der Binomialverteilung exakt berechnete
Wahrscheinlichkeit unverändert bleibt, aber bei der Approximation durch die
Normalverteilung sich eine oftmals bessere Näherung an den exakten Wert ergibt.

Berechnen Sie die Wahrscheinlichkeit, dass es an einem Tag bei mindestens $19$,
aber gleichzeitig nur höchstens $23$ Paketen zu Lieferverzögerungen kommt,
approximativ mittels einer Normalapproximation mit Stetigkeitskorrektur.

\begin{align*}
    X = \text{Anzahl Verspäteter Pakete}                                                            \\
    X \sim B(n, p)                                                                                  \\
    E(X) = 240 \cdot \frac{1}{12} = 20                                                              \\
    Var(X) = 240 \cdot \frac{1}{12} \cdot \frac{11}{12} = \frac{55}{3}                              \\
    \sigma(X) = \sqrt{\frac{55}{3}}                                                                 \\
    P(19 \geq X \geq 23) = P(X \leq 23) - P(X \leq 18)                                              \\\\
    P(23 \leq X) = P\left(\frac{X - E(X)}{\sigma} \leq \frac{23 - 20}{\sqrt{\frac{55}{3}}}\right)   \\
    \text{Stetigkeitskorrektur }                                                                    \\
    P(23 \leq X) = P\left(\frac{X - E(X)}{\sigma} \leq \frac{23.5 - 20}{\sqrt{\frac{55}{3}}}\right) \\
    \varphi\left(\frac{23.5 - 20}{\sqrt{\frac{55}{3}}}\right)                                       \\
    \approx \varphi(0.82) \overset{Tabelle}{\approx} 0.79389                                        \\\\
    P(18 \leq X) = P\left(\frac{X - E(X)}{\sigma} \leq \frac{18 - 20}{\sqrt{\frac{55}{3}}}\right)   \\
    \text{Stetigkeitskorrektur}                                                                     \\
    P(18 \leq X) = P\left(\frac{X - E(X)}{\sigma} \leq \frac{18.5 - 20}{\sqrt{\frac{55}{3}}}\right) \\
    \varphi\left(\frac{18.5 - 20}{\sqrt{\frac{55}{3}}}\right)                                       \\
    \approx \varphi\left(-0.35\right)                                                               \\
    = 1 - \varphi\left(0.35\right) \overset{Tabelle}{\approx} 1 - 0.63683                           \\
    = 0.36317                                                                                       \\\\
    P(19 \geq X \geq 23) = 0.79389 - 0.36317                                                        \\
    = 0.43072
\end{align*}

\section{Aufgabe: Sigma-Bereiche}

\subsection{a}

Bei einer einzeldosierten Arzneiform mit einem Sollgewicht von $\mu = 50mg$
gibt der Hersteller als Standardabweichung den Wert $\sigma = 1.5mg$ an. Zur
Qualitätskontrolle einer produzierten Charge werden $20$ ausgewählten Tabletten
die Gewichte $x_1, \dots, x_{20}$ mittels einer Analysewaage gewogen und das
arithmetische Mittel $\overline{x}_{20} = \frac{1}{20} \sum_{i = 1}^{20} x_i$
als Prüfgröße herangezogen. Es wird angenommen, dass die gemessenen Gewichte
Realisierungen von unabhängigen und identischen $N(50, 2.25)$-verteilten
Zufallsvariablen $X_1, \dots, X_{20}$ sind.

Berechnen Sie die Wahrscheinlichkeit, dass das zufällige Gewicht $X_1$ einer
einzelnen Tablette um höchstens eine Standardabweichung von $\sigma$ vom
Sollgewicht $\mu$ abweicht.

\begin{align*}
    P\left(\left|X_1 - \mu\right| \leq \sigma\right) = P\left(\left|X_1 - 50\right| \leq 1.5\right) \\
    -1.5 \leq X_1 - 50 \leq 1.5 \quad | + 50                                                        \\
    \Leftrightarrow -1.5 + 50 \leq X_1 \leq 1.5 + 50                                                \\
    \Leftrightarrow 48.5 \leq X_1 \leq 51.5                                                         \\
    X = \text{Gewicht einer Tablette in mg}                                                         \\
    P\left(48.5 \leq X_1 \leq 51.5\right) = P(X_1 \leq 51.5) - P(X_1 \leq 47.5)                     \\\\
    P(X_1 \leq 51.5) = P\left(\frac{X - E(X)}{\sigma} \leq \frac{51.5 - 50}{1.5}\right)             \\
    \varphi\left(\frac{51.5 - 50}{1.5}\right)                                                       \\
    = \varphi(1) \overset{Tabelle}{\approx} 0.84134                                                 \\\\
    P(X_1 \leq 48.5) = P\left(\frac{X - E(X)}{\sigma} \leq \frac{48.5 - 50}{1.5}\right)             \\
    \varphi(-1) = 1 - \varphi(1) = 1 - 0.84134 = 0.15866                                            \\
    P(48.5 \leq X_1 \leq 51.5)                                                                      \\
    = 0.84134 - 0.15866                                                                             \\
    = 0.68268 = 68.27\%
\end{align*}

\subsection{b}

Als Prüfsumme bei der Qualitätskontrolle wird nicht das Gewicht einer einzelnen
Tablette, sondern das mittlere Gewicht von $20$ Tabletten ermittel. Man kann
zeigen, dass die Zufallsvariable $\overline{X}_{20}$, die das arithmetische
Mittel der Gewichte beschreibt, wieder eine Normalverteilung besitzt. Berechen
Sie die Parameter dieser Normalverteilung, also dem Erwartungswert
$\overline{\mu} = E(\overline{X}_{20})$ und die Varianz $\overline{\sigma}^2 =
    Var(\overline{X}_{20})$.

\begin{align*}
    \overline{\sigma}^2 = Var(\overline{X}_{20})                                  \\
    = Var(\frac{1}{20} \sum_{i = 1}^{20}X_i)                                      \\
    = \frac{1}{20^2} \sum_{i = 1}^{20} \sigma^2                                   \\
    = \frac{1}{20} \sigma^2                                                       \\
    = \frac{\sigma^2}{20}                                                         \\
    = \frac{1.5}{20} = 0.1125 = \overline{\sigma}^2                               \\
    \overline{\mu} = E(\overline{X}_{20})                                         \\\\
    = E\left(\frac{1}{20}\sum_{i = 1}^{20}X_i\right)                              \\
    = \frac{1}{20} \sum_{i = 1}^{20} E(X_i)                                       \\
    = \frac{1}{20} \sum_{i = 1}^{20} \mu                                          \\
    = \mu                                                                         \\
    = 50 = \overline{\mu}                                                         \\\\
    \overline{X}_{20} \sim N(\overline{\mu}, \overline{\sigma}^2) = N(50, 0.1125) \\
\end{align*}

\subsection{c}

Es gibt $\overline{X}_{20} \sim N(50, 0.1125)$. Berechnen Sie die
Wahrscheinlichkeit, dass das arithmetische Mittel $\overline{X}_{20}$ um
höchstens eine Standardabweichung $\overline{\sigma}$ von Mittelwert
$\overline{\mu}$ abweicht.

\begin{align*}
    P\left(\left|\overline{X}_{20} - \overline{\mu}\right| \leq \overline{\sigma}\right)
    = P\left(\left|\overline{X}_{20} - 50\right| \leq \frac{0.75}{\sqrt{5}}\right)                                  \\
    -\frac{0.75}{\sqrt{5}} \leq X_{20} - 50 \leq \frac{0.75}{\sqrt{5}} \quad | + 50                                 \\
    -\frac{0.75}{\sqrt{5}} + 50 \leq X_{20} \leq \frac{0.75}{\sqrt{5}} + 50                                         \\
    \frac{1000 - 3\sqrt{5}}{20} \leq X_{20} \leq \frac{1000 + 3\sqrt{5}}{20}                                        \\
    P\left(\frac{1000 - 3\sqrt{5}}{20} \leq X_{20} \leq \frac{1000 + 3\sqrt{5}}{20}\right)                          \\
    = P\left(\frac{1000 + 3\sqrt{5}}{20} \leq X_{20}\right) - P\left(\frac{1000 - 3\sqrt{5}}{20} \leq X_{20}\right) \\
    P\left(\frac{1000 + 3\sqrt{5}}{20} \leq X_{20}\right)                                                           \\
    =P\left(\frac{X - E(X)}{\sigma} \leq \frac{\frac{1000 + 3\sqrt{5}}{20} - 50}{\sqrt{0.1125}}\right)              \\
    \varphi\left(\frac{\frac{1000 + 3\sqrt{5}}{20} - 50}{\sqrt{0.1125}}\right) \approx \varphi\left(1\right)        \\
    \overset{Tabelle}{\approx} 0.84134                                                                              \\\\
    P\left(\frac{1000 - 3\sqrt{5}}{20} \leq X_{20}\right)                                                           \\
    = P\left(\frac{X - E(X)}{\sigma} \leq \frac{\frac{1000 - 3\sqrt{5}}{20} - 50}{\sqrt{0.1125}}\right)             \\
    \varphi\left(\frac{\frac{1000 - 3\sqrt{5}}{20} - 50}{\sqrt{0.1125}}\right) = \varphi(-1)                        \\
    = 1 - \varphi(1) = 1 - 0.84134 = 0.15866                                                                        \\
    P\left(\frac{1000 - 3\sqrt{5}}{20} \leq X_{20} \leq \frac{1000 + 3\sqrt{5}}{20}\right)                          \\
    = 0.84134 - 0.15866                                                                                             \\
    = 0.68268 = 68.27\%
\end{align*}

\section{Aufgabe: Passagiere eines Transatlantikflugs}

Die Lufthansa Weiß aus Erfahrung, dass Passagiere mit einer Wahrscheinlichkeit
von $90\%$ zum Flug erschienen. Für einen Transatlantikflug hat sie 900 Tickets
verkauft. Wir bezeichnen mit $X$ die Anzahl an Passagiere, die zum Flug
erscheinen.

Bestimmen Sie näherungsweise die Wahrscheinlichkeit, dass nicht mehr als $819$
Passagiere zum Abflug erscheinen. Verwenden Sie dazu die Normalapproximation
und geben Sie das Ergebnis auf mindestens zwei Nachkommastellen genau an.

\begin{align*}
    X = \text{Anzahl passagiere die erscheinen}                                             \\
    B \sim (900, 0.9)                                                                       \\
    E(X) = 900 \cdot 0.9 = 810                                                              \\
    Var(X) = 810 \cdot 0.1 = 81                                                             \\
    \sigma = \sqrt{81} = 9                                                                  \\
    P(X \leq 819) = P\left(\frac{X - E(X)}{\sigma} \leq \frac{819 - 810}{9}\right)          \\
    \varphi\left(\frac{819 - 810}{9}\right) = \varphi(1) \overset{Tabelle}{\approx} 0.84134 \\\\
    Z = \frac{X - 810}{9} \\
    \left\{X \leq 819\right\} = \frac{819 \leq 810}{9} \\
    \frac{819}{9} \leq \frac{810}{9} \quad | \cdot 9\\
    819 \leq 810 \quad | \cdot 9\\
\end{align*}