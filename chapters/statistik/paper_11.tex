\chapter{Übungsblatt 11}

\section{Aufgabe 1}

\textit{Hinweis: Bisher haben wir z-Tests in der Regel auf binomialverteilte Zufallsvariablen
    angewandt, indem wir die zu Grunde liegende Binomialverteilung mittels des Satzes von de
    Moivre-Laplace mit einer Normalverteilung approximiert haben. Natürlich kann man diese Art
    von Test aber auch direkt auf eine normalverteilte Zufallsvariable anwenden. Das soll im
    Folgenden geübt werden.}

Das Gewicht von Brötchen einer Bäckerei (gemessen in Gramm) ist zufallsabhängig und wird durch eine normalverteilte Zufallsvariable $X \sim N(\mu, \sigma^2)$ beschrieben, deren Varianz $\sigma^2 = 49g^2$ bekannt sei.

Für $85$ (zufällig und unabhängig) ausgewählte Brötchen ergibt sich ein Durchschnittsgewicht (=arithmetisches Mittel) von  $\overline{x}_85 = 37.5g$.

\subsection{a}
Geben Sie ein Intervall an, in dem der Stichprobenmittelwert mit einer Wahrscheinlichkeit von $95\%$ enthalten ist. \textit{(Hinweis: Die Summe von $85$ normalverteilten Zufallsvariablen ist selbst auch normalverteilt. Was sind ihr Erwartungswert und ihre Standardabweichung?)}

\subsection{b}

Der Bäcker behauptet, dass \sout{der} das durchschnittliche Brötchengewicht genau $38g$ betrage. Prüfen Sie nun diese Hypothese an Hand der vorgegebenen Stichprobe zum Signifikanzniveau $\alpha = 0,05$.

\subsection{c}

Der Bäcker behauptet, dass seine Brötchen im Durchschnitt mindestens $38g$ wiegen. Prüfen Sie diese Hypothese an Hand der vorgegebenen Stichprobe zum Signifikanzniveau $\alpha = 0,05$.