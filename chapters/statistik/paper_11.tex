\chapter{Übungsblatt 11}

\section{Aufgabe 1}

\textit{Hinweis: Bisher haben wir z-Tests in der Regel auf binomialverteilte Zufallsvariablen
    angewandt, indem wir die zu Grunde liegende Binomialverteilung mittels des Satzes von de
    Moivre-Laplace mit einer Normalverteilung approximiert haben. Natürlich kann man diese Art
    von Test aber auch direkt auf eine normalverteilte Zufallsvariable anwenden. Das soll im
    Folgenden geübt werden.}

Das Gewicht von Brötchen einer Bäckerei (gemessen in Gramm) ist zufallsabhängig
und wird durch eine normalverteilte Zufallsvariable $X \sim N(\mu, \sigma^2)$
beschrieben, deren Varianz $\sigma^2 = 49g^2$ bekannt sei.

Für $85$ (zufällig und unabhängig) ausgewählte Brötchen ergibt sich ein
Durchschnittsgewicht (=arithmetisches Mittel) von $\overline{x}_{85} = 37.5g$.

\subsection{a}
Geben Sie ein Intervall an, in dem der Stichprobenmittelwert mit einer
Wahrscheinlichkeit von $95\%$ enthalten ist. \textit{(Hinweis: Die Summe von
    $85$ normalverteilten Zufallsvariablen ist selbst auch normalverteilt. Was sind
    ihr Erwartungswert und ihre Standardabweichung?)}

\begin{align*}
    KI = \left[\overline{x} - z_{1 - \frac{\alpha}{2}} \cdot \frac{\sigma}{\sqrt{n}}, \quad \overline{x} + z_{1 - \frac{\alpha}{2}} \cdot \frac{\sigma}{\sqrt{n}}\right] \\
    z_{0.975} = 1.96                                                                                                                                                     \\
    \frac{\sigma}{\sqrt{n}} = \frac{7}{\sqrt{85}}                                                                                                                        \\
    \textbf{Untere grenze}                                                                                                                                               \\
    37.5 - 1.96 \cdot \frac{7}{\sqrt{85}} \approx 36.0119                                                                                                                \\
    \textbf{Obere grenze}                                                                                                                                                \\
    37.5 + 1.96 \cdot \frac{7}{\sqrt{85}} \approx 38.9881                                                                                                                \\
    \Rightarrow [36.0119, 38.9881]
\end{align*}

\subsection{b}

Der Bäcker behauptet, dass \sout{der} das durchschnittliche Brötchengewicht
genau $38g$ betrage. Prüfen Sie nun diese Hypothese an Hand der vorgegebenen
Stichprobe zum Signifikanzniveau $\alpha = 0,05$.

\textit{Wir wissen nicht, welche der beiden Lösungsansätze der richtige ist :'(}

\begin{align*}
    KI = \left[\overline{x} - z_{1 - \frac{\alpha}{2}} \cdot \frac{\sigma}{\sqrt{n}}, \quad \overline{x} + z_{1 - \frac{\alpha}{2}} \cdot \frac{\sigma}{\sqrt{n}}\right] \\
    z_{0.975} = 1.96                                                                                                                                                     \\
    \frac{\sigma}{\sqrt{n}} = \frac{7}{\sqrt{85}}                                                                                                                        \\
    \textbf{Untere grenze}                                                                                                                                               \\
    38 - 1.96 \cdot \frac{7}{\sqrt{85}} \approx 36.5119                                                                                                                  \\
    \textbf{Obere grenze}                                                                                                                                                \\
    38 + 1.96 \cdot \frac{7}{\sqrt{85}} \approx 39.4881                                                                                                                  \\
    \Rightarrow [36.5119, 39.4881]                                                                                                                                       \\
    \text{37.5 ist enthalten, also wird die Nullhypothese nicht abgelehnt.}                                                                                              \\
\end{align*}

\textbf{Oder}

\textbf{Da 38 in dem aus a berechneten Intervall liegt, ist die Nullhypothese nicht abzulehnen.}

\subsection{c}

Der Bäcker behauptet, dass seine Brötchen im Durchschnitt mindestens $38g$
wiegen. Prüfen Sie diese Hypothese an Hand der vorgegebenen Stichprobe zum
Signifikanzniveau $\alpha = 0,05$.

\begin{align*}
    H_0 = E(X) \geq 38                                                                          \\
    H_1 = E(X) < 38                                                                             \\
    X \sim N(38, 49)                                                                            \\
    P(X \geq z | \mu \geq 38) \geq 0.95                                                         \\
    1 - P\left(\frac{X - E(X)}{\sigma} \leq \frac{z - 38}{\frac{7}{\sqrt{85}}}\right) \geq 0.95 \\
    z_{-0.95} = -1.65                                                                           \\
    38 - 1.65 \cdot \frac{7}{\sqrt{85}} \approx 36.75                                           \\
    \text{Die Nullhypothese ist abzulehnen, falls } X < 36.75                                   \\
    \text{Da } 37.5 \geq 36.75 \text{ ist, ist die Nullhypothese nicht abzulehnen}
\end{align*}