\chapter{Selfmade Prüfung 3}

\section{Lineare Abbildungen}

\subsection{Aufgabe 1}

Erstelle einen Orthogonalen Vektor zu\dots

\begin{align*}
    A = \begin{pmatrix}
            2 \\ 8 \\ 6
        \end{pmatrix} \quad B = \begin{pmatrix}
                                    5 \\ 9 \\ 20
                                \end{pmatrix}
\end{align*}

\begin{align*}
    A \times B = \begin{pmatrix}
                     8 \cdot 20 - 6 \cdot 9 \\
                     6 \cdot 5 - 2 \cdot 20 \\
                     2 \cdot 9 - 8 \cdot 5
                 \end{pmatrix} = \begin{pmatrix}
                                     106 \\ -10 \\ -22
                                 \end{pmatrix}
\end{align*}

\subsection{Aufgabe 2}

Ist $C$ Orthogonal zu $D$?

\begin{align*}
    C = \begin{pmatrix}
            1 \\ 3
        \end{pmatrix} \quad D = \begin{pmatrix}
                                    3 \\ 4
                                \end{pmatrix}
\end{align*}

\begin{align*}
    \left<C, D\right> \\
    \left< \begin{pmatrix}
               1 \\ 3
           \end{pmatrix}, \begin{pmatrix}
                              3 \\ 4
                          \end{pmatrix}\right> = 1 \cdot 3 + 3 \cdot 4 = 15
\end{align*}

\subsection{Aufgabe 3}

Invertiere Matrix $A = \begin{pmatrix}
        9 & 7 & 3 \\
        9 & 1 & 2 \\
        8 & 8 & 8
    \end{pmatrix}$

\subsection{Aufgabe 4}

Bestimme $Kern(B)$ und $\dim(Bild(B))$ von $B = \begin{pmatrix}
        4 & 11 \\
        7 & 10
    \end{pmatrix}$

\begin{align*}
    \begin{cases}
        \text{I:\@}  & 4x_1 + 11x_2 = 0 \\
        \text{II:\@} & 7x_1 + 10x_2 = 0 \\
    \end{cases}
\end{align*}

\begin{longtable}{p{10cm}}
    \hline
    \multicolumn{1}{c}{\textbf{Linearkombination}}                                         \\
    \hline
    \endfirsthead

    \hline
    \multicolumn{1}{c}{\tablename\ \thetable\ -- \textit{Fortführung von vorherier Seite}} \\
    \hline
    \multicolumn{1}{c}{\textbf{Linearkombination}}                                         \\
    \hline
    \endhead

    \hline
    \multicolumn{1}{r}{\textit{Fortsetzung siehe nächste Seite}}                           \\
    \endfoot

    \hline
    \endlastfoot

    $\displaystyle\begin{matrix}
                          4 & 11 \\
                          7 & 10 \\
                      \end{matrix}$                                                            \\\hline
    4II - 7I                                                                               \\\hline\pagebreak[0]

    $\displaystyle\begin{matrix}
                          4 & 11  \\
                          0 & -37 \\
                      \end{matrix}$                                                            \\\hline
    II : -37                                                                               \\\hline\pagebreak[0]

    $\displaystyle\begin{matrix}
                          4 & 11 \\
                          0 & 1  \\
                      \end{matrix}$                                                            \\\hline
    I - 11II                                                                               \\\hline\pagebreak[0]

    $\displaystyle\begin{matrix}
                          4 & 0 \\
                          0 & 1 \\
                      \end{matrix}$                                                            \\\hline
    I : 4                                                                                  \\\hline\pagebreak[0]

    $\displaystyle\begin{matrix}
                          1 & 0 \\
                          0 & 1 \\
                      \end{matrix}$                                                            \\\hline
    \\\hline\pagebreak[0]

\end{longtable}

\begin{align*}
    \begin{cases}
        \text{I:\@}  & x_1 = 0 \\
        \text{II:\@} & x_2 = 0
    \end{cases}              \\
    span = \left\{\begin{pmatrix}
                      0 \\ 0
                  \end{pmatrix}\right\} \\
    \dim(Bild(B)) = 2 - \dim(Kern(B)) = 2 - 0 = 2
\end{align*}

\section{Eigenwerte und Eigenvektoren}

Bestimme Eigenwerte und Eigenvektoren von $A = \begin{pmatrix}
        2 & 13 \\
        8 & 7
    \end{pmatrix}$.

\begin{align*}
    A - \lambda I =
    \begin{pmatrix}
        2 - \lambda & 13          \\
        8           & 7 - \lambda
    \end{pmatrix}                                                    \\
    \det(A - \lambda I) = (2 - \lambda) (7 - \lambda) - 8 \cdot 13               \\
    = 14 - 2\lambda -7\lambda +\lambda^2 - 104                                   \\
    \lambda^2 - 9\lambda -90                                                     \\
    \det(A - \lambda I) = 0                                                      \\
    \lambda^2 - 9\lambda -90 = 0 \quad | PQ                                      \\
    \lambda_{1, 2} = -\frac{-9}{2} \pm \sqrt{{\left(\frac{-9}{2}\right)}^2 + 90} \\
    \lambda_1 = 15 \quad \lambda_2 = -6                                          \\\\
    \text{Eigenvektoren für } \lambda_1 = 15                                     \\
    A - \lambda I \cdot \vec{x} = 0                                              \\
    \begin{cases}
        \text{I:\@}  & -13x_1 + 13x_2 = 0 \\
        \text{II:\@} & 8x_1 + -8x_2 = 0   \\
    \end{cases}
\end{align*}

\begin{longtable}{p{10cm}}
    \hline
    \multicolumn{1}{c}{\textbf{Linearkombination}}                                         \\
    \hline
    \endfirsthead

    \hline
    \multicolumn{1}{c}{\tablename\ \thetable\ -- \textit{Fortführung von vorherier Seite}} \\
    \hline
    \multicolumn{1}{c}{\textbf{Linearkombination}}                                         \\
    \hline
    \endhead

    \hline
    \multicolumn{1}{r}{\textit{Fortsetzung siehe nächste Seite}}                           \\
    \endfoot

    \hline
    \endlastfoot

    $\displaystyle\begin{matrix}
                          -13 & 13 \\
                          8   & -8 \\
                      \end{matrix}$                                                            \\\hline
    -13II - 8I                                                                             \\\hline\pagebreak[0]

    $\displaystyle\begin{matrix}
                          -13 & 13 \\
                          0   & 0  \\
                      \end{matrix}$                                                            \\\hline
    I : -13                                                                                \\\hline\pagebreak[0]

    $\displaystyle\begin{matrix}
                          1 & -1 \\
                          0 & 0  \\
                      \end{matrix}$                                                            \\\hline
    \\\hline\pagebreak[0]

\end{longtable}

\begin{align*}
    \begin{cases}
        \text{I:\@} & x_1 - x_2 = 0 \Leftrightarrow x_1 = x_2 \\
        \text{I:\@} & 0 = 0                                   \\
    \end{cases} \\
    x_2 = t | t \in \mathbb{R}                            \\
    \begin{pmatrix}
        t \\t
    \end{pmatrix}                                        \\
    t \cdot \begin{pmatrix}
                1 \\1
            \end{pmatrix}                                \\
    span = \left\{
    \begin{pmatrix}
        1 \\ 1
    \end{pmatrix}
    \right\}                                              \\\\
\end{align*}

\begin{align*}
    \text{Eigenvektoren für} \lambda_2 = -6 \\
    \begin{cases}
        \text{I:\@}  & 8x_1 + 13x_2 = 0 \\
        \text{II:\@} & 8x_1 + 13x_2 = 0
    \end{cases}         \\
\end{align*}

\begin{longtable}{p{10cm}}
    \hline
    \multicolumn{1}{c}{\textbf{Linearkombination}}                                         \\
    \hline
    \endfirsthead

    \hline
    \multicolumn{1}{c}{\tablename\ \thetable\ -- \textit{Fortführung von vorherier Seite}} \\
    \hline
    \multicolumn{1}{c}{\textbf{Linearkombination}}                                         \\
    \hline
    \endhead

    \hline
    \multicolumn{1}{r}{\textit{Fortsetzung siehe nächste Seite}}                           \\
    \endfoot

    \hline
    \endlastfoot

    $\displaystyle\begin{matrix}
                          8 & 13 \\
                          8 & 13 \\
                      \end{matrix}$                                                            \\\hline
    II - I                                                                                 \\\hline\pagebreak[0]

    $\displaystyle\begin{matrix}
                          8 & 13 \\
                          0 & 0  \\
                      \end{matrix}$                                                            \\\hline
    I : 8                                                                                  \\\hline\pagebreak[0]

    $\displaystyle\begin{matrix}
                          1 & \frac{13}{8} \\
                          0 & 0            \\
                      \end{matrix}$                                                         \\\hline
    \\\hline\pagebreak[0]

\end{longtable}

\begin{align*}
    \begin{cases}
        \text{I:\@}  & x_1 + \frac{13}{8}x_2 = 0 \Leftrightarrow x_1 = -\frac{13}{8}x_2 \\
        \text{II:\@} & 0 = 0
    \end{cases} \\
    x_2 = t | t \in \mathbb{R}                                                      \\
    \begin{pmatrix}
        -\frac{13}{8}t \\
        t
    \end{pmatrix}                                                                  \\
    t \cdot \begin{pmatrix}
                -\frac{13}{8} \\ 1
            \end{pmatrix}                                                       \\
    span = \left\{\begin{pmatrix}
                      -\frac{13}{8} \\ 1
                  \end{pmatrix}\right\}
\end{align*}

\section{Zufallsvariablen und Verteilungen}

Auf einer Party wird ein Spiel gespielt, wobei man ein Ziel treffen muss mit
einem Ball auf 5m entfernung. Aus erfahrung weißt du, dass du eine
Zielgenauigkeit hast von 20\%. Wann triffst du das Ziel?

\begin{align*}
    X \sim Geom(0.2)                        \\
    X = \text{Würfe bis zum ersten Treffer} \\
    E(X) = \frac{1}{p} = 5
\end{align*}

\section{Normalverteilung und z-Test}

Bei einer Runde russisch roulette wurden 2 Kugeln reingetan und 8 Plätze leer
gelassen. Du bist an der Reihe und kannst für 2 Punkte auf dein gegenüber
schießen (gibt nur punkte, wenn Kugel schießt) Oder 2 Punkte wenn du dich
selbst auswählst. Da es um dein leben geht, möchtest du mindestens $70\%$
sicher sein, dass ein leerer Platz ist bevor du dich selbst auswählst.

Skibidi Toilet

\section{Quiz}

Kann man jede Matrix invertieren? Wie könnte man beweisen ob sie Intervierbar
ist oder nicht?

Nein, eine Matrix mit einer det ungleich null ist invertierbar. gleich null
nicht.

Gibt es ein $p$ für $B(p) = B(1, p)$

Ja, jedes $p > 0$.

Seien $A, B$ und $C$ Matritzen aus $\mathbb{R}^3$, Gilt dann Wenn $A \cdot B =
    C$ dann auch $B \cdot A = C$

Nein