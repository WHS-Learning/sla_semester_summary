\chapter{Selfmade Prüfung 4}

\section{Lineare Abbildungen}

Sei eine Abbildung $f : A \rightarrow B$, mit $A \in \mathbb{R}^2$
\begin{align*}
    f\left((x_1, x_2, x_3)^T\right) = \begin{pmatrix}
                                          x_1 + x_2 + x_3 \\
                                          2x_1 + x_2 - 2x_3
                                      \end{pmatrix}
\end{align*}

Basis von $A$ sei $\left(\begin{pmatrix}
            1 \\ 1 \\ 1
        \end{pmatrix}, \begin{pmatrix}
            0 \\ 2 \\ 1
        \end{pmatrix}, \begin{pmatrix}
            1 \\ 2 \\ 1
        \end{pmatrix}\right)$ und die Basis von B sei $\left(\begin{pmatrix}
            1 \\ 1
        \end{pmatrix}, \begin{pmatrix}
            1 \\ 3
        \end{pmatrix}\right)$.

Berechnen Sie die Abbildungsmatrix $M^A_B(f)$. (Hinweis:
Denken Sie daran, dass die Basis von $A$ und $B$ nicht die Standardbasis ist.)

\section{Eigenwerte und Eigenvektoren}

Sei die Matrix A:

\begin{align*}
    A = \begin{pmatrix}
            1 & 2 & 3 & 4 \\
            0 & 1 & 0 & 3 \\
            0 & 0 & 2 & 0 \\
            0 & 0 & 4 & 0
        \end{pmatrix}
\end{align*}

\begin{enumerate}
    \item Welchen Grad hat das Characteristische Polynom von A?
    \item Berechnen Sie die Eigenwerte und Eigenvektoren von A.
\end{enumerate}

\section{Zufallsvariablen und Verteilungen}

Es sei folgendes hypothetisches Glücksspiel: In einem Beutel befinden sich 32
Kugeln. 6 Davon sind Rot, 18 sind Schwarz, und 8 sind Weiß. Sie dürfen für
einen Einsatz von 2€ 5 Kugeln ziehen. Ziehen Sie eine der Roten Kugeln gewinnen
Sie 0,50€. Ziehen Sie eine der Weißen Kugeln gewinnen Sie 0,25€.

\begin{enumerate}
    \item Handelt es sich um ein faires Spiel?
    \item mit welcher Wahrscheinlichkeit befindet sich unter 5 gezogenen Kugeln 2 Rote
          Kugeln?
\end{enumerate}

\section{Normalverteilung und z-Test}

Eine Firma stellt Mikrochips her, welche Sie in Sets von 100 verkauft. Die
Firma gibt an, dass nur 1\% ihrer verkauften Chips mit Fehlern behaftet sind.
Für ein Projekt benötigt Ihre Firma 98 Mikrochips.

Ihre Firma möchte nicht betuppt werden und plant die bestellten Chips auf Ihre
tatsächliche Fehlerquote zu prüfen. Man möchte sich zu 99\% sicher sein, damit
weitere Kooperation in Zukunft nichts mehr im Wege steht.

Geben Sie eine geeignete Entscheidungsregel für eine solche Prüfung an.

\section{Quiz}

Gibt es eine Drehmatrix $A \in \mathbb{R}^{n \times m}$, mit $n \in
    \mathbb{R}$, die nicht Teil der speziellen orthogonalen Gruppe $SO(n)$ ist?

Existiert eine Matrix vollen Rangs, deren Determinante 0 ist?

Können Sie die Parameter einer Wahrscheinlichkeit exakt aus einer Stichprobe
schätzen?