\chapter{Selfmade Prüfung 1}

\section{Lineare Abbildungen}

\subsection{Aufgabe 1}

Sei $M = \begin{pmatrix}
        3 & 5 & 9  \\
        3 & 7 & 11 \\
        2 & 3 & 1
    \end{pmatrix}$ und $v = \begin{pmatrix}
        1 \\ 8 \\ 7
    \end{pmatrix}$.

Berechnen Sie
\begin{enumerate}
    \item $M \cdot v$
    \item $v \times v$
    \item $v \cdot M$
\end{enumerate}

\subsection{Aufgabe 2}

Gegeben sind die Vektoren

\begin{align*}
    v_1 = \begin{pmatrix}
              1 \\ 2 \\ 3
          \end{pmatrix}, v_2 = \begin{pmatrix}
                                   2 \\ 9 \\ 5
                               \end{pmatrix}, v_3 = \begin{pmatrix}
                                                        0 \\ 0 \\ 1
                                                    \end{pmatrix}
\end{align*}

Prüfen Sie, ob diese eine Basis des $\mathbb{R}^3$ bilden.

\subsection{Aufgabe 3}

Ist die Matrix $A^{-1} = \begin{pmatrix}
        1  & -4 & -3 \\
        1  & -5 & -3 \\
        -1 & 6  & 4
    \end{pmatrix}$ die Inverse von $A = \begin{pmatrix}
        2  & 2  & 3 \\
        1  & -1 & 0 \\
        -1 & 2  & 1
    \end{pmatrix}$?

\section{Eigenwerte und Eigenvektoren}

\subsection{Aufgabe 1}

Sei $M = \begin{pmatrix}
        12 & 1 & -6 \\
        4  & 9 & -6 \\
        11 & 1 & -5
    \end{pmatrix}$.
Berechnen Sie alle Eigenwerte der Matrix und zu einen der Eigenwerte den Eigenvektor.

\section{Zufallsvariablen und Verteilungen}

Sie streiten mit ihren Kommulitoren Thorsten. Thorsten sagt, er schafft es
durchschnittlich 2 mal die erste Figur bei dem Spiel Mensch ärgere dich nicht
herauszuziehen.

Zur Erinnerung: Ist keine Figur auf dem Spielfeld, so dürfen Sie 3 mal würfeln.
Würfeln Sie eine 6, dürfen Sie die Figur aus dem Haus holen.

Prüfen Sie, ob Thorsten recht hat. Begründen Sie ihre Antwort.

Sie sollen Keinen z-Test durchführen.

\section{Normalverteilung und z-Test}

Prüfen Sie nun mit einem Konfidenzniveau von 10\%, ob Thorsten aus der obrigen Aufgabe recht hat.

\section{Quiz}

Hat jede Matrix Eigenvektoren?

Kann jede Matrix gedreht werden?

Kann jeder Vektor ein Eigenvektor einer beliebigen Matrix sein?