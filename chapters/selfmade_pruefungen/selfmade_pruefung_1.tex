\chapter{Selfmade Prüfung 1}

\section{Lineare Abbildungen}

\subsection{Aufgabe 1}

Sei $M = \begin{pmatrix}
        3 & 5 & 9  \\
        3 & 7 & 11 \\
        2 & 3 & 1
    \end{pmatrix}$ und $v = \begin{pmatrix}
        1 \\ 8 \\ 7
    \end{pmatrix}$.

Berechnen Sie
\begin{enumerate}
    \item $M \cdot v$
    \item $v \times v$
    \item $v \cdot M$
\end{enumerate}

\textbf{Antwort:}

\begin{align*}
    M \cdot v = \begin{pmatrix}
                    3 \cdot 1 + 5 \cdot 8 + 9 \cdot 7  \\
                    3 \cdot 1 + 7 \cdot 8 + 11 \cdot 7 \\
                    2 \cdot 1 + 3 \cdot 8 + 1 \cdot 7  \\
                \end{pmatrix} = \begin{pmatrix}
                                    106 \\ 136 \\ 33
                                \end{pmatrix} \\\\
    v \times v = \begin{pmatrix}
                     0 \\ 0 \\ 0
                 \end{pmatrix}                    \\\\
    v \cdot M = \text{Geht nicht}
\end{align*}

\subsection{Aufgabe 2}

Gegeben sind die Vektoren

\begin{align*}
    v_1 = \begin{pmatrix}
              1 \\ 2 \\ 3
          \end{pmatrix}, v_2 = \begin{pmatrix}
                                   2 \\ 9 \\ 5
                               \end{pmatrix}, v_3 = \begin{pmatrix}
                                                        0 \\ 0 \\ 1
                                                    \end{pmatrix}
\end{align*}

Prüfen Sie, ob diese eine Basis des $\mathbb{R}^3$ bilden.

\textbf{Antwort:}

\begin{align*}
    \det\left(\begin{pmatrix}
                  1 & 2 & 0 \\
                  2 & 9 & 0 \\
                  3 & 5 & 1
              \end{pmatrix}\right) = 5 \\
\end{align*}

Da $5 \neq 0$, sind die Vektoren Linear unabhängig und bilden damit eine Basis
des $\mathbb{R}^3$.

\subsection{Aufgabe 3}

Ist die Matrix $A^{-1} = \begin{pmatrix}
        1  & -4 & -3 \\
        1  & -5 & -3 \\
        -1 & 6  & 4
    \end{pmatrix}$ die Inverse von $A = \begin{pmatrix}
        2  & 2  & 3 \\
        1  & -1 & 0 \\
        -1 & 2  & 1
    \end{pmatrix}$?

\textbf{Antwort:}

\begin{align*}
    A \cdot A^{-1} = \begin{pmatrix}
                         1 \cdot 2 - 4 \cdot 1 - 3 \cdot -1  & 1 \cdot 2 - 4 \cdot -1 - 3 \cdot 2  & 1 \cdot 3 - 4 \cdot 0 - 3 \cdot 1  \\
                         1 \cdot 2 - 5 \cdot 1 - 3 \cdot -1  & 1 \cdot 2 - 5 \cdot -1 - 3 \cdot 2  & 1 \cdot 3 - 5 \cdot 0 - 3 \cdot 1  \\
                         -1 \cdot 2 + 6 \cdot 1 + 4 \cdot -1 & -1 \cdot 2 + 6 \cdot -1 + 4 \cdot 2 & -1 \cdot 3 + 6 \cdot 0 + 4 \cdot 1 \\
                     \end{pmatrix} \\
    =\begin{pmatrix}
         1 & 0 & 0 \\
         0 & 1 & 0 \\
         0 & 0 & 1
     \end{pmatrix}
\end{align*}

\section{Eigenwerte und Eigenvektoren}

\subsection{Aufgabe 1}

Sei $M = \begin{pmatrix}
        2  & 10 & -6  \\
        -2 & 26 & -14 \\
        -6 & 42 & -22
    \end{pmatrix}$.
Berechnen Sie alle Eigenwerte der Matrix und zu einen der Eigenwerte den Eigenvektor.

\textbf{Antwort:}

\begin{align*}
    M - \lambda I = \begin{pmatrix}
                        2 - \lambda & 10           & -6            \\
                        -2          & 26 - \lambda & -14           \\
                        -6          & 42           & -22 - \lambda
                    \end{pmatrix}                  \\
    \det(M - \lambda I) = -\lambda(\lambda^2 - 6\lambda + 8)                    \\
    -\lambda_1 = 0 \text{ oder } \lambda^2 - 6\lambda + 8 = 0                   \\
    \lambda^2 - 6\lambda + 8 = 0 \qquad | PQ                                    \\
    \lambda_{2, 3} = -\frac{-6}{2} \pm \sqrt{{\left(\frac{-6}{2}\right)}^2 - 8} \\
    \lambda_2 = 4                                                               \\
    \lambda_3 = 2                                                               \\\\
    \text{Eigenvektoren zu } \lambda_1 = 0                                      \\
    \begin{pmatrix}
        2 - 0 & 10     & -6    \\
        -2    & 26 - 0 & -14   \\
        -6    & 42     & -22-0
    \end{pmatrix}
\end{align*}

\begin{longtable}{p{10cm}}
    \hline
    \multicolumn{1}{c}{\textbf{Linearkombination}}                                         \\
    \hline
    \endfirsthead

    \hline
    \multicolumn{1}{c}{\tablename\ \thetable\ -- \textit{Fortführung von vorherier Seite}} \\
    \hline
    \multicolumn{1}{c}{\textbf{Linearkombination}}                                         \\
    \hline
    \endhead

    \hline
    \multicolumn{1}{r}{\textit{Fortsetzung siehe nächste Seite}}                           \\
    \endfoot

    \hline
    \endlastfoot

    $\displaystyle\begin{matrix}
                          2  & 10 & -6  \\
                          -2 & 26 & -14 \\
                          -6 & 42 & -22 \\
                      \end{matrix}$                                                            \\\hline
    II + I                                                                                 \\\hline\pagebreak[0]
    $\displaystyle\begin{matrix}
                          2  & 10 & -6  \\
                          0  & 36 & -20 \\
                          -6 & 42 & -22 \\
                      \end{matrix}$                                                            \\\hline
    III + 3I                                                                               \\\hline\pagebreak[0]
    $\displaystyle\begin{matrix}
                          2 & 10 & -6  \\
                          0 & 36 & -20 \\
                          0 & 72 & -40 \\
                      \end{matrix}$                                                            \\\hline
    III + 3I                                                                               \\\hline\pagebreak[0]
    $\displaystyle\begin{matrix}
                          2 & 10 & -6  \\
                          0 & 36 & -20 \\
                          0 & 72 & -40 \\
                      \end{matrix}$                                                            \\\hline
    72II - 36III                                                                           \\\hline\pagebreak[0]
    $\displaystyle\begin{matrix}
                          2 & 10 & -6  \\
                          0 & 36 & -20 \\
                          0 & 0  & 0   \\
                      \end{matrix}$                                                            \\\hline
    36II - 10II                                                                            \\\hline\pagebreak[0]
    $\displaystyle\begin{matrix}
                          2 & 0  & -16 \\
                          0 & 36 & -20 \\
                          0 & 0  & 0   \\
                      \end{matrix}$                                                            \\\hline
    -16II - -20I                                                                           \\\hline\pagebreak[0]
    $\displaystyle\begin{matrix}
                          2 & 0  & -16 \\
                          0 & 36 & 0   \\
                          0 & 0  & 0   \\
                      \end{matrix}$                                                            \\\hline
    I : 2                                                                                  \\\hline\pagebreak[0]
    II : 36                                                                                \\\hline\pagebreak[0]
    $\displaystyle\begin{matrix}
                          1 & 0 & -8 \\
                          0 & 1 & 0  \\
                          0 & 0 & 0  \\
                      \end{matrix}$                                                            \\\hline
\end{longtable}

\begin{align*}
    \begin{cases}
        \text{I:\@}  & x_1 -8x_3 = 0 \Leftrightarrow x_1 = 8x_3 \\
        \text{II:\@} & x_2 = 0
    \end{cases} \\
    x_3 = t | t \in \mathbb{R}                              \\
    \begin{pmatrix}
        8t \\ 0 \\ t
    \end{pmatrix}                                          \\
    t \begin{pmatrix}
          8 \\ 0 \\ 1
      \end{pmatrix}                                        \\
    span\left\{\begin{pmatrix}
                   8 \\ 0 \\ 1
               \end{pmatrix}\right\}
\end{align*}

\section{Zufallsvariablen und Verteilungen}

In einer Runde Poker (mit einem Standard 52 Karten-deck) haben Sie ein Royal
Flush (`Ass', `König', `Dame', `Bube', `10') mit dem Symbol Pik. Wie groß ist
dafür die Wahrscheinlichkeit? Ziehen sie bei der Berechnung nicht in betracht,
dass noch andere mit Ihnen Poker spielen. Sie Spielen für diese Aufgabe also
alleine.

\textbf{Antwort:}

\begin{align*}
    X \sim H(52, 5, 5) \\
    P(X = 5) = P\left(\frac{\begin{pmatrix}
                                    5 \\ 5
                                \end{pmatrix} \cdot \begin{pmatrix}
                                                        47 \\ 0
                                                    \end{pmatrix}}{\begin{pmatrix}
                                                                       52 \\ 5
                                                                   \end{pmatrix}}\right) = 0.000000385 \approx 0\%
\end{align*}

\section{Normalverteilung und z-Test}

Historisch regnete es im Sommer an durchschnittlich 20\% der Tage. Eine
aktuelle Studie untersucht, ob eine Klimaveränderung stattgefunden hat. Von 150
analysierten Sommertagen regnete es an 53 Tagen. Es soll beurteilt werden, ob
bei einer Irrtumswahrscheinlichkeit von höchstens 6\% von einem Klimawandel
ausgegangen werden kann.

\begin{align*}
    S_n \sim B(150, 0.2)                                         \\
    S_n = \text{Anzahl Regentage}                                \\
    H_0 = S_n = 30                                               \\
    H_1 = S_n \neq 30                                            \\
    E(S_n) = 150 \cdot 0.2 = 30                                  \\
    Var(S_n) = 30 \cdot 0.8 = 24                                 \\
    \sigma = \sqrt{24}                                           \\
    Z_{0.97} = 1.89                                              \\
    \text{Nullhypothese wird abgelehnt, falls }                  \\
    -1.89 < \frac{S_n - E(S_n)}{\sigma} < 1.89                   \\
    -1.89 < \frac{S_n - 30}{\sqrt{24}} < 1.89                    \\
    -1.89 \cdot \sqrt{24} + 30 < S_n < 1.89 \cdot \sqrt{24} + 30 \\
    20.74 < S_n < 39.26                                          \\
    21 < S_n < 39
\end{align*}

Da es an 53 Tagen geregnet hat, wird die Nullhypothese abgelehnt. Es hat also
ein Klimawandel stattgefunden.

\section{Quiz}

Hat jede Matrix (aus $\mathbb{R}^{m \times n}$) Eigenvektoren? Falls nein,
geben Sie eine Matrix die keinen hat.

\textbf{Antwort:}
Jede Drehmatrix, die nicht um 360 bzw 0 grad dreht. z.B. $\begin{pmatrix}
        0 & -1 \\
        1 & 0
    \end{pmatrix}$

Kann jeder Vektor ein Eigenvektor einer beliebigen Matrix sein? Falls nein,
geben Sie ein Beispiel.

\textbf{Antwort:}

\begin{align*}
    \begin{pmatrix}
        0 \\ 0 \\ 0
    \end{pmatrix}
\end{align*}