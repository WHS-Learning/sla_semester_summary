\chapter{Bonustest 7 - Zufallsvariablen und Verteilungen}

\section{Frage 1}

Die Zufallsvariable $X$ ist binomialverteilt mit den Parametern $n=20$ und $p= 0.2$. Berechnen Sie, auf zwei Nachkommastellen genau: 

\subsection{a}
$P(X = 2)$

\begin{align*}
    X \sim B(20, 0.2) \\
    P(X = 2) = \begin{pmatrix}
        20 \\ 2
    \end{pmatrix} \cdot 0.2^2 \cdot 0.8^{18} \approx 13.69\%
\end{align*}

\subsection{b}
$P(X \geq 12)$

\begin{align*}
    X \sim B(20, 0.2) \\
    P(X \leq 12) = \sum_{i = 0}^{12} \left(\begin{pmatrix}
        20 \\ i
    \end{pmatrix} \cdot 0.2^i \cdot 0.8^{20 - i}\right) \approx 100\% \\
    \text{Alternativ:} \\
    P(X \leq 12) = 1 - P(X > 12) = \\
    1 - \sum_{i = 12}^{20} \left(\begin{pmatrix}
        20 \\ i
    \end{pmatrix} \cdot 0.2^i \cdot 0.8^{20 - i}\right) \approx 100\%
\end{align*}

\subsection{c}
$P(X > 2)$

\begin{align*}
    X \sim B(20, 0.2) \\
    P(X > 2) = \sum_{i = 3}^{20} \left(\begin{pmatrix}
        20 \\ i
    \end{pmatrix} \cdot 0.2^i \cdot 0.8^{20 - i}\right) \approx 79.39\%\\
    \text{Alternativ: } \\
    P(X > 2) = 1 - P(X \leq 2) = \\
    1 - \sum_{2}^{0}  \left(\begin{pmatrix}
        20 \\ i
    \end{pmatrix} \cdot 0.2^i \cdot 0.8^{20 - i}\right) \approx 79.39\%
\end{align*}

\section{Frage 2}

Wie groß ist die Wahrscheinlichkeit, dass in einer Familie mit fünf Kindern\dots

Die Zufallsvariable ist Binomialverteilt, da sie in mehreren, voneinander unabhängigen Zufallsexperimenten zwischen Treffer und Niete unterscheidet.

$p = 0.5$ Junge und Mädchen haben modelliert gleiche Wahrscheinlichkeit

$n = 5$ 5 Kinder

\subsection{a}

genau drei Mädchen sind?

\begin{align*}
    X \sim B(5, 0.5) \\
    X = \text{Anzahl der Mädchen} \\
    P(X = 3) = \begin{pmatrix}
        5 \\ 3
    \end{pmatrix} \cdot 0.5^3 \cdot 0.5^2 = 31.25\%
\end{align*}

\subsection{b}

mindestens zwei Mädchen sind

\begin{align*}
    X \sim B(5, 0.5) \\
    X = \text{Anzahl der Mädchen} \\
    P(X \geq 2) = \sum_{i = 2}^{5} \left(\begin{pmatrix}
        5 \\ i
    \end{pmatrix} \cdot 0.5^i \cdot 0.5^{5 - i}\right) \approx 81.25\%
\end{align*}

\subsection{c}

kein Junge ist

\begin{align*}
    X \sim B(5, 0.5) \\
    X = \text{Anzahl der Jungen} \\
    P(X = 0) = \begin{pmatrix}
        5 \\ 0
    \end{pmatrix} \cdot 0.5^0 \cdot 0.5^5 \approx 3.13
\end{align*}

\subsection{d}

vier Mädchen und ein Junge sind? 

\begin{align*}
    X \sim B(5, 0.5) \\
    X = \text{Anzahl der Mädchen} \\
    P(X = 4) = \begin{pmatrix}
        5 \\ 4
    \end{pmatrix} \cdot 0.5^4 \cdot 0.5^1 = 15.63
\end{align*}

(gibt es 4 Mädchen muss es auch einen Jungen geben. Die Aussage ist also Äquivalen zu den Aussagen `genau vier Mädchen sind' und `genau ein Junge ist')

\section{Frage 3}

Beim Spiel `Mensch, ärgere dich nicht' muss zum Spielbeginn eine Sechs gewürfelt werden.

Wie hoch ist die Wahrscheinlichkeit, genau nach 3-maligem Würfeln mitspielen zu dürfen?

Hier wird nach der Wahrscheinlichkeit gefragt, nach \textit{genau} 3 malen eine 6 zu würfeln, also keine 6, keine 6, 6. Dies sind drei aneinander gereihte Bernoulliexperimente. 

\begin{align*}
    X \sim Bernoulli(\frac{1}{6}) \\
    X = \text{6 gewürfelt} \\
    P(X = 0) \cdot P(X = 0) \cdot P(X = 1) \\
    = \frac{5}{6} \cdot \frac{5}{6} \cdot \frac{1}{6} \approx 11.57
\end{align*}

\pagebreak

\section{Frage 4}

Bei einer unfairen Münze liegt die Wahrscheinlichkeit Kopf zu werfen bei $p = \frac{1}{5}$.

Wie hoch ist die Wahrscheinlichkeit, dass erst beim dritten Versuch Kopf geworfen wird?

Hier ist nach der Wahrscheinlichkeit gefragt, \textit{genau} nach dem dritten mal Kopf zu würfeln. Also sind hier 3 aneinander gekettete Bernoulliexperimente vertreten.

\begin{align*}
    X \sim Bernoulli(\frac{1}{5}) \\
    X = \text{Kopf geworfen} \\
    P(X = 0) \cdot P(X = 0) \cdot P(X = 1) = \\
    \frac{4}{5} \cdot \frac{4}{5} \cdot \frac{1}{5} = 12.8\%
\end{align*}

Wie hoch ist die Wahrscheinlichkeit, dass erst beim fünften Versuch Kopf geworfen wird?

\begin{align*}
    X \sim Bernoulli(\frac{1}{5}) \\
    X = \text{Kopf geworfen} \\
    P(X = 0) \cdot P(X = 0) \cdot P(X = 0) \cdot P(X = 0) \cdot P(X = 1) = \\
    \frac{4}{5} \cdot \frac{4}{5} \cdot \frac{4}{5} \cdot \frac{4}{5} \cdot \frac{1}{5} \approx 8.19\%
\end{align*}

\section{Frage 5}

Für eine Stelle haben sich $16$ Personen beworben, davon haben $5$ bereits Arbeitserfahrung, die übrigen $11$ noch nicht.

Es werden nun $6$ Personen per Losentscheid ausgewählt. Wie hoch ist die Wahrscheinlichkeit, dass genau $3$ Personen mit Arbeitserfahrung ausgewählt werden?

Die Zufallsvariable ist Hypergeometrisch verteilt, da aus einer Gesamtmenge eine kleinere Zielmenge gezogen wird.

\begin{align*}
    X \sim H(16, 5, 6) \\
    X = \text{Anzahl Personen mit Berufserfahrung} \\
    P(X = 3) = \frac{\begin{pmatrix}
        5 \\ 3
    \end{pmatrix} \cdot \begin{pmatrix}
        11 \\ 3
    \end{pmatrix}}{\begin{pmatrix}
        16 \\ 6
    \end{pmatrix}} \approx 20.60\%
\end{align*}

\section{Frage 6}

In einer Packung seien $500$ Nägel. Davon seien $10$ Nägel defekt. Wie groß ist die Wahrscheinlichkeit, dass unter $5$ zufällig aus der Packung gezogenen Nägel genau einer defekt ist?

Die Zufallsvariable ist Hypergeometrisch verteilt, da aus einer Gesamtmenge eine kleinere Zielmenge gezogen wird.

\begin{align*}
    X \sim H(500, 10, 5) \\
    X = \text{Anzahl Defekter Nägel} \\
    P(X = 1) = \frac{\begin{pmatrix}
        10 \\ 1
    \end{pmatrix} \cdot \begin{pmatrix}
        490 \\ 4
    \end{pmatrix}}{\begin{pmatrix}
        500 \\ 5
    \end{pmatrix}} \approx 9.30\%
\end{align*}