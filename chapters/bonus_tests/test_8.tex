\chapter{Bonustest 8 -  Lagemaße und Normalverteilung}

\section{Aufgabe 1}

Auf einer Kirmes steht ein Glücksrad mit $20$ gleichgroßen Feldern. Die Felder sind mit $1$ bis $20$ durchnummeriert. Innerhalb eines Jahrzehnts wird das Glücksrad $2000000$ Mal gedreht. Bezeichne $X$ wie oft dabei das Glücksrad auf der Zahl $18$ stehengeblieben ist.

\subsection{Ohne Stetigkeitskorrektur:}

\begin{align*}
    X \sim B(2000000, \frac{1}{20})
    E(X) = 2000000 \cdot \frac{1}{20} = 100000 \\
    Var(x) = E(x) \cdot \frac{19}{20} = 95000 \\
    \sigma = \sqrt{95000} \\
    P(X \leq 100012) = P\left(\frac{X - E(X)}{\sigma} \leq \frac{100012 - 100000}{\sqrt{95000}}\right) \\
    \Phi\left(\frac{100012 - 100000}{\sqrt{95000}}\right) \\
    \approx \Phi(0.04) \overset{Tabelle}{\approx} 51.60\%\\\\
    P(X \geq 100013) = 1 - P(X \leq 100012) \\
    = 1 - 0.51595 \approx 48.41 \% \\
    P(100012 < X < 100030) = P(X < 100030) - P(X < 100012) \\
    P(X < 100130) = P\left(\frac{X - E(X)}{\sigma} < \frac{100130 - 100000}{\sqrt{95000}}\right)\\
    \Phi(\frac{100030 - 100000}{\sqrt{95000}}) \\
    \approx \Phi(0.42) \overset{Tabelle}{\approx} 66.28\% \\
    \text{Da keine Stetigkeitskorrektur gemacht wird, ist } \\
    P(X \leq 100012) = P(X < 100012) = P(X > 100012) \\
    0.6628 - 0.5160 = 14.68\%
\end{align*}

\subsection{Mit Stetigkeitskorrektur:}

\begin{align*}
    X \sim B(2000000, \frac{1}{20}) \\
    E(X) = 100000 \quad | \quad Var(X) = 95000 \quad | \quad \sigma = \sqrt{95000} \\\\
    P(100012 < X < 100130) = P(100013 \leq X \leq 100129) \\
    \approx P(100012.5 \leq X_{\text{korr}} \leq 100129.5) \\
    = P(X_{\text{korr}} \leq 100129.5) - P(X_{\text{korr}} \leq 100012.5) \\\\
    P(X_{\text{korr}} \leq 100129.5) = \Phi\left(\frac{100129.5 - 100000}{\sqrt{95000}}\right) \\
    = \Phi\left(\frac{129.5}{\sqrt{95000}}\right) \\
    \approx \Phi(0.42) \overset{\text{Tabelle}}{\approx} 0.6628 \\\\
    P(X_{\text{korr}} \leq 100012.5) = \Phi\left(\frac{100012.5 - 100000}{\sqrt{95000}}\right) \\
    = \Phi\left(\frac{12.5}{\sqrt{95000}}\right) \\
    \approx \Phi(0.04) \overset{\text{Tabelle}}{\approx} 0.5160 \\\\
    \text{Ergebnis} = 0.6628 - 0.5160 = 0.1468 = 14.68\%
\end{align*}

\section{Frage 2}

Bei einer Meinungsumfrage werden $500$ Personen befragt. Zum Zeitpunkt der Umfrage sind $30\%$ der Bevölkerung Anhänger der Partei A. Wie groß ist die Wahrscheinlichkeit, dass bei der Umfrage zwischen $28\%$ und $32\%$ der Befragten Anhänger der Partei A sind?

\subsection{Ohne Stetigkeitskorrektur}

\begin{align*}
    X \sim B(0.3, 500) \\
    X = \text{Anzahl Anhänger von Partei A} \\
    E(X) = 500 \cdot 0.3 = 150 \\
    Var(X) = 150 \cdot 0.7 = 105 \\
    \sigma = \sqrt{105}\\
    28\% \Rightarrow 500 \cdot 0.28 = 140 \\
    32\% \Rightarrow 500 \cdot 0.32 = 160 \\
    P(140 \leq X \leq 160) = P(160 < X) - P(140 < X) \\
    P(160 < X) = P\left(\frac{X - E(X)}{\sigma} < \frac{160 - 150}{\sqrt{105}}\right)
    \Phi\left(\frac{160 - 150}{\sqrt{105}}\right) \\
    \approx \Phi(0.98) \overset{Tabelle}{\approx} 83.65 \%\\
    P(140 < X) = P\left(\frac{X - E(X)}{\sigma} < \frac{140 - 150}{\sqrt{105}}\right)
    \Phi\left(\frac{140 - 150}{\sqrt{105}}\right) \\
    \approx \Phi(-0.98) = 1 - \Phi(0.98) \overset{Tabelle}{\approx} 1 - 0.83646 \approx 16.35\% \\
    P(140 \leq X \leq 160) = 0.8365 - 0.1635 = 67.30\%
\end{align*}

\subsection{Mit Stetigkeitskorrektur}

\begin{align*}
    X \sim B(0.3, 500) \\
    X = \text{Anzahl Anhänger von Partei A} \\
    E(X) = 150 \\
    \sigma = \sqrt{105} \approx 10.247 \\
    \\
    P(140 \leq X \leq 160) \rightarrow P(139.5 \leq X_{stetig} \leq 160.5) \\
    \\
    P(X_{stetig} \leq 160.5) = P\left(\frac{X - E(X)}{\sigma} \leq \frac{160.5 - 150}{\sqrt{105}}\right) \\
    = \Phi\left(\frac{10.5}{\sqrt{105}}\right) \\
    \approx \Phi(1.02) \overset{Tabelle}{\approx} 84.61 \% \\
    \\
    P(X_{stetig} \leq 139.5) = P\left(\frac{X - E(X)}{\sigma} \leq \frac{139.5 - 150}{\sqrt{105}}\right) \\
    = \Phi\left(\frac{-10.5}{\sqrt{105}}\right) \approx \Phi(-1.02) = 1 - \Phi(1.02) \\
    \overset{Tabelle}{\approx} 1 - 0.8461 = 15.39 \% \\
    \\
    P(140 \leq X \leq 160) \approx 0.8461 - 0.1539 = 0.6922 = 69.22\%
\end{align*}

\section{Frage 3}

Ein Würfel trägt $3$ Einser, $2$ Zweier und eine Sechs. Er wird $1000$ mal geworfen. Berechnen Sie die folgenden Wahrscheinlichkeiten, mithilfe des Satzes von De Moivre-Laplace.

\subsection{Ohne Stetigkeitskorrektur}

\begin{align*}
    X \sim B(0.5, 1000) \\
    X = \text{Anzahl gewürfelter Einsen} \\
    E(X) = 1000 \cdot 0.5 = 500 \\
    Var(X) = 500 \cdot 0.5 = 250 \\
    \sigma = \sqrt{250} \\
    P(X < 520) = P\left(\frac{X - E(X)}{\sigma} < \frac{520 - 500}{\sqrt{250}}\right) \\
    \Phi\left(\frac{520 - 500}{\sqrt{250}}\right) \\
    \approx \Phi(1.26) \overset{Tabelle}{\approx} 89.62\%\\\\
    X \sim B(\frac{1}{3}, 1000) \\
    X = \text{Anzahl gewürfelter Zweien} \\
    E(X) = 1000 \cdot \frac{1}{3} = \frac{1000}{3} \\
    Var(X) = \frac{1000}{3} \cdot \frac{2}{3} = \frac{2000}{9} \\
    \sigma = \swarrow\frac{2000}{9} \\
    P(X > 300) = 1 - P(X \leq 300) \\
    = 1 - P\left(\frac{X - E(X)}{\sigma} \leq \frac{300 - \frac{1000}{3}}{\sqrt{\frac{2000}{9}}}\right) \\
    1 - \Phi\left(\frac{300 - \frac{1000}{3}}{\sqrt{\frac{2000}{9}}}\right) \\
    \approx \Phi(-2.24) \approx 1 - (1 - \Phi(2.24)) \\ 
    \overset{Tabelle}{\approx} 1 - (1 - 98.75)\% = 1 - 1 + 98.75\% = 98.75\%  \\\\
    X \sim B(\frac{1}{6}, 1000) \\
    X = \text{Anzahl gewürfelter Sechsen} \\
    E(X) = 1000 \cdot \frac{1}{6} = \frac{500}{3} \\
    Var(X) = \frac{500}{3} \cdot \frac{5}{6} = \frac{1250}{9} \\
    \sigma = \sqrt{\frac{1250}{9}} \\
    P(140 \geq X \geq 180) = P(X < 180) - P(X < 140) \\
    P(X < 180) = P\left(\frac{X - E(X)}{\sigma} < \frac{180 - \frac{500}{3}}{\sqrt{\frac{1250}{9}}}\right)
    \Phi\left(\frac{180 - \frac{500}{3}}{\sqrt{\frac{1250}{9}}}\right) \\
    \approx \Phi(1.13) = \overset{Tabelle}{\approx} 0.87076 = 87.08\%\\
    P(X < 140) = P\left(\frac{X - E(X)}{\sigma} < \frac{140 - \frac{500}{3}}{\sqrt{\frac{1250}{9}}}\right)
    \Phi\left(\frac{140 - \frac{500}{3}}{\sqrt{\frac{1250}{9}}}\right) \\
    \approx \Phi(-2.26) = 1 - \Phi(2.26) = 1 - 0.98809 = 0.0119 = 1.19\% \\
    P(140 \geq X \geq 180) = 0.87076 - 0.0119 = 0.8217 = 85.89\%
\end{align*}

\subsection{Mit Stetigkeitskorrektur}

\begin{align*}
    X \sim B(0.5, 1000) \\
    X = \text{Anzahl gewürfelter Einsen} \\
    E(X) = 1000 \cdot 0.5 = 500 \\
    Var(X) = 500 \cdot 0.5 = 250 \\
    \sigma = \sqrt{250} \\
    P(X < 520) = P(X \leq 519) \approx P(X \leq 519.5) \\
    P\left(\frac{X - E(X)}{\sigma} \leq \frac{519.5 - 500}{\sqrt{250}}\right) \\
    \Phi\left(\frac{519.5 - 500}{\sqrt{250}}\right) \\
    \approx \Phi(1.23) \overset{Tabelle}{\approx} 89.07\% \\\\
    X \sim B(\frac{1}{3}, 1000) \\
    X = \text{Anzahl gewürfelter Zweien} \\
    E(X) = 1000 \cdot \frac{1}{3} = \frac{1000}{3} \\
    Var(X) = \frac{1000}{3} \cdot \frac{2}{3} = \frac{2000}{9} \\
    \sigma = \sqrt{\frac{2000}{9}} \\
    P(X > 300) = P(X \geq 301) \approx P(X \geq 300.5) \\
    = 1 - P(X < 300.5) \\
    = 1 - P\left(\frac{X - E(X)}{\sigma} < \frac{300.5 - \frac{1000}{3}}{\sqrt{\frac{2000}{9}}}\right) \\
    1 - \Phi\left(\frac{300.5 - \frac{1000}{3}}{\sqrt{\frac{2000}{9}}}\right) \\
    \approx 1 - \Phi(-2.20) = \Phi(2.20) \\
    \overset{Tabelle}{\approx} 98.61\% \\\\
    X \sim B(\frac{1}{6}, 1000) \\
    X = \text{Anzahl gewürfelter Sechsen} \\
    E(X) = 1000 \cdot \frac{1}{6} = \frac{500}{3} \\
    Var(X) = \frac{500}{3} \cdot \frac{5}{6} = \frac{1250}{9} \\
    \sigma = \sqrt{\frac{1250}{9}} \\
    P(140 \leq X \leq 180) \approx P(139.5 \leq X \leq 180.5) \\
    P(X \leq 180.5) - P(X \leq 139.5) \\
    P(X \leq 180.5) = \Phi\left(\frac{180.5 - \frac{500}{3}}{\sqrt{\frac{1250}{9}}}\right) \\
    \approx \Phi(1.17) \overset{Tabelle}{\approx} 0.8790 = 87.90\%\\
    P(X \leq 139.5) = \Phi\left(\frac{139.5 - \frac{500}{3}}{\sqrt{\frac{1250}{9}}}\right) \\
    \approx \Phi(-2.31) = 1 - \Phi(2.31) \overset{Tabelle}{\approx} 1 - 0.9896 = 0.0104 = 1.04\% \\
    P(139.5 \leq X \leq 180.5) = 0.8790 - 0.0104 = 0.8686 = 86.86\%
\end{align*}

\section{Frage 4}

In einer Bonbon-Tüte gibt es grüne, blaue und rote Bonbons. Die Anzahl der blauen Bonbons kann mit einer Normalverteilung beschrieben werden. Hierbei liegt der Erwartungswert bei $24$ und die Standardabweichung bei $4$.

\subsection{Ohne Stetigkeitskorrektur}

Wie hoch ist die Wahrscheinlichkeit das in einer Tüte weniger als 19 blaue Bonbons sind?

\begin{align*}
    X \sim N(24, 4) \\
    X = \text{Anzahl blauer Bonbons} \\
    P(X < 19) = P\left(\frac{X - E(X)}{\sigma} < \frac{19 - 24}{4}\right) \\
    \Phi(\frac{19 - 24}{4}) \\
    \approx \Phi(-1.25) = 1 - \Phi(1.25) \overset{Tabelle}{\approx} 1 - 0.89435 = 0.10565 = 19.57\%
\end{align*}

Wie muss der Erwartungswert gewählt werden, so dass die Wahrscheinlichkeit bei weniger als $19$ blauen Bonbons bei $95\%$ liegt?

\begin{align*}
    \text{Die Wahrscheinlichkeit soll bei 95\% liegen.} \\
    \text{Der z-Wert für 0.95 ist ca. 1.65} \\
    1.65 = \frac{19 - E(X)}{4} \quad | \cdot 4  \\
    6.6 = 19 - E(X) \quad | +E(X) \quad | -6.6 \\
    E(X) = 12.4
\end{align*}

\subsection{Mit Stetigkeitskorrektur}

Wie hoch ist die Wahrscheinlichkeit das in einer Tüte weniger als 19 blaue Bonbons sind?

\begin{align*}
    X \sim N(24, 4) \\
    X = \text{Anzahl blauer Bonbons} \\
    \text{Gesucht ist }P(X < 19) \text{, mit Stetigkeitskorrektur wird daraus} P(X < 18.5) \\
    P(X < 18.5) = P\left(\frac{X - E(X)}{\sigma} > \frac{18.5 - 24}{4}\right) \\
    \Phi(\frac{18.5 - 24}{4}) \\
    \approx \Phi(-1.38) = 1 - \Phi(1.38) \overset{Tabelle}{\approx} 1 - 0.91621 = 0.0838 = 8.38\%
\end{align*}

Wie muss der Erwartungswert gewählt werden, so dass die Wahrscheinlichkeit bei weniger als $19$ blauen Bonbons bei $95\%$ liegt?

\begin{align*}
    \text{Gesucht ist  } P(X < 19) \text{, mit Stetigkeitskorrektur} P(X < 18.5) \\
    \text{Die Wahrscheinlichkeit soll bei 95\% liegen.} \\
    \text{Der z-Wert für 0.95 ist ca. 1.65} \\
    1.65 = \frac{18.5 - E(X)}{4} \quad | \cdot 4  \\
    6.6 = 18.5 - E(X) \quad | +E(X) \quad | -6.6 \\
    E(X) = 11.9
\end{align*}

\section{Frage 5}

Eine Maschine produziert 500 mm lange Schrauben mit einer Standardabweichung von 15 mm. Die Länge der Schrauben kann als normalverteilt angesehen werden.

\subsection{Ohne Stetigkeitskorrektur}

Wie hoch ist die Wahrscheinlichkeit, dass eine Schraube kürzer ist als 485 mm ist?

\begin{align*}
    X \sim N(500, 15) \\
    X = \text{Länge der Schraube} \\
    P(X < 485) = P\left(\frac{X - E(X)}{\sigma} < \frac{485 - 500}{15}\right) \\
    \Phi\left(\frac{485 - 500}{15}\right) \\
    \approx \Phi(-1) = 1 - \Phi(1) \overset{Tabelle}{\approx} 1 - 0.84134 = 0.1587 = 15.87\%
\end{align*}

Wie hoch ist die Wahrscheinlichkeit, dass eine Schraube höchstens 501 mm und mindestens 499 mm lang ist?

\begin{align*}
    X \sim N(500, 15) \\
    X = \text{Länge der Schraube} \\
    P(499 \leq X \leq 501) = P(X \leq 501) - P(X \leq 499) \\
    P(X \leq 501) = P\left(\frac{X - E(X)}{\sigma} < \frac{501 - 500}{15}\right) \\
    \Phi(\frac{501 - 500}{15}) \approx \Phi(0.07) \overset{Tabelle}{\approx} 0.52790 \approx 52.79\% \\
    P(X \leq 499) = P\left(\frac{X - E(X)}{\sigma} < \frac{499 - 500}{15}\right) \\
    \Phi\left(\frac{499-500}{15}\right) \approx \Phi(-0.07) = 1 - \Phi(0.07)\\
    = 1 - 0.52790 = 0.4721 \approx 47.21\%\\
    P(499 \leq X \leq 501) = 0.52790 - 0.4721 = 0.0558 \approx 5.58\%
\end{align*}

\section*{Berechnung mit Stetigkeitskorrektur}

Eine Maschine produziert 500 mm lange Schrauben mit einer Standardabweichung von 15 mm. Die Länge der Schrauben kann als normalverteilt angesehen werden.

\subsection{Mit Stetigkeitskorrektur}

Wie hoch ist die Wahrscheinlichkeit, dass eine Schraube kürzer ist als 485 mm ist?

\begin{align*}
    X \sim N(500, 15) \\
    X = \text{Länge der Schraube} \\
    \text{Aus P(X < 485) wird P(X $\leq$ 484), mit Korrektur also P(X < 484.5)} \\
    P(X < 484.5) = P\left(\frac{X - E(X)}{\sigma} < \frac{484.5 - 500}{15}\right) \\
    \Phi\left(\frac{-15.5}{15}\right) \\
    \approx \Phi(-1.03) = 1 - \Phi(1.03) \overset{Tabelle}{\approx} 1 - 0.84849 = 0.15151 = 15.15\%
\end{align*}

Wie hoch ist die Wahrscheinlichkeit, dass eine Schraube höchstens 501 mm und mindestens 499 mm lang ist?

\begin{align*}
    X \sim N(500, 15) \\
    X = \text{Länge der Schraube} \\
    \text{Aus P(499 $\leq$ X $\leq$ 501) wird mit Korrektur P(498.5 $\leq$ X $\leq$ 501.5)} \\
    P(498.5 \leq X \leq 501.5) = P(X \leq 501.5) - P(X \leq 498.5) \\
    \\
    P(X \leq 501.5) = P\left(\frac{X - E(X)}{\sigma} < \frac{501.5 - 500}{15}\right) \\
    \Phi(\frac{1.5}{15}) = \Phi(0.1) \overset{Tabelle}{\approx} 0.53983 \approx 53.98\% \\
    \\
    P(X \leq 498.5) = P\left(\frac{X - E(X)}{\sigma} < \frac{498.5 - 500}{15}\right) \\
    \Phi\left(\frac{-1.5}{15}\right) = \Phi(-0.1) = 1 - \Phi(0.1)\\
    = 1 - 0.53983 = 0.46017 \approx 46.02\%\\
    \\
    P(498.5 \leq X \leq 501.5) = 0.53983 - 0.46017 = 0.07966 \approx 7.97\%
\end{align*}

\section{Frage 6}

Waschmaschinen sollen für einen Waschgang durchschnittlich $45l$ Wasser verbrauchen. Ein Hersteller will erreichen, dass bei höchstens $5\%$ seiner Maschinen der Wasserverbrauch größer als $55l$ ist.
Welche Standardabweichung darf die Maschine (höchstens) haben, wenn man voraussetzt, dass der Wasserverbrauch normalverteilt ist?

\begin{align*}
    1.65 = \frac{55 - 45}{\sigma} \quad |\cdot \sigma
    \Leftrightarrow 1.65\sigma = 10 \quad |: 1.65
    \Leftrightarrow \sigma = 6.06
\end{align*}

\section{Frage 7}

Wie hoch ist die Wahrscheinlichkeit dafür, maximal $110$ Mal eine sechs zu würfeln bei $600$ Würfen?

Hierbei handelt es sich um ein Zufallsexperiment das auf einer Binomialverteilung basiert. Da das $n$ sehr hoch ist, würde sich die Anwendung vom Satz von Moivre-Laplace anbieten, um einen Näherungswert der Lösung zu erhalten. Sei $X$ die Anzahl der gewürfelten sechsen.

Zuerst sollte man jedoch prüfen ob die Näherung ausreichend `gut' ist. Dies tut man mit der Faustregel:  $n \cdot p \cdot (1 - p) > 9$.

Falls dies der Fall ist, fährt man fort mit der Berechnung des Erwartungswerts sowie der Standardabwechung.

\begin{align*}
    E(X) = \mu = 600 \cdot \frac{1}{6} = 100 \\
    \sqrt{Var(X)} = \sigma = \sqrt{100 \cdot \frac{5}{6}} \approx 9.13
\end{align*}

In dieser Aufgabe wird nach der Wahrscheinlichkeit für $P(X \leq 10)$ gefragt. Wir formen diese Ungleichung nun um. Ziehen hier auf beiden Seiten $\mu$ ab und teilen durch $\sigma$. So erhalten wir:

\begin{align*}
    P(X \leq 110) = P\left(\frac{X - E(X)}{\sigma} \leq \frac{110 - \mu}{\sigma}\right) 
\end{align*}

Nach dem Satz von Moivre-Laplace gilt nun:

\begin{align*}
    P\left(\frac{X - E(X)}{\sigma} \leq \frac{110 - \mu}{\sigma}\right) \approx \Phi\left(\frac{110 - \mu}{\sigma}\right) \\
    \Phi\left(\frac{110 - \mu}{\sigma}\right) = \Phi(\frac{110 - 100}{9.13}) \approx \Phi(1.10)
\end{align*}

Nun können Sie den ausgerechneten Wert für $\Phi$ in der Tablle der Werte für die Standardnormalverteilung nachschlagen.

\begin{align*}
    P(X \leq 110) = 86.43\%
\end{align*}

\section{Frage 8}

Gegeben sei die diskrete Zufallsvariable $Y$, die die Werte $1,2,3,4$ annehmen kann und die folgende Zähldichte hat:

\begin{align*}
    f(y) = P(Y = y) \\
    = \begin{cases}
        \frac{(5 - y) \cdot y}{20} & \text{Für } y \in \left\{1, 2, 3, 4\right\}\\
        0 & \text{sonst}
    \end{cases}
\end{align*}

Berechnen Sie die Wahrscheinlichkeit, dass die Zufallsvariable $Y$ genau den Wert $3$ annimmt und den Erwartungswert von $Y$.

\begin{align*}
    P(Y = 3) = \frac{(5 - 3) \cdot 3}{20} = 0.3 = 30\% \\
    \frac{1 + 2 + 3 + 4}{4} = 2.5
\end{align*}

\section{Frage 9}

Einer ihrer Kommilitonen schreibt ein Programm, das eine (natürliche) Zufallszahl von 1 bis 100 generieren kann. Er macht daraus ein Spiel, das auf folgenden Regeln basiert:

\begin{itemize}
    \item es wird pro Versuch $1$€ gesetzt
    \item bei einer $55$ erhalten Sie $100$€
    \item bei einer $11$, $22$ oder $33$ erhalten Sie $3$€
    \item bei einer $44$, $66$, $77$ oder $88$ erhalten Sie $2$€
    \item bei einer $99$, $1$ oder $100$ erhalten Sie $1$€
    \item bei jeder andern Zahl behält der Kommilitone den Einsatz und Sie erhalten nichts.
\end{itemize}

Ist dieses Spiel fair?

\begin{align*}
    (100 - 1) \cdot \frac{1}{100} + (3 - 1) \cdot \frac{3}{100}\\
    + (2 - 1) \cdot\frac{4}{100} + (1 - 1) \cdot \frac{3}{100}\\
    + (0 - 1) \cdot \frac{89}{100} = 0.2 \neq 0 \\
    \Rightarrow \text{das Spiel ist unfair.}
\end{align*}

\section{Frage 10}

Ein Torwart hält den Ball bei einem Elfmeter mit einer Wahrscheinlichkeit von $65.0\%$. Nun muss er nach einer Verlängerung ins Tor (Zufallsvariable $x=5$ Elfmeter). Wie viele gehaltene Bälle kann man erwarten?

\begin{align*}
    E(X) = 5 \cdot  0.65 = 3.25 \\
    Var(X) = 3.25 \cdot 0.35 \approx 1.14
\end{align*}