\chapter{Bonustest 8 -  Lagemaße und Normalverteilung}

\section{Aufgabe 1}

Auf einer Kirmes steht ein Glücksrad mit $20$ gleichgroßen Feldern. Die Felder sind mit $1$ bis $20$ durchnummeriert. Innerhalb eines Jahrzehnts wird das Glücksrad $2000000$ Mal gedreht. Bezeichne $X$ wie oft dabei das Glücksrad auf der Zahl $18$ stehengeblieben ist.

\subsection{Ohne Stetigkeitskorrektur:}

\begin{align*}
    X \sim B(2000000, \frac{1}{20})
    E(X) = 2000000 \cdot \frac{1}{20} = 100000 \\
    Var(x) = E(x) \cdot \frac{19}{20} = 95000 \\
    \sigma = \sqrt{95000} \\
    P(X \leq 100012) = P\left(\frac{X - E(X)}{\sigma} \leq \frac{100012 - 100000}{\sqrt{95000}}\right) \\
    \Phi\left(\frac{100012 - 100000}{\sqrt{95000}}\right) \\
    \approx \Phi(0.04) \overset{Tabelle}{\approx} 51.60\%\\\\
    P(X \geq 100013) = 1 - P(X \leq 100012) \\
    = 1 - 0.51595 \approx 48.41 \% \\
    P(100012 < X < 100030) = P(X < 100030) - P(X < 100012) \\
    P(X < 100130) = P\left(\frac{X - E(X)}{\sigma} < \frac{100130 - 100000}{\sqrt{95000}}\right)\\
    \Phi(\frac{100030 - 100000}{\sqrt{95000}}) \\
    \approx \Phi(0.42) \overset{Tabelle}{\approx} 66.28\% \\
    \text{Da keine Stetigkeitskorrektur gemacht wird, ist } \\
    P(X \leq 100012) = P(X < 100012) = P(X > 100012) \\
    0.6628 - 0.5160 = 14.68\%
\end{align*}

\subsection{Mit Stetigkeitskorrektur:}

\begin{align*}
    X \sim B(2000000, \frac{1}{20}) \\
    E(X) = 100000 \quad | \quad Var(X) = 95000 \quad | \quad \sigma = \sqrt{95000} \\\\
    P(100012 < X < 100130) = P(100013 \leq X \leq 100129) \\
    \approx P(100012.5 \leq X_{\text{korr}} \leq 100129.5) \\
    = P(X_{\text{korr}} \leq 100129.5) - P(X_{\text{korr}} \leq 100012.5) \\\\
    P(X_{\text{korr}} \leq 100129.5) = \Phi\left(\frac{100129.5 - 100000}{\sqrt{95000}}\right) \\
    = \Phi\left(\frac{129.5}{\sqrt{95000}}\right) \\
    \approx \Phi(0.42) \overset{\text{Tabelle}}{\approx} 0.6628 \\\\
    P(X_{\text{korr}} \leq 100012.5) = \Phi\left(\frac{100012.5 - 100000}{\sqrt{95000}}\right) \\
    = \Phi\left(\frac{12.5}{\sqrt{95000}}\right) \\
    \approx \Phi(0.04) \overset{\text{Tabelle}}{\approx} 0.5160 \\\\
    \text{Ergebnis} = 0.6628 - 0.5160 = 0.1468 = 14.68\%
\end{align*}

\section{Frage 2}

Bei einer Meinungsumfrage werden $500$ Personen befragt. Zum Zeitpunkt der Umfrage sind $30\%$ der Bevölkerung Anhänger der Partei A. Wie groß ist die Wahrscheinlichkeit, dass bei der Umfrage zwischen $28\%$ und $32\%$ der Befragten Anhänger der Partei A sind?

\subsection{Ohne Stetigkeitskorrektur}

\begin{align*}
    X \sim B(0.3, 500) \\
    X = \text{Anzahl Anhänger von Partei A} \\
    E(X) = 500 \cdot 0.3 = 150 \\
    Var(X) = 150 \cdot 0.7 = 105 \\
    \sigma = \sqrt{105}\\
    28\% \Rightarrow 500 \cdot 0.28 = 140 \\
    32\% \Rightarrow 500 \cdot 0.32 = 160 \\
    P(140 \leq X \leq 160) = P(160 < X) - P(140 < X) \\
    P(160 < X) = P\left(\frac{X - E(X)}{\sigma} < \frac{160 - 150}{\sqrt{105}}\right)
    \Phi\left(\frac{160 - 150}{\sqrt{105}}\right) \\
    \approx \Phi(0.98) \overset{Tabelle}{\approx} 83.65 \%\\
    P(140 < X) = P\left(\frac{X - E(X)}{\sigma} < \frac{140 - 150}{\sqrt{105}}\right)
    \Phi\left(\frac{140 - 150}{\sqrt{105}}\right) \\
    \approx \Phi(-0.98) = 1 - \Phi(0.98) \overset{Tabelle}{\approx} 1 - 0.83646 \approx 16.35\% \\
    P(140 \leq X \leq 160) = 0.8365 - 0.1635 = 67.30\%
\end{align*}

\subsection{Mit Stetigkeitskorrektur}

\begin{align*}
    X \sim B(0.3, 500) \\
    X = \text{Anzahl Anhänger von Partei A} \\
    E(X) = 150 \\
    \sigma = \sqrt{105} \approx 10.247 \\
    \\
    P(140 \leq X \leq 160) \rightarrow P(139.5 \leq X_{stetig} \leq 160.5) \\
    \\
    P(X_{stetig} \leq 160.5) = P\left(\frac{X - E(X)}{\sigma} \leq \frac{160.5 - 150}{\sqrt{105}}\right) \\
    = \Phi\left(\frac{10.5}{\sqrt{105}}\right) \\
    \approx \Phi(1.02) \overset{Tabelle}{\approx} 84.61 \% \\
    \\
    P(X_{stetig} \leq 139.5) = P\left(\frac{X - E(X)}{\sigma} \leq \frac{139.5 - 150}{\sqrt{105}}\right) \\
    = \Phi\left(\frac{-10.5}{\sqrt{105}}\right) \approx \Phi(-1.02) = 1 - \Phi(1.02) \\
    \overset{Tabelle}{\approx} 1 - 0.8461 = 15.39 \% \\
    \\
    P(140 \leq X \leq 160) \approx 0.8461 - 0.1539 = 0.6922 = 69.22\%
\end{align*}

\section{Frage 3}

Ein Würfel trägt $3$ Einser, $2$ Zweier und eine Sechs. Er wird $1000$ mal geworfen. Berechnen Sie die folgenden Wahrscheinlichkeiten, mithilfe des Satzes von De Moivre-Laplace.

\subsection{Ohne Stetigkeitskorrektur}

\begin{align*}
    X \sim B(0.5, 1000) \\
    X = \text{Anzahl gewürfelter Einsen} \\
    E(X) = 1000 \cdot 0.5 = 500 \\
    Var(X) = 500 \cdot 0.5 = 250 \\
    \sigma = \sqrt{250} \\
    P(X < 520) = P\left(\frac{X - E(X)}{\sigma} < \frac{520 - 500}{\sqrt{250}}\right) \\
    \Phi\left(\frac{520 - 500}{\sqrt{250}}\right) \\
    \approx \Phi(1.26) \overset{Tabelle}{\approx} 89.62\%\\\\
    X \sim B(\frac{1}{3}, 1000) \\
    X = \text{Anzahl gewürfelter Zweien} \\
    E(X) = 1000 \cdot \frac{1}{3} = \frac{1000}{3} \\
    Var(X) = \frac{1000}{3} \cdot \frac{2}{3} = \frac{2000}{9} \\
    \sigma = \swarrow\frac{2000}{9} \\
    P(X > 300) = 1 - P(X \leq 300) \\
    = 1 - P\left(\frac{X - E(X)}{\sigma} \leq \frac{300 - \frac{1000}{3}}{\sqrt{\frac{2000}{9}}}\right) \\
    1 - \Phi\left(\frac{300 - \frac{1000}{3}}{\sqrt{\frac{2000}{9}}}\right) \\
    \approx \Phi(-2.24) \approx 1 - (1 - \Phi(2.24)) \\ 
    \overset{Tabelle}{\approx} 1 - (1 - 98.75)\% = 1 - 1 + 98.75\% = 98.75\%  \\\\
    X \sim B(\frac{1}{6}, 1000) \\
    X = \text{Anzahl gewürfelter Sechsen} \\
    E(X) = 1000 \cdot \frac{1}{6} = \frac{500}{3} \\
    Var(X) = \frac{500}{3} \cdot \frac{5}{6} = \frac{1250}{9} \\
    \sigma = \sqrt{\frac{1250}{9}} \\
    P(140 \geq X \geq 180) = P(X < 180) - P(X < 140) \\
    P(X < 180) = P\left(\frac{X - E(X)}{\sigma} < \frac{180 - \frac{500}{3}}{\sqrt{\frac{1250}{9}}}\right)
    \Phi\left(\frac{180 - \frac{500}{3}}{\sqrt{\frac{1250}{9}}}\right) \\
    \approx \Phi(1.13) = \overset{Tabelle}{\approx} 0.87076 = 87.08\%\\
    P(X < 140) = P\left(\frac{X - E(X)}{\sigma} < \frac{140 - \frac{500}{3}}{\sqrt{\frac{1250}{9}}}\right)
    \Phi\left(\frac{140 - \frac{500}{3}}{\sqrt{\frac{1250}{9}}}\right) \\
    \approx \Phi(-2.26) = 1 - \Phi(2.26) = 1 - 0.98809 = 0.0119 = 1.19\% \\
    P(140 \geq X \geq 180) = 0.87076 - 0.0119 = 0.8217 = 85.89\%
\end{align*}

\subsection{Mit Stetigkeitskorrektur}

\begin{align*}
    X \sim B(0.5, 1000) \\
    X = \text{Anzahl gewürfelter Einsen} \\
    E(X) = 1000 \cdot 0.5 = 500 \\
    Var(X) = 500 \cdot 0.5 = 250 \\
    \sigma = \sqrt{250} \\
    P(X < 520) = P(X \leq 519) \approx P(X \leq 519.5) \\
    P\left(\frac{X - E(X)}{\sigma} \leq \frac{519.5 - 500}{\sqrt{250}}\right) \\
    \Phi\left(\frac{519.5 - 500}{\sqrt{250}}\right) \\
    \approx \Phi(1.23) \overset{Tabelle}{\approx} 89.07\% \\\\
    X \sim B(\frac{1}{3}, 1000) \\
    X = \text{Anzahl gewürfelter Zweien} \\
    E(X) = 1000 \cdot \frac{1}{3} = \frac{1000}{3} \\
    Var(X) = \frac{1000}{3} \cdot \frac{2}{3} = \frac{2000}{9} \\
    \sigma = \sqrt{\frac{2000}{9}} \\
    P(X > 300) = P(X \geq 301) \approx P(X \geq 300.5) \\
    = 1 - P(X < 300.5) \\
    = 1 - P\left(\frac{X - E(X)}{\sigma} < \frac{300.5 - \frac{1000}{3}}{\sqrt{\frac{2000}{9}}}\right) \\
    1 - \Phi\left(\frac{300.5 - \frac{1000}{3}}{\sqrt{\frac{2000}{9}}}\right) \\
    \approx 1 - \Phi(-2.20) = \Phi(2.20) \\
    \overset{Tabelle}{\approx} 98.61\% \\\\
    X \sim B(\frac{1}{6}, 1000) \\
    X = \text{Anzahl gewürfelter Sechsen} \\
    E(X) = 1000 \cdot \frac{1}{6} = \frac{500}{3} \\
    Var(X) = \frac{500}{3} \cdot \frac{5}{6} = \frac{1250}{9} \\
    \sigma = \sqrt{\frac{1250}{9}} \\
    P(140 \leq X \leq 180) \approx P(139.5 \leq X \leq 180.5) \\
    P(X \leq 180.5) - P(X \leq 139.5) \\
    P(X \leq 180.5) = \Phi\left(\frac{180.5 - \frac{500}{3}}{\sqrt{\frac{1250}{9}}}\right) \\
    \approx \Phi(1.17) \overset{Tabelle}{\approx} 0.8790 = 87.90\%\\
    P(X \leq 139.5) = \Phi\left(\frac{139.5 - \frac{500}{3}}{\sqrt{\frac{1250}{9}}}\right) \\
    \approx \Phi(-2.31) = 1 - \Phi(2.31) \overset{Tabelle}{\approx} 1 - 0.9896 = 0.0104 = 1.04\% \\
    P(139.5 \leq X \leq 180.5) = 0.8790 - 0.0104 = 0.8686 = 86.86\%
\end{align*}

\section{Frage 4}

In einer Bonbon-Tüte gibt es grüne, blaue und rote Bonbons. Die Anzahl der blauen Bonbons kann mit einer Normalverteilung beschrieben werden. Hierbei liegt der Erwartungswert bei $24$ und die Standardabweichung bei $4$.

\begin{align*}
    X \sim N(24, 4) \\
    X = Anzahl blauer Bonbons \\
    P(X > 19) = P\left(\frac{X - E(X)}{\sigma} > \frac{19 - 24}{4}\right) \\
    \Phi(\frac{19 - 24}{4}) \\
    \approx \Phi(-1.25) = 1 - \Phi(1.25) \overset{Tabelle}{\approx} 1 - 0.89435 = 0.10565 = 19.57\%
\end{align*}

Wie muss der Erwartungswert gewählt werden, so dass die Wahrscheinlichkeit bei weniger als $19$ blauen Bonbons bei $95\%$ liegt?

\begin{align*}
    
\end{align*}