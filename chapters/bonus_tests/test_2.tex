\chapter{Bonustest 2 - Skalarprodukt, Kreuzprodukt, lineare Gleichungssystem, Matzitzen und lineare Abbildungen}

\section{Frage 1}

Bilden die Vektoren 

\[
\vec{a} = \begin{pmatrix}
    6 \\ 0 \\ 7 \\ -7
\end{pmatrix}, \vec{b} = \begin{pmatrix}
    1 \\3 \\ -1 \\ 0
\end{pmatrix}, \vec{c} = \begin{pmatrix}
    -2 \\ 3 \\ 0 \\ 5
\end{pmatrix} \text{ und } \vec{d} = \begin{pmatrix}
    1 \\ -1 \\ 0 \\ 1
\end{pmatrix}
\]

Eine basis des $\mathbb{R}^4$?

Hierfür müssen die Vektoren auf Lineare unabhängigkeit geprüft werden

\begin{align*}
    x_1 \begin{pmatrix}
        6 \\ 0 \\ 7 \\ -7
    \end{pmatrix} + x_2 \begin{pmatrix}
        1 \\3 \\ -1 \\ 0
    \end{pmatrix} + x_3 \begin{pmatrix}
        -2 \\ 3 \\ 0 \\ 5
    \end{pmatrix} + x_4 \begin{pmatrix}
        1 \\ -1 \\ 0 \\ 1
    \end{pmatrix} = \begin{pmatrix}
        0 \\ 0 \\ 0 \\ 0
    \end{pmatrix}
\end{align*}

\begin{longtable}{p{10cm}}
    \hline
    \multicolumn{1}{c}{\textbf{Linearkombination}} \\
    \hline
    \endfirsthead

    \hline
    \multicolumn{1}{c}{\tablename\ \thetable\ -- \textit{Fortführung von vorherier Seite}} \\
    \hline
    \multicolumn{1}{c}{\textbf{Linearkombination}} \\
    \hline
    \endhead

    \hline
    \multicolumn{1}{r}{\textit{Fortsetzung siehe nächste Seite}} \\
    \endfoot

    \hline
    \endlastfoot

    $\displaystyle\begin{matrix}
        6 & 1 & -2 & 1 \\
        0 & 3 & 3 & -1 \\
        7 & -1 & 0 & 0 \\
        -7 & 0 & 5 & 1   
    \end{matrix}$\\\hline
    III + IV \\\hline\pagebreak[0]
    $\displaystyle\begin{matrix}
        6 & 1 & -2 & 1 \\
        0 & 3 & 3 & -1 \\
        0 & -1 & -5 & -1 \\
        -7 & 0 & 5 & 1   
    \end{matrix}$\\\hline
    6IV + 7I \\\hline\pagebreak[0]
    $\displaystyle\begin{matrix}
        6 & 1 & -2 & 1 \\
        0 & 3 & 3 & -1 \\
        0 & -1 & -5 & -1 \\
        0 & 7 & 16 & 13  
    \end{matrix}$\\\hline
    IV + 7III \\\hline\pagebreak[0]
    $\displaystyle\begin{matrix}
        6 & 1 & -2 & 1 \\
        0 & 3 & 3 & -1 \\
        0 & -1 & -5 & -1 \\
        0 & 0 & -19 & 6  
    \end{matrix}$\\\hline
    3III + II \\\hline\pagebreak[0]
    $\displaystyle\begin{matrix}
        6 & 1 & -2 & 1 \\
        0 & 3 & 3 & -1 \\
        0 & 0 & -12 & -4 \\
        0 & 0 & -19 & 6  
    \end{matrix}$\\\hline
    -12IV + 19III \\\hline\pagebreak[0]
    $\displaystyle\begin{matrix}
        6 & 1 & -2 & 1 \\
        0 & 3 & 3 & -1 \\
        0 & 0 & -12 & -4 \\
        0 & 0 & 0 & -148  
    \end{matrix}$\\\hline
    IV : -148 \\\hline\pagebreak[0]
    $\displaystyle\begin{matrix}
        6 & 1 & -2 & 1 \\
        0 & 3 & 3 & -1 \\
        0 & 0 & -12 & -4 \\
        0 & 0 & 0 & 1  
    \end{matrix}$\\\hline
    III + 4IV \\\hline\pagebreak[0]
    $\displaystyle\begin{matrix}
        6 & 1 & -2 & 1 \\
        0 & 3 & 3 & -1 \\
        0 & 0 & -12 & 0 \\
        0 & 0 & 0 & 1  
    \end{matrix}$\\\hline
    II + IV \\\hline\pagebreak[0]
    $\displaystyle\begin{matrix}
        6 & 1 & -2 & 1 \\
        0 & 3 & 3 & 0 \\
        0 & 0 & -12 & 0 \\
        0 & 0 & 0 & 1  
    \end{matrix}$\\\hline
    I - IV \\\hline\pagebreak[0]
    $\displaystyle\begin{matrix}
        6 & 1 & -2 & 0 \\
        0 & 3 & 3 & 0 \\
        0 & 0 & -12 & 0 \\
        0 & 0 & 0 & 1  
    \end{matrix}$\\\hline
    III : -12 \\\hline\pagebreak[0]
    $\displaystyle\begin{matrix}
        6 & 1 & -2 & 0 \\
        0 & 3 & 3 & 0 \\
        0 & 0 & 1 & 0 \\
        0 & 0 & 0 & 1  
    \end{matrix}$\\\hline
    II - 3III \\\hline\pagebreak[0]
    $\displaystyle\begin{matrix}
        6 & 1 & -2 & 0 \\
        0 & 3 & 0 & 0 \\
        0 & 0 & 1 & 0 \\
        0 & 0 & 0 & 1  
    \end{matrix}$\\\hline
    I + 2III \\\hline\pagebreak[0]
    $\displaystyle\begin{matrix}
        6 & 1 & 0 & 0 \\
        0 & 3 & 0 & 0 \\
        0 & 0 & 1 & 0 \\
        0 & 0 & 0 & 1  
    \end{matrix}$\\\hline
    II : 3 \\\hline\pagebreak[0]
    $\displaystyle\begin{matrix}
        6 & 1 & 0 & 0 \\
        0 & 1 & 0 & 0 \\
        0 & 0 & 1 & 0 \\
        0 & 0 & 0 & 1  
    \end{matrix}$\\\hline
    I - II \\\hline\pagebreak[0]
    $\displaystyle\begin{matrix}
        6 & 0 & 0 & 0 \\
        0 & 1 & 0 & 0 \\
        0 & 0 & 1 & 0 \\
        0 & 0 & 0 & 1  
    \end{matrix}$\\\hline
    I : 6 \\\hline\pagebreak[0]
    $\displaystyle\begin{matrix}
        1 & 0 & 0 & 0 \\
        0 & 1 & 0 & 0 \\
        0 & 0 & 1 & 0 \\
        0 & 0 & 0 & 1  
    \end{matrix}$\\\hline
\end{longtable}

Da das Gleichungssystem bestimmt und nicht unterbestimmt ist, bilden die Vektoren eine Basis des $\mathbb{R}^4$.

\section{Frage 2}

Seien $a, b \in \mathbb{R}$ Parameter und 

\[
    \vec{u}= \begin{pmatrix}
        -4 \\ a \\1
    \end{pmatrix}, \vec{v} = \begin{pmatrix}
        3 \\ 1 \\ b
    \end{pmatrix}, \vec{w} = \begin{pmatrix}
        2 \\ 2 \\ -4
    \end{pmatrix}
\]

Vektoren im $\mathbb{R}^3$.

Bestimmen Sie Werte der Parameter $a$ und $b$, sodass $\vec{u}, \vec{v}$ und $\vec{w}$ linear abhängig sind. Geben Sie bei mehreren Lösungen eine für $a$ und $b$ beispielhaft an.

Ein möglicher Lösungsansatz besteht darin, die Bedingung für lineare Abhängigkeit zu nutzen. Drei Vektoren im $\mathbb{R}^3$ sind genau dann linear abhängig, wenn die Determinante der Matrix, die aus diesen Vektoren als Spalten gebildet wird, gleich Null ist.

Die aufgestellte Matrix lautet:
\[
A = \begin{pmatrix}
    -4 & 3 & 2 \\
    a & 1 & 2 \\
    1 & b & -4
\end{pmatrix}
\]
Die Determinante dieser Matrix wird berechnet:
\begin{align*}
    \det(A) &= (-4 \cdot 1 \cdot -4) + (3 \cdot 2 \cdot 1) + (2 \cdot a \cdot b) - (1 \cdot 1 \cdot 2) - (b \cdot 2 \cdot -4) - (-4 \cdot a \cdot 3) \\
    &= 16 + 6 + 2ab - 2 - (-8b) - (-12a) \\
    &= 20 + 2ab + 8b + 12a
\end{align*}
Für lineare Abhängigkeit muss die Determinante Null sein:
\[
    12a + 8b + 2ab + 20 = 0
\]
Diese Gleichung lässt sich durch 2 dividieren, um sie zu vereinfachen:
\[
    6a + 4b + ab + 10 = 0
\]
Es gibt unendlich viele Paare $(a, b)$, die diese Gleichung erfüllen. Um eine beispielhafte Lösung anzugeben, kann ein Wert für einen der Parameter frei gewählt und der andere berechnet werden.

\textbf{Beispiel 1: $a=0$}
Wird $a=0$ in die Gleichung eingesetzt, ergibt sich:
\begin{align*}
    6(0) + 4b + (0)b + 10 &= 0 \\
    4b + 10 &= 0 \\
    4b &= -10 \\
    b &= -\frac{10}{4} = -\frac{5}{2}
\end{align*}
Ein mögliches Wertepaar ist $a=0$ und $b = -2.5$.

\textbf{Beispiel 2: $b=1$}
Wird $b=1$ in die Gleichung eingesetzt, ergibt sich:
\begin{align*}
    6a + 4(1) + a(1) + 10 &= 0 \\
    7a + 14 &= 0 \\
    7a &= -14 \\
    a &= -2
\end{align*}
Ein weiteres mögliches Wertepaar ist $a=-2$ und $b=1$.

\section{Frage 3}

Bestimmen Sie das folgende Kreuzprodukt

\begin{align*}
    \begin{pmatrix}
        1 \\ 9 \\ -2
    \end{pmatrix} \times \begin{pmatrix}
        2 \\ -1 \\ -1
    \end{pmatrix} \\
    \begin{pmatrix}
        9 \cdot -1 - -2 \cdot -1 \\
        -2 \cdot 2 - 1 \cdot -1 \\
        1 \cdot -1 - 9 \cdot 2
    \end{pmatrix} \\
    \begin{pmatrix}
        -9 - 2\\
        -2 - 1 \\
        -1 - 18
    \end{pmatrix} \\
    \begin{pmatrix}
        -11 \\ -3 \\ -19
    \end{pmatrix}
\end{align*}

\section{Frage 4}

Bestimmen Sie das folgende Kreuzprodukt

\begin{align*}
    \begin{pmatrix}
        1 \\ 3 \\ -2
    \end{pmatrix} \times \begin{pmatrix}
        2 \\ 2 \\ 2
    \end{pmatrix} \\
    \begin{pmatrix}
        3 \cdot 2 - -2 \cdot 2 \\
        -2  \cdot 2 - 1 \cdot 2 \\
        1 \cdot 2 - 3 \cdot 2
    \end{pmatrix} \\
    \begin{pmatrix}
        6 - -4 \\
        -4 - 2 \\
        2 - 6
    \end{pmatrix} \\
    \begin{pmatrix}
        10 \\ -6 \\ -4
    \end{pmatrix}
\end{align*}

\section{Frage 5}

Ein Tetraeder wird als Modell eines Methanmoleküls verwendet. Dabei stellen die vier Eckpunkte die vier Wasserstoffatome und der Punkt $C(1,1,1)$ das Kohlenstoffatom dar.

Die Wasserstoffatome befinden sich im Tetraeder an den Eckpnkten $H_1(2,0,0), H_2(0,2,0), H_3(0,0,2)$ und $H_4(2,2,2)$.

Für diese Aufgabe ist die Abbildung des Moleküls irrelevant.

Der winkel $\alpha$ zwischen den Strecken $\overline{CH_1}$ und $\overline{CH_2}$ wird Bindungswinkel genannt. Berechnen Sie den Bindungswinkel im Methanmolekül in Grad und auf zwei Nachkommastellen genau.

\begin{align*}
    \overline{CH_1} = H_1 - C \\
    = \begin{pmatrix}
        2 \\ 0 \\ 0
    \end{pmatrix} - \begin{pmatrix}
        1 \\ 1 \\ 1
    \end{pmatrix} \\
    = \begin{pmatrix}
        1 \\ -1 \\ -1
    \end{pmatrix} \\\\
    \overline{CH_2} = H_2 - C \\
    = \begin{pmatrix}
        0 \\ 2 \\ 0
    \end{pmatrix} - \begin{pmatrix}
        1 \\ 1 \\ 1
    \end{pmatrix} \\
    = \begin{pmatrix}
        -1 \\ 1 \\ -1
    \end{pmatrix} \\\\ \pagebreak[2]
    \cos(\theta) = \frac{\left\langle \overline{CH_1}, \overline{CH_2} \right\rangle}{\left|\overline{CH_1}\right| \cdot \left|\overline{CH_2}\right|} \\\\
    \left\langle \overline{CH_1}, \overline{CH_2} \right\rangle  \\
    = \left\langle \begin{pmatrix}
        1 \\ -1 \\ -1
    \end{pmatrix}, \begin{pmatrix}
        -1 \\ 1 \\ -1
    \end{pmatrix} \right\rangle  \\
    = 1 \cdot -1 + -1 \cdot 1 + -1 \cdot -1 \\
    = -1 + -1 + 1 \\
    = -1 \\\\ \pagebreak[2]
    \left|\overline{CH_1}\right| \\
    = \left|\begin{pmatrix}
        1 \\ -1 \\ -1
    \end{pmatrix}\right| \\
    = \sqrt{1^2 + (-1)^2 + (-1)^2} \\
    = \sqrt{1 + 1 + 1} \\
    \sqrt{3} \\\\ \pagebreak[2]
    \left|\overline{CH_2}\right| \\
    = \left|\begin{pmatrix}
        -1 \\ 1 \\ -1
    \end{pmatrix}\right| \\
    = \sqrt{(-1)^2 + 1^2 + (-1)^2} \\
    = \sqrt{1 + 1 + 1} \\
    \sqrt{3} \\\\ \pagebreak[2]
    \cos(\theta) = \frac{\left\langle \overline{CH_1}, \overline{CH_2} \right\rangle}{\left|\overline{CH_1}\right| \cdot \left|\overline{CH_2}\right|} \\
    = \frac{-1}{\sqrt{3} \cdot \sqrt{3}} \\
    \arccos(\frac{-1}{\sqrt{3} \cdot \sqrt{3}}) \cdot \frac{180}{\pi} \\
    \approx 109.47^\circ
\end{align*}

\section{Frage 6}

Gegeben sei die Abbildung $T : \mathbb{R}^3 \rightarrow \mathbb{R}^3$ mit $T(x,y,z) = \begin{bmatrix}
    2x - y - z \\
    z \\
    x + y
\end{bmatrix}$

Berechnen Sie die Abbildungsmatrix $A$ von $T$ bezüglich der Standardbasis $\mathbb{R}^3$.

Die Spalten der Abbildungsmatrix \(A\) sind die Bilder der Standardbasisvektoren unter der linearen Abbildung \(T\). Die Standardbasis des \(\mathbb{R}^3\) besteht aus den Vektoren:
\[
e_1 = \begin{pmatrix} 1 \\ 0 \\ 0 \end{pmatrix}, \quad
e_2 = \begin{pmatrix} 0 \\ 1 \\ 0 \end{pmatrix}, \quad
e_3 = \begin{pmatrix} 0 \\ 0 \\ 1 \end{pmatrix}
\]

Die Bilder dieser Basisvektoren werden wie folgt berechnet:
\begin{enumerate}
    \item \textbf{Erste Spalte:}
    \[
        T(e_1) = T(1,0,0) = \begin{pmatrix} 2(1) - 0 - 0 \\ 0 \\ 1 + 0 \end{pmatrix} = \begin{pmatrix} 2 \\ 0 \\ 1 \end{pmatrix}
    \]
    
    \item \textbf{Zweite Spalte:}
    \[
        T(e_2) = T(0,1,0) = \begin{pmatrix} 2(0) - 1 - 0 \\ 0 \\ 0 + 1 \end{pmatrix} = \begin{pmatrix} -1 \\ 0 \\ 1 \end{pmatrix}
    \]
    
    \item \textbf{Dritte Spalte:}
    \[
        T(e_3) = T(0,0,1) = \begin{pmatrix} 2(0) - 0 - 1 \\ 1 \\ 0 + 0 \end{pmatrix} = \begin{pmatrix} -1 \\ 1 \\ 0 \end{pmatrix}
    \]
\end{enumerate}

Zusammensetzen der Spalten ergibt die Abbildungsmatrix \(A\):
\[
A = \begin{pmatrix} 2 & -1 & -1 \\ 0 & 0 & 1 \\ 1 & 1 & 0 \end{pmatrix}
\]

\section{Frage 7} 

Sei $\mathbb{P}_n(\mathbb{R})$ der Vektorraum der reellen Polynome vom Grad kleiner oder gleich $n$.

Sei $T : \mathbb{P}_3 \rightarrow \mathbb{P}_2(\mathbb{R})$ die Abbilfung mit $T(f) := f'$.

Berechnen sie die Dimension vom $Bild(T)$.

Gegeben ist der Vektorraum $\mathbb{P}_n(\mathbb{R})$ der reellen Polynome vom Grad kleiner oder gleich $n$.
Die lineare Abbildung $T: \mathbb{P}_3(\mathbb{R}) \to \mathbb{P}_2(\mathbb{R})$ ist definiert durch die Ableitung:
\[ T(f) = f' \]
Gesucht ist die Dimension des Bildes von $T$, geschrieben als $\dim(\text{Bild}(T))$.

\subsection*{Anwendung des Dimensionssatzes}
Der Dimensionssatz (auch Rang-Defekt-Satz) für eine lineare Abbildung $T: V \to W$ lautet:
\[ \dim(V) = \dim(\text{Kern}(T)) + \dim(\text{Bild}(T)) \]

\begin{enumerate}
    \item \textbf{Dimension des Urbildraums bestimmen:}
    Der Urbildraum ist $V = \mathbb{P}_3(\mathbb{R})$. Eine Basis für diesen Raum ist $\{1, x, x^2, x^3\}$. Die Anzahl der Basisvektoren ist 4.
    \[ \dim(\mathbb{P}_3) = 4 \]

    \item \textbf{Dimension des Kerns bestimmen:}
    Der Kern von $T$ enthält alle Polynome $f \in \mathbb{P}_3(\mathbb{R})$, für die $T(f) = 0$ gilt.
    \[ T(f) = f' = 0 \]
    Dies ist genau dann der Fall, wenn das Polynom $f$ eine Konstante ist, d.h. $f(x) = c$ für ein $c \in \mathbb{R}$. Der Kern besteht also aus allen konstanten Polynomen.
    \[ \text{Kern}(T) = \{c \mid c \in \mathbb{R}\} = \text{span}\{1\} \]
    Eine Basis des Kerns ist das Polynom $\{1\}$. Die Dimension des Kerns ist somit:
    \[ \dim(\text{Kern}(T)) = 1 \]

    \item \textbf{Dimension des Bildes berechnen:}
    Durch Umstellen des Dimensionssatzes ergibt sich:
    \[ \dim(\text{Bild}(T)) = \dim(\mathbb{P}_3) - \dim(\text{Kern}(T)) \]
    Einsetzen der bekannten Werte:
    
    \[ \dim(\text{Bild}(T)) = 4 - 1 = 3 \]
\end{enumerate}
