\chapter{Bonustest 2 - Skalarprodukt, Kreuzprodukt, lineare Gleichungssystem, Matzitzen und lineare Abbildungen}

\section{Aufgabe 1}

Bilden die Vektoren 

\[
\vec{a} = \begin{pmatrix}
    6 \\ 0 \\ 7 \\ -7
\end{pmatrix}, \vec{b} = \begin{pmatrix}
    1 \\3 \\ -1 \\ 0
\end{pmatrix}, \vec{c} = \begin{pmatrix}
    -2 \\ 3 \\ 0 \\ 5
\end{pmatrix} \text{ und } \vec{d} = \begin{pmatrix}
    1 \\ -1 \\ 0 \\ 1
\end{pmatrix}
\]

Eine basis des $\mathbb{R}^4$?

Hierfür müssen die Vektoren auf Lineare unabhängigkeit geprüft werden

\begin{align*}
    x_1 \begin{pmatrix}
        6 \\ 0 \\ 7 \\ -7
    \end{pmatrix} + x_2 \begin{pmatrix}
        1 \\3 \\ -1 \\ 0
    \end{pmatrix} + x_3 \begin{pmatrix}
        -2 \\ 3 \\ 0 \\ 5
    \end{pmatrix} + x_4 \begin{pmatrix}
        1 \\ -1 \\ 0 \\ 1
    \end{pmatrix} = \begin{pmatrix}
        0 \\ 0 \\ 0 \\ 0
    \end{pmatrix}
\end{align*}

\begin{longtable}{p{10cm}}
    \hline
    \multicolumn{1}{c}{\textbf{Linearkombination}} \\
    \hline
    \endfirsthead

    \hline
    \multicolumn{1}{c}{\tablename\ \thetable\ -- \textit{Fortführung von vorherier Seite}} \\
    \hline
    \multicolumn{1}{c}{\textbf{Linearkombination}} \\
    \hline
    \endhead

    \hline
    \multicolumn{1}{r}{\textit{Fortsetzung siehe nächste Seite}} \\
    \endfoot

    \hline
    \endlastfoot

    $\displaystyle\begin{matrix}
        6 & 1 & -2 & 1 \\
        0 & 3 & 3 & -1 \\
        7 & -1 & 0 & 0 \\
        -7 & 0 & 5 & 1   
    \end{matrix}$\\\hline
    III + IV \\\hline\pagebreak[0]
    $\displaystyle\begin{matrix}
        6 & 1 & -2 & 1 \\
        0 & 3 & 3 & -1 \\
        0 & -1 & -5 & -1 \\
        -7 & 0 & 5 & 1   
    \end{matrix}$\\\hline
    6IV + 7I \\\hline\pagebreak[0]
    $\displaystyle\begin{matrix}
        6 & 1 & -2 & 1 \\
        0 & 3 & 3 & -1 \\
        0 & -1 & -5 & -1 \\
        0 & 7 & 16 & 13  
    \end{matrix}$\\\hline
    IV + 7III \\\hline\pagebreak[0]
    $\displaystyle\begin{matrix}
        6 & 1 & -2 & 1 \\
        0 & 3 & 3 & -1 \\
        0 & -1 & -5 & -1 \\
        0 & 0 & -19 & 6  
    \end{matrix}$\\\hline
    3III + II \\\hline\pagebreak[0]
    $\displaystyle\begin{matrix}
        6 & 1 & -2 & 1 \\
        0 & 3 & 3 & -1 \\
        0 & 0 & -12 & -4 \\
        0 & 0 & -19 & 6  
    \end{matrix}$\\\hline
    -12IV + 19III \\\hline\pagebreak[0]
    $\displaystyle\begin{matrix}
        6 & 1 & -2 & 1 \\
        0 & 3 & 3 & -1 \\
        0 & 0 & -12 & -4 \\
        0 & 0 & 0 & -148  
    \end{matrix}$\\\hline
    IV : -148 \\\hline\pagebreak[0]
    $\displaystyle\begin{matrix}
        6 & 1 & -2 & 1 \\
        0 & 3 & 3 & -1 \\
        0 & 0 & -12 & -4 \\
        0 & 0 & 0 & 1  
    \end{matrix}$\\\hline
    III + 4IV \\\hline\pagebreak[0]
    $\displaystyle\begin{matrix}
        6 & 1 & -2 & 1 \\
        0 & 3 & 3 & -1 \\
        0 & 0 & -12 & 0 \\
        0 & 0 & 0 & 1  
    \end{matrix}$\\\hline
    II + IV \\\hline\pagebreak[0]
    $\displaystyle\begin{matrix}
        6 & 1 & -2 & 1 \\
        0 & 3 & 3 & 0 \\
        0 & 0 & -12 & 0 \\
        0 & 0 & 0 & 1  
    \end{matrix}$\\\hline
    I - IV \\\hline\pagebreak[0]
    $\displaystyle\begin{matrix}
        6 & 1 & -2 & 0 \\
        0 & 3 & 3 & 0 \\
        0 & 0 & -12 & 0 \\
        0 & 0 & 0 & 1  
    \end{matrix}$\\\hline
    III : -12 \\\hline\pagebreak[0]
    $\displaystyle\begin{matrix}
        6 & 1 & -2 & 0 \\
        0 & 3 & 3 & 0 \\
        0 & 0 & 1 & 0 \\
        0 & 0 & 0 & 1  
    \end{matrix}$\\\hline
    II - 3III \\\hline\pagebreak[0]
    $\displaystyle\begin{matrix}
        6 & 1 & -2 & 0 \\
        0 & 3 & 0 & 0 \\
        0 & 0 & 1 & 0 \\
        0 & 0 & 0 & 1  
    \end{matrix}$\\\hline
    I + 2III \\\hline\pagebreak[0]
    $\displaystyle\begin{matrix}
        6 & 1 & 0 & 0 \\
        0 & 3 & 0 & 0 \\
        0 & 0 & 1 & 0 \\
        0 & 0 & 0 & 1  
    \end{matrix}$\\\hline
    II : 3 \\\hline\pagebreak[0]
    $\displaystyle\begin{matrix}
        6 & 1 & 0 & 0 \\
        0 & 1 & 0 & 0 \\
        0 & 0 & 1 & 0 \\
        0 & 0 & 0 & 1  
    \end{matrix}$\\\hline
    I - II \\\hline\pagebreak[0]
    $\displaystyle\begin{matrix}
        6 & 0 & 0 & 0 \\
        0 & 1 & 0 & 0 \\
        0 & 0 & 1 & 0 \\
        0 & 0 & 0 & 1  
    \end{matrix}$\\\hline
    I : 6 \\\hline\pagebreak[0]
    $\displaystyle\begin{matrix}
        1 & 0 & 0 & 0 \\
        0 & 1 & 0 & 0 \\
        0 & 0 & 1 & 0 \\
        0 & 0 & 0 & 1  
    \end{matrix}$\\\hline
\end{longtable}

Da das Gleichungssystem bestimmt und nicht unterbestimmt ist, bilden die Vektoren eine Basis des $\mathbb{R}^4$.

\section{Aufgabe 2}

Seien $a, b \in \mathbb{R}$ Parameter und 

\[
    \vec{u}= \begin{pmatrix}
        -4 \\ a \\1
    \end{pmatrix}, \vec{v} = \begin{pmatrix}
        3 \\ 1 \\ b
    \end{pmatrix}, \vec{w} = \begin{pmatrix}
        2 \\ 2 \\ -4
    \end{pmatrix}
\]

Vektoren im $\mathbb{R}^3$.

Bestimmen Sie Werte der Parameter $a$ und $b$, sodass $\vec{u}, \vec{v}$ und $\vec{w}$ linear abhängig sind. Geben Sie bei mehreren Lösungen eine für $a$ und $b$ beispielhaft an.

\begin{align*}
    x_1 \vec{u}  + x_2 \vec{v} + x_3 \vec{w} = \vec{0} \\
    x_1 \begin{pmatrix}
        -4 \\ a \\ 1
    \end{pmatrix} + x_2 \begin{pmatrix}
        3 \\ 1 \\ b
    \end{pmatrix} + x_3 \begin{pmatrix}
        2 \\ 2 \\ -4
    \end{pmatrix} = \begin{pmatrix}
        0 \\ 0 \\0
    \end{pmatrix} \\
    \begin{pmatrix}
        -4x_1 \\ ax_1 \\ x_1
    \end{pmatrix} + \begin{pmatrix}
        3x_2 \\ x_2 \\ bx_2
    \end{pmatrix} + \begin{pmatrix}
        2x_3 \\ 2x_3 \\ -4x_3
    \end{pmatrix} \\
    \begin{cases}
        \text{I:\@} & -4x_1 + 3x_2 + 2x_3 = 0 \\
        \text{II:\@} & ax_1 + x_2 + 2x_3 = 0 \\
        \text{III:\@} & x_1 + bx_2 -4x_3 = 0
    \end{cases} \\
\end{align*}