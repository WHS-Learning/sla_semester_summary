\chapter{Bonustest 3 - Invertieren von Matzitzen}

\section{Frage 1}

Bestimmen Sie die Umkehrabbildung für

\[
    A = \begin{pmatrix}
        3 & 4 & 3 \\
        4 & 1 & 1 \\
        2 & 1 & 2
    \end{pmatrix}
\]

zunächst muss geprüft werden, ob die Matrx invertierbar ist. Eine Matrix,
welche Invertierhar hat, besitzt eine $\det(A) \neq 0$.

\begin{align*}
    \det(A) = 3 \cdot 1 \cdot 2 + 4 \cdot 1 \cdot 2 + 3 \cdot 4 \cdot 1 \\
    - 2 \cdot 1 \cdot 3 - 1 \cdot 1 \cdot 3 - 2 \cdot 4 \cdot 4         \\
    = 6 + 8 + 12 - 6 - 3 - 32                                           \\
    = -15
\end{align*}

Da $\det(A) \neq 0$, ist die Matrix invertierbar. Nun kann die Matrix
invertiert werden. Hierfür wird das \nameref{gauss_jordan_verfahren} verwendet.

\begin{longtable}{p{4cm}|p{3cm}}

    \hline
    \multicolumn{1}{c|}{\textbf{Matrix}} & \multicolumn{1}{c}{\textbf{Einheitsmatrix}}     \\
    \hline
    \endfirsthead

    \hline
    \multicolumn{2}{c}{\tablename\ \thetable\ -- \textit{Fortführung von vorherier Seite}} \\
    \hline
    \multicolumn{1}{c|}{\textbf{Matrix}} & \multicolumn{1}{c}{\textbf{Einheitsmatrix}}     \\
    \hline
    \endhead

    \hline
    \multicolumn{2}{r}{\textit{Fortsetzung siehe nächste Seite}}                           \\
    \endfoot

    \hline
    \endlastfoot

    $\displaystyle\begin{matrix}
                          3 & 4 & 3 \\
                          4 & 1 & 1 \\
                          2 & 1 & 2
                      \end{matrix}$         &
    $\displaystyle\begin{matrix}
                          1 & 0 & 0 \\
                          0 & 1 & 0 \\
                          0 & 0 & 1
                      \end{matrix}$                                                            \\\hline
    \multicolumn{2}{p{\dimexpr4cm+3cm+2\tabcolsep\relax}}{Operation: II - 2III}            \\\hline\pagebreak[0]

    $\displaystyle\begin{matrix}
                          3 & 4  & 3  \\
                          0 & -1 & -3 \\
                          2 & 1  & 2
                      \end{matrix}$         &
    $\displaystyle\begin{matrix}
                          1 & 0 & 0  \\
                          0 & 1 & -2 \\
                          0 & 0 & 1
                      \end{matrix}$                                                            \\\hline
    \multicolumn{2}{p{\dimexpr4cm+3cm+2\tabcolsep\relax}}{Operation: 3III - 2I}            \\\hline\pagebreak[0]

    $\displaystyle\begin{matrix}
                          3 & 4  & 3  \\
                          0 & -1 & -3 \\
                          0 & -5 & 0
                      \end{matrix}$         &
    $\displaystyle\begin{matrix}
                          1  & 0 & 0  \\
                          0  & 1 & -2 \\
                          -2 & 0 & 3
                      \end{matrix}$                                                            \\\hline
    \multicolumn{2}{p{\dimexpr4cm+3cm+2\tabcolsep\relax}}{Operation: I + 4II}              \\\hline\pagebreak[0]

    $\displaystyle\begin{matrix}
                          3 & 0  & -9 \\
                          0 & -1 & -3 \\
                          0 & -5 & 0
                      \end{matrix}$         &
    $\displaystyle\begin{matrix}
                          1  & 4 & -8 \\
                          0  & 1 & -2 \\
                          -2 & 0 & 3
                      \end{matrix}$                                                            \\\hline
    \multicolumn{2}{p{\dimexpr4cm+3cm+2\tabcolsep\relax}}{Operation: III - 5II}            \\\hline\pagebreak[0]

    $\displaystyle\begin{matrix}
                          3 & 0  & -9 \\
                          0 & -1 & -3 \\
                          0 & 0  & 15
                      \end{matrix}$         &
    $\displaystyle\begin{matrix}
                          1  & 4  & -8 \\
                          0  & 1  & -2 \\
                          -2 & -5 & 13
                      \end{matrix}$                                                            \\\hline
    \multicolumn{2}{p{\dimexpr4cm+3cm+2\tabcolsep\relax}}{Operation: 15II + 3III}          \\\hline\pagebreak[0]

    $\displaystyle\begin{matrix}
                          3 & 0   & -9 \\
                          0 & -15 & 0  \\
                          0 & 0   & 15
                      \end{matrix}$         &
    $\displaystyle\begin{matrix}
                          1  & 4  & -8 \\
                          -6 & 0  & 9  \\
                          -2 & -5 & 13
                      \end{matrix}$                                                            \\\hline

    \multicolumn{2}{p{\dimexpr4cm+3cm+2\tabcolsep\relax}}{Operation: 15I + 9III}           \\\hline\pagebreak[0]
    $\displaystyle\begin{matrix}
                          45 & 0   & 0  \\
                          0  & -15 & 0  \\
                          0  & 0   & 15
                      \end{matrix}$         &
    $\displaystyle\begin{matrix}
                          -3 & 15 & -3 \\
                          -6 & 0  & 9  \\
                          -2 & -5 & 13
                      \end{matrix}$                                                            \\\hline

    \multicolumn{2}{p{\dimexpr4cm+3cm+2\tabcolsep\relax}}{Operation: I : 45}               \\\hline\pagebreak[0]
    \multicolumn{2}{p{\dimexpr4cm+3cm+2\tabcolsep\relax}}{Operation: II : -15}             \\\hline\pagebreak[0]
    \multicolumn{2}{p{\dimexpr4cm+3cm+2\tabcolsep\relax}}{Operation: III : 15}             \\\hline\pagebreak[0]
    $\displaystyle\begin{matrix}
                          0 & 0 & 0 \\
                          0 & 0 & 0 \\
                          0 & 0 & 0
                      \end{matrix}$         &
    $\displaystyle\begin{matrix}
                          -\frac{1}{15} & \frac{1}{3}  & -\frac{1}{15} \\
                          \frac{2}{5}   & 0            & -\frac{3}{5}  \\
                          -\frac{2}{15} & -\frac{1}{3} & \frac{13}{15}
                      \end{matrix}$                             \\\hline
\end{longtable}

\[
    A^{-1} = \begin{pmatrix}
        -\frac{1}{15} & \frac{1}{3}  & -\frac{1}{15} \\
        \frac{2}{5}   & 0            & -\frac{3}{5}  \\
        -\frac{2}{15} & -\frac{1}{3} & \frac{13}{15}
    \end{pmatrix}
\]

\section{Frage 2}

Bestimmen Sie die Inverse Matrix für:

\[
    A = \begin{pmatrix}
        1 & 2 & 3 \\
        1 & 2 & 3 \\
        2 & 1 & 2
    \end{pmatrix}
\]

zunächst muss geprüft werden, ob die Matrx invertierbar ist. Eine Matrix,
welche Invertierhar hat, besitzt eine $\det(A) \neq 0$.

\begin{align*}
    \det(A) = 1 \cdot 2 \cdot 2 + 2 \cdot 3 \cdot 2 + 3 \cdot 1 \cdot 1 \\
    - 2 \cdot 2 \cdot 3 - 1 \cdot 3 \cdot 1 - 2 \cdot 1 \cdot 2         \\
    = 4 + 12 + 3 - 12 - 3 - 4                                           \\
    = 0
\end{align*}

Die Matrix besitzt also \textit{keine inverse}.

\section{Frage 3}

Bestimmen Sie die inverse Matrix für:

\[
    A = \begin{pmatrix}
        1 & 1 & 2 \\
        2 & 1 & 3 \\
        4 & 2 & 1
    \end{pmatrix}
\]

\begin{align*}
    \text{Die erweiterte Matrix:}                \\
    \begin{pmatrix}
        1 & 1 & 2 & 1 & 0 & 0 \\
        2 & 1 & 3 & 0 & 1 & 0 \\
        4 & 2 & 1 & 0 & 0 & 1
    \end{pmatrix}                        \\
    II - 2I                                      \\
    III - 4I                                     \\
    \begin{pmatrix}
        1 & 1  & 2  & 1  & 0 & 0 \\
        0 & -1 & -1 & -2 & 1 & 0 \\
        0 & -2 & -7 & -4 & 0 & 1
    \end{pmatrix}                     \\
    -II                                          \\
    III + 2II                                    \\
    \begin{pmatrix}
        1 & 1 & 2  & 1 & 0  & 0 \\
        0 & 1 & 1  & 2 & -1 & 0 \\
        0 & 0 & -5 & 0 & -2 & 1
    \end{pmatrix}                      \\
    I - II                                       \\
    \begin{pmatrix}
        1 & 0 & 1  & -1 & 1  & 0 \\
        0 & 1 & 1  & 2  & -1 & 0 \\
        0 & 0 & -5 & 0  & -2 & 1
    \end{pmatrix}                     \\
    \frac{-1}{5}III                              \\
    \begin{pmatrix}
        1 & 0 & 1 & -1 & 1           & 0           \\
        0 & 1 & 1 & 2  & -1          & 0           \\
        0 & 0 & 1 & 0  & \frac{2}{5} & \frac{1}{5}
    \end{pmatrix}   \\
    I - III \text{und} II - III                  \\
    \begin{pmatrix}
        1 & 0 & 0 & -1 & \frac{3}{5}  & \frac{1}{5} \\
        0 & 1 & 0 & 2  & -\frac{7}{5} & \frac{1}{5} \\
        0 & 0 & 1 & 0  & \frac{2}{5}  & \frac{1}{5}
    \end{pmatrix}  \\
    A^{-1} =     \begin{pmatrix}
                     -1 & \frac{3}{5}  & \frac{1}{5} \\
                     2  & -\frac{7}{5} & \frac{1}{5} \\
                     0  & \frac{2}{5}  & \frac{1}{5}
                 \end{pmatrix} \\
\end{align*}

\section{Frage 4}

Bestimmen Sie die inverse Matrix für:

\[
    A = \begin{pmatrix}
        1 & 1 & 2 \\
        2 & 2 & 5 \\
        1 & 4 & 1
    \end{pmatrix}
\]

\begin{align*}
    \text{Die erweiterte Matrix:}                          \\
    \begin{pmatrix}
        1 & 1 & 2 & 1 & 0 & 0 \\
        2 & 2 & 5 & 0 & 1 & 0 \\
        1 & 4 & 1 & 0 & 0 & 1
    \end{pmatrix}                                  \\
    II - 2I                                                \\
    III - I                                                \\
    \begin{pmatrix}
        1 & 1 & 2  & 1  & 0 & 0 \\
        0 & 0 & 1  & -2 & 1 & 0 \\
        0 & 3 & -1 & -1 & 0 & 1
    \end{pmatrix}                                \\
    \frac{1}{3}III                                         \\
    II \leftrightarrow III                                 \\
    \begin{pmatrix}
        1 & 1 & 2            & 1            & 0 & 0           \\
        0 & 1 & -\frac{1}{3} & -\frac{1}{3} & 0 & \frac{1}{3} \\
        0 & 0 & 1            & -2           & 1 & 0
    \end{pmatrix}  \\
    I - II                                                 \\
    \begin{pmatrix}
        1 & 1 & \frac{7}{3}  & \frac{4}{3}  & 0 & -\frac{1}{3} \\
        0 & 1 & -\frac{1}{3} & -\frac{1}{3} & 0 & \frac{1}{3}  \\
        0 & 0 & 1            & -2           & 1 & 0
    \end{pmatrix} \\
    I - \frac{7}{3}                                        \\
    II + \frac{1}{3}III                                    \\
    \begin{pmatrix}
        1 & 1 & 0 & 6  & -\frac{7}{3} & -\frac{1}{3} \\
        0 & 1 & 0 & -1 & \frac{1}{3}  & \frac{1}{3}  \\
        0 & 0 & 1 & -2 & 1            & 0
    \end{pmatrix}           \\
    A^{-1} = \begin{pmatrix}
                 6  & -\frac{7}{3} & -\frac{1}{3} \\
                 -1 & \frac{1}{3}  & \frac{1}{3}  \\
                 -2 & 1            & 0
             \end{pmatrix}
\end{align*}

\section{Frage 5}

Bestimmen Sie die Inverse $A^{-1}$ der Matrix

\[
    A = \begin{pmatrix}
        1 & 8  & -5 \\
        0 & 2  & 16 \\
        1 & -1 & 3
    \end{pmatrix}
\]

zunächst muss geprüft werden, ob die Matrx invertierbar ist. Eine Matrix,
welche Invertierhar hat, besitzt eine $\det(A) \neq 0$.

\begin{align*}
    \det(A) = 1 \cdot 2 \cdot 3 + 8 \cdot 16 \cdot 1 + (-5) \cdot 0 \cdot (-1) \\
    - 1 \cdot 2 \cdot (-5) - (-1) \cdot 16 \cdot 1 - 3 \cdot 0 \cdot 8         \\
    = 6 + 128 + 0 - (-10) - (-16) - 0                                          \\
    = 160
\end{align*}

Da $\det(A) \neq 0$, ist die Matrix invertierbar. Nun kann die Matrix
invertiert werden. Hierfür wird das \nameref{gauss_jordan_verfahren} verwendet.

\begin{longtable}{p{4cm}|p{3cm}}

    \hline
    \multicolumn{1}{c|}{\textbf{Matrix}} & \multicolumn{1}{c}{\textbf{Einheitsmatrix}}     \\
    \hline
    \endfirsthead

    \hline
    \multicolumn{2}{c}{\tablename\ \thetable\ -- \textit{Fortführung von vorherier Seite}} \\
    \hline
    \multicolumn{1}{c|}{\textbf{Matrix}} & \multicolumn{1}{c}{\textbf{Einheitsmatrix}}     \\
    \hline
    \endhead

    \hline
    \multicolumn{2}{r}{\textit{Fortsetzung siehe nächste Seite}}                           \\
    \endfoot

    \hline
    \endlastfoot

    $\displaystyle\begin{matrix}
                          1 & 8  & -5 \\
                          0 & 2  & 16 \\
                          1 & -1 & 3
                      \end{matrix}$         &
    $\displaystyle\begin{matrix}
                          1 & 0 & 0 \\
                          0 & 1 & 0 \\
                          0 & 0 & 1
                      \end{matrix}$                                                            \\\hline

    \multicolumn{2}{p{\dimexpr4cm+3cm+2\tabcolsep\relax}}{Operation: III - I}              \\\hline\pagebreak[0]
    $\displaystyle\begin{matrix}
                          1 & 8  & -5 \\
                          0 & 2  & 16 \\
                          0 & -9 & 8
                      \end{matrix}$         &
    $\displaystyle\begin{matrix}
                          1  & 0 & 0 \\
                          0  & 1 & 0 \\
                          -1 & 0 & 1
                      \end{matrix}$                                                            \\\hline

    \multicolumn{2}{p{\dimexpr4cm+3cm+2\tabcolsep\relax}}{Operation: I - 4II}              \\\hline\pagebreak[0]
    $\displaystyle\begin{matrix}
                          1 & 0  & -69 \\
                          0 & 2  & 16  \\
                          0 & -9 & 8
                      \end{matrix}$         &
    $\displaystyle\begin{matrix}
                          1  & -4 & 0 \\
                          0  & 1  & 0 \\
                          -1 & 0  & 1
                      \end{matrix}$                                                            \\\hline

    \multicolumn{2}{p{\dimexpr4cm+3cm+2\tabcolsep\relax}}{Operation: 2III + 9II}           \\\hline\pagebreak[0]
    $\displaystyle\begin{matrix}
                          1 & 0 & -69 \\
                          0 & 2 & 16  \\
                          0 & 0 & 160
                      \end{matrix}$         &
    $\displaystyle\begin{matrix}
                          1  & -4 & 0 \\
                          0  & 1  & 0 \\
                          -2 & 9  & 2
                      \end{matrix}$                                                            \\\hline

    \multicolumn{2}{p{\dimexpr4cm+3cm+2\tabcolsep\relax}}{Operation: 10II - III}           \\\hline\pagebreak[0]
    $\displaystyle\begin{matrix}
                          1 & 0  & -69 \\
                          0 & 20 & 0   \\
                          0 & 0  & 160
                      \end{matrix}$         &
    $\displaystyle\begin{matrix}
                          1  & -4 & 0  \\
                          -2 & 1  & -2 \\
                          -2 & 9  & 2
                      \end{matrix}$                                                            \\\hline

    \multicolumn{2}{p{\dimexpr4cm+3cm+2\tabcolsep\relax}}{Operation: 160I + 69III}         \\\hline\pagebreak[0]
    $\displaystyle\begin{matrix}
                          160 & 0  & 0   \\
                          0   & 20 & 0   \\
                          0   & 0  & 160
                      \end{matrix}$         &
    $\displaystyle\begin{matrix}
                          22 & -19 & 138 \\
                          -2 & 1   & -2  \\
                          -2 & 9   & 2
                      \end{matrix}$                                                           \\\hline

    \multicolumn{2}{p{\dimexpr4cm+3cm+2\tabcolsep\relax}}{Operation: I : 160}              \\\hline\pagebreak[0]
    \multicolumn{2}{p{\dimexpr4cm+3cm+2\tabcolsep\relax}}{Operation: II : 20}              \\\hline\pagebreak[0]
    \multicolumn{2}{p{\dimexpr4cm+3cm+2\tabcolsep\relax}}{Operation: III : 160}            \\\hline\pagebreak[0]
    $\displaystyle\begin{matrix}
                          1 & 0 & 0 \\
                          0 & 1 & 0 \\
                          0 & 0 & 1
                      \end{matrix}$         &
    $\displaystyle\begin{matrix}
                          \frac{11}{80} & -\frac{19}{160} & \frac{69}{180} \\
                          -\frac{1}{10} & \frac{1}{20}    & -\frac{1}{10}  \\
                          -\frac{1}{80} & \frac{9}{160}   & \frac{1}{80}
                      \end{matrix}$                         \\\hline
\end{longtable}

Nun können die Elemente abgelesen werden: $A^{-1}_{12} = -\frac{19}{160},
    A^{-1}_{22} = \frac{1}{20}, A^{-1}_{31} = -\frac{1}{80}$.

\section{Frage 6}

Bestimmen Sie die Inverse von $M = \begin{pmatrix}
        3 & -2 \\ -1 & 1
    \end{pmatrix}$

Als erstes sollte bestimmt werden, ob die Matrix eine Inverse besitzt. Sie
besitzt eine Inverse, wenn $\det(M) \neq 0$.

\begin{align*}
    \det(M) = 3 \cdot 1 - (-1 \cdot -2) = 3 - 2 = 1
\end{align*}

Die Matrix besitzt also eine Inverse. Diese kann nun über das Gauß-jordan
verfahren ermittelt werden

\begin{longtable}{p{4cm}|p{3cm}}

    \hline
    \multicolumn{1}{c|}{\textbf{Matrix}} & \multicolumn{1}{c}{\textbf{Einheitsmatrix}}     \\
    \hline
    \endfirsthead

    \hline
    \multicolumn{2}{c}{\tablename\ \thetable\ -- \textit{Fortführung von vorherier Seite}} \\
    \hline
    \multicolumn{1}{c|}{\textbf{Matrix}} & \multicolumn{1}{c}{\textbf{Einheitsmatrix}}     \\
    \hline
    \endhead

    \hline
    \multicolumn{2}{r}{\textit{Fortsetzung siehe nächste Seite}}                           \\
    \endfoot

    \hline
    \endlastfoot

    $\displaystyle\begin{matrix}
                          3  & -2 \\
                          -1 & 1
                      \end{matrix}$         &
    $\displaystyle\begin{matrix}
                          1 & 0 \\
                          0 & 1
                      \end{matrix}$                                                            \\\hline

    \multicolumn{2}{p{\dimexpr4cm+3cm+2\tabcolsep\relax}}{Operation: 3II + I}              \\\hline\pagebreak[0]

    $\displaystyle\begin{matrix}
                          3 & -2 \\
                          0 & 1
                      \end{matrix}$         &
    $\displaystyle\begin{matrix}
                          1 & 0 \\
                          1 & 3
                      \end{matrix}$                                                            \\\hline

    \multicolumn{2}{p{\dimexpr4cm+3cm+2\tabcolsep\relax}}{Operation: I + 2II}              \\\hline\pagebreak[0]

    $\displaystyle\begin{matrix}
                          3 & 0 \\
                          0 & 1
                      \end{matrix}$         &
    $\displaystyle\begin{matrix}
                          3 & 6 \\
                          1 & 3
                      \end{matrix}$                                                            \\\hline

    \multicolumn{2}{p{\dimexpr4cm+3cm+2\tabcolsep\relax}}{Operation: I : 3}                \\\hline\pagebreak[0]

    $\displaystyle\begin{matrix}
                          1 & 0 \\
                          0 & 1
                      \end{matrix}$         &
    $\displaystyle\begin{matrix}
                          1 & 2 \\
                          1 & 3
                      \end{matrix}$                                                            \\\hline
\end{longtable}

Die Inverse $M^{-1}$ ist also $\begin{pmatrix}
        1 & 2 \\ 1 & 3
    \end{pmatrix}$

\section{Frage 7}

Bestimmen Sie die Inverse von $M = \begin{pmatrix}
        -5 & 1 & -3 \\
        9  & 1 & 2  \\
        -3 & 0 & -1
    \end{pmatrix}$

Als erstes sollte bestimmt werden, ob die Matrix eine Inverse besitzt. Sie
besitzt eine Inverse, wenn $\det(M) \neq 0$

\begin{align*}
    \det(M) = -5 \cdot 1 \cdot -1 + 1 \cdot 2 \cdot -3 + -3 \cdot 9 \cdot 0 \\
    - (-3 \cdot 1 \cdot -3) - (0 \cdot 2 \cdot -5) - (-1 \cdot 9 \cdot 1)   \\
    = 5 - 6 +0 - 9 + 0 + 9 = -10
\end{align*}

Die Matrix besitzt also eine Inverse. Diese kann über das Gauß-jordan verfahren
berechnet werden.

\begin{longtable}{p{4cm}|p{3cm}}

    \hline
    \multicolumn{1}{c|}{\textbf{Matrix}} & \multicolumn{1}{c}{\textbf{Einheitsmatrix}}     \\
    \hline
    \endfirsthead

    \hline
    \multicolumn{2}{c}{\tablename\ \thetable\ -- \textit{Fortführung von vorherier Seite}} \\
    \hline
    \multicolumn{1}{c|}{\textbf{Matrix}} & \multicolumn{1}{c}{\textbf{Einheitsmatrix}}     \\
    \hline
    \endhead

    \hline
    \multicolumn{2}{r}{\textit{Fortsetzung siehe nächste Seite}}                           \\
    \endfoot

    \hline
    \endlastfoot

    $\displaystyle\begin{matrix}
                          -5 & 1 & -3 \\
                          9  & 1 & 2  \\
                          -3 & 0 & -1
                      \end{matrix}$         &
    $\displaystyle\begin{matrix}
                          1 & 0 & 0 \\
                          0 & 1 & 0 \\
                          0 & 0 & 1
                      \end{matrix}$                                                            \\\hline

    \multicolumn{2}{p{\dimexpr4cm+3cm+2\tabcolsep\relax}}{Operation: II + 3III}            \\\hline\pagebreak[0]
    $\displaystyle\begin{matrix}
                          -5 & 1 & -3 \\
                          0  & 1 & -1 \\
                          -3 & 0 & -1
                      \end{matrix}$         &
    $\displaystyle\begin{matrix}
                          1 & 0 & 0 \\
                          0 & 1 & 3 \\
                          0 & 0 & 1
                      \end{matrix}$                                                            \\\hline
    \multicolumn{2}{p{\dimexpr4cm+3cm+2\tabcolsep\relax}}{Operation: 5III - 3I}            \\\hline\pagebreak[0]
    $\displaystyle\begin{matrix}
                          -5 & 1  & -3 \\
                          0  & 1  & -1 \\
                          0  & -3 & 4
                      \end{matrix}$         &
    $\displaystyle\begin{matrix}
                          1  & 0 & 0 \\
                          0  & 1 & 3 \\
                          -3 & 0 & 5
                      \end{matrix}$                                                            \\\hline
    \multicolumn{2}{p{\dimexpr4cm+3cm+2\tabcolsep\relax}}{Operation: I - II}               \\\hline\pagebreak[0]
    $\displaystyle\begin{matrix}
                          -5 & 0  & -2 \\
                          0  & 1  & -1 \\
                          0  & -3 & 4
                      \end{matrix}$         &
    $\displaystyle\begin{matrix}
                          1  & -1 & -3 \\
                          0  & 1  & 3  \\
                          -3 & 0  & 5
                      \end{matrix}$                                                            \\\hline
    \multicolumn{2}{p{\dimexpr4cm+3cm+2\tabcolsep\relax}}{Operation: III + 3II}            \\\hline\pagebreak[0]
    $\displaystyle\begin{matrix}
                          -5 & 0 & -2 \\
                          0  & 1 & -1 \\
                          0  & 0 & 1
                      \end{matrix}$         &
    $\displaystyle\begin{matrix}
                          1  & -1 & -3 \\
                          0  & 1  & 3  \\
                          -3 & 3  & 14
                      \end{matrix}$                                                            \\\hline
    \multicolumn{2}{p{\dimexpr4cm+3cm+2\tabcolsep\relax}}{Operation: II + III}             \\\hline\pagebreak[0]
    $\displaystyle\begin{matrix}
                          -5 & 0 & -2 \\
                          0  & 1 & 0  \\
                          0  & 0 & 1
                      \end{matrix}$         &
    $\displaystyle\begin{matrix}
                          1  & -1 & -3 \\
                          -3 & 4  & 17 \\
                          -3 & 3  & 14
                      \end{matrix}$                                                            \\\hline
    \multicolumn{2}{p{\dimexpr4cm+3cm+2\tabcolsep\relax}}{Operation: I + 2III}             \\\hline\pagebreak[0]
    $\displaystyle\begin{matrix}
                          -5 & 0 & 0 \\
                          0  & 1 & 0 \\
                          0  & 0 & 1
                      \end{matrix}$         &
    $\displaystyle\begin{matrix}
                          -5 & 5 & 25 \\
                          -3 & 4 & 17 \\
                          -3 & 3 & 14
                      \end{matrix}$                                                            \\\hline
    \multicolumn{2}{p{\dimexpr4cm+3cm+2\tabcolsep\relax}}{Operation: I : -5}               \\\hline\pagebreak[0]
    $\displaystyle\begin{matrix}
                          0 & 0 & 0 \\
                          0 & 1 & 0 \\
                          0 & 0 & 1
                      \end{matrix}$         &
    $\displaystyle\begin{matrix}
                          1  & -1 & -5 \\
                          -3 & 4  & 17 \\
                          -3 & 3  & 14
                      \end{matrix}$                                                            \\\hline
\end{longtable}

Die Inverse $M^{-1}$ ist also $\begin{pmatrix}
    1 & -1 & -5 \\
    -3 & 4 & 17 \\
    -3 & 3 & 14
\end{pmatrix}$