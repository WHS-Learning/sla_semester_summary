\chapter{Bonustest 6 - Eigenwerte und Eigenvektoren}

\section{Frage 1}

Gegeben ist die Matrix $M = \begin{pmatrix}
        -2 & 3 \\ -4 & 5
    \end{pmatrix}$. Berechnen Sie das charakteristische Polynom $p(x)$ der Matrix $M$.

\begin{align*}
    M - \lambda I = \begin{pmatrix}
                        -2 & 3 \\
                        -4 & 5
                    \end{pmatrix} - \begin{pmatrix}
                                        \lambda & 0       \\
                                        0       & \lambda
                                    \end{pmatrix}                                \\
    = \begin{pmatrix}
          -2 - \lambda & 3           \\
          -4           & 5 - \lambda
      \end{pmatrix}                                                     \\\\
    \det\left(M - \lambda I\right) = (-2 - \lambda) \cdot (5 - \lambda) - -4 \cdot 3 \\
    = -10 + 2\lambda - 5\lambda + \lambda^2 + 12                                     \\
    = \lambda^2 - 3\lambda + 2
\end{align*}

\section{Frage 2}

Gegeben ist die Matrix $M = \begin{pmatrix}
        9 & 0 \\ -27 & 18
    \end{pmatrix}$. Bestimmen Sie die Menge aller Eigenwerte der Matrix $M$.

\begin{align*}
    M - \lambda I = \begin{pmatrix}
                        9 & 0 \\ -27 & 18
                    \end{pmatrix} - \begin{pmatrix}
                                        \lambda & 0       \\
                                        0       & \lambda
                                    \end{pmatrix}                                        \\
    = \begin{pmatrix}
          9 - \lambda & 0 \\ -27 & 18 - \lambda
      \end{pmatrix}                                                   \\
    \det\left(M - \lambda I\right) = \det\left(\begin{pmatrix}
                                                       9 - \lambda & 0 \\ -27 & 18 - \lambda
                                                   \end{pmatrix}\right)          \\
    = (9 - \lambda) \cdot (18 - \lambda) - 27 \cdot 0 = 162 - 9\lambda -18\lambda +\lambda^2 \\
    = \lambda^2 -27\lambda + 162                                                             \\\\
    \det(M - \lambda I) = 0                                                                  \\
    \lambda^2 - 27\lambda + 162 = 0 \quad | pq                                               \\
    p = -27, q = 162                                                                         \\
    -\frac{-27}{2} \pm \sqrt{{\left(\frac{-27}{2}\right)}^2 - 162}                           \\
    \lambda_{1} = 18 \quad \lambda_{2} = 9
\end{align*}

\section{Frage 3}

Gegeben ist die Matrix $M = \begin{pmatrix}
        15 & 0 \\ -45 & 30
    \end{pmatrix}$. Berechnen Sie die Eigenvektoren der Matrix $M$.

\begin{align*}
    M - \lambda I = \begin{pmatrix}
                        -    15 & 0 \\ -45 & 30
                    \end{pmatrix} - \begin{pmatrix}
                                        \lambda & 0       \\
                                        0       & \lambda
                                    \end{pmatrix}                                   \\
    = \begin{pmatrix}
          15 - \lambda & 0            \\
          -45           & 30 - \lambda
      \end{pmatrix}                                                      \\\\
    \det\left(M - \lambda I\right) = (15 - \lambda) \cdot (30 - \lambda) - -45 \cdot 0 \\
    = +450 - 15\lambda - 30\lambda +\lambda^2 \\
    = \lambda^2 - 45\lambda + 450 \\\\
    \det(M - \lambda I) = 0 \\
    \lambda^2 - 45\lambda + 450 = 0 \quad | pq \\
    p = -45 \quad q = 450 \\
    -\frac{-45}{2} \pm \sqrt{{\left(\frac{-45}{2}\right)}^2 - 450} \\
    \lambda_1 = 30 \quad \lambda_2 = 15 \\\\
    \text{Eigenwerte für }\lambda = 30 \\
    \begin{pmatrix}
        15 - 30 & 0 \\
        -45 & 30 - 30
    \end{pmatrix}\\
    \begin{pmatrix}
        -15 & 0 \\
        -45 & 0
    \end{pmatrix} = \begin{pmatrix}
        0 \\ 0
    \end{pmatrix} \\
    \begin{cases}
        \text{I:\@} & -15x_1 = 0 \\
        \text{II:\@} & 30x_1 = 0 
    \end{cases}
    x_1 = 0 \\
    x_2 = t \in \mathbb{R}\\
    \begin{pmatrix}
        0 \\ t
    \end{pmatrix} = t \begin{pmatrix}
        0 \\ 1
    \end{pmatrix} \\
    span = \left\{\begin{pmatrix}
        0 \\ 1
    \end{pmatrix}\right\}\\\\
    \text{Eigenwerte für } \lambda = 15 \\
    \begin{pmatrix}
        15 - 15 & 0 \\
        -45 & 30 - 15
    \end{pmatrix} \\
    \begin{pmatrix}
        0 & 0 \\
        -45 & 15
    \end{pmatrix} = \begin{pmatrix}
        0 \\ 0
    \end{pmatrix} \\
    \begin{cases}
        \text{I:\@} & 0 = 0 \\
        \text{II:\@} & -45x_1 + 15x_2 = 0 \quad | :-15 | -x_2 \Leftrightarrow 3x_1 = x_2
    \end{cases} \\
    x_1 = t \in \mathbb{R} \\
    \begin{pmatrix}
        t \\
        3t
    \end{pmatrix} \\
    t \begin{pmatrix}
        1 \\ 3
    \end{pmatrix}
    span = \left\{\begin{pmatrix}
        1 \\ 3
    \end{pmatrix}\right\}
\end{align*}