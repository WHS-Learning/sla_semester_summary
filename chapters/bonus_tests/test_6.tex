\chapter{Bonustest 6 - Eigenwerte und Eigenvektoren}

\section{Frage 1}

Gegeben ist die Matrix $M = \begin{pmatrix}
        -2 & 3 \\ -4 & 5
    \end{pmatrix}$. Berechnen Sie das charakteristische Polynom $p(x)$ der Matrix $M$.

\begin{align*}
    M - \lambda I = \begin{pmatrix}
                        -2 & 3 \\
                        -4 & 5
                    \end{pmatrix} - \begin{pmatrix}
                                        \lambda & 0       \\
                                        0       & \lambda
                                    \end{pmatrix}                                \\
    = \begin{pmatrix}
          -2 - \lambda & 3           \\
          -4           & 5 - \lambda
      \end{pmatrix}                                                     \\\\
    \det\left(M - \lambda I\right) = (-2 - \lambda) \cdot (5 - \lambda) - -4 \cdot 3 \\
    = -10 + 2\lambda - 5\lambda + \lambda^2 + 12                                     \\
    = \lambda^2 - 3\lambda + 2
\end{align*}

\section{Frage 2}

Gegeben ist die Matrix $M = \begin{pmatrix}
        9 & 0 \\ -27 & 18
    \end{pmatrix}$. Bestimmen Sie die Menge aller Eigenwerte der Matrix $M$.

\begin{align*}
    M - \lambda I = \begin{pmatrix}
                        9 & 0 \\ -27 & 18
                    \end{pmatrix} - \begin{pmatrix}
                                        \lambda & 0       \\
                                        0       & \lambda
                                    \end{pmatrix}                                        \\
    = \begin{pmatrix}
          9 - \lambda & 0 \\ -27 & 18 - \lambda
      \end{pmatrix}                                                   \\
    \det\left(M - \lambda I\right) = \det\left(\begin{pmatrix}
                                                       9 - \lambda & 0 \\ -27 & 18 - \lambda
                                                   \end{pmatrix}\right)          \\
    = (9 - \lambda) \cdot (18 - \lambda) - 27 \cdot 0 = 162 - 9\lambda -18\lambda +\lambda^2 \\
    = \lambda^2 -27\lambda + 162                                                             \\\\
    \det(M - \lambda I) = 0                                                                  \\
    \lambda^2 - 27\lambda + 162 = 0 \quad | pq                                               \\
    p = -27, q = 162                                                                         \\
    -\frac{-27}{2} \pm \sqrt{{\left(\frac{-27}{2}\right)}^2 - 162}                           \\
    \lambda_{1} = 18 \quad \lambda_{2} = 9
\end{align*}

\section{Frage 3}

Gegeben ist die Matrix $M = \begin{pmatrix}
        15 & 0 \\ -45 & 30
    \end{pmatrix}$. Berechnen Sie die Eigenvektoren der Matrix $M$.

\begin{align*}
    M - \lambda I = \begin{pmatrix}
                        -    15 & 0 \\ -45 & 30
                    \end{pmatrix} - \begin{pmatrix}
                                        \lambda & 0       \\
                                        0       & \lambda
                                    \end{pmatrix}                                   \\
    = \begin{pmatrix}
          15 - \lambda & 0            \\
          -45           & 30 - \lambda
      \end{pmatrix}                                                      \\\\
    \det\left(M - \lambda I\right) = (15 - \lambda) \cdot (30 - \lambda) - -45 \cdot 0 \\
    = +450 - 15\lambda - 30\lambda +\lambda^2 \\
    = \lambda^2 - 45\lambda + 450 \\\\
    \det(M - \lambda I) = 0 \\
    \lambda^2 - 45\lambda + 450 = 0 \quad | pq \\
    p = -45 \quad q = 450 \\
    -\frac{-45}{2} \pm \sqrt{{\left(\frac{-45}{2}\right)}^2 - 450} \\
    \lambda_1 = 30 \quad \lambda_2 = 15 \\\\
    \text{Eigenwerte für }\lambda = 30 \\
    \begin{pmatrix}
        15 - 30 & 0 \\
        -45 & 30 - 30
    \end{pmatrix}\\
    \begin{pmatrix}
        -15 & 0 \\
        -45 & 0
    \end{pmatrix} = \begin{pmatrix}
        0 \\ 0
    \end{pmatrix} \\
    \begin{cases}
        \text{I:\@} & -15x_1 = 0 \\
        \text{II:\@} & 30x_1 = 0 
    \end{cases}
    x_1 = 0 \\
    x_2 = t \in \mathbb{R}\\
    \begin{pmatrix}
        0 \\ t
    \end{pmatrix} = t \begin{pmatrix}
        0 \\ 1
    \end{pmatrix} \\
    span = \left\{\begin{pmatrix}
        0 \\ 1
    \end{pmatrix}\right\}\\\\
    \text{Eigenwerte für } \lambda = 15 \\
    \begin{pmatrix}
        15 - 15 & 0 \\
        -45 & 30 - 15
    \end{pmatrix} \\
    \begin{pmatrix}
        0 & 0 \\
        -45 & 15
    \end{pmatrix} = \begin{pmatrix}
        0 \\ 0
    \end{pmatrix} \\
    \begin{cases}
        \text{I:\@} & 0 = 0 \\
        \text{II:\@} & -45x_1 + 15x_2 = 0 \quad | :-15 | -x_2 \Leftrightarrow 3x_1 = x_2
    \end{cases} \\
    x_1 = t \in \mathbb{R} \\
    \begin{pmatrix}
        t \\
        3t
    \end{pmatrix} \\
    t \begin{pmatrix}
        1 \\ 3
    \end{pmatrix}
    span = \left\{\begin{pmatrix}
        1 \\ 3
    \end{pmatrix}\right\}
\end{align*}

\section{Frage 4}
Gegeben ist die Matrix $M = \begin{pmatrix}
    4 & 0 \\ -1 & -3
\end{pmatrix}$.

\begin{align*}
    M - \lambda I
    \begin{pmatrix}
        4 & 0 \\ -1 & -3
    \end{pmatrix} - \begin{pmatrix}
        \lambda & 0 \\
        0 & \lambda
    \end{pmatrix} = \begin{pmatrix}
        4 - \lambda & 0 \\ -1 & -3 - \lambda
    \end{pmatrix}\\
    \det\left(M - \lambda I\right) \\
    \det\left(\begin{pmatrix}
        4 - \lambda & 0 \\ -1 & -3 - \lambda
    \end{pmatrix}\right) = (4 - \lambda) \cdot (-3 - \lambda)\\
    = -12 - 4\lambda + 3\lambda + \lambda^2 \\
    = \lambda^2 - \lambda - 12\\
    \text{Das charakteristische Polynom ist } \lambda^2 - 7\lambda + 13 \\
    \det(M - \lambda I) = 0 \\
    \lambda^2 - \lambda - 12 = 0 \quad | pq\\
    p = -1 \quad q = -12 \\
    \lambda_{1, 2} = -\frac{-1}{2} \pm \sqrt{{\left(\frac{-1}{2}\right)}^2 + 12} \\
    \lambda_1 = 4 \\
    \lambda_2 = -3 \\
    \text{Die Eigenwerte sind: }4, -3 \\\\
    \text{Eigenvektoren für } \lambda = 4 \\
    M - \lambda I = \vec{0} \\
    \begin{pmatrix}
        4 - 4 & 0 \\ -1 & -3 - 4
    \end{pmatrix} = \begin{pmatrix}
        0 \\ 0
    \end{pmatrix} \\
    \begin{cases}
        \text{I:\@} & 8x_1 = 0 \Leftrightarrow x_1 = 0\\
        \text{II:\@} & -1x_1 -7x_2 = 0 \quad | + x_1 \Leftrightarrow -7x_2 = x_1
    \end{cases} \\
    x_1 = t | t \in \mathbb{R} \\
    \begin{pmatrix}
        t \\ -7t 
    \end{pmatrix} \\
    t \begin{pmatrix}
        1 \\ -7
    \end{pmatrix} \\
    span = \left\{\begin{pmatrix}
        1 \\ -7
    \end{pmatrix}\right\} \\\\
    \text{Eigenvektoren für } \lambda = -3 \\
    M - \lambda I = \vec{0} \\
    \begin{pmatrix}
        4 + 3 & 0 \\ -1 & -3 + 3
    \end{pmatrix} = \begin{pmatrix}
        0 \\ 0
    \end{pmatrix} \\
    \begin{cases}
        \text{I:\@} & 7x_1 = 0 \Leftrightarrow x_1 = 0 \\
        \text{II:\@} & -1x_1 = 0 \Leftrightarrow x_1 = 0
    \end{cases} \\
    x_2 = t | t \in \mathbb{R} \\
    \begin{pmatrix}
        0 \\ t
    \end{pmatrix} \\
    t \begin{pmatrix}
        0 \\ 1
    \end{pmatrix} \\
    span = \left\{\begin{pmatrix}
        0 \\ 1
    \end{pmatrix}\right\}
\end{align*}

\section{Frage 5}

Gegeben ist die Matrix$M = \begin{pmatrix}
    3 & -9 & 9 \\
    -9 & 15 & -9 \\
    9 & -9 & 15
\end{pmatrix}$
\begin{align*}
    M - \lambda I \\
    \begin{pmatrix}
    3 & -9 & 9 \\
    -9 & 15 & -9 \\
    9 & -9 & 15
\end{pmatrix} - \begin{pmatrix}
    \lambda & 0 & 0 \\
    0 & \lambda & 0 \\
    0 & 0 & \lambda
\end{pmatrix} \\
\det(M - \lambda I) = \det\begin{pmatrix}
    3 - \lambda & -9 & 9 \\
    -9 & 15 - \lambda & -9 \\
    9 & -9 & 15 - \lambda
\end{pmatrix} \\
\\
= (3 - \lambda) \cdot (15 - \lambda) \cdot (15 - \lambda) + -9 \cdot -9 \cdot 9 + 9 \cdot -9 \cdot -9 \\
\quad - 9 \cdot (15 - \lambda) \cdot 9 - -9 \cdot -9 \cdot (3 - \lambda) - -9 \cdot -9 \cdot (15 - \lambda) \\
\\
= (3 - \lambda) \cdot (\lambda^2 - 30\lambda + 225) + 729 + 729 \\
\quad - 81 \cdot (15 - \lambda) - 81 \cdot (3 - \lambda) - 81 \cdot (15 - \lambda) \\
\\
= (3\lambda^2 - 90\lambda + 675 - \lambda^3 + 30\lambda^2 - 225\lambda) + 1458 \\
\quad - 1215 + 81\lambda - 243 + 81\lambda - 1215 + 81\lambda \\
\\
= (-\lambda^3 + 33\lambda^2 - 315\lambda + 675) + 1458 - 2673 + 243\lambda \\
\\
= -\lambda^3 + 33\lambda^2 - 315\lambda + 2133 - 2673 + 243\lambda \\
\\
P(\lambda) = -\lambda^3 + 33\lambda^2 - 72\lambda - 540 \\\\
\det(M - \lambda I) = 0 \\
-\lambda^3 + 33\lambda^2 - 72\lambda - 540 = 0 \quad | \cdot(-1)\\
\Leftrightarrow \lambda^3 - 33\lambda^2 + 72\lambda + 540 = 0\\
\text{Nullstellen müssen ausprobiert werden.} \\
\text{Wähle }\lambda = -3 \\
\Leftrightarrow (-3)^3 - 33 \cdot (-3)^2 + 72 \cdot (-3) + 540 = 0 \\
\Leftrightarrow -27 - 297 - 216 + 540 = 0 \\
\Leftrightarrow -540 + 540 = 0 \\ 
\Leftrightarrow 0 = 0 \\
\lambda_1 = -3\\\\
(\lambda^3 - 33\lambda^2 + 72\lambda + 540) : (\lambda + 3) \\
\phantom{-}(\lambda^3 + 3\lambda^2) \\
\hline
\quad -36\lambda^2 + 72\lambda \\
\quad -(-36\lambda^2 - 108\lambda) \\
\hline
\quad \quad \quad \quad 180\lambda + 540 \\
\quad \quad \quad -(180\lambda + 540) \\
\hline
\quad \quad \quad \quad \quad \quad \quad 0 \\
\\
\text{Resultat: } \lambda^2 - 36\lambda + 180 \\
\\
\lambda^2 - 36\lambda + 180 = 0 \\
\lambda_{2,3} = -(\frac{-36}{2}) \pm \sqrt{(\frac{-36}{2})^2 - 180} \\
\lambda_{2,3} = 18 \pm \sqrt{(-18)^2 - 180} \\
\lambda_{2,3} = 18 \pm \sqrt{324 - 180} \\
\lambda_{2,3} = 18 \pm \sqrt{144} \\
\lambda_{2,3} = 18 \pm 12 \\
\\
\lambda_2 = 18 + 12 = 30 \\
\lambda_3 = 18 - 12 = 6 \\
\text{Die Eigenwerte sind: } -3, 6, 30 \\\\
\text{Eigenvektoren für } \lambda_1 = -3 \\
M - \lambda_1 I = \vec{0} \\
\begin{pmatrix}
    3 - (-3) & -9 & 9 \\ -9 & 15 - (-3) & -9 \\ 9 & -9 & 15 - (-3)
\end{pmatrix} = \begin{pmatrix} 0 \\ 0 \\ 0 \end{pmatrix} \\
\begin{pmatrix}
    6 & -9 & 9 \\ -9 & 18 & -9 \\ 9 & -9 & 18
\end{pmatrix} = \begin{pmatrix} 0 \\ 0 \\ 0 \end{pmatrix} \\
\begin{cases}
    \text{I:\@} & 6x_1 - 9x_2 + 9x_3 = 0 \Leftrightarrow 2x_1 - 3x_2 + 3x_3 = 0 \\
    \text{II:\@} & -9x_1 + 18x_2 - 9x_3 = 0 \Leftrightarrow x_1 - 2x_2 + x_3 = 0 \\
    \text{III:\@} & 9x_1 - 9x_2 + 18x_3 = 0 \Leftrightarrow x_1 - x_2 + 2x_3 = 0
\end{cases} \\
\text{Aus III: } x_1 = x_2 - 2x_3 \\
\text{in II eingesetzt: } (x_2 - 2x_3) - 2x_2 + x_3 = 0 \\
\Leftrightarrow -x_2 - x_3 = 0 \\
\Leftrightarrow x_2 = -x_3 \\
\text{damit ist } x_1 = (-x_3) - 2x_3 = -3x_3 \\
x_3 = t \quad | t \in \mathbb{R} \\
\begin{pmatrix} -3t \\ -t \\ t \end{pmatrix} = t \begin{pmatrix} -3 \\ -1 \\ 1 \end{pmatrix} \\
\text{span} = \left\{\begin{pmatrix} -3 \\ -1 \\ 1 \end{pmatrix}\right\} \\\\
\text{Eigenvektoren für } \lambda_2 = 6 \\
M - \lambda_2 I = \vec{0} \\
\begin{pmatrix}
    3 - 6 & -9 & 9 \\ -9 & 15 - 6 & -9 \\ 9 & -9 & 15 - 6
\end{pmatrix} = \begin{pmatrix} 0 \\ 0 \\ 0 \end{pmatrix} \\
\begin{pmatrix}
    -3 & -9 & 9 \\ -9 & 9 & -9 \\ 9 & -9 & 9
\end{pmatrix} = \begin{pmatrix} 0 \\ 0 \\ 0 \end{pmatrix} \\
\begin{cases}
    \text{I:\@} & -3x_1 - 9x_2 + 9x_3 = 0 \Leftrightarrow -x_1 - 3x_2 + 3x_3 = 0 \\
    \text{II:\@} & -9x_1 + 9x_2 - 9x_3 = 0 \Leftrightarrow -x_1 + x_2 - x_3 = 0 \\
    \text{III:\@} & 9x_1 - 9x_2 + 9x_3 = 0 \Leftrightarrow x_1 - x_2 + x_3 = 0
\end{cases} \\
\text{Aus II: } x_2 = x_1 + x_3 \\
\text{in I eingesetzt: } -x_1 - 3(x_1 + x_3) + 3x_3 = 0 \\
\Leftrightarrow -x_1 - 3x_1 - 3x_3 + 3x_3 = 0 \\
\Leftrightarrow -4x_1 = 0 \\
\Leftrightarrow x_1 = 0 \\
\text{damit ist } x_2 = 0 + x_3 = x_3 \\
x_3 = t \quad | t \in \mathbb{R} \\
\begin{pmatrix} 0 \\ t \\ t \end{pmatrix} = t \begin{pmatrix} 0 \\ 1 \\ 1 \end{pmatrix} \\
\text{span} = \left\{\begin{pmatrix} 0 \\ 1 \\ 1 \end{pmatrix}\right\} \\\\
\text{Eigenvektoren für } \lambda_3 = 30 \\
M - \lambda_3 I = \vec{0} \\
\begin{pmatrix}
    3 - 30 & -9 & 9 \\ -9 & 15 - 30 & -9 \\ 9 & -9 & 15 - 30
\end{pmatrix} = \begin{pmatrix} 0 \\ 0 \\ 0 \end{pmatrix} \\
\begin{pmatrix}
    -27 & -9 & 9 \\ -9 & -15 & -9 \\ 9 & -9 & -15
\end{pmatrix} = \begin{pmatrix} 0 \\ 0 \\ 0 \end{pmatrix} \\
\begin{cases}
    \text{I:\@} & -27x_1 - 9x_2 + 9x_3 = 0 \Leftrightarrow -3x_1 - x_2 + x_3 = 0 \\
    \text{II:\@} & -9x_1 - 15x_2 - 9x_3 = 0 \Leftrightarrow 3x_1 + 5x_2 + 3x_3 = 0 \\
    \text{III:\@} & 9x_1 - 9x_2 - 15x_3 = 0 \Leftrightarrow 3x_1 - 3x_2 - 5x_3 = 0
\end{cases} \\
\text{Aus I: } x_3 = 3x_1 + x_2 \\
\text{in III eingesetzt: } 3x_1 - 3x_2 - 5(3x_1 + x_2) = 0 \\
\Leftrightarrow 3x_1 - 3x_2 - 15x_1 - 5x_2 = 0 \\
\Leftrightarrow -12x_1 - 8x_2 = 0 \\
\Leftrightarrow x_2 = -\frac{3}{2}x_1 \\
\text{damit ist } x_3 = 3x_1 + (-\frac{3}{2}x_1) = \frac{3}{2}x_1 \\
\text{Wähle } x_1 = 2t \quad | t \in \mathbb{R} \text{ (um Brüche zu vermeiden)} \\
\Rightarrow x_2 = -3t, \quad x_3 = 3t \\
\begin{pmatrix} 2t \\ -3t \\ 3t \end{pmatrix} = t \begin{pmatrix} 2 \\ -3 \\ 3 \end{pmatrix} \\
\text{span} = \left\{\begin{pmatrix} 2 \\ -3 \\ 3 \end{pmatrix}\right\}
\end{align*}