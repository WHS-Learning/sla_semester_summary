\chapter{Bonustest 5 - geometrische Interpretation linearer Abbildungen}

\section{Frage 1}

Wählen Sie die richtige Achse für die entsprechende Drehmatrix.

\begin{align*}
    \begin{matrix}
        A = \begin{pmatrix}
                \cos(a) & -\sin(a) & 0 \\
                \sin(a) & \cos(a)  & 0 \\
                0       & 0        & 1
            \end{pmatrix} & \text{Drehung um die Z-Achse} \\
        B = \begin{pmatrix}
                1 & 0       & 0        \\
                0 & \cos(a) & -\sin(a) \\
                0 & \sin(a) & \cos(a)
            \end{pmatrix} & \text{Drehung um die X-Achse} \\
        C = \begin{pmatrix}
                \cos(a) & 0 & -\sin(a) \\
                0       & 1 & 0        \\
                \sin(a) & 0 & \cos(a)
            \end{pmatrix} & \text{Drehung um die Y-Achse}
    \end{matrix}
\end{align*}

\section{Frage 2}

Wählen Sie die richtigen Lösungen.

\begin{align*}
    \begin{matrix}
        \begin{pmatrix}
            -1 & 0  \\
            0  & -1
        \end{pmatrix} & \text{orthogonale Matrix aber keine Drehmatrix} \\
        \begin{pmatrix}
            0  & 1 \\
            -1 & 0
        \end{pmatrix} & \text{Drehmatrix}                               \\
        \begin{pmatrix}
            -1 & 0 \\
            0  & 0
        \end{pmatrix} & \text{Weder Drehmatrix noch orthogonale Matrix}
    \end{matrix}
\end{align*}

\section{Frage 3}

Kreuzen Sie die richtigen Aussagen an.

\begin{tabular}{llp{0.8\textwidth}}
    \makebox[0pt][l]{$\square$}\raisebox{.15ex}{\hspace{0.1em}$\checkmark$}           & a & Inverse Matrizen zu orthogonalen Matrizen sind ebenfalls orthogonale Matrizen.                 \\
    \makebox[0pt][l]{$\square$}\raisebox{.15ex}{\hspace{0.1em}$\checkmark$}           & b & Alle Drehmatrizen sind orthogonale Matrizen.                                                   \\
    \makebox[0pt][l]{$\square$}\raisebox{.15ex}{\hspace{0.1em}$\phantom{\checkmark}$} & c & $\mathbb{R}^{2 \times 2}$ ist eine Untergruppe von $O(n)$.                                     \\
    \makebox[0pt][l]{$\square$}\raisebox{.15ex}{\hspace{0.1em}$\checkmark$}           & d & Multipliziert man zwei orthogonale Matrizen, so ist das Ergebnis auch eine orthogonale Matrix. \\
\end{tabular}

\section{Frage 4}

Eine orthogonale Matrix $A$ ist eine Matrix, für die gilt: $A^{-1} = A^T$

Wahr

\section{Frage 5}

Sei $D_{\frac{\pi}{2}}: \mathbb{R}^2 \rightarrow \mathbb{R}^2$ die lineare
Abbildung, die jeden Vektor aus $\mathbb{R}^2$ um $\frac{\pi}{2}$ dreht.

Stellen Sie die Matrix zu $D_{\frac{\pi}{2}}$ auf.

\begin{align*}
    \begin{pmatrix}
        \cos(\alpha) & -\sin(\alpha) \\
        \sin(\alpha) & \cos(\alpha)
    \end{pmatrix} = \begin{pmatrix}
                        \cos(\frac{\pi}{2}) & -\sin(\frac{\pi}{2}) \\
                        \sin(\frac{\pi}{2}) & \cos(\frac{\pi}{2})
                    \end{pmatrix} = \begin{pmatrix}
                                        0 & -1 \\
                                        1 & 0
                                    \end{pmatrix}
\end{align*}

Sei $v = \begin{pmatrix}
        1 \\ 1
    \end{pmatrix}$. Bestimmen Sie $D_{\frac{\pi}{2}}(v)$.

\begin{align*}
    \begin{pmatrix}
        0 & -1 \\
        1 & 0
    \end{pmatrix} \cdot \begin{pmatrix}
                            1 \\ 1
                        \end{pmatrix} = \begin{pmatrix}
                                            -1 \\ 1
                                        \end{pmatrix}
\end{align*}

Bestimmen Sie $\left<v, D_{\frac{\pi}{2}}(v)\right>$.

\begin{align*}
    \left< \begin{pmatrix}
               1 \\ 1
           \end{pmatrix}, \begin{pmatrix}
                              -1 \\ 1
                          \end{pmatrix}\right> = 1 \cdot -1 + 1 \cdot 1 = 0
\end{align*}

\section{Frage 6}

Sei $D_{\frac{\pi}{3}}: \mathbb{R}^2 \rightarrow \mathbb{R}^2$ die lineare
Abbildung, die jeden Vektor aus $\mathbb{R}^2$ um $\frac{\pi}{3}$ dreht.

Stellen Sie die Matrix zu $D_{\frac{\pi}{3}}$ auf.

\begin{align*}
    \begin{pmatrix}
        \cos(\alpha) & -\sin(\alpha) \\
        \sin(\alpha) & \cos(\alpha)
    \end{pmatrix} = \begin{pmatrix}
                        \cos(\frac{\pi}{3}) & -\sin(\frac{\pi}{3}) \\
                        \sin(\frac{\pi}{3}) & \cos(\frac{\pi}{3})
                    \end{pmatrix} =  \begin{pmatrix}
                                         \frac{1}{2}        & -\frac{\sqrt{3}}{2} \\
                                         \frac{\sqrt{3}}{2} & \frac{1}{2}
                                     \end{pmatrix}
\end{align*}

Sei $v = \begin{pmatrix}
        2 \\ 1
    \end{pmatrix}$. Bestimmen Sie $D_{\frac{\pi}{3}}(v)$.

\begin{align*}
    \begin{pmatrix}
        \frac{1}{2}        & -\frac{\sqrt{3}}{2} \\
        \frac{\sqrt{3}}{2} & \frac{1}{2}
    \end{pmatrix} \cdot \begin{pmatrix}
                            2 \\ 1
                        \end{pmatrix} = \begin{pmatrix}
                                            1 - \frac{\sqrt{3}}{2} \\
                                            \sqrt{3} + \frac{1}{2}
                                        \end{pmatrix}
\end{align*}

Bestimmen Sie $\left<v, D_{\frac{\pi}{3}}(v)\right>$.

\begin{align*}
    \left<\begin{pmatrix}
              2 \\ 1
          \end{pmatrix}, \begin{pmatrix}
                             1 - \frac{\sqrt{3}}{2} \\
                             \sqrt{3} + \frac{1}{2}
                         \end{pmatrix}\right> = 2 \cdot \left(1 - \frac{\sqrt{3}}{2}\right) + 1 \cdot \left(\sqrt{3} + \frac{1}{2}\right) = \frac{5}{2}
\end{align*}