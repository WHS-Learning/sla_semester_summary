\chapter{Determinante}
\label{determinante}

Die Determinante ist eine spezielle Zahl, die einer quadratischen Matrix zugeordnet wird. Sie liefert wichtige Informationen über die Matrix und die lineare Transformation, die sie repräsentiert. Geometrisch kann die Determinante einer $n \times n$ Matrix als der Skalierungsfaktor des $n$-dimensionalen Volumens des Parallelepipeds interpretiert werden, das von den Spaltenvektoren (oder Zeilenvektoren) der Matrix aufgespannt wird. Ist die Determinante gleich null, so ist das Volumen null. Dies bedeutet, dass die Vektoren, welche die Matrix bilden, linear abhängig sind und die Transformation den Raum auf eine niedrigere Dimension abbildet (z.B. eine Ebene auf eine Gerade oder einen Punkt). Eine Determinante ist nur für quadratische Matrizen definiert.

\section{Determinante einer $2 \times 2$ Matrix}

Um die Determinante einer $2 \times 2$ Matrix $A = \begin{pmatrix} a & b \\ c & d \end{pmatrix}$ zu berechnen, werden ganz einfach die Diagonalen multipliziert und subtrahiert.
\[
    \det(A) = a \cdot d - c \cdot b
\]

\section{Determinanten von nicht $2 \times 2$ Matrizen}

Um die Determinante einer nicht $2 \times 2$ Matrix zu berechnen, gibt es im Wesentlichen drei Herangehensweisen:

\subsection{Regel von Sarrus}

Die Regel von Sarrus funktioniert \textbf{nur bei $3 \times 3$} Matrizen.

\begin{tikzpicture}[
    every node/.style={minimum size=0.7cm},
    positive/.style={blue, thick},
    negative/.style={red, thick, dashed}
  ]
  \matrix (m) [matrix of math nodes,
               nodes in empty cells,
               left delimiter=(, right delimiter=),
               column sep=0.7cm, row sep=0.7cm, % Abstand zwischen Zellen
               ampersand replacement=\&] % Wichtig für tikz-matrix
  {
    6 \& 17 \& 24 \\
    9 \& 32 \& 6 \\
    -5 \& -1 \& 47 \\
  };

  % Erweiterte Spalten (nur für die Visualisierung der Linien)
  \node at ($(m-1-3.east)!0.5!(m-1-3.east) + (0.7cm,0)$) (m-1-4) {$6$};
  \node at ($(m-2-3.east)!0.5!(m-2-3.east) + (0.7cm,0)$) (m-2-4) {$9$};
  \node at ($(m-3-3.east)!0.5!(m-3-3.east) + (0.7cm,0)$) (m-3-4) {$-5$};

  \node at ($(m-1-4.east)!0.5!(m-1-4.east) + (0.7cm,0)$) (m-1-5) {$17$};
  \node at ($(m-2-4.east)!0.5!(m-2-4.east) + (0.7cm,0)$) (m-2-5) {$32$};
  \node at ($(m-3-4.east)!0.5!(m-3-4.east) + (0.7cm,0)$) (m-3-5) {$-1$};

  % Vertikale Trennlinie (optional)
  \draw[gray, thin] ($(m-1-3.east)!0.5!(m-2-3.east) + (0.35cm,0.35cm)$) -- ($(m-3-3.east) + (0.35cm,-0.35cm)$);


  % Positive Diagonalen (blau, durchgezogen)
  \draw[positive] (m-1-1.center) -- (m-2-2.center) -- (m-3-3.center); % a-e-i
  \draw[positive] (m-1-2.center) -- (m-2-3.center) -- (m-3-4.center); % b-f-g (auf erweiterter Spalte)
  \draw[positive] (m-1-3.center) -- (m-2-4.center) -- (m-3-5.center); % c-d-h (auf erweiterter Spalte)

  % Negative Diagonalen (rot, gestrichelt)
  \draw[negative] (m-1-3.center) -- (m-2-2.center) -- (m-3-1.center); % c-e-g
  \draw[negative] (m-1-4.center) -- (m-2-3.center) -- (m-3-2.center); % a-f-h (auf erweiterter Spalte)
  \draw[negative] (m-1-5.center) -- (m-2-4.center) -- (m-3-3.center); % b-d-i (auf erweiterter Spalte)

\end{tikzpicture}

\bigskip % Etwas Abstand

Die Zahlen an den Linien werden multipliziert. Blaue Linien werden miteinander addiert und Rote subtrahiert.

\begin{align*}
    (6 \cdot 32 \cdot 47) + (17 \cdot 6 \cdot (-5)) + (24 \cdot 9 \cdot (-1)) \\
    - ((-5) \cdot 32 \cdot 24) - ((-1) \cdot 6 \cdot 6) - (47 \cdot 9 \cdot 17) \\
    = 9024 - 510 - 216 - (-3840) - (-36) - 7191 \\
    = 9024 - 510 - 216 + 3840 + 36 - 7191 \\
    = 4983
\end{align*}

\subsection{Leibniz-Formel (nicht prüfungsrelevant)}

Die Leibniz-Formel ist eine allgemeine Definition der Determinante für eine $n \times n$-Matrix $A=(a_{ij})$. Die Determinante wird als eine Summe über alle möglichen Permutationen der Spaltenindizes berechnet:
\[ \det(A) = \sum_{\sigma \in S_n} \text{sgn}(\sigma) \prod_{i=1}^{n} a_{i, \sigma(i)} \]
Hierbei ist:
\begin{itemize}
    \item $S_n$ die Menge aller Permutationen der Zahlen $\{1, 2, \ldots, n\}$. Eine Permutation $\sigma$ ist eine bijektive Abbildung dieser Menge auf sich selbst; $\sigma(i)$ ist das $i$-te Element in einer Anordnung der Zahlen $1, \ldots, n$.
    \item $\text{sgn}(\sigma)$ das Signum (Vorzeichen) der Permutation $\sigma$. Es ist $+1$, wenn die Permutation durch eine gerade Anzahl von Vertauschungen zweier Elemente (Transpositionen) aus der Grundreihenfolge $(1, 2, \ldots, n)$ hervorgeht. Es ist $-1$, wenn sie durch eine ungerade Anzahl von Vertauschungen hervorgeht. Alternativ kann das Signum über die Anzahl $k$ der Fehlstände (Inversionen) der Permutation berechnet werden als $\text{sgn}(\sigma) = (-1)^k$. Ein Fehlstand ist ein Paar von Indizes $(i,j)$ derart, dass $i < j$ ist, aber $\sigma(i) > \sigma(j)$ gilt.
    \item $\prod_{i=1}^{n} a_{i, \sigma(i)}$ das Produkt von $n$ Matrixelementen $a_{1, \sigma(1)} \cdot a_{2, \sigma(2)} \cdot \ldots \cdot a_{n, \sigma(n)}$. Dabei wird aus jeder Zeile $i$ genau ein Element ausgewählt, und zwar aus der Spalte $\sigma(i)$, sodass auch aus jeder Spalte genau ein Element stammt.
\end{itemize}
Für $n=2$ und $A = \begin{pmatrix} a_{11} & a_{12} \\ a_{21} & a_{22} \end{pmatrix}$ ergibt sich: $S_2$ enthält die Permutationen $\sigma_1 = (1,2)$ (Identität, $\sigma_1(1)=1, \sigma_1(2)=2$) und $\sigma_2 = (2,1)$ (Vertauschung, $\sigma_2(1)=2, \sigma_2(2)=1$).
\begin{itemize}
    \item $\sigma_1 = (1,2)$: Anzahl Fehlstände = 0. $\text{sgn}(\sigma_1)=(-1)^0=+1$. Term: $a_{1\sigma_1(1)}a_{2\sigma_1(2)} = a_{11}a_{22}$.
    \item $\sigma_2 = (2,1)$: Ein Fehlstand: $(1,2)$, da $1<2$ aber $\sigma_2(1)=2 > \sigma_2(2)=1$. $\text{sgn}(\sigma_2)=(-1)^1=-1$. Term: $a_{1\sigma_2(1)}a_{2\sigma_2(2)} = a_{12}a_{21}$.
\end{itemize}
Somit ist $\det(A) = (+1) \cdot a_{11}a_{22} + (-1) \cdot a_{12}a_{21} = a_{11}a_{22} - a_{12}a_{21}$. Dies entspricht der bekannten Formel für $2 \times 2$-Matrizen.

\subsubsection*{Beispiel für eine $3 \times 3$-Matrix}
Betrachtet sei die Matrix $A = \begin{pmatrix} 1 & 2 & 3 \\ 0 & 4 & 5 \\ 1 & 0 & 6 \end{pmatrix}$.
Für $n=3$ gibt es $3! = 6$ Permutationen in $S_3$. Die Elemente sind $a_{11}=1, a_{12}=2, a_{13}=3, a_{21}=0, a_{22}=4, a_{23}=5, a_{31}=1, a_{32}=0, a_{33}=6$.

\begin{enumerate}
    \item $\sigma_1 = (1,2,3)$ (d.h. $1 \mapsto 1, 2 \mapsto 2, 3 \mapsto 3$). Fehlstände: 0. $\text{sgn}(\sigma_1)=+1$. \\
    Produkt: $a_{11}a_{22}a_{33} = 1 \cdot 4 \cdot 6 = 24$. Term: $(+1) \cdot 24 = 24$.
    \item $\sigma_2 = (1,3,2)$ (d.h. $1 \mapsto 1, 2 \mapsto 3, 3 \mapsto 2$). Fehlstand: $(2,3)$ (weil $2<3$, aber $\sigma_2(2)=3 > \sigma_2(3)=2$). Anzahl Fehlstände: 1. $\text{sgn}(\sigma_2)=-1$. \\
    Produkt: $a_{11}a_{23}a_{32} = 1 \cdot 5 \cdot 0 = 0$. Term: $(-1) \cdot 0 = 0$.
    \item $\sigma_3 = (2,1,3)$ (d.h. $1 \mapsto 2, 2 \mapsto 1, 3 \mapsto 3$). Fehlstand: $(1,2)$ (weil $1<2$, aber $\sigma_3(1)=2 > \sigma_3(2)=1$). Anzahl Fehlstände: 1. $\text{sgn}(\sigma_3)=-1$. \\
    Produkt: $a_{12}a_{21}a_{33} = 2 \cdot 0 \cdot 6 = 0$. Term: $(-1) \cdot 0 = 0$.
    \item $\sigma_4 = (2,3,1)$ (d.h. $1 \mapsto 2, 2 \mapsto 3, 3 \mapsto 1$). Fehlstände: $(1,3)$ (weil $1<3$, $\sigma_4(1)=2 > \sigma_4(3)=1$); $(2,3)$ (weil $2<3$, $\sigma_4(2)=3 > \sigma_4(3)=1$). Anzahl Fehlstände: 2. $\text{sgn}(\sigma_4)=+1$. \\
    Produkt: $a_{12}a_{23}a_{31} = 2 \cdot 5 \cdot 1 = 10$. Term: $(+1) \cdot 10 = 10$.
    \item $\sigma_5 = (3,1,2)$ (d.h. $1 \mapsto 3, 2 \mapsto 1, 3 \mapsto 2$). Fehlstände: $(1,2)$ (weil $1<2$, $\sigma_5(1)=3 > \sigma_5(2)=1$); $(1,3)$ (weil $1<3$, $\sigma_5(1)=3 > \sigma_5(3)=2$). Anzahl Fehlstände: 2. $\text{sgn}(\sigma_5)=+1$. \\
    Produkt: $a_{13}a_{21}a_{32} = 3 \cdot 0 \cdot 0 = 0$. Term: $(+1) \cdot 0 = 0$.
    \item $\sigma_6 = (3,2,1)$ (d.h. $1 \mapsto 3, 2 \mapsto 2, 3 \mapsto 1$). Fehlstände: $(1,2)$ (weil $1<2$, $\sigma_6(1)=3 > \sigma_6(2)=2$); $(1,3)$ (weil $1<3$, $\sigma_6(1)=3 > \sigma_6(3)=1$); $(2,3)$ (weil $2<3$, $\sigma_6(2)=2 > \sigma_6(3)=1$). Anzahl Fehlstände: 3. $\text{sgn}(\sigma_6)=-1$. \\
    Produkt: $a_{13}a_{22}a_{31} = 3 \cdot 4 \cdot 1 = 12$. Term: $(-1) \cdot 12 = -12$.
\end{enumerate}
Die Determinante ist die Summe dieser Terme:
\[ \det(A) = 24 + 0 + 0 + 10 + 0 - 12 = 22 \]
Dies ist dasselbe Ergebnis, das man mit der Regel von Sarrus für $3 \times 3$-Matrizen erhalten würde. Die Leibniz-Formel ist jedoch allgemeingültig für Matrizen beliebiger quadratischer Größe, während die Sarrus-Regel nur für $3 \times 3$-Matrizen gilt.

\subsection{Kästchenregel}

Bei der Kästchenregel, auch bekannt als Determinantenformel für Blockmatrizen, kann die Determinante einer Matrix berechnet werden, indem die Matrix in Blöcke (Kästchen) unterteilt wird. Dies ist besonders nützlich, wenn einige der Blöcke Nullmatrizen sind.
Für eine Blockmatrix der Form
\[ M = \begin{pmatrix} A & B \\ 0 & D \end{pmatrix} \quad \text{oder} \quad M = \begin{pmatrix} A & 0 \\ C & D \end{pmatrix} \]
wobei $A$ und $D$ quadratische Matrizen sind (und $0$ eine Nullmatrix passender Größe ist), ist die Determinante gegeben durch
\[ \det(M) = \det(A) \cdot \det(D) \]
Diese Regel vereinfacht die Berechnung erheblich, wenn Matrizen in eine solche obere oder untere Blockdreiecksform gebracht werden können. Die Kästchenregel bietet sich besonders bei $\mathbb{R}^{n \times n}$ Matrizen an, bei denen $n \geq 4$ ist und eine geeignete Blockstruktur vorliegt oder erzeugt werden kann.

\begin{longtable}{p{10cm}}

    \hline
    \multicolumn{1}{c}{\textbf{Linearkombination}} \\
    \hline
    \endfirsthead

    \hline
    \multicolumn{1}{c}{\tablename\ \thetable\ -- \textit{Fortführung von vorherier Seite}} \\
    \hline
    \multicolumn{1}{c}{\textbf{Linearkombination}} \\
    \hline
    \endhead

    \hline
    \multicolumn{1}{r}{\textit{Fortsetzung siehe nächste Seite}} \\
    \endfoot

    \hline
    \endlastfoot

    $\displaystyle\begin{pmatrix} % Matrix statt array für pmatrix-Stil
    1 & 0 & 2 & 0 \\
    1 & 2 & 9 & 0 \\
    2 & 0 & 1 & 5 \\
    1 & 1 & 1 & 1
    \end{pmatrix}$\\\hline
    I und IV tauschen (Vorzeichenwechsel für Determinante: $\cdot (-1)$) \\\hline\pagebreak[0] % Hinweis auf Vorzeichenwechsel


    $\displaystyle\begin{pmatrix}
    1 & 1 & 1 & 1 \\
    1 & 2 & 9 & 0 \\
    2 & 0 & 1 & 5 \\
    1 & 0 & 2 & 0
    \end{pmatrix}$\\\hline
    II - I $\rightarrow$ II \\ % Korrekte Notation für Zeilenumformung
    III - 2I $\rightarrow$ III \\
    IV - I $\rightarrow$ IV \\\hline\pagebreak[0]

    $\displaystyle\begin{pmatrix}
    1 & 1 & 1 & 1 \\
    0 & 1 & 8 & -1 \\
    0 & -2 & -1 & 3 \\
    0 & -1 & 1 & -1
    \end{pmatrix}$\\\hline
    III + 2II $\rightarrow$ III \\
    IV + II $\rightarrow$ IV \\\hline\pagebreak[0]

    $\displaystyle\begin{pmatrix}
    1 & 1 & 1 & 1 \\
    0 & 1 & 8 & -1 \\
    0 & 0 & 15 & 1 \\
    0 & 0 & 9 & -2
    \end{pmatrix}$\\\hline
    % Hier ist die Matrix B aus dem Beispiel erreicht, aber der Gauß-Algorithmus könnte weitergeführt werden.
    % Die gezeigte Matrix ist B = \left( \begin{array}{cc|cc} 1 & 1 & 1 & 1 \\ 0 & 1 & 8 & -1 \\ \hline 0 & 0 & 15 & 1 \\ 0 & 0 & 9 & -2 \end{array} \right)
    % und nicht die ursprüngliche Matrix A.
    % Die Determinante dieser umgeformten Matrix ist (-1) * det(ursprüngliche Matrix).

\end{longtable}

Die im Beispiel umgeformte Matrix (nennen wir sie $M'$) nach Zeilentausch und Umformungen ist:
\[
M' = \begin{pmatrix}
    1 & 1 & 1 & 1 \\
    0 & 1 & 8 & -1 \\
    0 & 0 & 15 & 1 \\
    0 & 0 & 9 & -2
    \end{pmatrix}
\]
Diese Matrix hat eine obere Dreiecksblockstruktur, wenn man sie betrachtet als:
\[
M' = \left( \begin{array}{cc|cc}
1 & 1 & 1 & 1 \\
0 & 1 & 8 & -1 \\
\hline
0 & 0 & 15 & 1 \\
0 & 0 & 9 & -2
\end{array} \right)
= \begin{pmatrix} A' & B' \\ 0 & D' \end{pmatrix}
\]
wobei $A' = \begin{pmatrix} 1 & 1 \\ 0 & 1 \end{pmatrix}$ und $D' = \begin{pmatrix} 15 & 1 \\ 9 & -2 \end{pmatrix}$.
Die Determinante ist $\det(M') = \det(A') \cdot \det(D')$.

\bigskip

\begin{tikzpicture}[
    every node/.style={minimum size=0.7cm},
    positive/.style={blue, thick},
    negative/.style={red, thick, dashed},
    block/.style={draw, very thick, rounded corners, inner sep=3pt}
  ]
  \matrix (m) [matrix of math nodes,
               nodes in empty cells,
               column sep=0.7cm, row sep=0.7cm,
               ampersand replacement=\&]
  {
    1 \& 1 \& 1 \& 1 \\ % Zeile 1 von M'
    0 \& 1 \& 8 \& -1 \\ % Zeile 2 von M'
    0 \& 0 \& 15 \& 1 \\ % Zeile 3 von M'
    0 \& 0 \& 9 \& -2 \\ % Zeile 4 von M'
  };

  \draw[gray, thin] (m-2-1.south west) -- (m-2-4.south east);
  \draw[gray, thin] (m-1-2.north east) -- (m-4-2.south east);

  \node[block, fit=(m-1-1) (m-2-2), label={[yshift=-0.1cm]below:$\det(A')$}] (blockA) {};
  \node[block, fit=(m-3-3) (m-4-4), label={[yshift=-0.1cm]below:$\det(D')$}] (blockD) {};

  \draw[positive, shorten >=2pt, shorten <=2pt] (m-1-1.center) -- (m-2-2.center); % 1*1
  % keine negative Diagonale für A', da 0*1=0

  \draw[positive, shorten >=2pt, shorten <=2pt] (m-3-3.center) -- (m-4-4.center); % 15*(-2)
  \draw[negative, shorten >=2pt, shorten <=2pt] (m-3-4.center) -- (m-4-3.center); % 1*9

  \node[anchor=west, text width=7cm, at={($(m-2-4.east)+(0.5cm,0)$)}] (formel)
    {
     $\det(A') = (1 \cdot 1 - 1 \cdot 0) = 1$ \\
     $\det(D') = (15 \cdot (-2) - 1 \cdot 9) = -30 - 9 = -39$ \\
     $\det(M') = \det(A') \cdot \det(D') = 1 \cdot (-39) = -39$
    };
\end{tikzpicture}

Da die Matrix $M'$ aus der ursprünglichen Matrix durch einen Zeilentausch und elementare Zeilenumformungen, die die Determinante nicht ändern (Addition eines Vielfachen einer Zeile zu einer anderen), hervorgegangen ist, gilt: $\det(\text{ursprüngliche Matrix}) = (-1) \cdot \det(M') = (-1) \cdot (-39) = 39$.

\section{Transponiertheit}
\begin{itemize}
    \item Die Transponierte einer Matrix \(A\), bezeichnet als \(A^T\), entsteht durch Spiegelung der Elemente an der Hauptdiagonalen.
    \item Das Element \(A_{m,n}\) der Ursprungsmatrix wird zum Element \(A_{n,m}\) der transponierten Matrix.
    \item Eine \(m \times n\)-Matrix wird zu einer \(n \times m\)-Matrix.
    \item \textbf{Beispiel:}
    \[
    A = \begin{pmatrix}
        1 & 2 & 3 \\
        4 & 5 & 6
    \end{pmatrix}
    \rightarrow
    A^T = \begin{pmatrix}
        1 & 4 \\
        2 & 5 \\
        3 & 6
    \end{pmatrix}
    \]
\end{itemize}