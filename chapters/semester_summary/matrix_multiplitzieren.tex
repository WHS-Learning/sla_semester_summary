\chapter{Matritzen Multiplitzieren}
\label{matrix_multiplitzieren}

Zwei Matritzen $A$ und $B$ dürfen nur miteinander Multiplitziert werden, wenn
$A$ so viele Spalten hat, wie $B$ Zeilen. Sein die Matritzen $A = \begin{pmatrix}
        1 & 3 \\
        0 & 1 \\
        1 & 3
    \end{pmatrix}, B = \begin{pmatrix}
        1 & 2 \\
        3 & 4
    \end{pmatrix}$, so werden sie wie folgt Multiplitziert

\begin{align*}
    A \cdot B                                       \\
    \begin{pmatrix}
        1 & 3 \\
        0 & 1 \\
        1 & 3
    \end{pmatrix} \cdot \begin{pmatrix}
                            1 & 2 \\
                            2 & 4
                        \end{pmatrix}              \\
    = \begin{pmatrix}
          1 \cdot 1 + 3 \cdot 2 & 1 \cdot 3 + 3 \cdot 4 \\
          0 \cdot 1 + 1 \cdot 2 & 0 \cdot 3 + 1 \cdot 4 \\
          1 \cdot 1 + 3 \cdot 2 & 1 \cdot 3 + 3 \cdot 4
      \end{pmatrix} \\
    = \begin{pmatrix}
          7 & 15 \\
          2 & 4  \\
          7 & 15
      \end{pmatrix}
\end{align*}