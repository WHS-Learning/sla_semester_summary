\chapter{Linearkombinationen}

Eine Linearkombination von Vektoren ist eine Summe dieser Vektoren, wobei jeder Vektor zuvor mit einem Skalar (einer reellen Zahl) multipliziert wird. Das Ergebnis einer solchen Operation ist wiederum ein Vektor.

Ein Vektor $\vec{x}$ wird als Linearkombination der Vektoren $\vec{v}_1, \vec{v}_2, \ldots, \vec{v}_n$ bezeichnet, falls Skalare $k_1, k_2, \ldots, k_n \in \mathbb{R}$ existieren, sodass gilt:

\[
    \vec{x} = k_1 \cdot \vec{v}_1 + k_2 \cdot \vec{v}_2 + \ldots + k_n \cdot \vec{v}_n
\]

Um festzustellen, ob ein gegebener Vektor $\vec{x}$ als Linearkombination anderer Vektoren dargestellt werden kann und um die entsprechenden Skalare $k_i$ zu bestimmen, wird in der Regel ein lineares Gleichungssystem aufgestellt und gelöst.

\section{Beispiel}

Gegeben sind die Vektoren

\[
    \vec{u} = \begin{pmatrix} 2 \\ 4 \\ 1 \end{pmatrix}, \quad \vec{v} = \begin{pmatrix} 0 \\ 0 \\ 1 \end{pmatrix}, \quad \text{und} \quad \vec{w} = \begin{pmatrix} 1 \\ 2 \\ 0 \end{pmatrix}.
\]

Es soll geprüft werden, ob $\vec{w}$ eine Linearkombination von $\vec{u}$ und $\vec{v}$ ist. Gesucht sind also Skalare $a, b \in \mathbb{R}$, sodass die Gleichung $a \cdot \vec{u} + b \cdot \vec{v} = \vec{w}$ erfüllt ist.

Einsetzen der Vektorkomponenten führt zu:

\[
    a \begin{pmatrix} 2 \\ 4 \\ 1 \end{pmatrix} + b \begin{pmatrix} 0 \\ 0 \\ 1 \end{pmatrix} = \begin{pmatrix} 1 \\ 2 \\ 0 \end{pmatrix}
\]

Daraus ergibt sich das folgende lineare Gleichungssystem:
\begin{align*}
    \begin{cases}
        \text{I:\@} & a \cdot 2 + b \cdot 0 = 1 \\
        \text{II:\@} & a \cdot 4 + b \cdot 0 = 2 \\
        \text{III:\@} & a \cdot 1 + b \cdot 1 = 0
    \end{cases} \\
    \begin{cases}
        \text{I:\@} & a \cdot 2 = 1 \quad | : 2 \Leftrightarrow a = \frac{1}{2}\\
        \text{II:\@} & a \cdot 4 = 2 \quad | : 4 \Leftrightarrow a = \frac{1}{2}\\
        \text{III:\@} & a \cdot 1 + b \cdot 1 = 0
    \end{cases} \\
    \text{Einsetzen von } a=\frac{1}{2} \text{ in Gleichung III:} \\
    \begin{cases}
        \text{I:\@} & a = \frac{1}{2}\\
        \text{II:\@} & a = \frac{1}{2}\\
        \text{III:\@} & \frac{1}{2} \cdot 1 + b \cdot 1 = 0 \Leftrightarrow \frac{1}{2} + b = 0 \quad |- \frac{1}{2} \Leftrightarrow b = -\frac{1}{2}
    \end{cases}
\end{align*}
Die Skalare sind $a = \frac{1}{2}$ und $b = -\frac{1}{2}$. Damit ist die Bedingung erfüllt und der Vektor $\vec{w}$ lässt sich als Linearkombination der Vektoren $\vec{u}$ und $\vec{v}$ darstellen:
$$ \vec{w} = \frac{1}{2} \vec{u} - \frac{1}{2} \vec{v} $$
Zur Überprüfung kann das Ergebnis eingesetzt werden:
$$ \frac{1}{2} \begin{pmatrix} 2 \\ 4 \\ 1 \end{pmatrix} - \frac{1}{2} \begin{pmatrix} 0 \\ 0 \\ 1 \end{pmatrix} = \begin{pmatrix} 1 \\ 2 \\ \frac{1}{2} \end{pmatrix} - \begin{pmatrix} 0 \\ 0 \\ \frac{1}{2} \end{pmatrix} = \begin{pmatrix} 1 \\ 2 \\ 0 \end{pmatrix} = \vec{w} $$