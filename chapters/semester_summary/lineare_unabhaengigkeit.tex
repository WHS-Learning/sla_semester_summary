\chapter{Lineare Unabhängigkeit}

\section{Zwei Vektoren}

Zwei Vektoren $\vec{a}$ und $\vec{b}$, die beide nicht der Nullvektor sind,
werden als linear \textbf{abhängig} bezeichnet, wenn einer ein skalares
Vielfaches des anderen ist. Das bedeutet, es existiert eine Zahl (Skalar) $k
    \in \mathbb{R}$, für die $\vec{a} = k \cdot \vec{b}$ gilt.

Sind die Vektoren nicht linear abhängig, so sind sie linear
\textbf{unabhängig}. Äquivalent dazu kann die lineare
Unabhängigkeit/Abhängigkeit über die Linearkombination zum Nullvektor geprüft
werden: Die Vektoren $\vec{a}$ und $\vec{b}$ sind linear \textbf{unabhängig},
wenn die Vektorgleichung
\[ x_1 \vec{a} + x_2 \vec{b} = \vec{0} \]
nur die triviale Lösung $x_1 = 0$ und $x_2 = 0$ besitzt. Gibt es hingegen eine
nichttriviale Lösung (bei der $x_1$ oder $x_2$ oder beide ungleich Null sind),
sind die Vektoren linear \textbf{abhängig}.

\subsection*{Beispiel}
Gegeben seien die Vektoren
\[
    \vec{a} = \begin{pmatrix} 2 \\ 0 \\ 1 \end{pmatrix}, \quad \vec{b} = \begin{pmatrix} 4 \\ 6 \\ 8 \end{pmatrix}.
\]
Es wird geprüft, ob ein Skalar $k$ existiert, sodass $\vec{a} = k \cdot
    \vec{b}$:
\begin{align*}
    \begin{pmatrix} 2 \\ 0 \\ 1 \end{pmatrix} & = k \cdot \begin{pmatrix} 4 \\ 6 \\ 8 \end{pmatrix}           \\
    \Rightarrow \quad                         & \begin{cases}
                                                    2 = k \cdot 4 \quad \Rightarrow k = \frac{2}{4} = \frac{1}{2} \\
                                                    0 = k \cdot 6 \quad \Rightarrow k = \frac{0}{6} = 0           \\
                                                    1 = k \cdot 8 \quad \Rightarrow k = \frac{1}{8}
                                                \end{cases}
\end{align*}
Da die Werte für $k$ ($\frac{1}{2}$, $0$, $\frac{1}{8}$) nicht übereinstimmen, existiert kein einheitlicher Skalar $k$, der alle drei Gleichungen gleichzeitig erfüllt. Somit ist $\vec{a}$ kein skalares Vielfaches von $\vec{b}$. Die Vektoren $\vec{a}$ und $\vec{b}$ sind daher linear \textbf{unabhängig}.

Zur Überprüfung mit dem alternativen Ansatz $x_1 \vec{a} + x_2 \vec{b} =
    \vec{0}$:
\begin{align*}
    x_1 \begin{pmatrix} 2 \\ 0 \\ 1 \end{pmatrix} + x_2 \begin{pmatrix} 4 \\ 6 \\ 8 \end{pmatrix} & = \begin{pmatrix} 0 \\ 0 \\ 0 \end{pmatrix} \\
    \Rightarrow \quad                                                                             & \begin{cases}
                                                                                                        2x_1 + 4x_2 = 0           \\
                                                                                                        \phantom{0x_1} + 6x_2 = 0 \\
                                                                                                        \phantom{2}x_1 + 8x_2 = 0
                                                                                                    \end{cases}
\end{align*}
Aus der zweiten Gleichung ($6x_2 = 0$) folgt direkt $x_2 = 0$.
Setzt man $x_2 = 0$ in die erste Gleichung ein, erhält man $2x_1 + 4 \cdot 0 = 0 \Rightarrow 2x_1 = 0 \Rightarrow x_1 = 0$.
Die dritte Gleichung ($x_1 + 8x_2 = 0$) ist mit $x_1=0$ und $x_2=0$ ebenfalls erfüllt ($0 + 8 \cdot 0 = 0$).
Da $x_1=0$ und $x_2=0$ die einzige Lösung ist (triviale Lösung), sind die Vektoren $\vec{a}$ und $\vec{b}$ linear \textbf{unabhängig}.

\section{Beliebig viele Vektoren}

Eine Menge von $n$ Vektoren $\vec{v}_1, \vec{v}_2, \dots, \vec{v}_n$ heißt
linear \textbf{unabhängig}, wenn die Vektorgleichung
\[
    x_1 \vec{v}_1 + x_2 \vec{v}_2 + \dots + x_n \vec{v}_n = \vec{0}
\]
ausschließlich die triviale Lösung $x_1 = x_2 = \dots = x_n = 0$ besitzt.
Existiert hingegen mindestens eine nichttriviale Lösung (d.h. mindestens ein
Koeffizient $x_i \neq 0$), so sind die Vektoren linear \textbf{abhängig}.

\subsection*{Beispiel}
Es soll geprüft werden, ob die folgenden Vektoren linear unabhängig oder
abhängig sind:
\[
    \vec{a} = \begin{pmatrix} 1 \\ 2 \\ 1 \end{pmatrix}, \quad \vec{b} = \begin{pmatrix} -2 \\ 1 \\ 3 \end{pmatrix}, \quad \vec{c} = \begin{pmatrix} 4 \\ 3 \\ -1 \end{pmatrix}.
\]
Dazu wird der Ansatz $x_1 \vec{a} + x_2 \vec{b} + x_3 \vec{c} = \vec{0}$
verfolgt. Dies führt zu dem homogenen linearen Gleichungssystem:
\begin{align*}
    1x_1 - 2x_2 + 4x_3 & = 0 \\
    2x_1 + 1x_2 + 3x_3 & = 0 \\
    1x_1 + 3x_2 - 1x_3 & = 0
\end{align*}
Dieses System wird mithilfe des \nameref{gauss_jordan_verfahren} (Gauß-Jordan-Elimination) gelöst. Betrachtet wird die Koeffizientenmatrix:

\begin{longtable}{p{10cm}}
    \hline
    \multicolumn{1}{c}{\textbf{Linearkombination}}                                                                                                                                                         \\
    \hline
    \endfirsthead

    \hline
    \multicolumn{1}{c}{\tablename\ \thetable\ -- \textit{Fortführung von vorherier Seite}}                                                                                                                 \\
    \hline
    \multicolumn{1}{c}{\textbf{Linearkombination}}                                                                                                                                                         \\
    \hline
    \endhead

    \hline
    \multicolumn{1}{r}{\textit{Fortsetzung siehe nächste Seite}}                                                                                                                                           \\
    \endfoot

    \hline
    \endlastfoot

    $\displaystyle\begin{matrix}
                          1 & -2 & 4  \\
                          2 & 1  & 3  \\
                          1 & 3  & -1
                      \end{matrix}$                                                                                                                                                                            \\\hline
    Operation: III - I                                                                                                                                                                                     \\\hline\pagebreak[0]
    $\displaystyle\begin{matrix}
                          1 & -2 & 4 \\
                          2 & 1  & 3 \\
                          0 & -5 & 5
                      \end{matrix}$                                                                                                                                                                            \\\hline
    Operation: II - 2I                                                                                                                                                                                     \\\hline\pagebreak[0]
    $\displaystyle\begin{matrix}
                          1 & -2 & 4  \\
                          0 & 5  & -5 \\
                          0 & -5 & 5
                      \end{matrix}$                                                                                                                                                                            \\\hline
    Operation: III + II                                                                                                                                                                                    \\\hline\pagebreak[0]
    $\displaystyle\begin{matrix}
                          1 & -2 & 4  \\
                          0 & 5  & -5 \\
                          0 & 0  & 0
                      \end{matrix}$                                                                                                                                                                            \\\hline
    Hier ist jetzt eine Nullzeile entstanden. Das bedeutet nur, dass das Gleichungssystem nicht eindeutig lösbar ist. Diese muss nicht mehr betrachtet werden, da sie als einzige Aussage $0 = 0$ liefert. \\\hline\pagebreak[0]
    Operation: 5I + 2II                                                                                                                                                                                    \\\hline\pagebreak[0]
    $\displaystyle\begin{matrix}
                          1 & 0 & 10 \\
                          0 & 5 & -5
                      \end{matrix}$                                                                                                                                                                            \\\hline
    Operation: II + 2I                                                                                                                                                                                     \\\hline\pagebreak[0]
    $\displaystyle\begin{matrix}
                          1 & 0 & 10 \\
                          0 & 5 & 0
                      \end{matrix}$                                                                                                                                                                            \\\hline
\end{longtable}

Aus der reduzierten Zeilenstufenform lassen sich die folgenden Gleichungen
ablesen:
\begin{align*}
    \text{I:\@} \quad x_1 + 2x_3 & = 0 \quad \Rightarrow \quad x_1 = -2x_3 \\
    \text{II:\@} \quad x_2 - x_3 & = 0 \quad \Rightarrow \quad x_2 = x_3
\end{align*}
Setzt man $x_3 = t$, wobei $t \in \mathbb{R}$ ein freier Parameter ist, so erhält man die allgemeine Lösung des Gleichungssystems:
\[
    \begin{pmatrix} x_1 \\ x_2 \\ x_3 \end{pmatrix} = \begin{pmatrix} -2t \\ t \\ t \end{pmatrix} = t \begin{pmatrix} -2 \\ 1 \\ 1 \end{pmatrix}.
\]
Da es nichttriviale Lösungen gibt (z.B. für $t=1$ ergibt sich die Lösung
$x_1=-2, x_2=1, x_3=1$), sind die Vektoren $\vec{a}, \vec{b}, \vec{c}$ linear
\textbf{abhängig}.

Lineare abhängigkeit kann auch über die \nameref{determinante} geprüft werden.
Hierbei sind die Vektoren Linear abhängig, wenn $\det(A) = 0$ ist, wobei $A$
die Matrixkombination aus den Vektoren ist. In dem o.g. Beispiel ist $A = \begin{pmatrix}
        1 & -2 & 4  \\
        2 & 1  & 3  \\
        1 & 3  & -1
    \end{pmatrix}$