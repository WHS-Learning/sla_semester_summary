\chapter{Definitheit}

Eine quadratische Matrix \(A \in \mathbb{R}^{n \times n}\) kann eine der folgenden Eigenschaften haben:
\begin{itemize}
    \item Positiv definit
    \item Positiv semidefinit
    \item Negativ definit
    \item Negativ semidefinit
    \item Indefinit
\end{itemize}

\section{Bestimmung über Eigenwerte}
Die Definitheit lässt sich anhand der Vorzeichen der Eigenwerte \(\lambda_i\) einer Matrix bestimmen.
\begin{itemize}
    \item \textbf{Positiv definit}, wenn alle \(\lambda_i > 0\).
    \item \textbf{Positiv semidefinit}, wenn alle \(\lambda_i \geq 0\).
    \item \textbf{Negativ definit}, wenn alle \(\lambda_i < 0\).
    \item \textbf{Negativ semidefinit}, wenn alle \(\lambda_i \leq 0\).
    \item \textbf{Indefinit}, wenn es sowohl positive als auch negative Eigenwerte gibt.
\end{itemize}

\subsection{Beispiel}
Bei einer Dreiecksmatrix stehen die Eigenwerte auf der Hauptdiagonalen. Die Matrix
\[ A = \begin{pmatrix}
    1 & 4 & 2 \\
    0 & 2 & 4 \\
    0 & 0 & 3
\end{pmatrix} \]
hat die Eigenwerte \(\lambda_1 = 1, \lambda_2 = 2, \lambda_3 = 3\). Da alle Eigenwerte größer als Null sind, ist die Matrix \textbf{positiv definit}.

\section{Hurwitz-Kriterium}
Um die aufwändige Berechnung der Eigenwerte zu umgehen, kann das Hurwitz-Kriterium (auch Kriterium der Hauptminoren) genutzt werden. Dabei werden die Determinanten der führenden Hauptminoren \(\Delta_k\) betrachtet.
\begin{center}
    \begin{tikzpicture}[
        coeff/.style={draw, minimum width=1.2cm, minimum height=0.75cm, anchor=center, font=\footnotesize},
        zerocoeff/.style={minimum width=1.2cm, minimum height=0.75cm, anchor=center, font=\footnotesize}
    ]
        \matrix[matrix of nodes, 
                nodes={coeff}, 
                ampersand replacement=\&, 
                column sep=-\pgflinewidth,
                row sep=-\pgflinewidth
            ] (H) {
        $a_{1,1}$ \& $a_{1,2}$ \& $a_{1,3}$ \& $\dots$ \& $a_{1,n}$ \\
        $a_{2,1}$ \& $a_{2,2}$ \& $a_{2,3}$ \& $\dots$ \& $a_{2,n}$ \\
        $a_{3,1}$ \& $a_{3,2}$ \& $a_{3,3}$ \& $\dots$ \& $a_{3,n}$ \\
        $\vdots$  \& $\vdots$  \& $\vdots$  \& $\ddots$ \& $\vdots$  \\
        $a_{n,1}$ \& $a_{n,2}$ \& $a_{n,3}$ \& $\dots$ \& $a_{n,n}$ \\
        };

        \draw[blue, thick, rounded corners=1pt] (H-1-1.north west) rectangle (H-1-1.south east);
        \node at (H-1-1.north east) [above left, blue, scale=0.7, xshift=-2mm, yshift=1mm] {$\Delta_1$};
        \node at (H-1-1.north east) [above left, red, scale=0.7, xshift=12mm, yshift=1mm] {$\Delta_2$};
        \node at (H-1-1.north east) [above left, green!60!black, scale=0.7, xshift=30mm, yshift=1mm] {$\Delta_3$};
        \node at (H-1-1.north east) [above left, purple!80!black, scale=0.7, xshift=65mm, yshift=1mm] {$\Delta_n$};

        \draw[red, thick, rounded corners=1pt] (H-1-1.north west) rectangle (H-2-2.south east);
        \draw[green!60!black, thick, rounded corners=1pt] (H-1-1.north west) rectangle (H-3-3.south east);
        \draw[purple!80!black, thick, rounded corners=1pt] (H-1-1.north west) rectangle (H-5-5.south east);

    \end{tikzpicture}
    \captionof{figure}{Schematische Darstellung der Hauptminoren \(\Delta_k\).}
\end{center}
Die Vorzeichenfolge der Determinanten \(\det(\Delta_k)\) ist entscheidend:
\begin{itemize}
    \item \textbf{Positiv definit}: Alle Vorzeichen sind positiv. (\texttt{+, +, +, \dots})
    \item \textbf{Negativ definit}: Die Vorzeichen alternieren, beginnend mit negativ. (\texttt{-, +, -, +, \dots}) \newline
            \textbf{Achtung}: Die Vorzeichenfolgen \texttt{+, -, +, -, \dots} und \texttt{-, -, -, -, \dots} sind kontraintuitiv \textbf{NICHT} Negativ definit.
    \item \textbf{Indefinit}: Alle anderen Vorzeichenfolgen.
    \item \textbf{Negativ bzw Positiv semidefinit}: Wenn durch auftauchende Nullen die Zeichenkette unterbrochen wird (z.B. \texttt{-, +, 0, +, 0, 0, -, \dots}), \textit{kann} die Matrix Positiv bzw Negativ semidefinit sein. \textbf{Sie kann aber auch Indefinit sein!} Hier müssen die Eigenwerte ausgerechnet werden, um die Definitheit sicher bestimmen zu können.
\end{itemize}
\textbf{Wichtig:} Noch einmal; treten Nullen bei den Determinanten auf, kann dieses vereinfachte Kriterium nicht zuverlässig zwischen semidefinit und indefinit unterscheiden.

\subsection{Beispiel}
Untersuchung der Matrix \[ A = \begin{pmatrix} -2 & 1 & 0 \\ 1 & -2 & 1 \\ 0 & 1 & -2 \end{pmatrix} \]
\begin{itemize}
    \item \(\det(\Delta_1) = -2\)
    \item \(\det(\Delta_2) = \begin{vmatrix} -2 & 1 \\ 1 & -2 \end{vmatrix} = 3\)
    \item \(\det(\Delta_3) = \det(A) = -4\)
\end{itemize}
Die Vorzeichenfolge ist \texttt{-, +, -}. Die Matrix ist somit \textbf{negativ definit}.