\chapter{Arithmetisches Mittel, Median und Moduls}

\section{Arithmetisches Mittel}

Das Arithmetische Mittel beschreibt den Durchschnitt. Beispielsweise ist der
Notendurchschnitt das Arithmetische Mittel.

\subsection{Beispiel}

\begin{table}[htbp]
    \centering
    \begin{tabular}{|c|c|c|c|c|c|c|}
        \hline
        \textbf{Note}   & 1 & 2 & 3 & 4 & 5 & 6 \\
        \hline
        \textbf{Anzahl} & 5 & 3 & 3 & 7 & 2 & 1 \\
        \hline
    \end{tabular}
    \label{tab:noten_einfach}
\end{table}

Das Arithmetische Mittel wird hier berechnet, indem die Anzahl mit der
Wertigkeit multiplitziert und durch die gesammtanzahl geteilt wird

\begin{align*}
    \frac{5 \cdot 1 + 3 \cdot 2 + 3 \cdot 3 + 7 \cdot 4 + 2 \cdot 5 + 1 \cdot 6}{21} \approx 3.0476
\end{align*}

\section{Median}

Der Durchschnitt kann von der Realität stark abweichen. Beispielsweise hat eine
Stadt $999$ Einwohler, welche je $2000€$, und einen Einwohler, welcher
$2000000$ verdient. Das Durchschnittliche einkommen dieser Stadt ist $3998€$,
allerdings verdient kein Einwohner annährend so viel Geld.

Der Median beschreibt den mittleren Wert einer sortierten Folge. Gibt es keine
Mitte, so wird das Arithmetische Mittel der beiden mittleren Werte gebildet.

\subsection{Beispiel}

Eine Stadt hat folgende Temperaturhistorie in grad Celsius:

\texttt{12, 16, 17, 21, 25, 29, 32, 36}

Das Median hier ist $\frac{21 + 25}{2} = 23$.

\section{Modus}

Der Modus ist der Wert mit der größten Häufigkeit.