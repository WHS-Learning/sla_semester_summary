\chapter{Matrizen Invertieren}

Die Inverse einer quadratischen Matrix \(A\) ist eine Matrix \(A^{-1}\), sodass
das Produkt von \(A\) und \(A^{-1}\) die Einheitsmatrix \(I\) ergibt:
\[ AA^{-1} = A^{-1}A = I \]
Eine Methode zur Berechnung der Inversen einer Matrix ist das
\nameref{gauss_jordan_verfahren}. Bei diesem Verfahren wird die zu
invertierende Matrix \(A\) um die Einheitsmatrix \(I\) derselben Dimension
erweitert, sodass eine Matrix \([A|I]\) entsteht. Durch elementare
Zeilenumformungen wird der linke Teil \(A\) in die reduzierte Zeilenstufenform
überführt. Wenn \(A\) in die Einheitsmatrix \(I\) transformiert werden kann,
ist die resultierende rechte Seite die Inverse \(A^{-1}\), also \([I|A^{-1}]\).

Eine notwendige und hinreichende Bedingung für die Invertierbarkeit einer
quadratischen Matrix ist, dass ihre \nameref{determinante} ungleich Null ist.
Ist \(\det(A) = 0\), so ist die Matrix singulär und besitzt keine Inverse.

\section{Beispiel}

Es soll die Inverse der Matrix \(A\) bestimmt werden:
\[ A = \begin{pmatrix}
        1 & 2 & 3 \\
        0 & 1 & 4 \\
        5 & 6 & 0
    \end{pmatrix} \]
Zuerst wird die Determinante von \(A\) berechnet, um die Invertierbarkeit zu
prüfen. Nach der Regel von Sarrus (für 3x3-Matrizen):
\begin{align*}
    \det(A) & = (1 \cdot 1 \cdot 0) + (2 \cdot 4 \cdot 5) + (3 \cdot 0 \cdot 6)       \\
            & \quad - (5 \cdot 1 \cdot 3) - (6 \cdot 4 \cdot 1) - (0 \cdot 0 \cdot 2) \\
            & = 0 + 40 + 0 - 15 - 24 - 0                                              \\
            & = 1
\end{align*}
Da \(\det(A) = 1 \neq 0\), ist die Matrix \(A\) invertierbar.

Nun wird das Gauß-Jordan-Verfahren angewendet. Die erweiterte Matrix ist:
\[ \left[ A | I \right] = \left[ \begin{array}{ccc|ccc}
            1 & 2 & 3 & 1 & 0 & 0 \\
            0 & 1 & 4 & 0 & 1 & 0 \\
            5 & 6 & 0 & 0 & 0 & 1
        \end{array} \right] \]

\begin{longtable}{p{4cm}|p{3cm}}

    \hline
    \multicolumn{1}{c|}{\textbf{Matrix}} & \multicolumn{1}{c}{\textbf{Inverse}}            \\
    \hline
    \endfirsthead

    \hline
    \multicolumn{2}{c}{\tablename\ \thetable\ -- \textit{Fortführung von vorherier Seite}} \\
    \hline
    \multicolumn{1}{c|}{\textbf{Matrix}} & \multicolumn{1}{c}{\textbf{Inverse}}            \\
    \hline
    \endhead

    \hline
    \multicolumn{2}{r}{\textit{Fortsetzung siehe nächste Seite}}                           \\
    \endfoot

    \hline
    \endlastfoot

    $\displaystyle\begin{matrix}
                          1 & 2 & 3 \\
                          0 & 1 & 4 \\
                          5 & 6 & 0
                      \end{matrix}$         &
    $\displaystyle\begin{matrix}
                          1 & 0 & 0 \\
                          0 & 1 & 0 \\
                          0 & 0 & 1 \\
                      \end{matrix}$                                                            \\\hline
    \multicolumn{2}{p{\dimexpr4cm+3cm+2\tabcolsep\relax}}{Operation: III - 5I}             \\\hline\pagebreak[0]
    $\displaystyle\begin{matrix}
                          1 & 2  & 3   \\
                          0 & 1  & 4   \\
                          0 & -4 & -15
                      \end{matrix}$         &
    $\displaystyle\begin{matrix}
                          1  & 0 & 0 \\
                          0  & 1 & 0 \\
                          -5 & 0 & 1 \\
                      \end{matrix}$                                                            \\\hline
    \multicolumn{2}{p{\dimexpr4cm+3cm+2\tabcolsep\relax}}{Operation: III + 4II}            \\\hline\pagebreak[0]
    $\displaystyle\begin{matrix}
                          1 & 2 & 3 \\
                          0 & 1 & 4 \\
                          0 & 0 & 1
                      \end{matrix}$         &
    $\displaystyle\begin{matrix}
                          1  & 0 & 0 \\
                          0  & 1 & 0 \\
                          -5 & 4 & 1 \\
                      \end{matrix}$                                                            \\\hline
    \multicolumn{2}{p{\dimexpr4cm+3cm+2\tabcolsep\relax}}{Operation: I - 2II}              \\\hline\pagebreak[0]
    $\displaystyle\begin{matrix}
                          1 & 0 & -5 \\
                          0 & 1 & 4  \\
                          0 & 0 & 1
                      \end{matrix}$         &
    $\displaystyle\begin{matrix}
                          1  & -2 & 0 \\
                          0  & 1  & 0 \\
                          -5 & 4  & 1 \\
                      \end{matrix}$                                                            \\\hline
    \multicolumn{2}{p{\dimexpr4cm+3cm+2\tabcolsep\relax}}{Operation: II - 4III}            \\\hline\pagebreak[0]
    $\displaystyle\begin{matrix}
                          1 & 0 & -5 \\
                          0 & 1 & 0  \\
                          0 & 0 & 1
                      \end{matrix}$         &
    $\displaystyle\begin{matrix}
                          1  & -2  & 0  \\
                          20 & -15 & -4 \\
                          -5 & 4   & 1  \\
                      \end{matrix}$                                                            \\\hline
    \multicolumn{2}{p{\dimexpr4cm+3cm+2\tabcolsep\relax}}{Operation: I + 5III}             \\\hline\pagebreak[0]
    $\displaystyle\begin{matrix}
                          1 & 0 & 0 \\
                          0 & 1 & 0 \\
                          0 & 0 & 1
                      \end{matrix}$         &
    $\displaystyle\begin{matrix}
                          -24 & 18  & 5  \\
                          20  & -15 & -4 \\
                          -5  & 4   & 1  \\
                      \end{matrix}$                                                           \\\hline
\end{longtable}

Die linke Seite ist nun die Einheitsmatrix. Die rechte Seite ist die Inverse
\(A^{-1}\).
\[ A^{-1} = \begin{pmatrix}
        -24 & 18  & 5  \\
        20  & -15 & -4 \\
        -5  & 4   & 1
    \end{pmatrix} \]

\section{Umkehrabbildung}

Eine Matrix \(A \in \mathbb{R}^{n \times n}\) kann als eine lineare Abbildung
von \(\mathbb{R}^n\) nach \(\mathbb{R}^n\) interpretiert werden. Ein Vektor
\(\vec{x} \in \mathbb{R}^n\) wird durch die Matrix \(A\) auf einen Vektor
\(\vec{y} \in \mathbb{R}^n\) abgebildet:
\[ \vec{y} = A\vec{x} \]
Diese Abbildung \(f: \mathbb{R}^n \rightarrow \mathbb{R}^n\), \(\vec{x} \mapsto
A\vec{x}\) ist bijektiv (also sowohl injektiv als auch surjektiv) genau dann,
wenn die Matrix \(A\) invertierbar ist.

Wenn die Matrix \(A\) invertierbar ist, existiert eine Umkehrabbildung
\(f^{-1}: \mathbb{R}^n \rightarrow \mathbb{R}^n\). Diese Umkehrabbildung bildet
den Vektor \(\vec{y}\) zurück auf den ursprünglichen Vektor \(\vec{x}\) ab. Die
Abbildungsmatrix dieser Umkehrabbildung ist die Inverse \(A^{-1}\):
\[ \vec{x} = A^{-1}\vec{y} \]
Somit macht die Umkehrabbildung die ursprüngliche Abbildung rückgängig. Wendet
man beide Abbildungen nacheinander an, erhält man die identische Abbildung, bei
der jeder Vektor auf sich selbst abgebildet wird:
\[ A^{-1}(A\vec{x}) = (A^{-1}A)\vec{x} = I\vec{x} = \vec{x} \]
\[ A(A^{-1}\vec{y}) = (AA^{-1})\vec{y} = I\vec{y} = \vec{y} \]
Die Existenz einer Umkehrabbildung ist also direkt an die Invertierbarkeit der
Abbildungsmatrix \(A\) geknüpft, was wiederum bedeutet, dass \(\det(A) \neq 0\)
sein muss.

\subsection{Beispiel zur Umkehrabbildung}
Betrachten wir die Matrix \(A\) aus dem vorherigen Beispiel und ihre Inverse
\(A^{-1}\):
\[ A = \begin{pmatrix}
        1 & 2 & 3 \\
        0 & 1 & 4 \\
        5 & 6 & 0
    \end{pmatrix}, \quad
    A^{-1} = \begin{pmatrix}
        -24 & 18  & 5  \\
        20  & -15 & -4 \\
        -5  & 4   & 1
    \end{pmatrix} \]
Sei \(\vec{x} = \begin{pmatrix} 1 \\ 1 \\ 1 \end{pmatrix}\). Die Abbildung mit \(A\) ergibt:
\[ \vec{y} = A\vec{x} = \begin{pmatrix}
        1 & 2 & 3 \\
        0 & 1 & 4 \\
        5 & 6 & 0
    \end{pmatrix}
    \begin{pmatrix} 1 \\ 1 \\ 1 \end{pmatrix} =
    \begin{pmatrix}
        1 \cdot 1 + 2 \cdot 1 + 3 \cdot 1 \\
        0 \cdot 1 + 1 \cdot 1 + 4 \cdot 1 \\
        5 \cdot 1 + 6 \cdot 1 + 0 \cdot 1
    \end{pmatrix} =
    \begin{pmatrix} 6 \\ 5 \\ 11 \end{pmatrix} \]
Nun wird die Umkehrabbildung mit \(A^{-1}\) auf \(\vec{y}\) angewendet, um
\(\vec{x}\) zurückzuerhalten:
\[ A^{-1}\vec{y} = \begin{pmatrix}
        -24 & 18  & 5  \\
        20  & -15 & -4 \\
        -5  & 4   & 1
    \end{pmatrix}
    \begin{pmatrix} 6 \\ 5 \\ 11 \end{pmatrix} =
    \begin{pmatrix}
        -24 \cdot 6 + 18 \cdot 5 + 5 \cdot 11 \\
        20 \cdot 6 - 15 \cdot 5 - 4 \cdot 11  \\
        -5 \cdot 6 + 4 \cdot 5 + 1 \cdot 11
    \end{pmatrix} \]
\[ = \begin{pmatrix}
        -144 + 90 + 55 \\
        120 - 75 - 44  \\
        -30 + 20 + 11
    \end{pmatrix} =
    \begin{pmatrix} 1 \\ 1 \\ 1 \end{pmatrix} = \vec{x} \]
Dies demonstriert, wie die durch \(A^{-1}\) repräsentierte Umkehrabbildung den
Vektor \(\vec{y}\) auf den ursprünglichen Vektor \(\vec{x}\) zurückführt.