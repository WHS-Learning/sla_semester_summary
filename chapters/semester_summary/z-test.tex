\chapter{Der Z-Test (100\% Klau\-sur\-auf\-gabe)}

Der Z-Test ist ein grundlegendes statistisches Verfahren zur Überprüfung von
Hypothesen über Populationsparameter. Er ermöglicht es, basierend auf einer
Stichprobe zu beurteilen, ob ein beobachtetes Ergebnis durch Zufall entstanden
ist oder ob es eine signifikante Abweichung von einer angenommenen
Grundgesamtheit darstellt.

Die Anwendbarkeit des Z-Tests beruht maßgeblich auf dem \textbf{Zentralen
    Grenzwertsatz}. Dieser Satz besagt, dass die Verteilung von
Stichprobenmittelwerten (oder Summen) sich einer Normalverteilung annähert,
wenn der Stichprobenumfang ausreichend groß ist. Dies gilt unabhängig von der
ursprünglichen Verteilung der Population und ist entscheidend für die
Durchführung parametrischer Tests wie des Z-Tests.

\section{Grundlagen eines Hypothesentests}

Jeder statistische Hypothesentest folgt einer standardisierten Struktur, die
aus den folgenden wesentlichen Komponenten besteht.

\subsection{Nullhypothese ($H_0$)}
Die Nullhypothese repräsentiert die ursprüngliche Annahme oder den bestehenden
Zustand, der vorläufig als wahr angenommen wird. Ihr Ziel ist es, diese Annahme
zu widerlegen. Häufig postuliert $H_0$, dass kein Effekt, kein Unterschied oder
keine Beziehung in der Grundgesamtheit vorliegt.
\begin{itemize}
    \item \textit{Beispiel Münzwurf:} $H_0$: Die Münze ist fair; die Wahrscheinlichkeit für „Kopf“ ist $p = 0.5$.
\end{itemize}

\subsection{Alternativhypothese ($H_1$)}
Die Alternativhypothese stellt die Annahme dar, die der Nullhypothese
entgegensteht. Sie wird akzeptiert, wenn die Nullhypothese aufgrund der
Testergebnisse abgelehnt wird. $H_1$ formuliert in der Regel den Effekt, den
Unterschied oder die Beziehung, die in den Daten vermutet wird.
\begin{itemize}
    \item \textit{Beispiel Münzwurf:} $H_1$: Die Münze ist nicht fair ($p \neq 0.5$) oder sie ist zugunsten von „Kopf“ gezinkt ($p > 0.5$).
\end{itemize}

\subsection{Signifikanzniveau ($\alpha$) und Konfidenzniveau ($1-\alpha$)}
Die Festlegung dieser Werte erfolgt vor der Testdurchführung und bestimmt die
Kriterien für die Ablehnung der Nullhypothese.
\begin{itemize}
    \item \textbf{Das Signifikanzniveau $\alpha$} (Alpha) ist die vorab festgelegte maximale Wahrscheinlichkeit, die Nullhypothese fälschlicherweise abzulehnen, obwohl sie in Wirklichkeit wahr ist (Fehler 1. Art). Übliche Werte sind $\alpha = 0.05$ (5\%) oder $\alpha = 0.01$ (1\%).
    \item \textbf{Das Konfidenzniveau}, definiert als $1-\alpha$, gibt die Wahrscheinlichkeit an, eine wahre Nullhypothese korrekt beizubehalten.
\end{itemize}
Ein Ergebnis wird als \textbf{statistisch signifikant} betrachtet, wenn die Wahrscheinlichkeit seines Auftretens unter Annahme der Gültigkeit von $H_0$ kleiner als $\alpha$ ist. In diesem Fall wird die Nullhypothese abgelehnt.

\section{Systematisches Vorgehen beim Z-Test}
Ein Hypothesentest mittels Z-Test folgt einer klaren Abfolge von Schritten:
\begin{enumerate}
    \item \textbf{Hypothesenformulierung:} Präzise Definition der Nullhypothese ($H_0$) und der Alternativhypothese ($H_1$). Festlegung, ob ein ein- oder zweiseitiger Test erforderlich ist.
    \item \textbf{Signifikanzniveauwahl:} Bestimmung des akzeptablen Risikos für einen Fehler 1. Art ($\alpha$).
    \item \textbf{Voraussetzungsprüfung:} Verifikation der Anwendbarkeit des Z-Tests (z.B. Normalverteilungsannahme, bekanntes $\sigma$ oder ausreichend großer Stichprobenumfang $n$).
    \item \textbf{Teststatistikberechnung:} Ermittlung des Z-Werts auf Basis der Stichprobendaten.
    \item \textbf{Entscheidungsfindung:} Vergleich des berechneten Z-Werts mit den kritischen Werten (aus der Z-Tabelle) oder Analyse des p-Werts im Verhältnis zu $\alpha$.
    \item \textbf{Schlussfolgerung:} Ablehnung oder Beibehaltung von $H_0$ und Interpretation der Ergebnisse im Kontext der ursprünglichen Fragestellung.
\end{enumerate}

\section{Anwendungsbeispiel 1: Einseitiger Z-Test}

Ein Unternehmen strebt eine maximale Materialfehlerquote von 20\% an. Eine
Qualitätskontrolle von 100 Bauteilen zeigt 15 Defekte. Es soll überprüft
werden, ob die angenommene Fehlerhäufigkeit bei einer Irrtumswahrscheinlichkeit
von höchstens 10\% aufrechterhalten werden kann.

\begin{align*}
    X                  & = \begin{cases}
                               0 & \text{Kein Materialfehler} \\
                               1 & \text{Materialfehler}
                           \end{cases}                                                              \\
    S_n                & = \sum_{i = 1}^{100} X_i \sim B(100, p) \quad (\text{Summe der Fehler ist binomialverteilt})  \\
    H_0:               & p \leq 0.2 \quad (\text{Fehleranteil ist höchstens 20\%})                                     \\
    H_1:               & p > 0.2 \quad (\text{Fehleranteil ist größer als 20\%})                                       \\
    \alpha             & = 0.10 \quad (\text{Signifikanzniveau von 10\%})                                              \\
                       & \rightarrow \text{Für den Test wird der Randfall von } H_0 \text{ betrachtet: } p = 0.2       \\
    E(S_n)             & = n \cdot p = 100 \cdot 0.2 = 20                                                              \\
    \sigma(S_n)        & = \sqrt{n \cdot p \cdot (1-p)} = \sqrt{100 \cdot 0.2 \cdot 0.8} = \sqrt{16} = 4               \\
    \intertext{Der standardisierte Z-Wert mittels Normalapproximation ist:}
    Z                  & = \frac{S_n - E(S_n)}{\sigma(S_n)} = \frac{S_n - 20}{4} \overset{\text{ldm}}{\approx} N(0, 1) \\
    \intertext{Die Nullhypothese wird abgelehnt, wenn der berechnete Z-Wert den kritischen Wert überschreitet. Für $\alpha=0.10$ bei einem einseitig rechten Test ist der kritische Z-Wert ca. $1.28$. Der Annahmebereich für $S_n$ ist daher:}
    \frac{S_n - 20}{4} & \leq 1.28                                                                                     \\
    S_n                & \leq 1.28 \cdot 4 + 20                                                                        \\
    S_n                & \leq 5.12 + 20                                                                                \\
    S_n                & \leq 25.12                                                                                    \\
    \intertext{Beobachtetes Ergebnis:}
    S_n                & = 15                                                                                          \\
\end{align*}
Da $15 \leq 25.12$, liegt das beobachtete Ergebnis im Annahmebereich von $H_0$. Der berechnete Z-Wert ist $Z = \frac{15 - 20}{4} = -1.25$. Da $-1.25$ nicht größer als $1.28$ ist (d.h., nicht im kritischen Bereich liegt), wird $H_0$ nicht abgelehnt.
\textbf{Schlussfolgerung:} Die Nullhypothese kann nicht abgelehnt werden. Es liegen keine statistisch signifikanten Belege (bei $\alpha=10\%$) vor, die darauf hindeuten, dass die Fehlerquote über 20\% liegt. Die angenommene Fehlerhäufigkeit kann aufrechterhalten werden.

\section{Anwendungsbeispiel 2: Zweiseitiger Z-Test}

Historisch regnete es im Sommer an durchschnittlich 20\% der Tage. Eine
aktuelle Studie untersucht, ob eine Klimaveränderung stattgefunden hat. Von 150
analysierten Sommertagen regnete es an 53 Tagen. Es soll beurteilt werden, ob
bei einer Irrtumswahrscheinlichkeit von höchstens 6\% von einem Klimawandel
ausgegangen werden kann.

\begin{align*}
    X                          & = \text{Indikator für Regentag}                                                               \\
    S_n                        & = \sum_{i = 1}^{150}X_i \sim B(150, p)                                                        \\
    H_0:                       & p = 0.2 \quad (\text{Anteil der Regentage ist 20\%})                                          \\
    H_1:                       & p \neq 0.2 \quad (\text{Anteil der Regentage weicht von 20\% ab})                             \\
    \alpha                     & = 0.06 \quad (\text{Signifikanzniveau von 6\%})                                               \\
    \mu                        & = n \cdot p = 150 \cdot 0.2 = 30                                                              \\
    \sigma                     & = \sqrt{n \cdot p \cdot (1-p)} = \sqrt{150 \cdot 0.2 \cdot 0.8} = \sqrt{24} \approx 4.899     \\
    \intertext{Der standardisierte Z-Wert für die Stichprobe ist:}
    Z                          & = \frac{S_n - \mu}{\sigma} = \frac{S_n - 30}{\sqrt{24}} \overset{\text{ldm}}{\approx} N(0, 1) \\
    \intertext{Für einen zweiseitigen Test mit $\alpha=0.06$ (was 3\% in jedem Endbereich entspricht), sind die kritischen Z-Werte ca. $\pm 1.88$. Der Annahmebereich für $H_0$ liegt zwischen diesen Werten:}
    -1.88 \cdot \sqrt{24} + 30 & \leq S_n \leq 1.88 \cdot \sqrt{24} + 30                                                       \\
    -9.21 + 30                 & \leq S_n \leq 9.21 + 30                                                                       \\
    20.79                      & \leq S_n \leq 39.21                                                                           \\
    \intertext{Gerundet auf ganze Tage (da $S_n$ eine diskrete Zählgröße ist):}
    \intertext{!Man würde die 20.79 \textbf{immer} aufrunden und die 39.21 \textbf{immer} abrunden!}
    21                         & \leq S_n \leq 39                                                                              \\
    \intertext{Beobachtetes Ergebnis:}
    S_n                        & = 53                                                                                          \\
    \intertext{Da $53 > 39$, liegt das beobachtete Ergebnis außerhalb des Annahmebereichs von $H_0$. Der berechnete Z-Wert ist $Z = \frac{53-30}{\sqrt{24}} \approx 4.695$. Da $4.695 > 1.88$ (und somit im Ablehnungsbereich liegt), wird $H_0$ abgelehnt.}
\end{align*}
\textbf{Schlussfolgerung:} Die Nullhypothese wird abgelehnt. Es kann bei einem Signifikanzniveau von 6\% statistisch gesichert davon ausgegangen werden, dass der Anteil der Regentage von 20\% abweicht, was auf einen Klimawandel hindeutet.