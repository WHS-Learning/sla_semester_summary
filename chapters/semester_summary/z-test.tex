\chapter{Z-Test}

Der Z-Test basiert auf den Satz von de Moivre-Laplace.

Wenn jemand eine Münze wirft und festgestellt werden soll, ob diese gezinkt
ist, wie oft muss das selbe Ergebnis geworfen werden?

Ist die Münze nach den Münzwurf \textit{Kopf, Kopf, Kopf, Kopf, Kopf, Kopf,
    Kopf, Zahl, Kopf, Kopf, Kopf} gezinkt? Hundertprozentig sicher kann man sich
hier nie sein, egal wie oft die Münze geworfen wird.

\section{Hypothesentests}

\subsection{Nullhypothese}

Die Hypothese über ein Wahrscheinlichkeitsverteilung, die grprüft wird und
gegebenenfalls abgelehnt werden soll. Bezeichnung: $H_0$.

Häufig: $H_0: E(X) = \mu$

\subsection{Alternativhypothese}

Die Hypothese über die entsprechende Wahrscheinlichkeit, die als Alternative zu
$H_0$ angenommen wird. Bezeichnung: $H_1$.

\subsection{Konfidenzniveau}

Das Konfidenzniveau beschreibt die Wahrscheinlichkeit, mit der (gegebenenfalls)
das Ablehnen der Nullhypothese $H_0$ mit Alternativhypothese $H_1$ zu Recht
erfolgt. Ein häufiger Wert ist hier $95\%$.

Wird die Nullhypothese $H_0$ mit einem Konfidenzniveau von $95\%$ abgelehnt, so
ist das Ergebnis unter der Nullhypothese unter $5\%$. Daher kann die
Nullhypothese nicht verworfen werden.

Wenn die Wahrscheinlichkeit für das Eintreten des Testergebnisses oder eines
noch extremeren Ergebnisses über $5\%$ beträgt, so kann die Nullhypothese nicht
verworfen werden. Dies bedeutet nicht, dass diese wahr sein muss.

\subsection{Beispiel: Münze}

Um zu prüfen, ob eine Münze gezinkz ist, soll sie 100 mal geworfen werden.
Zuvor soll ermittelt werden, ab wie vielen "Zahl"-Würfen die Münze als gezinkt
angenommen wird.

0 = Kopf; 1 = Zahl

Einzelner Münzwurf: $X_1 ~ B\left(\frac{1}{2}\right), (i \in \left\{1, \dots,
    100\right\})$

\begin{align*}
    S_n = \sum_{i=1}^{100} X_i ~ B\left(100, \frac{1}{2}\right) \\
    E(S_n) = 100 \cdot \frac{1}{2} = 50                         \\
    Var(S_n) = 100 \cdot \frac{1}{2} \cdot \frac{1}{2} = 25     \\
    \sigma (S_n) = \sqrt{25} = 5                                \\
    \frac{S_n - 50}{5} \approx N(0,1)
\end{align*}

Bestimme Annahne- und Ablehnungsbereich für eine $N(0,1)$-Verteilte
Zuvallsfahl:

Dafür:

Nullhypothese:
\begin{align*}
    H_0: E(X) = \mu_0 = 0
\end{align*}

Alternativhypothese:
\begin{align*}
    H_1: \text{Münze zeigt Überproportional häufig Kopf} \\
    \text{Überproportional viele Treffer}                \\
    E(X) < 0                                             \\
    z_0 = 1.65 \text{ bzw } z_0 = -1.65                  \\
\end{align*}

d.h. Entscheidungsregel: Falls $\frac{S_n - 50}{5} < -1.65$: $H_1$ wird
abgelehnt.

Falls $\frac{S_n - 50}{5} \geq -1.65$: $H_1$ wird nicht abgelehnt.

Also:

\begin{align*}
    H_0 \text{ wird abgelehnt}               \\
    \Leftrightarrow S_n - 50 < -1.65 \cdot 5 \\
    \Leftrightarrow S_n < -1.65 \cdot 5 + 50 \\
    \Leftrightarrow S_n < 41.75
    \text{d.h. } S_n \leq 41                 \\
    H_0 \text{ wird nicht abgelehnt}         \\
    S_n \geq -1.65 \cdot 5 + 50              \\
    S_n \geq 41.75                           \\
    \text{d.h. } S_n \geq 42
\end{align*}

Ergebnis des Münzwurfs 80 mal Zahl

\begin{align*}
    S_n = 80    \\
    S_n \geq 42 \\
\end{align*}

liegt im Annahmebereich, d.h. die Nullhypothese wird nicht abgelehnt.

\subsection{Allgemeines Vorgehen beim appoximativen Binomialtest}

\begin{itemize}
    \item Nullhypothese: $H_0: S_n ~ B(n, p)$
    \item Bestimme $\mu = E(S_n) = n \cdot p$ und $\sigma = \sqrt{Var(S_n)} = \sqrt{n
                  \cdot p \cdot (1 - p)}$
    \item
\end{itemize}

\section{Vorgehen beim klassischen Z-Test}

\begin{itemize}
    \item Vorraussetzung: $X ~ N(\mu, \sigma)$
    \item Nullhypothese: $\mu = \mu_0$ oder $\mu \geq \mu_0$ oder $\mu \leq \mu_0$.
          Hierbei wird die Standardabweichung und die Normalverteilungsannahme als sicher
          erwartet und der Mittelwert in Frage gestellt.
    \item Bestimme an Hnad der Standardnormalverteilungstabelle einen Wert, für den das
          Signifikanzniveau überschritten wird
\end{itemize}

\subsection{Beispiel}

Schokoladdentafeln, 100g Gewicht laut Verpackung

\begin{align*}
    X: \text{Gewicht der Schokoladdentafeln in gramm}  \\
    X ~ N(\mu, \sigma)                                 \\
    \text{Normale PRoduktionsabweichung }              \\
    \sigma = 1g                                        \\
    \text{Nullhypothese: } H_0: \mu = 100g             \\
    \text{Alternativhypothese: } H_1: \mu < 100g       \\
    \text{Signifikanzniveau: } 95\%                    \\
    \text{Da die Schokoladdentafeln}                   \\
    \text{Normalverteilt sind, reicht}                 \\
    \text{es eine Schokoladdentafel zu testen.}        \\
    \text{Eine Tafel wird zufällig}                    \\
    \text{ausgewählt und gewogen}                      \\
    \phi(1.96) \approx 97.5\%                          \\
    \Rightarrow \phi(-1.96) \approx 2.5\%              \\
    \text{Annahmebereich: } -1.96 \leq z \leq 1.96     \\
    \Leftrightarrow -1.96 + 100 \leq x \leq 1.96 + 100 \\
    \Leftrightarrow 98.04 \leq x \leq 101.96
\end{align*}

\subsection{Noch ein Beispiel}

\begin{align*}
    X = \begin{cases}
            1, & \text{falls eine Testperson die Riegel kennt} \\
            0, & \text{sonst}
        \end{cases} \\
    S_n = \sum_{i = 1}^{500} X_i ~ B(500, p)               \\
    H_0: p \leq 0.5                                        \\
    \rightarrow \text{Extremster Wert: } p = 0.5           \\
    E(S_n) = 500 \cdot 0.5 = 250                           \\
    \sigma(S_n) = \sqrt{125}                               \\
    \frac{S_n - 250}{\sqrt{125}} \approx N(0, 1)           \\
    \text{Nullhypothese wird nicht abgelehnt, falls}       \\
    \frac{S_n - 250}{\sqrt{125}} \leq 1.65                 \\
    \Leftrightarrow S_n \leq 165 \cdot \sqrt{125} + 250    \\
    \Leftrightarrow S_n \leq 268.45                        \\
    \text{beobachtetes Ergebnis:}                          \\
    S_n = 271 > 268                                        \\
    \text{Nullhypothese wird abgelehnt}
\end{align*}