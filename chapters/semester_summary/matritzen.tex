\chapter{Matrizen}

Eine Matrix ist eine rechteckige Anordnung von Zahlen oder anderen mathematischen Objekten, die in Zeilen und Spalten organisiert sind. Die Notation $M \in \mathbb{R}^{m \times n}$ wird hier verwendet, um eine Matrix zu beschreiben, welche $n$ Zeilen (Höhe) und $m$ Spalten (Breite) besitzt. Die einzelnen Elemente der Matrix werden mit $a_{ij}$ bezeichnet, wobei $i$ den Zeilenindex (von $1$ bis $n$) und $j$ den Spaltenindex (von $1$ bis $m$) angibt.

Eine solche Matrix $M$ mit $n$ Zeilen und $m$ Spalten hat die allgemeine Form:
\[
   M = \begin{pmatrix}
   a_{11} & a_{12} & \cdots & a_{1m} \\
   a_{21} & a_{22} & \cdots & a_{2m} \\
   \vdots & \vdots & \ddots & \vdots \\
   a_{n1} & a_{n2} & \cdots & a_{nm}
   \end{pmatrix}
\]
Eine Matrix kann auch als eine Zusammenfassung von Spaltenvektoren (wenn jede Spalte als Vektor betrachtet wird) oder Zeilenvektoren (wenn jede Zeile als Vektor betrachtet wird) angesehen werden.

\section{Matrix-Vektor-Produkt}

Das Produkt einer Matrix $M \in \mathbb{R}^{n \times m}$ (also $n$ Zeilen, $m$ Spalten) mit einem Spaltenvektor $V \in \mathbb{R}^{m \times 1}$ (also ein Vektor mit $m$ Elementen) ist ein neuer Spaltenvektor $W \in \mathbb{R}^{n \times 1}$ (also ein Vektor mit $n$ Elementen).
Die Multiplikation ist nur definiert, wenn die Anzahl der Spalten der Matrix $M$ gleich der Anzahl der Zeilen (Elemente) des Vektors $V$ ist. Hier müssen also beide die Dimension $m$ aufweisen.

Sei $M = \begin{pmatrix}
a_{11} & a_{12} & \cdots & a_{1m} \\
a_{21} & a_{22} & \cdots & a_{2m} \\
\vdots & \vdots & \ddots & \vdots \\
a_{n1} & a_{n2} & \cdots & a_{nm}
\end{pmatrix}$ und $V = \begin{pmatrix} v_1 \\ v_2 \\ \vdots \\ v_m \end{pmatrix}$.

Das Matrix-Vektor-Produkt $MV$ wird berechnet, indem das Skalarprodukt jeder Zeile der Matrix $M$ mit dem Vektor $V$ gebildet wird:
\[
   MV = \begin{pmatrix}
   a_{11}v_1 + a_{12}v_2 + \cdots + a_{1m}v_m \\
   a_{21}v_1 + a_{22}v_2 + \cdots + a_{2m}v_m \\
   \vdots \\
   a_{n1}v_1 + a_{n2}v_2 + \cdots + a_{nm}v_m
   \end{pmatrix} = \begin{pmatrix} \sum_{j=1}^{m} a_{1j}v_j \\ \sum_{j=1}^{m} a_{2j}v_j \\ \vdots \\ \sum_{j=1}^{m} a_{nj}v_j \end{pmatrix}
\]
Das Ergebnis ist ein Vektor mit $n$ Elementen.

\section{Diagonalmatrix}

Eine Diagonalmatrix ist eine quadratische Matrix, d.h. sie besitzt gleich viele Zeilen wie Spalten ($n=m$). Bei einer Diagonalmatrix sind alle Elemente außerhalb der Hauptdiagonalen (die von links oben nach rechts unten verläuft) gleich Null. Nur die Elemente auf der Hauptdiagonalen können von Null verschieden sein.

Eine Diagonalmatrix $D \in \mathbb{R}^{n \times n}$ hat die Form:
\[
D = \begin{pmatrix}
d_{11} & 0 & \cdots & 0 \\
0 & d_{22} & \cdots & 0 \\
\vdots & \vdots & \ddots & \vdots \\
0 & 0 & \cdots & d_{nn}
\end{pmatrix}
\]
Beispiel einer $3 \times 3$ Diagonalmatrix (hier ist $n=m=3$):
\[
   D = \begin{pmatrix}
   1 & 0 & 0 \\
   0 & 2 & 0 \\
   0 & 0 & 7
   \end{pmatrix}
   \in \mathbb{R}^{3 \times 3}
\]
Das Matrix-Vektor-Produkt mit einer Diagonalmatrix ist besonders einfach zu berechnen. Jedes Element des Vektors wird mit dem entsprechenden Diagonalelement der Matrix multipliziert:
Sei $V = \begin{pmatrix}v_1\\v_2\\v_3\end{pmatrix}$. Dann ist
\[
   D V = \begin{pmatrix}
   1 & 0 & 0 \\
   0 & 2 & 0 \\
   0 & 0 & 7
   \end{pmatrix}
   \begin{pmatrix}v_1\\v_2\\v_3\end{pmatrix}
   = \begin{pmatrix}1 \cdot v_1\\2 \cdot v_2\\7 \cdot v_3\end{pmatrix}
\]

\subsection{Einheitsmatrix}

Die Einheitsmatrix, oft mit $I$ oder $I_n$ (wobei $n$ die Dimension angibt) bezeichnet, ist eine spezielle Diagonalmatrix. Bei der Einheitsmatrix sind alle Elemente auf der Hauptdiagonalen gleich Eins, und alle anderen Elemente sind Null. Sie ist ebenfalls quadratisch.

Beispiel der $3 \times 3$ Einheitsmatrix $I_3$:
\[
   I_3 = \begin{pmatrix}
   1 & 0 & 0 \\
   0 & 1 & 0 \\
   0 & 0 & 1
   \end{pmatrix}
   \in \mathbb{R}^{3 \times 3}
\]
Die Multiplikation einer Matrix oder eines Vektors mit der Einheitsmatrix passender Dimension verändert diese nicht (d.h. $IM = M$ und $Iv = v$). Sie ist das neutrale Element der Matrizenmultiplikation.

\section{Drehmatrizen}

Drehmatrizen sind quadratische Matrizen, die verwendet werden, um Vektoren um einen Ursprung (in 2D) oder eine Achse (in 3D) um einen bestimmten Winkel zu drehen.

\subsection*{Drehmatrix in $\mathbb{R}^2$}
Eine Drehung eines Vektors in der Ebene ($\mathbb{R}^2$) um den Ursprung um den Winkel $\alpha$ (üblicherweise gegen den Uhrzeigersinn) wird durch die folgende $2 \times 2$-Matrix $R(\alpha)$ beschrieben:
\[
   R(\alpha) = \begin{pmatrix}
       \cos(\alpha) & -\sin(\alpha) \\
       \sin(\alpha) & \cos(\alpha)
   \end{pmatrix}
\]
Wird ein Vektor $v = \begin{pmatrix} x \\ y \end{pmatrix}$ mit dieser Matrix multipliziert, $v' = R(\alpha)v$, so ist $v'$ der um $\alpha$ gedrehte Vektor.

\subsection*{Drehmatrizen in $\mathbb{R}^3$}
Im dreidimensionalen Raum ($\mathbb{R}^3$) erfolgen Drehungen typischerweise um die Koordinatenachsen ($x, y, z$). Die entsprechenden Drehmatrizen sind $3 \times 3$-Matrizen.

\subsubsection*{Drehung um die $x$-Achse}
Eine Drehung um den Winkel $\alpha$ um die $x$-Achse wird durch die Matrix $R_x(\alpha)$ repräsentiert:
\[
   R_x(\alpha) = \begin{pmatrix}
       1 & 0 & 0 \\
       0 & \cos(\alpha) & -\sin(\alpha) \\
       0 & \sin(\alpha) & \cos(\alpha)
   \end{pmatrix}
\]

\subsubsection*{Drehung um die $y$-Achse}
Eine Drehung um den Winkel $\alpha$ (oder $\beta$ zur Unterscheidung) um die $y$-Achse wird durch die Matrix $R_y(\alpha)$ repräsentiert:
\[
   R_y(\alpha) = \begin{pmatrix}
       \cos(\alpha) & 0 & \sin(\alpha) \\
       0 & 1 & 0 \\
       -\sin(\alpha) & 0 & \cos(\alpha)
   \end{pmatrix}
\]

\subsubsection*{Drehung um die $z$-Achse}
Eine Drehung um den Winkel $\alpha$ (oder $\gamma$) um die $z$-Achse wird durch die Matrix $R_z(\alpha)$ repräsentiert:
\[
   R_z(\alpha) = \begin{pmatrix}
       \cos(\alpha) & -\sin(\alpha) & 0 \\
       \sin(\alpha) & \cos(\alpha) & 0 \\
       0 & 0 & 1
   \end{pmatrix}
\]

\subsection{Drehung um mehrere Achsen}
Wenn ein Vektor in $\mathbb{R}^3$ nacheinander um mehrere Achsen gedreht werden soll, können die entsprechenden Drehmatrizen miteinander multipliziert werden (siehte \nameref{matrix_multiplitzieren}). Das Ergebnis dieser Matrizenmultiplikation ist eine einzelne Matrix, die die Gesamttransformation (also die verkettete Drehung) darstellt.
Wird beispielsweise zuerst eine Drehung $R_1$ und danach eine Drehung $R_2$ auf einen Vektor $v$ angewendet, so ist der transformierte Vektor $v' = R_2 (R_1 v) = (R_2 R_1) v$. Die Gesamttransformationsmatrix ist $R_{ges} = R_2 R_1$.
Es ist wichtig zu beachten, dass die Matrizenmultiplikation im Allgemeinen nicht kommutativ ist, d.h. die Reihenfolge der Multiplikation (und somit der Drehungen) ist entscheidend ($R_2 R_1 \neq R_1 R_2$ im Allgemeinen). Die Matrix, die der zuerst auszuführenden Drehung entspricht, steht im Produkt rechts.