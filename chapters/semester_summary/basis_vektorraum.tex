\chapter{Basis von Vektorräumen}

Eine Basis eines Vektorraums $V$ ist eine Menge von Vektoren $\mathcal{B} =
    \{v_1, v_2, \dots, v_n\}$, die zwei grundlegende Eigenschaften erfüllen:
\begin{itemize}
    \item Die Vektoren in $\mathcal{B}$ sind linear unabhängig.
    \item Die Vektoren in $\mathcal{B}$ spannen den Vektorraum $V$ auf (d.h., jeder
          Vektor in $V$ lässt sich als Linearkombination der Vektoren in $\mathcal{B}$
          darstellen).
\end{itemize}
Jeder Vektor im Vektorraum $V$ kann eindeutig als Linearkombination der Basisvektoren dargestellt werden. Alle Basen eines gegebenen Vektorraums haben dieselbe Anzahl von Elementen. Diese Anzahl wird als die Dimension des Vektorraums bezeichnet, geschrieben als $\dim(V)$.

Ein Vektorraum besitzt im Allgemeinen unendlich viele verschiedene Basen
(sofern der zugrundeliegende Körper, wie z.B. $\mathbb{R}$, unendlich viele
Elemente enthält). Für den häufig betrachteten Vektorraum $\mathbb{R}^n$ bilden
die sogenannten Standardeinheitsvektoren $e_1, e_2, \dots, e_n$ eine spezielle
Basis, die als Standardbasis oder kanonische Basis bekannt ist. Beispielsweise
ist für $\mathbb{R}^3$ die Standardbasis gegeben durch $\left\{\begin{pmatrix} 1 \\ 0 \\ 0 \end{pmatrix}, \begin{pmatrix} 0 \\ 1 \\ 0 \end{pmatrix}, \begin{pmatrix} 0 \\ 0 \\ 1 \end{pmatrix}\right\}$.

Die Dimension des Vektorraums $\mathbb{R}^n$ ist $n$. Folglich muss jede Basis
des $\mathbb{R}^n$ aus genau $n$ Vektoren bestehen.
\begin{itemize}
    \item Eine Menge von mehr als $n$ Vektoren im $\mathbb{R}^n$ ist immer linear
          abhängig und kann daher keine Basis bilden. Beispielsweise können vier Vektoren
          im $\mathbb{R}^3$ keine Basis des $\mathbb{R}^3$ bilden, da sie zwangsläufig
          linear abhängig sind.
    \item Eine Menge von weniger als $n$ Vektoren im $\mathbb{R}^n$ kann den Raum
          $\mathbb{R}^n$ nicht vollständig aufspannen. Beispielsweise können zwei
          Vektoren im $\mathbb{R}^3$ höchstens eine Ebene aufspannen, aber nicht den
          gesamten dreidimensionalen Raum.
\end{itemize}

\section{Bedingungen für eine Basis im $\mathbb{R}^n$}

Eine Menge von Vektoren $\{v_1, v_2, \dots, v_k\}$ aus dem Vektorraum
$\mathbb{R}^n$ bildet genau dann eine Basis für den $\mathbb{R}^n$, wenn die
folgenden beiden Bedingungen erfüllt sind:
\begin{enumerate}
    \item Die Vektoren $v_1, v_2, \dots, v_k$ sind linear unabhängig.
    \item Die Anzahl der Vektoren $k$ ist gleich der Dimension des Raumes, also $k=n$.
\end{enumerate}
Es ist wichtig zu beachten: Wenn bekannt ist, dass $k=n$ (d.h., die Anzahl der Vektoren entspricht der Dimension des Raumes $\mathbb{R}^n$), dann ist die Bedingung der linearen Unabhängigkeit bereits ausreichend. Alternativ ist auch die Bedingung, dass die $n$ Vektoren den Raum $\mathbb{R}^n$ aufspannen, ausreichend. In der Praxis wird oft die lineare Unabhängigkeit von $n$ Vektoren überprüft.

\section{Beispiel: Überprüfung einer Basis im $\mathbb{R}^3$}

Um zu prüfen, ob die Vektoren $\left\{\begin{pmatrix}
        1 \\ 0 \\ 2
    \end{pmatrix}, \begin{pmatrix}
        3 \\ 2 \\ 1
    \end{pmatrix}, \begin{pmatrix}
        1 \\ 1 \\ 1
    \end{pmatrix}\right\}$ eine Basis des $\mathbb{R}^3$ bilden, müssen diese auf lineare Unabhängigkeit geprüft werden. Da es sich um drei Vektoren im $\mathbb{R}^3$ handelt, ist die Anzahl der Vektoren gleich der Dimension des Raumes ($n=3$). Somit genügt es, die lineare Unabhängigkeit nachzuweisen.

Dazu wird das homogene lineare Gleichungssystem $c_1 \begin{pmatrix} 1 \\ 0 \\ 2 \end{pmatrix} + c_2 \begin{pmatrix} 3 \\ 2 \\ 1 \end{pmatrix} + c_3 \begin{pmatrix} 1 \\ 1 \\ 1 \end{pmatrix} = \begin{pmatrix} 0 \\ 0 \\ 0 \end{pmatrix}$ betrachtet. Dies führt auf die Koeffizientenmatrix:

\begin{longtable}{p{10cm}}
    \hline
    \multicolumn{1}{c}{\textbf{Linearkombination / Gauß-Algorithmus}}                       \\
    \hline
    \endfirsthead

    \hline
    \multicolumn{1}{c}{\tablename\ \thetable\ -- \textit{Fortführung von vorheriger Seite}} \\
    \hline
    \multicolumn{1}{c}{\textbf{Linearkombination / Gauß-Algorithmus}}                       \\
    \hline
    \endhead

    \hline
    \multicolumn{1}{r}{\textit{Fortsetzung siehe nächste Seite}}                            \\
    \endfoot

    \hline
    \endlastfoot

    $\displaystyle\begin{matrix}
                          1 & 3 & 1 \\
                          0 & 2 & 1 \\
                          2 & 1 & 1
                      \end{matrix}$                                                             \\\hline
    Operation: III - 2I                                                                     \\\hline\pagebreak[0]
    $\displaystyle\begin{matrix}
                          1 & 3  & 1  \\
                          0 & 2  & 1  \\
                          0 & -5 & -1
                      \end{matrix}$                                                             \\\hline
    Operation: 2III + 5II                                                                   \\\hline\pagebreak[0]
    $\displaystyle\begin{matrix}
                          1 & 3 & 1 \\
                          0 & 2 & 1 \\
                          0 & 0 & 3
                      \end{matrix}$                                                             \\\hline
    Operation: 2I - 3II                                                                     \\\hline\pagebreak[0]
    $\displaystyle\begin{matrix}
                          1 & 0 & -1 \\
                          0 & 2 & 1  \\
                          0 & 0 & 3
                      \end{matrix}$                                                             \\\hline
    Operation: I + $\frac{1}{3}$III                                                         \\\hline\pagebreak[0]
    $\displaystyle\begin{matrix}
                          1 & 0 & 0 \\
                          0 & 2 & 1 \\
                          0 & 0 & 3
                      \end{matrix}$                                                             \\\hline
    Operation: II - $\frac{1}{3}$III                                                        \\\hline\pagebreak[0]
    $\displaystyle\begin{matrix}
                          1 & 0 & 0 \\
                          0 & 2 & 0 \\
                          0 & 0 & 3
                      \end{matrix}$                                                             \\\hline
    Operation: II : 2                                                                       \\\hline\pagebreak[0]
    Operation: III : 3                                                                      \\\hline\pagebreak[0]
    $\displaystyle\begin{matrix}
                          1 & 0 & 0 \\
                          0 & 1 & 0 \\
                          0 & 0 & 1
                      \end{matrix}$                                                             \\\hline
\end{longtable}

Da die reduzierte Zeilenstufenform der Koeffizientenmatrix die Einheitsmatrix
ist, hat das homogene lineare Gleichungssystem nur die triviale Lösung $c_1 =
    0, c_2 = 0, c_3 = 0$. Dies bedeutet, dass die Vektoren linear unabhängig sind.
Weil es sich um drei linear unabhängige Vektoren im dreidimensionalen Raum
$\mathbb{R}^3$ handelt (Anzahl der Vektoren = Dimension des Raumes), bilden sie
eine Basis des $\mathbb{R}^3$.