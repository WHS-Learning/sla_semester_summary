\chapter{Eigenwerte und Eigenvektoren}
Die Multiplikation einer quadratischen Matrix $A$ mit einem ihrer Eigenvektoren
$\vec{v}$ resultiert in einem skalierten Vielfachen des selben Vektors. Der
Skalierungsfaktor wird als Eigenwert $\lambda$ bezeichnet.

\section{Eigenwerte Berechnen}
Die Eigenwerte $\lambda$ einer Matrix $A$ werden durch das Nullsetzen der
Determinante von $(A - \lambda I)$ ermittelt.
\[
    \det(A - \lambda I) = 0
\]
Die resultierenden Nullstellen für $\lambda$ sind die Eigenwerte der Matrix.

\section{Eigenvektoren Berechnen}
Für jeden Eigenwert $\lambda_i$ wird der zugehörige Eigenvektor $\vec{x}$ durch
das Lösen des folgenden linearen Gleichungssystems gefunden:
\[
    (A - \lambda_i I) \vec{x} = \vec{0}
\]
Der Lösungsraum (Spannraum) dieses Systems stellt die Menge der Eigenvektoren
zum jeweiligen Eigenwert $\lambda_i$ dar.