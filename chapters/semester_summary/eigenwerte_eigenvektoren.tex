\chapter{Eigenwerte und Eigenvektoren}

Die Multiplikation einer Matrix mit einem Vektor ergibt einen Vektor. Für Quadratische Matritzen gibt es Vektoren, welche, wenn sie an eine Matrix multiplitziert werden, wieder den selben Vektor ergeben, sodass man diese nur mit einen Faktor multiplitzieren muss. Einen solchen Vektor nennt man Eigenvektor.

\section{Eigenwerte Berechnen}

Um die Eigenwerte zu berechnen, muss die Matrix $A$ mit der Einheitsmatrix mit $\lambda$ multiplitziert, subtrahiert werden $A - \lambda I$. Hiervon muss dann die Determinante gleich 0 gesetzt werden.

Die Resultierenden Ergebnisse sind die Eigenwerte.

\begin{align*}
    A - \lambda \cdot I
\end{align*}

\section{Eigenvektoren}